\documentclass[12pt]{article}
\usepackage[greek,main=german]{babel}
\usepackage[utf8]{inputenc}
\usepackage{csquotes}
\usepackage{blindtext}
\usepackage{titlesec}
\usepackage[right=2.5 cm, left=2.5 cm, top=2.5 cm, bottom=2.5 cm]{geometry}
\usepackage[onehalfspacing]{setspace}
\usepackage{ragged2e}
\usepackage[T1]{fontenc}
\usepackage{hyperref}
\usepackage{enumitem}
\usepackage{newtxtext}
\usepackage{setspace}
\setstretch{1.5}

\usepackage{blindtext}
\usepackage{fancyhdr}
\renewcommand{\headrulewidth}{0pt}
\usepackage[citestyle=verbose-ibid, bibstyle=authoryear, dashed=false, backend=biber]{biblatex}

\newcommand*{\zitatblock}[1]{%
    \begin{quote}
    \fontsize{10}{12}\selectfont
    \setlength{\parskip}{1.0em}
    #1
    \end{quote}
}


\addbibresource{literatur.bib}

\onehalfspacing
\begin{document}
\pagenumbering{gobble}
\begin{titlepage}
	\begin{center}
		\vspace*{1cm}
		
		\textbf{\LARGE Die \enquote{Zwei}-Welten in der platonischen Ideenlehre}
		
		\vspace{0.5cm}
		\large
		
		Bachelorarbeit\\
		zur Erlangung des akademischen Grades\\
		Bachelor of Arts (B.A.)\\
		im Fach Philosophie\\
		\vspace {1.5cm}
	\end{center}
	\vspace{2.5cm}
	\raggedright
		Universität Augsburg\\
		Philosophisch-Sozialwissenschaftliche Fakultät\\
		Institut für Philosophie\\
		
		\vspace{1.5cm}
		
		\begin{tabular}{@{}ll}
			\makebox[2.5cm][l]{eingereicht von:} & \hspace{2cm} Florian Wittek\\
			& \hspace{2cm} geboren am 5. März 1999\\
			& \hspace{2cm} in Augsburg\\
			& \hspace{2cm} 1617430\\
		\end{tabular}\\	
		\vspace{1.5cm}

		Erstprüferin/Erstprüfer: \hspace{2cm} Prof. Dr. Georg Gasser\\
		Zweitprüferin/Zweitprüfer: \hspace{1.4cm} Dr. Vorname Nachname

		\vfill
		Augsburg, den \hspace{2.4cm} XX. Monat 2023
		
	
\end{titlepage}
\newpage
\pagenumbering{gobble}
\section*{Vorwort}
Hier käme das Vorwort
\newpage
\tableofcontents
\newpage
\pagestyle{fancy}
\fancyhf{}
\fancyfoot[R]{\thepage}
\pagenumbering{arabic}
\setcounter{page}{1}
\justifying
\section*{Timaios Dialog}
Englisch:   http://classics.mit.edu/Plato/timaeus.html\\
Deutsch:    http://www.zeno.org/Philosophie/M/Platon/Timaios\\
Mit Zitation: http://www.opera-platonis.de/Timaios.pdf


\section*{Notizen}
Alt-Akademische Erledigung der okzidentalen Metaphysik:
Timaios 27c1-53c4 
Philebos 25a3-b1, 31c8-c11

Wichtige Stellen im Timaios 27c1-53c4
Es gebe ein Schönes an und für sich und ein Gutes und Großes (Phaidon 100b)
Nenne es nun Anwesenheit oder Gemeinschaft [\dots], dass vermöge des Schönen alle schönen Dinge schön werden. (Phaidon 100D)
Das ist schwierig in der Hinsicht, ob hier schon von den schönen Dingen gesprochen wird, also dass bereits in ihrem Sein die Dinge schön sind und erst durch die Idee des Schönen erkannt werden können, dass sie schön sind, weil es hier in dem Satz so strukturiert wurde, dass alle schönen Dinge erst schön werden, also voher eigentlich schon als schön verstanden werden dürften. 
Also die Unterscheidung der Erkenntnis des Schönen in oder an den Dingen gegen die ontologische Deutung, dass die Dinge erst dadurch existieren.

Wenn man den dialektischen Ansprüchen Platons gerecht werden möchte, dahingehend dass man diese bis zum Ende hindurchgeht und sich auch über die der Dialektik inhärenten Dialektik, also die Anwendung der Dialektik auf sich selbst, bewusst wird und dieser eben konsequent bleibt, so bleibt am Ende nichts anderes übrig, als dass es die Aufgabe ist, dass sich die Arbeit der Philosophie gerade damit beschäftigen müsste, dass sich die Dialektik mit dem Thema der Ideen als deren Werkzeug auseinandergehalten werden muss. Dabei gilt es also, dass sich auch in Verbindung mit dem Verständnis von Einem und Vielen ein einheitliches Bild zusammenführen lassen muss.\\
Dazu mehr bei Kutschera 2002 S.186f. \footcite[vgl.][S. 186f.]{Kutschera}
\zitatblock{\enquote{Im \emph{Sophisten} wird betont, dass wir in jeder Aussage Begriffe miteinander verknüpfen und dass es Aufgabe der Dialektik ist, zu untersuchen, welche Begriffe bzw. Ideen sich miteinander verbinden lassen und welche nicht. Es geht dabei um eine Untersuchung von Begriffsverhältnissen, und dazu gehört insbesondere auch die Teilhabe einer Idee an einer anderen.}\footcite[][S. 186f.]{Kutschera}}
Getrennt kann nur werden, was vorher zusammengesetzt war (Phaed. 92E, 78C)

We grenzt man die Arbeit von dem Thema der Unsterblichkeit der Seele ab, so dass man zwar das Thema behandel kann, aber nicht in die Frage nach dem Tod und der Schau der Ideen vor dem Eintritt in die Erfahrungswelt abzuschweifen und dennoch das Thema fruchtbar zu machen.

\section*{Literatur Notizen. Nicht eingeordnet}
\subsection*{Natorp Platons Ideenelehre}
Wichtig bei Gottfired Martin\nocite{NatorpIdeenlehre}
\subsubsection*{Der Staat}
Interne Gliederung nicht im Inhaltsverzeichnis: S. 183 Dialektische Begründung der Lehre von den drei Seelenteilen (436-441), S. 185 Direkte Einführung der Ideenlehre (475-486), S. 188 Die Idee des Guten (502-518), S. 201 Der Erkenntnisweg zum Unvedingten (521-534) Wissenschaften und Dialektik
Es fehlt der letzte Teil 596 im Unterkapitel
\zitatblock{\enquote{Im Phaedo wurde zwar schon alles Sein zuletzt begründet in den Grundlegungen des Denkens. Aber hier sollen wir uns gar etwas denken, das über beides,das Denken und das gedachte Sein hinaus liegt. Aber doch wiederum liegt es im Bereiche, in der Gattung des Denkbaren [\dots]. Es ist das Letzte zwar unter dem Erkennbaren (517B), nur eben noch zu erblicken, aber doch erblickt man es, und muss dann zu dem Schluss kommen, dass es der Grund ist von allem Rechten und Schönen, im sichtbaren Reich der erzeugende Grund des Lichts und des Herrn, der Sonne, im Reich des Denkens selbst als Herrscher Wahrheit und Vernunft verleihend}\footcite[][S. 191]{NatorpIdeenlehre}}
\enquote{Wie also ist es gleichwohl über das Sein und über das Denken hinaus? Jedenfalls insofern es das letzte begründende Prinzip des Seins wie des Erkennens ist.}\footcite[][S. 191]{NatorpIdeenlehre}
\enquote{Es vertritt, nicht eine (besondere) Setzung des Denkens, mithin nciht ein (besonderes) Sein noch eine (besondere) Erkenntnis, sondern die Denksetzung selbst, als letztbegründendes Prinzip alles besonderen Seins, aller besonderen Erkenntnis}\footcite[][S. 192]{NatorpIdeenlehre}
\zitatblock{\enquote{Nicht \emph{ein} letztes logisches Prinzip, sonder \emph{das} Prinzip des Logischen selbst und überhaupt, in welchem alle besondere Denksetzung und damit alles besondere Sein [\dots] zuletzt zu begründen ist. [\dots] Das Gesetz ist es allgemein, welches den Gegenstand konstituiert; dieses Gesetzt selbst, dass im Gesetz der Gegenstand zu begründen, ist somit übergegenständlich, auch über alle besonderen Gesetz, nicht ein, sondern \emph{das} Gesetz; woraus zugleich klar wird, inwiefern dies letzte Prinzip sogar über die Erkenntnis der Wissenschaft hinaus ist.}\footcite[vgl.][S. 194f.]{NatorpIdeenlehre}}
\subsubsection*{Parmenides}
Wenn man auf erkannte Dinge nicht verzichten will, dann schieben sich die Ideen an die Stelle der Dinge, womit aber die Methodenbedeutung, die Natorp in den Ideen sieht, verloren geht, womit man die reinsten Begriffe nicht aus, sondern an der Erfahrung gewinnen kann.\footcite[vgl.][S. 222f.]{NatorpIdeenlehre} \enquote{Wem nun die Erfahrung aufhörte Problem zu sein, wem also auch die Idee selbst sich nicht ferner am Problem der Erfahrung als Wissenschaft entwickeln konnte, dem musste die Ideenwelt erstarren zu etwas wie einer andern Welt gegebener Dinge[\dots]}\footcite[vgl.][S. 223]{NatorpIdeenlehre}
\enquote{Die Fehlmeinung, die as den Ideen Dinge macht [\dots]}\footcite[][S. 225]{NatorpIdeenlehre}
\subsection*{Reine Begriffshandbücher}
\enquote{Trennung und Teilnahme sind die zwei Hauptbegrife der Dialektik. Getrennt kann nur werden, was vorher zusammengesetzt war (Phaed. 92 E, 78 C)}\footcite[][S. 349]{Perls}
Erklärung zu Phaed. 66B: \enquote{Die doxa ist die Verbindung einer Wahrnehmung mit einer Idee. Also ist ihre eine Hälfte nicht ohne die körperliche Wahrnehmung möglich.}\footcite[][S. 350]{Perls}
\enquote{Allerdings hat sich Platon nirgends näher darüber geäußert, wie er die Teilhabe ontologisch im einzelnen aufgefasst hat.}\footcite[][S. 172]{Gigon75}
Ausdehnung der Relation zwischen Erfahrungswelt und Ideen über den Bereich der ethischen Begriffe hinaus, auf die Totalität der Erfahrungswelt. \footcite[vgl.][S. 172]{Gigon75} \enquote{Überall wo eine Vielheit ähnlicher Dinge oder Phänomene auf ein vorgeordnete Einheit hinweist, deutet er nun diese Einheit als Idee.}\footcite[][S. 172]{Gigon75}
In den Spätdialogen \emph{Sophistes, Politikos} und \emph{Philebos} werden nicht mehr die ontologische Dimension der Ideen betrachtet, sondern nur Allgemeinbegriffe, die lediglich die Geordnetheit der Erfahrungswelt aufgezeigt wird und nicht mehr unveränderliche und urbildlich für sich selbst bestehende Wesenheiten gemeint.\footcite[vgl.][S. 174]{Gigon75}

\zitatblock{\enquote{Die Frage nach der Möglichkeit des Wissensgewinns führt bei Platon also zu einem ontologischen und anthropologischen Dualismus: Aus der Auffassung, dass die Welt des sinnenfälligen Werdens keine sichere Erkenntnis vermitteln kann, folgert er die Gegebenheit eines welttranszendenten Bereichs rein idealer Wesenswahrheiten, die nur der geistigen Erkenntnis des Denkens zugänglich sind und von der Seele immer schon apriorisch gewusst werden.}\footcite[][S. 99]{ThurnerDualismus}}
Es wird von Platon das Wort Chorismos nie in Bezug auf Ideen und Einzeldinge verwendet. Er verwendet es lediglich innerhalb der Seelenlehre in der Bestimmung des Todes als \enquote{Erlösung und Trennung der Seele vom Leib} (Phaidon 67d)\footcite[vgl.][S. 282]{ThurnerTrennung} 
Thurner beginnt mit dem Tiamios Dialog (27d-47e) als Impuls für die zwei Welten. \footcite[vgl.][S. 283]{ThurnerTrennung}
\zitatblock{\enquote{Die Sinnendinge haben den Charakter von Abbildern, weil sie vom Weltbildner (dem DEMIURG) nach dem Vorbild (paradeigma) der rein geistigen Ideen (noêta) gestaltet worden sind (Ti 29a—31 b). Dies impliziert ein selbstständiges Sein dieser Urbilder vor und jenseits der Sinnendinge. Zwischen dem Bereich der rein intelligiblen Urbilder und ihrer sinnlich-materiellen Ähnlichkeiten vermittelt die „WELTSEELE" (psychê tu pantos), die vom Demiurgen durch eine Mischung von unveränderlich Unteilbarem und körperlich Teilbarem zusammengesetzt wurde (Ti 35a).}\footcite[][S. 283]{ThurnerTrennung}}
Fraglich wird allerdings, wenn man es so formuliert, dass \zitatblock{\enquote{[d]ie Unterscheidung zwischen dem geistigen Bereich idealen Seins und dem Bereich des Sinnlichen, das sich zwischen Sein und Nicht-sein befindet}\footcite[vgl.][S. 283]{ThurnerTrennung},} dann wird sehr schnell klar, dass es unmöglich ist, dass etwas zwischen Sein und Nicht-Sein zu verorten ist und damit völlig aus jeglicher Logik fällt. Dabei hat Thurner wohl im Sinn, dass das Sonnengleichnis verschiedene Seinsebenen skizziert. \zitatblock{\enquote{[\dots] eine Unterscheidung der Gesamtheit des Seins in zwei unterschiedlich bestimmte Seinsbereiche, die aber durch ein Urbild-Abbild-Verhältnis miteinander verbunden sind}\footcite[vgl.][S. 284]{ThurnerTrennung}} Diese Unterscheidung ist nicht auf die Erkenntnisstufen bezogen!
Es gibt eine Stelle in Pol. 477b, wo die Meinung als etwas zwischen Seiendem und Nicht-Seiendem gesetzt wird. Dies kommt aber daher, dass Erkenntnis auf das Seiende sich bezieht und Unkenntnis sich auf das Nichtseiende. Diese Behandlung des Nicht-Seienden und der Möglichkeit von falschen Aussagen (also auch Meinungen) kommt erst im Sophistes zu einem Abschluss, dass das Nicht-Seiende dennoch ist, in Abhängigkeit vom Seienden. Daher ist ein \enquote{Dazwischen} nicht zulässig. 
\enquote{Diese absolute Transzendenz des Guten bestimmt Platon näher, indem er am Schluss des Sonnengleichnisses über das Gute sagt, es sei \enquote{jenseits des Seins, dieses an Würde und Kraft überragend}(509b)}\footcite[][S. 284]{ThurnerTrennung}
\enquote{Das abschließende Höhlengleichnis veranschaulicht die anthropologische, vor allem das Leben der Philosophen betreffenden Konsequenzen der Platonischen Ontologie zweier Seinsbereiche}\footcite[][S. 284]{ThurnerTrennung} Dieser Punkt wird nicht weiter ausgeführt, also dass es zwei Seinsbereiche im Höhlengleichnis zu geben scheint. Wird hier der Unterschied im Höhlengleichnis zwischen den Dingen im gesamten Gleichnis und den entsprechenden Erkenntnisstufen gesetzte oder das drinnen und draußen von der Höhle unterschieden?
\zitatblock{\enquote{Nicht zuletzt im Sonnengleichnis wird deutlich, dass die Platonische Tendenz zur geistigen „Abtrennung" jener bleibenden, idealen Wesenseigenschaften, die auf der Stufe der unklaren Sinneserkenntnis noch als „unabgetrennt" (achöristos; vgl. Resp 524c) wahrgenommen werden, nur richtig verstanden wird, wenn man sie als die eine komplementäre Hälfte eines dialektischen Gedankenzusammenhanges sieht. Platon bringt dadurch zum Ausdruck, dass die idealen Prinzipien des Seins ihre Begründungsfunktion nur dann erfüllen können, wenn sie dem von ihnen Begründeten ontologisch überlegen sind. Paradoxerweise sind die Ideen nur dann und deshalb als TEILHABE-Ursache in den werdenden Sinnendingen anwesend (vgl. Phlb 26e; Ti 28a), weil sie diese zugleich überragen.}\footcite[][S. 284f.]{ThurnerTrennung}}
Dies ist nicht genau genug formuliert worden, sodass nicht klar ist, was hiermit gemeint sein soll und wie es schlussendlich gemeint und zu verstehen sein soll.

methexis/Teilhabe der Dinge an den Ideen


Nochmals einsehen Platonisches Philosophieren 70/CD 3067 K89

\subsection*{David Ross: Plato's theory of ideas (1951) 70/CD 3067 R823}
Ideenkritik im Parmenides ist eine Selbstkritik Platons (aus Martin S.151) S. 84



\subsection*{Rethinking Plato and Platonism 70/CD 3067 W878 R4 }
Vielversprechend

\subsection*{Platons Philosophie II Kutschera 70/CD 3067 K97-2}
Parmenides Kapitel S.164ff
\subsubsection*{Vorhaben Kutscheras}
6.2 und 6.3 Eingehen auf ersten Teil des Parmenides. 6.4 Überleitung zum zweiten Teil. 6.5 Grundgedanken seiner Interpretation des zweiten Teils. 6.7 Probleme mit seiner Interpretation.\\
\enquote{Der \emph{Parmenides} ist neben dem \emph{Timaios} [\dots] sicher jener Dialog, der am schwierigsten zu verstehen ist.}\footcite[][S. 161]{Kutschera}
Der Einstieg in den zweiten Teil des Dialogs ist, dass der junge Sokrates behauptet, dass die Paradoxien von Zenon sich auflösen ließen, wenn man Ideen von ihren empirischen Instanzen unterscheide.\footcite[vgl.][S. 161]{Kutschera}
Es muss sich erst noch weiter in der Philosphie geübt werden, was auch an der Übung getan wird, dass Parmenides dies an dem Beispiel des Einen tut. Dabei stellt Parmenides aus der Existenz des Einen, wie aus der Nicht-Existenz des Einen eine Kette von Widerspruchen ab. Es folgt: \enquote{Es selbst (Das Eine) wie die anderen sind, sowohl für sich wie in Beziehung aufeinandner, ales auf alle Weise und sind es nicht, und scheinen es zu sein und scheinen es nicht zu sein.}\footcite[vgl.][S. 162]{Kutschera}

Kutschera setzt S1: \textbf{Sind F und G gegensätzliche Eigenschaften, so gilt nicht: F hat die Eigenschaften G}
Seite 169. Dem Gedanken des Chorismos liegt zunächst die Unterscheidung der Ideen von ihren empirischen Instanzen zugrunde (130b2-3) (Parmenides). Mit Verweis auf Eutyphron 5d1-2 
\enquote{Für Platon, für den Ideen Gegenstände waren, gilt [eine Idee ist verschieden von all ihrem empirischen Instanzen], weil eine Idee das ihren empirischen Instanzen Gemeinsame repräsentiert und der Grund für deren Sosein ist.}\footcite[vgl.][S. 169]{Kutschera} Somit S2:\textbf{Eine Idee ist verschieden von all ihren empirischen Instanzen} und S3:\textbf{Die Existenz und Beschaffenheit der Ideen hängt nicht von der Existenz und Beschaffenheit empirischer Dinge ab.} Unabhängikeit der Ideen vom Entstehen und Vergehen ihrer körperlichen Instanzen.\footcite[vgl.][S. 167]{Kutschera}

Ausschlaggebend ist im Parmenides die Stelle 133c7 ff.\footcite[vgl.][S. 179]{Kutschera}
 



\section{Einleitung}
Ich gehe davon aus, dass es kein Chorismos zwischen Ideen und Sinnesdingen gibt. Sofern damit gemint ist, dass es zwei Welten gibt, in denen jeweils Ideen und Dinge sich aufhalten. 
Man könnte hier einen Anfang setzten, wenn man ganz einfach beginnt, also damit, dass mit den Basics angefangen werden muss.\\
Die Transzendenz in Platon(s Dialektik) 
Die Frage nach dem Einen, den Ideen\footcite[][]{Staudacher} und den das Seiende übersteigende Ideen, folglich transzendent.\footcite[][]{Bordt}
Womöglich auch damit in Verbindung die Dialektik, dass die verschiedenen Stufen der Ideen und der Sinnesdingen überwunden werden, bis hin zur obersten Stufe, welche damit dann auch noch überwunden werden kann, oder auch nicht, da die Prinzipien der Dialektik nicht mehr auf diesen Bereich anwendbar sind oder eben schon noch, aber bei der nächsten Stufe dann nicht mehr?\\
Siehe dazu\footcite[vgl.][S. 104ff]{Hirschberger}

Es kann nur eine Welt geben, nicht mehrere nebeneinander (Tim. 31a2-31b2)
unterstützt von Vogel in der Weise, dass er festhält, \zitatblock{\enquote{As for present-day philosophy, it shows the tendency to eliminate the existence of a \enquote{transcendent} reality, in so far as this is meant to be a reality, existing \enquote{somewhere beyond} the world in which we live: There is one reality only, this one here and now. Certainly, it can be analysed into its intelligible forms, and this may be called a metaphysic of immanency}\footcite[][S. 161]{Vogel}}
35c sehr wichtig!!!
\zitatblock{\enquote{There are four grounds on which Plato is usually qualified as a dualist: (1) his position in metaphysics, usually refered to as the two worlds theory; (2) his radical antithesis of soul and body, as it is commonly understood; (3) the doctrine of two ultimate principles, which he appears to have held at least in his later years; (4) the so-called cosmic dualism, attributed to him by early Christian writers and still ascribed to him by some present-day scholars}\footcite[][S. 159]{Vogel}}
Die Forderung von Vogel ist sehr hoch angesetzt, denn \enquote{[w]hat i \emph{do} want is to give to our philosophers a correct view of what Plato held, and thus, by means of a true picture, contribute something to prevent misunderstandings and clear the way to a true metaphysic}\footcite[vgl.][S. 161]{Vogel}
Es wird wohl schwierig sein, den Gedanken, dass die Seele in der Transzendenz die Ideen bereits geschaut hat, zu verbinden, da hier das Moment der Transzendenz nicht aufgelöst werden kann, ohne auch noch die Seelenlehre zu betrachten.
Die zugrunde liegende Frage ist eigentlich, wie kann die Idee des Schönen, Guten, Gerechten an den schönen, guten, gerechten Dingen teilhaben?\footcite[vgl.][S. 16]{Martin73}

Die Aufteilung im Liniengleichnis muss so aufgefasst werden, dass der größte Teil dem untersten Teil zugesprochen werden muss. Dies macht daher Sinn, da sich das Liniengleichnis auf einen Zielpunkt hin konstruiert, die Idee des Guten. Bei einer Betrachtung von oben herab also muss gelten, dass sich von dem obersten Punkt aus, also der Spitze her - wie eine Art Definitionsbaum - immer mehr Varianten darunter fallen (Man siehe hierzu auch die Dihairese im Sophistes), dieses Prinzip auch für den weiteren Weg \enquote{nach unten} gelten muss, sodass am Ende eine sehr viel breitere Basis besteht und der große Teil des Liniengleichnisses auf der untersten Ebene zu verorten ist. Hiermit entsteht also eine pyramidenähnliche Form. Pyramidenähnlich daher, weil der mittlere Teil nicht ganz einer perfekten Pyramide entsprechen würde, da die zweite und dritte Stufe im Liniengleichnis die gleiche Fläche zukommen müsste. Dies wird weiterhin dadurch unterstützt, dass im Höhlengleichnis am Grunde der Höhle die Schatten an der Höhlenwand jeweils anders interpretiert werden können und damit eine unzählbare Vielheit an Möglichkeiten besteht. Außerdem wird durch jedes Wackeln Zucken des Feuers, welches das Licht auf die Gegenstände wirft, die wiederum den Schatten auf die Höhlenwand werfen, umso zahlreicher. Dabei noch nicht beachtet, dass das Wenden, Drehen und Zusammensetzten der Gegenstände von denjenigen, welche die Gegenstände hinter der Mauer her tragen, nochmals die Zahl der Möglichkeiten anhebt.\\
Diese Ansichtsweise findet ebenfalls Halt, wenn man sich den pythagoreischen Tetraktys ansieht und die im platonischen Denken stark vertretene Dialektik des Einen und Vielen, welche vom Einen beginnt und in das Viele mündet.\\
Problem bei dieser Arbeit wird sein, dass ich bereits ein eigenes Verständnis von dem habe, wie Platon zu lesen sein sollte, oder eben wie Platon zu verstehen und auszulegen ist. 

Noch bevor man die Ideenlehre angehen könnte, noch bevor man sich dem widmen kann, was mit wem in Verbindung steht oder stehen kann und was nicht, muss sich über die Verhältnisse von Einem und Vielem in aller ihrer Formen zugewandt werden.
Die eigentliche Frage, die es zu lösen gilt ist die von der Möglichkeit von unveränderlichen Dingen, die für ales Werdende den Grund angeben. Wo ist der Baum, wenn er gefällt wird, verarbeitet wird, verbrannt wird. Dies ist vermutlich eher eine aristotelische Frage, als eine platonische, wenn man dies so formuliert.
\subsection*{Platon Handbuch Horn Müller Söder}
Zu Transzendenz S. 347ff\\
Verschiedene Arten von Aufstiegen. Aufstieg zu etwas Hinreichendem im \emph{Phaidon} (101d5-e1), der Aufstieg zur Idee des Schönen in der Diotima-Rede des Sokrates im \emph{Symposion} (211b5-d1), der Aufstieg zum nicht-vorausgestzten Anfang im Liniengleichnis der \emph{Politeia} (VI 511b3-7); der Aufstieg zur Idee des Guten im Höhlengleichnis der \emph{Politeia} (VII 515c6-516b7); der Aufsieg zum über-himmlischen Ort im Seelen-Mythos des \emph{Phaidros}(246d6-248b5). All diese Aufstiege schließen das Transzendieren ihrer jeweiligen Anfangs- und Zwischenstationen ein. (Aus dem Lateinischen \emph{transcendere} \enquote{übersteigen}, \enquote{überschreiten}) 
Damit ging die Annahme voraus, dass Platon \enquote{Philosophie als Transzendieren} (Hafwassen 1998) porträtiere. 
Es werden in der Regel den Ideen im Allgemeinen (als Entitäten) oder der Idee des Guten im Besonderen Transzendenz zugesprochen.\\
Raum-Zweittranszendenz und im Verhältnis zu ihren sinnlich wahrnehmbaren Partizipanten. Der Idee des Guten wird speziell noch Seinstranszendenz zugeschrieben.\footcite[vgl.][S. 347]{StrobelTranszendenz}
\enquote{[\dots], dass Ideen nicht räumlich lokalisert werden können und die Prädikate, die auf sie zutreffen, von Zeitbezügen frei sind.}\footcite[vgl.][S. 347]{StrobelTranszendenz}
Es bleibt die Frage nach der Transzendenz gegenüber den sinnlich wahrnehmbaren Partizipanten. Es hängt davon ab, wie man diese These verstehen mag.\\
\emph{Nicht-Immanenz}: Eine gegebene Idee \emph{F} ist nicht in/an den Sinnendingen, die F sind.\\
\emph{Unabhängige Existenz}: Eine gegebene Idee \emph{F} kann existieren, ohne dass ein Sinnesding, das F ist, existiert, aber umgekehrt kann kein Sinnending, das F it, existieren, ohne dass die Idee \emph{F} existiert\footcite[][S.348]{StrobelTranszendenz}

Es bedarf einer eindeutigen und ausführlichen Erläuterung der Fragestellung, um diese an dem Rest des Textes erarbeiten zu können. Es geht dabei darum, dass die Ausgangsfrage im Grunde alle Aspekte der platonischen Philosophie unter sich fasst, sodass man alles - ausgeklammert sei die Ethik - behandeln müsste.
\section{Zwei-Welten-Theorie}
Überleitung zum Teil dessen, dass es im Höhlen- und Liniengleichnis zu dem Verstädnis kommt, dass man in vielen Lehrwerken von zwei Welten spricht, da es einfacher ist dies so darzustellen. Es muss sich aber hier dem Thema zugewandt werden, ob es sinnvoll ist von diesen zwei Welten sprechen zu können. Bzw. zu welchem Grad dies möglich ist oder auf welcher Ebene man von zwei Welten sprechen kann und ab welchem Punkt nicht mehr. Es gilt daher diese beiden Positionen deutlich voneinander zu trennen und darzustellen, um sich dann einer der beiden näher zu widmen.

\section{Darstellung der ersten Theorie (Wie wird von zwei Welten gesprochen?)}
Es gibt keinen direkte Autor, der eine wirkliche Zwei-Welten Theorie in die Ideenlehre hineinliest, aber der Einstieg in die Darstellung der Lehre wird meistens immer so gemacht, dass man Spielraum für Interpretation hat und eine Verwendung von Zwei Welten einfacher ist und die Darstellung abkürzt 
In Büchern wie von Jörg Disse wird Platon so dargestellt, als würde man von zwei getrennten Welten sprechen\footcite[vgl.][S. 22 und 28]{DisseMetaphysik} (Ist nicht eingescannt)
\enquote{Die Ideen bilden bei Platon einen von der Sinnenwelt eindeutig getrennten Wirklichkeitsbereich}\footcite[][S. 31]{DisseMetaphysik}
Es wird jedoch versucht zu unterscheiden zwischen der ersten Bedeutung der Trennung (chorismos) als eine räumliche Trennung. Allerdings wird daraufhin lediglich festgestellt, dass Platon keine weiter Erklärung diesbezüglich liefert, da er auch nicht mit den Begriffen der Transzendenz und Immanenz hantiert.\footcite[vgl.][S. 34f]{DisseMetaphysik} Man muss aber zu Gute halten, dass das Problem erkannt wird, wenn man von der Transzendenz der Ideen im Gegensatz zur raumzeitlichen Welt spricht.\footcite[vgl.][S. 35]{DisseMetaphysik} Also dass nicht mehr über etwas gesprochen werden kann, wenn man diese Sache außerhalb des logischen Raumes verortet, wenn man ihr die absolute Transzendenz zuspricht.\\
\textbf{Was sind die Stellen im Original?}\\
Dies liegt wohl an dem heutigen Verständnis davon, wenn man von Welten - also auch zwei Welten - spricht. Es wird dieses Verhältnis lediglich als räumlich vorgestellt, was automatisch zu der falschen Annahme von zwei Welten neben- oder übereinander führt.
Das vermutlich ursprüngliche Problem hierbei liegt daran, dass man in der Behandlung des Themas folgendermaßen beginnt: \enquote{Den apriorischen Begriffen unseres Geistes korrespondieren entsprechende Gegenstände. Diese Gegenstandswelt interessiert Platon ebenso wie die Frage nach der Quelle der Wahrheit.}\footcite[][S. 97]{Hirschberger}
Was hier deutlich wird, ist die chronologisch verdrehte Betrachtungsweise dieses Problems. Das heißt, dass aus Sicht Kants und auch mit kantischen Begriffen herangegangen wird, um platonische Philosophie zu beschreiben und zu erklären. Diese These bedarf weitaus mehr Erklärung, die aber nicht geliefert wird. Es wird sich damit begnügt, dass Platon hier so erklärt wird, dass bei Kant nur die Formen a priori sind, bei Platon auch die Inhalte und Platon damit als reiner Rationalist bezeichnet wird.\footcite[vgl.][S.96]{Hirschberger} 
Nochmal ansehen, was davor geschrieben wird, da es heißt, \enquote{Nur ein mangelder Metaphysik- und Transzendenzbegriff - \enquote{Metaphysik}: das schlechthin unzugängliche \enquote{Jenseitige}- führt zu der Zweiweltentheorie eines totalen Chorismos, wo in Wirklichkeit nur ein modaler gemeint war, eine \enquote{Trennung} des Seins nach seinem Wesen in Gegründetes und Gründendes. Es ist eine Modifizierung, der es ebensosehr auf die Trennung wie auf die Einheit ankam}\footcite[][S. 96]{Hirschberger}
Ähnliches findet sich auch bei Thurner:\zitatblock{\enquote{Die Frage nach der Möglichkeit des Wissensgewinns führt bei Platon also zu einem ontologischen und anthropologischen Dualismus: Aus der Auffassung, dass die Welt des sinnenfälligen Werdens keine sichere Erkenntnis vermitteln kann, folgert er die Gegebenheit eines welttranszendenten Bereichs rein idealer Wesenswahrheiten, die nur der geistigen Erkenntnis des Denkens zugänglich sind und von der Seele immer schon apriorisch gewusst werden.}\footcite[][S. 99]{ThurnerDualismus}}
Zitat aus Phaidon 79d: Die Seele verhält sich zu jenen Gegenständen immer in derselben Weise, da sie eben damit etwas erfasst, das selbst auch von dieser Art ist. Diesen Gegenständen komme es zu, \enquote{niemals in keiner Weise, irgendwie auch nur die geringste Veränderung zu erleiden. (Phaidon 78d)}\footcite[][S. 97]{Hirschberger}
Hierin liegt der Grund, dass nicht scharf genug differenziert wird und die Definition nicht eingehalten wird. Also die verschiedenen Ebenen oder Stufen der Ideen, die jeweils als eigenständig also absolut unveränderlich und ewig gelten und dann in den Bereich des Denkens eintreten und somit nicht mehr die absolute Veränderlichkeit aufweisen können dadurch aber gedacht werden können und eben in den Objekten anwesend sein können.\footcite[vgl.][S. 180f.]{Kutschera}
Es wird von ihm das Verständnis gezeichnet, dass die Ideen und die Welt der Ideen eine \enquote{Ideale Wirklichkeit} darstellen und auch nach dem möglichen Vergehen der materiellen Welt immer noch existieren mögen, da man feststellen kann, dass die gelieferten Beispiele von mathematischen Ideen die Frage nach dem Anfang ihrer Existenz nicht beantworten können.\footcite[vgl.][S. 99]{Hirschberger} Keine Verweise auf das Original. Dabei auch schwierig ist folgende Formulierung:\enquote{Ferner bilden sie die obersten Strukturpläne der Welt, ohne ihrerseits davon abhängig zu sein. Sie sind das Sein des Seienden.}\footcite[][S. 99]{Hirschberger} Diese Zuschreibung ist eigentlich nur der Idee des Guten vorbehalten. Die Frage ist, wenn man dieser Formulierung kurz nachgehen würde, wären nur die Sinnesdinge Seiend, die Ideen hingegen nicht.
Direkt im Anschluss wird damit auch den Dingen die Möglichkeit abgesprochen den \enquote{reinen Wert und Wesen selbst} zu erreichen.\footcite[vgl.][S. 100]{Hirschberger} (Phaidon 75b)
Gänzlich unbrauchbar wird diese Ansicht, als Hirschberger es wie folgt formuliert:
\zitatblock{\enquote{Auch wegen dieser unausschöpfbar reichen, zeugenden Fruchtbarkeit ist die Ideenwelt die stärkere Wirklchkeit. Darum unterscheidet also Platon die Ideenwelt [\dots] von der sichtbaren Welt [\dots] und erblickt nur in jener die wahre und eigentliche Welt, in dieser aber bloß ein Abbild, das in der Mitte steht zwischen Sein und Nichtsein.}\footcite[vgl.][S. 100]{Hirschberger}}
Dies ist nicht zulässig so stehen zu lassen. Egal wie man es ausdrücken möchte. Wenn eine Welt - egal ob sinnlich oder geistig - zwischen Sein und Nichtsein steht, dann kann man dieser Welt überhaupt nichts abgewinnen, also nicht einmal eine Aussage über diese Welt treffen, wenn ihr Sein \emph{und} Nichtsein abgesprochen werden. 
Gerade mit dieser Formulierung würde diese Welt aus aller Wirklichkeit fallen und nicht exsitieren können.
Es wird jedoch eingräumt, dass es sich nicht um eine total voneinander getrennt Zweiweltentheorie handelt, da es für Platon eine Einheit des Seins gibt,\footcite[vgl][S. 100]{Hirschberger} allerdings kann damit nicht das vorher aufgestellt revidiert oder richtig eingeordnet werden.\\
Eine interessante Sache, die von Hirschberger hervorgebracht wird, ist die Unterscheidung von zwei Möglichkeiten der Diairesis entweder von oben nach unten, wie es im \emph{Sophistes} durchgeführt worden ist, aber auch von unten nach oben, indem man das Allgemeine aus dem Individuellen heraushebt, um schlussendlich an dem obersten Absoluten anzukommen.\footcite[vgl.][S. 106f.]{Hirschberger} 
Es geht mit dieser Dialektik darum, dass es um die Erklärung des gesamten Seins durch Aufweis der Strukturidee der Welt geht.\footcite[vgl.][S. 107]{Hirschberger}
\enquote{Und schließlich geht es in ihr, sofern sie das ganze Sein zusammenschaut und in ihm überall die Parusie der Idee des Guten entdeckt, um den Nachweis der Fußspur Gottes im All.}\footcite[][S. 107]{Hirschberger}
\enquote{So ist für Platon Dialektik im eigentlichen Sinn viel mehr als nur Logik, sie ist immer Metaphysik und wird als solche zugleich zur Grundlage der Ethik, Pädagogik und Politik}\footcite[][S. 108]{Hirschberger}
Oder wie es Burgin zusammenfasst: \zitatblock{\enquote{In his theory, Plato assumed that the physical world was the sensible realm, as people could grasp it with their five senses, while the world of Ideas was the intelligible realm, as people could comprehend it only with their intellect.}\footcite[][S. 179]{Burgin}}

\subsubsection*{Gottfried Martin: Platons Ideenlehre (1973) (70/CD 3067 M381) (Bereits eingescannt und auf Festplatte)}
37 Zwei Weisen des Seins (Phaidon 79A)
\enquote{Auch hier setzt Platon mit der ausdrücklichen Formulierug ein: zwei Weisen. (Politeia 509D Liniengleichnis)}\footcite[][S. 39]{Martin73}
Die Frage ist die Unterscheidung und die unterschiedliche Bedeutung von eidos und idea, welche von Martin nicht als strikt voneinander trennbar aufgefasst wird.\footcite[vgl.][S. 39]{Martin73} in Anlehnung an Ritter.\\
Unterscheidung der Ideen und Dingen in ihrer Erfahrbarkeit, also sichtbare Dinge und unsichtbare nur denkbare Ideen. 40 (Parm. 130B) Zweite Unterscheidung vom Sein her. Ideen sind unvergänglich, Dinge sind unter ständigem Wandel begriffen. \footcite[vgl.][S. 41]{Martin73}
42 Die Ideen sind ewig, immerwährend. Wie sind sie dann im Sophistes in Bewegung. Das geht wohl damit einher, dass man die Ideen als denkbar verstehen muss, also auch verstanden werden müssen. Dabei muss das Element der Bewegung bzw. Veränderung auftreten, da man sonst nicht von einer Erkenntnis sprechen kann. Das Ding (auch die Idee) kann erst dadurch als erkannt bezeichnet werden, wenn man eine Veränderung der Idee von nicht erkannt zu erkannt festlegen kann. (Phaidros 247e, Sophistes 240b, 246b, 248a, 249d)
Die Zweiweltentheorie (bei Platon) S. 89ff.
Diese Zweiweltentheorie wird so dargestellt, dass der Transzendenzgedanke im Phaidros (Götterreise), Symposion, Phaidon (Unsterblichkeit der Seele) und Politeia an den Mythoserzählungen angelegt wird und damit diese Überweltlichkeit als eine Zweiweltlichkeit interpretiert wird.\footcite[vgl.][S. 88ff.]{Martin73}
\enquote{Die Seele trennt sich im Tode vom Leib. Sie wird gerichtet werden. Ihr Ziel ist der Weg nach oben. Dies darf nicht in einem humanitären Entwicklungssinn als eine Höherentwicklung verstanden werden, sondern es meint in der Tat die dort oben liegende Welt des reinen Seins.}\footcite[][S. 93]{Martin73}
\enquote{So erweisen sich auch unter diesem Gesichtspunkt die vier großen Ideendialoge, der Phaidros, das Symposion, der Phadion und die Politeia als einheitlich. Die Welt hier unten ist nicht die wahre Wirklichkeit, die wahre Wirklichkeit ist die Welt der Ideen dort oben. Ich darf wiederholen: Ich will nicht sagen, dass dies Platons eigentliche Überzeugung und dass diese Zweiweltentheorie das letzte Wort Platons ist. Man muss sich vielmehr vor Augen halten, dass diese räumliche Darstellung des Unterschieds zwischen Ideen und Dingen eine Notwendigkeit unseres Denkens und Sprechens ist.}\footcite[][S. 96]{Martin73}
Natorp S. 75 warum Aristoteles Platon falsch zu verstehen scheint, da er alles für Bare Münze hält (158) Bei Natorp Fußnote 142 Phaidon, 231 Parmenides, 278ff Sophistes
\enquote{Lässt man scih von einer naheliegenden Bedeutung des Wortes \enquote{Metaphysik} leiten und versteht man unter Metaphysik die Lehre von dem, was hinter der Natur liegt, dann ist die Ideenlehre des Phaidon der Anfang und der Urspurng der Metaphysik. Heir werden [\dots] die Idn als etwas verstanden, was hinter und über der Natur liegt.[\dots] Was ist dies transzendente Sein der Ideen?}\footcite[vgl.][S. 128]{Martin73}
Nimmt man die Lehre von den transzendenten Realitäten als Ziel einer Philosophie, dann sind Phaidon und Phaidros Anfang und Ursprung. Wenn man aber die Frage: Was ist Sein? nimmt, dann wird der Sophiestes zu einem zentralen Dialog\footcite[vgl.][S. 129]{Martin73}
Probelmatisch wird es dann wieder, wenn es heißt \enquote{Erst nach dieser Entdeckung der Ideenlehre kann Platon das Sein in zwei Weisen des Seins differenzieren, und wir hatten die ausdrückliche Differenzierung in Phaidon (79a) und in der Politeia (509d) gefunden.}\footcite[][S. 131]{Martin73}
Es wird aber dann eingeräumt, wenn man behauptet die Ideen sind und sie alleine sind das Sein, nicht zulässig ist, da somit keine Dinge sein könnten. Daher muss es irgendeine Form geben, bei der den Dingen ein Sein zukommt, auch wenn es ein leicht abgeleitetes Sein ist.\footcite[vlg.][S. 131]{Martin73}
\zitatblock{\enquote{Nach [der Anamnesislehre] ist Erkenntnis immer eine Wiedererinnerung. Das heißt doch, und Platon sagt dies auch ausdrücklich, daß die Seele die Ideen in einem früheren Leben vor der Geburt kennengelernt haben muß. Dies wiederum kann doch nur heißen, daß die Ideen nicht in dieser Welt sind}\footcite[][S. 160]{Martin73}}
Diese Auffassung ist nicht haltbar. Dies hat damit zu tun, dass mit dieser Ansicht einige Implikationen einhergehen müssen, auf die wir keinen Zugriff oder keinen Zugang haben. Was ich damit meine ist, dass man, wenn man davon ausgeht, dass die Seele die Ideen vor der Geburt in einer \emph{anderen} Welt kennengelernt oder geschaut hat, dass wir eigentlich nichts über diese Welt aussagen können. Dabei bedarf es der Annahme, dass wir entweder sagen, man kann nur etwas über diese Welt aussagen, das haltbar bleibt, wenn wir eine nochmals übergeordnete Welt annehmen, die die Ideen und Sinneswelt verbindet, auf die allerdings wiederum zugegriffen werden kann, da sonst keine logsichen Schlüsse und Aussagen möglich wären. Oder man muss in irgendeiner Weise beweisen können, dass wir einen logischen Zugang zu diese anderen Welt aufweisen, wodurch man im eigentlichen Sinne durchaus Aussagen treffen könnte, dies aber nicht getan wird. Zumindest nur bis zu diese Punkt.
Dies ist sehr eng an das Konzept des Einen und Vielen geknüpft, ergo der Dialektik. Der Wissenschaft des Auseinanderhaltens und Zusammenführens von Begriffen.

Natürlich muss man diese beiden Bereiche dahingehend differenzieren und eine gedankliche Grenze setzen, da nur so von beiden in ihrer isolierten Form gesprochen werden kann. Nur dadurch ist es möglich, dass die beiden Bereiche aufeinander Auswirkung und überhaupt Wirkung haben.
Die Frage dabei lauter also von Martin, wie das CHORIS im griechischen gebraucht wird. Entwerder eine rein räumliche Trennung, was gerade dann für das Problem sprechen würde, oder ob auch andere Trennungen oder Teilungen/Sonderungen damit gemeint sein könnten. (S. 163ff.)
Die andere Möglichkeit ist die begriffliche Trennung von Dingen, die wohl am meisten von Platon vertreten wird\footcite[vgl.][S. 165]{Martin73}. Beispiele: Buch 10 der Politeia: 595A, Laches 195A, Eutydemos 289C
Es liegt bei Platon faktisch, bei Aristoteles dann explizit ein räumlicher, ein zeitlicher und ein begrifflicher Chorsimos vor.\footcite[vgl.][S. 166]{Martin73}
\enquote{In der Kritik der Ideenlehre geht Aristotles immer davon aus, dass der Chorismos bei Platon immer rein räumlich gemeint ist.}\footcite[][S. 166]{Martin73}
Räumliche Trennung der Begriffe im Höhlengleichnis auf welcher Ebene? Es lassen sich mehrere Ebenen aufzeigen. Siehe Liniengleichnis. Die Trennlinie ist scharf, aber bis zu welchem Grad darf man die Auflösung des Schärfegrades, mit dem man das Bild betrachtet, drehen, bis man den Blick dafür verliert, dass es sich alles dennoch in einer Welt abspielt, denn wie könnte dann der Aufstieg überhaupt erfolgen.
Es stellt sich bei der Methexis die Frage, \enquote{ob es sich um eine Teilnahme der Dinge an den Ideen oder um eine Anwesenheit der Ideen in den Dingen oder um eine Gemeinschaft der Ideen und der Dinge handelt.}\footcite[vgl.][S. 170]{Martin73}

Martin bleibt dabei, dass Platon nicht ausreichende Texte überliefert hat, aus denen hervorgeht, was im Parmenides über den Chorismos und die Methexis (Anteilhabe) vorgetragen worden ist, um diese aufgeworfenen Probleme lösen zu können.\footcite[vgl.][S. 173f.]{Martin73}
\subsubsection*{Martins Chorismos in den vier großen Ideendialoge}

\begin{itemize}
    \item {Phaidros schwer bestritten werden, dass es Chorsimos gibt}
    \item {Symposion mit dem Aufstieg zur Idee des Schönen}
    \item {Politeia mit dem Höhlengleichnis, aber auch das Sonnen und Liniengleichnis}
    \item {Phaidon. Chorsimos also Trennung von Ideen und Dingen über die Anamnesislehre, also Trennung in der Erkennbarkeit und Widererinnerung}
    \item {Parmenides (faktische Reflexion). Explizite Behandlung des Chorismos Problems. Hier wird auch das Wort \enquote{choris} verwendet (130BCD)}
    \item {Sophistes (methodische Reflexion)}
\end{itemize}
\footcite[vgl.][S. 160]{Martin73}


\subsection{Welche Probleme gibt es mit dieser Theorie}
Wenn man von dieser \enquote{eigentlichen} Wriklichkeit oder der \enquote{eigentlichen} Welt spricht, kommt man in der Darstellungsform und der Argumentation vom Weg ab. Die Phaidon Stelle sagt zwar, dass die Sinnenwelt zwar danach strebt, wie die Ideenwelt zu sein, allerdings dahinter zurück bleibt, hier aber nicht von dem ewigen Teil des Menschen - der Seele - gesprochen wird, welcher eben auch, wie die Ideen, ewig ist und somit diesen Zugang auf erkenntnistheorischer Ebene besitzt und somit zum einen Zugang zu dieser Ebene besitzt und zum anderen auch diese mögliche Angleichung haben kann. 
Zum anderen ist es so, wenn man von dieser eigentlichen Wirklichkeit spricht, es den Anschein erweckt, dass man diese Wirklichkeit als eigentliches oder wahres Seiendes bezeichnet. Umgekehrt ließe das nur zu, dass die Sinnenwelt nicht wahres Seiendes oder uneigentliches Wirkliches ist. Somit bliebe uns nichts anders übrig als der Sinnenwelt einen Grad des Seienden abzusprechen. Problematisch ist es in der Hinsicht, dass wir uns in dieser Sinnenwelt befinden und dies unser \enquote{Startpunkt} ist, wie es im Höhlengleichnis beschrieben wird. Ebenfalls ist es so, dass wir, selbst wenn man die Höhle verlassen und die Sonne gesehen hat, die Aufgabe haben zurück in die Höhle zu gehen.
Außerdem ist damit noch nicht geklärt inwiefern die Anteilhabe und die Anwesenheit von Sinnen- und Ideenwelt geregelt ist.
Es ist ebenfalls so, dass es einfacher ist von unten nach oben zu gehen, also von den Sinnendingen anzufangen, um dann zu den Ideen hoch zu gehen, als von oben herunter zu konsturieren. Man siehe hier mögicherweise die Konstruktion des Staates selbst, wo mit den Bauern begonnen wird und erst darauf aufbauend der Staat nach oben konstruiert wird. Daher ist fraglich, inwiefern diese \enquote{wahre} oder \enquote{wirkliche} Ideenwelt oder -ebene besser oder hilfreicher in der Hinsicht ist, von welchem Bezugspunkt man ausgeht, der einem zur Verfügung steht. Dagegen spricht allerdings die Vorgehensweise im Sophistes der Diairesis von Begriffen, wie etwa dem Angelfischer. Hierbei wird von oben nach unten vorgegangen, um am Ende der Untersuchung beim Angelfischer zu landen. Der Weg nach oben ist äußerst mühsam und schmerzlich. Der Weg nach unten allerdings ist, wie im Sophistes gesehen, immer wieder zu einem anderen Ergenbnis gekommen, sodass immer wieder von Vorne begeonnen werden musste.

Es  wirkt fast so, als würde man versuchen die beiden Seinsbereiche, die man identifiziert hat miteinander zu verbinden, ohne ein Drittes zu setzten, das die beiden Bereiche verbindet, was zu dem altbekannten Problem des infiniten Regresses gelangt, wo man wiederum ein drittes benötigt, um das erste Dritte mit einem der ursprünglichen zwei zu verbinden.

\section{Darstellung der zweiten Theorie (Es gibt nur eine Welt)}
Es wird an der Stelle eigentlich bereits deutlich, wenn in der Politeia in die drei Gleichnisse eingeleitet wird. Dies geschieht einzig und alleine über die Sinne. Also wie kann die Sinnenwelt als minderwertig angesehen werden, wenn sie doch für die Erarbeitung und Erklärung für die Ideenwelt als erstes herangezogen wird.
Wenn diese beiden Welten so voneinander getrennt sein mögen, dann wäre es doch sicherlich nicht ratsam auf diese Weise in die Lehre über die Ideen auf diese Weise einzusteigen. Es sind also die Sinnesdinge und deren Wahrnehmung, welche eigentlich nicht für die wahre Erkenntnis geeignet ist, die den Beginn und auch den einfachsten Zugang (ob es auch der einzige Zugang ist, ist noch fraglich) kennzeichnen, was nur heißen kann, dass man diese \enquote{Welt}, wie man am Ende sehen wird, nicht einfach aufgeben kann und sich somit nur noch in der Ideenwelt aufzuhalten sucht.
\section{Genauere Darstellung davon, dass man nicht von zwei Welten sprechen sollte (Größter Teil der Arbeit mit wahrscheinlich noch mehr Untertitel)}
\subsection{Chorismos}
Wie wird der Begriff des Chorismos wörtlich gebraucht, welche Bedeutung hat das? Wie 


\subsubsection{Methexis und Parousia von Ideen und Sinnesdingen}
Hier muss von zwei unterschiedlichen Bereichen gesprochen werden, damit man von einer Bezugnahmen diesen Ausmaßes sprechen kann. 
\subsection{Einheit-Vielheit Grundproblematiken bei diesem Thema der zwei Welten}
Es müsste ersteinmal grundlegend die Notwendigkeit der Verbindung von der Ideenlehre und dem Einen und Vielen hergestellt werden. 
Das Grundproblem, das hier leider auch zugrunde liegt ist die Verschränkung des Einen und Vielen, wenn man von den Dingen und Ideen spricht, die sich eben in diesen Begriffen jeweils unterschiedlich ausführen lassen. Dabei kommt auch noch die Frage nach der Verschiedenheit und der Einheit desselbigen hinzu.
Es sollte hier allerdings nicht zu sehr eingegangen werden, da dies sonst viel zu viel werden würde.
Siehe hierzu Miglioris Ausführungen zu Philebos und Parmenides\footcite[vgl.][S. 110ff.]{Migliori}
Die Dinge können auf verschiedenen Ebenen als Eines und Vieles beschrieben werden, bzw. kann Einheit und Vielheit an einem Einzelding beschrieben werden.\footcite[vgl.][S. 112]{Migliori}

\section{Wie wird das von Autoren formuliert}
Wie das im Verlaufe wohl von Halfwassen formuliert wird, ist es hier nicht das Ziel die Einheit als noch ein vor das Seiende gestelltes Prinzip zu formulieren.\footnote{vgl. Halfwassen Auf den Spuren des Einen S. 99}
\zitatblock{\enquote{Wenn ferner die Einheit von etwas der Grund seines Seins ist, dann ist jedes etwas auch in dem Grade seiend, indem es Eines ist. Je einheitlicher etwas ist, desto seiender (mallon on, Politeia 515 D 3) ist es dann auch. Erst sein henologischer Ansatz erlaubt Platon die Graduierung von Sein, die der Eleatismus noch nicht kennt. Einheit als Grund des Seins generiert den ontologischen Komparativ und damit die Grundlage der Ideenlehre, der zufolge die einheitliche Wesenheit von etwas seiender ist als ihre vielen individuellen Instanziierungen, und zwar genau darum, weil die eine Schönheit selbst oder die eine Gerechtigkeit selbst den vielen Fällen erscheinender Schönheit oder Gerechtigkeit als die diese Vielheit begründende Einheit zugrunde liegt (Politeia 476 A, 479 A — 480 A, 507 B).}\footcite[][S. 99f.]{halfwassen2015spuren}}
\subsection*{Auf den Spuren des Einen Jens Halfwassen  75/BF 1495 H169}
Seite 91f Platons Metaphysik des Einen.\\
Siehe \footcite[][]{halfwassen2015spuren}
Drei Prinzipien in \enquote{Aufstieg zum Einen Kapitel I und II erläutert:} \zitatblock{1. Jedes seiende existiert als dasjenige, was es jeweils ist, genau aus dem Grudne, weil es Eines ist.\\2. Die Gesamtheit aller einzelnen Seienden bildet die Einheit eines Ganzen. Einheit charakterisiert also nicht nur jedes einzelne Seiende, sondern ebenso die Totalität des Seins.\\3. Das Prinzip der Einheit des Ganzen und zugleich der Einheit jedes einzelnen Seienden ist \emph{das Eine selbst}. Als der Einheit-verleihende Ursprung ist das Eine das Absolute, durch das alles Seiende Eines und kraft seiner Einheit auch seiend ist.\footcite[][S. 91]{halfwassen2015spuren}}

\enquote{Platonische Dialektik ist der Versuch, die Vielfalt der grundlegenden Voraussetzungen unserer denkenden Bezugnahme auf Wriklichkeit auf einen einzigen absoluten Urgrund zurückzuführen: sie ist also die Suche nach dem Absoluten als dem unbedinten Ursprung und Urgrund des Ganzen der Wirklichkeit.}\footcite[][S. 95]{halfwassen2015spuren}
Das Wesen des Guten ist für Platon \emph{die reine Einheit}. Die Wesensbestimmung des Absoluten als reine Einhit ist grundlegend für Platons Prinzipientheorie, die darum den Charackter einer Metaphysik des Einen hat. Erst von ihr aus lässt sich auch verstehen, wie das Absolute Sein und Wassein, Erkennen und Erkennbarkeit zugleich begründet.\footcite[][S. 96]{halfwassen2015spuren}(Fußnote zu Krämer S. 474ff, 535-551)
\enquote{Die Bestimmung des Absoluten als reine Einheit, die absolute Transzendenz des Einen selbst und die Ansetzung eines eigenen Prinzips für die Vielheit sind die drei grundlegenden Thesen der Platonischen Metaphysik des Einen.}\footcite[][S. 96]{halfwassen2015spuren}

\enquote{[\dots] [D]ie Einheit ist die grundlegende Bedingung für das Sein und die Denkbarkeit alles Seienden.}\footcite[vgl.][S. 97]{halfwassen2015spuren} Politeia 478B12f und Parm 144C4-5
\enquote{Auch das Gegenteil des Einen, das Viele, denken wir immer schon und notwendig als Einheit, nämlich als geeinte Vielheit und das bedeutet als einheitliches Ganzes aus vielen elementaren Einheiten, so dass der Gedanke des Vielen in doppelter Weise Einheit voraussetzt.}\footcite[][S. 97]{halfwassen2015spuren} Parm 157C-158B
Bedeutung von Einheit, in der Vielheit enthalten ist, wie Ganzheit, Einheitlichkeit alsEinheit in der Vielheit oder Identität\footcite[vgl.][S. 97]{halfwassen2015spuren} Soph 254D Parm 139D 4-5
Auch der Gedanke von Werden, Nichtsein und Nichts wird unter dem Gedanken des Einen gefasst \footcite[vgl.][S. 97]{halfwassen2015spuren}

\enquote{Denn wer meint, die Wirklichkeit könne auch aus einer einheitlosen Vielheit unverbundener Einzeldinge besten, der nimmt eben damit Denkbestimmungen wie Wirklichkeit, Vielheit und Einzelnes als realitätshaltig in Anspruch und setzt somit genau das voraus, was er bestreiten will, nämlich die Einheit von Denken und Sein}[][S. 98]{halfwassen2015spuren}

\enquote{Alles Seiende und Denkbare ist also nur darum seiend und denkbar, weil es einheitlich ist, und zwar in der Weise, daß sein Charakter als Einheit die Grundlage seiner Denkbarkeit [\dots] bildet. Daraus folgt zugleich, dass Einheit das Kriterium der Unterscheidung von Sein und Nichtsein ist und der Maßstab, an dem Seiendes von höheren und geringeren Seinsgrad messbar wird.}\footcite[vgl.][S. 99]{halfwassen2015spuren}
Hier wird von verschiedenen Seinstufen gesprochen, die auf dem Kriterium der Einheit besteht. Welche Stufen sind hiermit gemeint?\\
Über das Eine selbst kann man nichts mehr aussagen, wie über alles weitere Seiende, da es selber nicht mehr bestimmt ist und eigentlich nur das Prinzip für die Bestimmung für alles weitere ist. Würde man über das Eine selbst so zu sprechen versuchen nimmt man es wiederum aus seiner selbst heraus, da das Denken diese Prinzipien braucht, um etwas zu denken, wie es bisher erklärt worden ist. Ausführung in Parmenides 137C-142A oder Aufstieg zum Einen 282ff und Kapitel XI:
\zitatblock{\enquote{Wird das Eine nur in sich selbst betrachtet, dann weist es als reine Einheit jedwede Bestimmung strikt von sich ab; es steht als solches jenseits aller Bestimmungen, weil jede denkbare Bestimmung es in die Vielheit hineinziehen würde. Man kann darum nichts von ihm aussagen, noch nicht einmal, daß es ist oder daß es Eines ist, weil es damit bereits eine Zweiheit wäre (141 E); die duale Struktur der Prädikation verfehlt prinzipiell die reine Einfachheit des Absoluten. Platon spricht dem absolut Einen darum systematisch alle Fundamentalbestimmungen ab, auch Sein, Einssein, Erkennbarkeit und Sagbarkeit}\footcite[][S. 101]{halfwassen2015spuren}}
Damit ist allerdings der Schritt weg von dem Thema gemacht worden und man befindet sich hier in der reinen Metaphysik und der Transzendenz des Einen. 
Damit lässt sich aber sagen, dass das Eine die Seins- und Erkenntnistranszendenz aufweist. \enquote{Das Eine selbst ist ebensosehr jenseits des Geistes ud der Erkennnis, wie es jenseits des Seins ist; es übersteigt den Zusmamenhang von Denken und Sein, indem es das Prinzip dieses Zusammenhangs ist; und es begründet ihn gerade kraft seiner Transzendenz.}\footcite[][S. 102]{halfwassen2015spuren}
Es folgt das zweite Prinzip, das Prinzip der Vielheit

Um das Seiende in seiner Vielheit ableiten zu können, braucht es ein Prinzip nach dem Einen, welches erfüllt, wenn es spezifisch eben die Vielheit generiert, was die reine Einheit absolut von sich ausschließt \footcite[vgl.][S. 103]{halfwassen2015spuren} Dieses Viele ist noch kein bestimmtes, sondern ist \emph{vorseiend}, aber nicht überseiend wie das Eine.\footcite[vgl.][S. 103]{halfwassen2015spuren} Dies ist auch die \enquote{unbestimmte Zweiheit}
Hier liegt jetzt der übergreifende Punkt und die Überleitung zur \enquote{Ideenlehre}. Aus diesen beiden Prinzipien gehen alle ontologischen Fundamentalbestimmungen hervor, welche die Struktur des Ideenbereichs und damit die ganze Welt des Seienden konstituieren.\footcite[vgl.][S. 104]{halfwassen2015spuren}

Was genau soll man hiermit anfangen? \enquote{Darüber hinaus ist auch das Zusammenwirken der beiden Prinzipien eine Form von Einheit und bedarf eines einigenden Grundes (vgl. Philebos 27 B mit 30 AB; Aristoteles, Metaphysik 1075 b 17—20), der nur das Eine selbst sein kann. Ferner konstituieren die Prinzipien das Sein als die Einheit eines Ganzen (Ev öXov, Parmenides 157 E 4, 158 A 7), als das seiende Eine, in dem alle entfaltete Vielheit einbegriffen bleibt}\footcite[][S. 106]{halfwassen2015spuren}
Es bleibt also noch ein übergreifenderes Prnizip, das die Einheit aus der Einheit und der Vielheit bildet. Hier heißt es, dass es wieder das Eine selbst sein muss.

Die aufschlussreichste Passus des Dialogwerks ist die Gleichnissequenz im 6. und 7. Buch der Politeia mit anschließenden Ausführungen über das Verhältnis von mathemtischer und Propädeutik und Dialektik.\footcite[vgl.][S. 135]{halfwassen2015spuren} Mit Verweis auf Krämer Arete bei Platon und Aristoteles 135-145, 473-480, 533ff
\enquote{[\dots] warum das Gute als letztes Seins-, Erkenntnis- und Wertprinzip selbst noch jenseits des Seins stehen muss.}\footcite[vgl.][S. 136]{halfwassen2015spuren} Mit Verweis auf Politeia 509b, Parmenides 141e und Aufstieg zum Einen 19ff, 188ff, 221ff, 257ff, 277ff, 302ff und 392ff. 
\enquote{Denn schon die Aussage, dass das Eine \emph{ist}, enthält ja eine Zweiheit: nämlich die Zweiheit von Einheit und Sein, aus der sich alle anderen Grundbestimmungen des Seienden ableiten lassen, wie die 2. Hypothesis des \emph{Parmenides} lehrt.}\footcite[][S. 136f.]{halfwassen2015spuren}

\enquote{Die Aussagen der \emph{Politeia} sprechen darum entscheidend für eine monistische Deutung der Prinzipienlehre}\footcite[][S. 137]{halfwassen2015spuren}
\zitatblock{\enquote{Eine monistische Deutung der Prinzipienlehre kann freilich von vornherein keine \emph{Eliminierung} der sie durchgehen bestimmenden Bipolarität bedeuten, sondern nur ihre \emph{Relativierung} insofern, als das Vielheitsprinzip dem Einen nicht gleichursprünglich und gleichmächtig gegenüberstehen kann.}\footcite[][S. 138]{halfwassen2015spuren}}


\subsection*{Der Aufstieg zum Einen 75/BF 2371 H169 Halfwassen}
Es ist zwar mit Plotin verbunden, bzw. daher kommt die Frage des Buches, aber ab Seite 220 geht es mit Platons Transzendenz los, aber auch die Einleitung ist sehr gut.
Stufen der Einheit S. 41
Seinstranszendenz, Geisttranszendenz, Erkenntnistranszendenz S. 150ff
Sehr sehr gut\footcite[][]{halfwassenaufstieg2006}
\subsection*{Szlezák Platonsiches Philosophieren}
\subsubsection*{Der Urspurng der Geistmetaphysik Halfwassen}
Rekapitualtion des Buches von Krämer mit demselben Titel. Was hat sich in den 35 Jahren zwischen diesen beiden Büchern getan? \footcite[vgl.][S. 50]{HalfwassenGeistmetaphysik}

Endlich hebt auch für Platon und Speusipp die Bipolarität der Prinzipien die Absolutheit des Einen keineswegs auf, ist auch hier das zweite Prinzip kein zweites Absolutes, sondern nur die Entfaltungsbasis des Absoluten.\footcite[][S. 53]{HalfwassenGeistmetaphysik}

\zitatblock{\enquote{Auf der Grundlage der skizzierten Henologie basiert Plotins Metaphysik des Geistes, die zwei große Themenkomplexe umfaßt, nämlich erstens die Konstitution des Geistes in seinem Hervorgang aus dem überseienden Einen und zweitens die immanente Struktur der Selbstbeziehung des absoluten Denkens, dessen Inhalte die reinen Wesenheiten des Seienden, also die Ideen sind, in deren untrennbarer Einheit sich der Geist intellektuell selbst anschaut}\footcite[][S. 54]{HalfwassenGeistmetaphysik}}

die dritte (in Plotins Zählung die vierte) Hypothesis des >>Parmenides<<, in der Platon zeigt, wie die von sich selbst her unbestimmte und seinslose Vielheit durch die Teilhabe an dem überseienden absoluten Einen zum in sich selbst bestimmten und vollendeten Einen und Ganzen (Ev öxov téXetov) des seienden Ideenkosmos wird\footcite[vgl.][S. 54f.]{HalfwassenGeistmetaphysik}

\zitatblock{\enquote{Denn halten wir fest: der Platonismus interpretiert das Sein aufgrund seiner Vermittlungsstruktur von Einheit und Vielheit als Geist, er analysiert die Struktur der Selbstbeziehung des Denkens, und er faßt den Zusammenhang der Ideen in der Einheit des Denkens als das Urbild und den aktiv bestimmenden Grund der geordneten Welt, konzipiert also das denkende Selbstverhältnis zugleich als Begründung der Welt}\footcite[][S. 60]{HalfwassenGeistmetaphysik}}

\subsubsection*{Szlezák Halfwassen Monismus Dualismus in Platons Prinzipienlehre}

Halfwassen identifiziert die bekanntlich zwei letzten Prinzipien Platons innerakademischer Prinzipienlehre, auf deren Zusammenwirken alles Seiende zurückgeführt wird: das absolute Eine (auto to en) und die unbestimmte Zweiheit (aoristos dyas)\footcite[vgl.][S. 67]{HalfwassenMonismusDualismus}

Er hält auch fest, dass Krämer in seinem Buch \enquote{Der Ursprung der Geistmetapyhsik} (1964) \enquote{[\dots] nicht mit einer Einheit der Gegensätze bei Platon, sondern im Sinne der neuplatonischen Platondeutung mit einer Zurückführung des Vielheitsprinzips auf das Eine als allbegrüdendes Ur-Prinzips.}\footcite[][S. 68]{HalfwassenMonismusDualismus} (Krämer Seite 332-334; 1964) 
\zitatblock{\enquote{Das Dialektikprogramm der Politeia beschreibt deutlich den Aufstieg zu Einem Unbedingten und Absoluten, das Urgrund von allem ist. Dies scheint einen irreduziblen Prinzipiendualsimus auszuschließen: denn wenn dem Einen die Veilehit als gleichursprüngliches und unabhängiges Prinzip gegenüberstünde}\footcite[][S. 70f.]{HalfwassenMonismusDualismus}} Politeia 511b, 533c
Damit ist zwar mit Halfwassen nicht vollständig auszuschließen, dass es lediglich den Monismus in dieser Weise in der Deutung der Prinzipienlehre gibt, sondern sich gerade in Anlehnung an Krämer, dass \enquote{[d]ie monistische Lösung entspricht einem Rückgriff hinter den Gegensatz der beiden Prinzipien, ohne ihn aufzuheben}\footcite[vgl.][S. 333]{Krämer1964Geistmetaphysik}
Daraufhin verweis auf Parmenides 8 Hypothesen, in denen Einheit und Vielheit zueinander in jedem Verhältnis untersucht werden.\\
8 Hypothesen des Parmenides nach Halfwassen:
\begin{itemize}
    \item {(2. Hypothese) Begründung des Seins vom ersten Prnizip her als die Entfaltung der Eiheit in die Vielheit durch die entzweiende und aufschließende Kraft der Zweiheit}\footcite[vgl.][S. 72]{HalfwassenMonismusDualismus}
    \item {(3.Hypothese) vom zweiten Prinzip her als die Begrenzung des Unbegrenzten durch die Einheit-setzende Macht des Einen}\footcite[vgl.][S. 72]{HalfwassenMonismusDualismus}
    \item {1. und 4. Hypothese thematisieren das Eine und die Vielheit an sich selbst, in ihrer Unbezüglichkeit. Den Prinzipien müssen an ihnen selbst alle grundlegenden Seinsbestimmungen, die Prädikate des seienden Einen sind, und schließlich auch das Sein selbst sowie Erkennbarkeit und Sagbarkeit abgesprochen werden (Parmenides 141)}\footcite[vgl.][S. 72]{HalfwassenMonismusDualismus}
    \item {(8. Hypothese) Vielheit getrennt von dem Einenauch nicht mehr Vielheit ist, sondern gar nichts.\footcite[vgl.][S. 73]{HalfwassenMonismusDualismus}}
\end{itemize}
\enquote{Wenn unsere Deutung richtig ist, dann verbindet Platons Prinzipienlehre einen Monismus in der Reduktino zum Absoluten mit einem Dualismus in der Deduktion es Seienden.}\footcite[][S. 79]{HalfwassenMonismusDualismus}


Das hier ist Halfwassen\\
Lé Timée de Platon contributions à l'histoire de sa réception = Platos Timaios- Das hier ist das Buch, aus dem Hans zitiert hat. Also \enquote{Der Demiurg von J. Halfwassen}
\subsection*{Hans Joachim Krämer. Arete bei Platon und Aristoteles in Platonisches Philosophieren 70/CD 3067 K89}
peras (Ende)\\
meros (Teil)
mega-mikron (Das Groß-Kleine, Das zweite Prinzip neben dem en/apeiron (Einen))
253f.: Das Prinzip der Zahlen ist noch vor dem der Ideen, da dieses Prinzip des Einen und Vielen erst die Ideen bedingen. 

\subsubsection*{Szlezák Maurizio Migliori Dialektik in Platonisches Philosophieren 70/CD 3067 K89}
\zitatblock{\enquote{Diese Verbindung von Ganzem und Teil lässt uns das verstehen, was das Herz einer jeden Dialektik ist: die Möglichkeit, die Identität der Gegensätze zu behaupten, ohne den Satz vom Widerspruch zu negieren.}\footcite[][S. 150]{Migliori}}

\subsection*{Die letzte denkbare Einheit (agathon) Robert Wallisch}
Es wird auf Paul Natorps Platons Ideenlehre zitiert, die ich in der Sophistes Arbeit zitiert habe. 
Erste Arbeitshypothese: \zitatblock{\enquote{Wenn die Ideenzu Beginn des sechsten Buches gedachte konstante Einheiten waren, welche die unstete Vielheit bewältigen - gleichsam als immergültige Bündelungen des Vielen zu noetischen Einheiten - so muss das agathon als eine noch höhere, eine letzte Instanz, welche ihrerseits die Ideen bedingt, d.h. den Ideen in analoger Weise übergeordnet ist wie die Ideen dem konreten Vielen, als letzte denkbare Einheit angesprochen werden}\footcite[][S. 10]{Wallisch}}

Das agathon ist im noetischen Bereich des Erkennens wie die Sonne, die als ein Drittes die Möglichkeit des Erkennens und des Denkens erst schafft.\footcite[vgl.][S. 10]{Wallisch} 
\enquote{Nur durch das agathon existieren die Ideen als reale Gegebenheiten, als Dinge des Denkens}\footcite[][S. 11]{Wallisch} und erzeugt somit die Ideen erst. Dabei wird eine Grenzüberschreitung angesprochen, die das Gute (agathon) jenseits der Seite der Dinge verortet. Damit sind Ideen und Dinge auf der einen Seite, das agathon, die Transzendenz des Guten auf der anderen Seite \footcite[vgl.][S. 11]{Wallisch}
Entscheidend dabei ist der Begriff der ousia
\enquote{Unnennbar viele sind die Sinnesdinge; doch auch die Ideen, welche die Vielen durch noetische Bündelung zu Einheiten bewältigen, sind selbst wiederum \textbf{viele} Einheiten und keinesfalls eine letzte denkbare Einheit [\dots]}\footcite[vgl.][S. 12]{Wallisch}
Im Gegensatz zu Natorp nennt Wallisch das agathon nicht eine absolute Einheit oder eine Grundprädikation, einletztes Objekt der Grundwissenschaft, also ein keinesfalls losgelöstes, ein auf die Welt anwenbares Gesetz des Denkens [\dots], sondern vilemehr die jedes Denken präzedierende ontologische Gegebenheit, die durch Erweiterung auf das Ganze zu erlangen ist.\footcite[vgl.][S. 14]{Wallisch}
Er hält auoerdem fest, dass \enquote{[\dots] nicht eine Teilung in zwei Welten, sondern die Unterscheidung verschiedener gnoselogischer Zugänge zu derselben Welt intendiert ist.}\footcite[vgl.][S. 15]{Wallisch}
\zitatblock{\enquote{Wenn wir nun die Ideen als nicht transzendent aufgefasst haben, so darf darunter keinesfalls eine Leugnung ihrer Sonderung von den Sinnesdingen verstanden werden. Platons Texte lassen keinen Zweifel daran zu, dass die Ideen als real und getrennt von den konkreten Dingen existierend zu denken sind. Tatsächlich ist die platonische Idee eine gnoseologische Gegebenheit und somit als reale Entität anzusprechen, die von den Sinesdingen gesondert existiert und auf die Sinnesdinge Wirkung hat.}\footcite[][S. 17]{Wallisch}}
Es wird mit Platons Texten auf Rep. 579a ff, Phaidon 74a-76e, 78a, Philebos 15a,b und Timaios 51c-e, 52a hingewiesen
\enquote{Weil die Ideen, noetische Einheiten, immer noch viele sind, befinden sie sich diesseits des ontologischen Chorismos, der die transzendente totale Einheit von den Vielen trennt.}\footcite[][S. 17]{Wallisch}
Tim 51b7-c5 Stelle zeigt \enquote{unzweideutig}, dass die Sonderung der Ideen erkenntnistheorisch und nicht ontologisch aufzufassen ist.\footcite[vgl.][S. 19]{Wallisch}
Ein weitere Punkt wird von Wallisch so ausgearbeitet, dass er - genauso wie Natorp - die Ideen mehr als Methode erscheint, als als Ziel\footcite[vgl.][S. 26]{Wallisch} Herangezogen wird dafür die Stelle Phaidon 99e4-100a5, wo die Ideen als eine \enquote{Hypothese} verstanden wird, die als gedanklich eingesetzte Stütze verstanden wird und somit nicht als Objekt von Erkenntnis verstanden werden dürfen.\footcite[vgl.][S. 26]{Wallisch} Es wird dabei betont, dass es auf die ontologische Wahrheit konzipiert worden ist und nicht in Hinblick auf die Erkennbarkeit der konkreten Sinnesdingen\footcite[vgl.][S. 27]{Wallisch}

\subsection{Häufig verwendete Stellen aus dem Original (Platon) von Halfwassen}
Darstellung an denjenigen Stellen, die von den Autoren immer wieder genannt werden und somit am häufigsten verwendet werden. Es müssen erst noch die Stellen gesichtet werden. Stellen von Halfwassen genannt:
\begin{itemize}
    \item {Politeia 506de, 509c-d 510b7, 511b6, 511b7, 517c2-d, 524ff, 533c, 534b}
    \item {Parmenides 141e, 142bff, 142eff, 143bff, 165e-166c}
    \item {Philebos 14cff, 23eff, 27b 30ab, 28c, 30d}
    \item {2. Brief 312e}
\end{itemize}

Zur absoluten Transzendenz des Einen und Guten im einzelnen Halfwassen (1992) 19ff, 188ff, 221ff, 257ff, 302ff und Krämer (1969)\\
Es wird viel auf Aristoteles Metaphysik verwiesen.

\subsubsection{Politeia (Gleichnisse)}
509c-d 510b7, 511b6, 511b7, 517c2-d, 524ff, 533c, 534b, 596a-597e (drei Seinsweisen, Bild Naturding Idee)
506de: Sokrates kann an dieser Stelle nicht über das Gute selbst sprechen, sondern versucht es in diesem Anlauf mit einem Sprössling des Guten, da Sokrates es selbst zu diesem Zeitpunkt nicht schaffen könnte.
\zitatblock{\enquote{Dass es eine Vielheit von Schönem, sagte ich, eine Vielheit von Gutem und so überhaupt von allen Dingen gäbe, räumen wir ein und bezeichnen es auch näher in der Rede. Auch bekanntlich ein Schönes an sich, ein Gutes an sich, und so überhaupt in Bezug auf alles, was wir erst eine Vielheit von jedem hinstellten, das stellen wir ann wieder, um in einem einzigen Begiff hin, als wenn die Vielheit einer Einheit wäre, und nennen es das Wesen von jedem.}} (Pol. 507b-c)
\zitatblock{\enquote{Unter dieser Sonne also, fuhr ich fort, denke dir, verstehe ich den Sprössling des Guten, der von dem eigentlichen wesenhaften Guten als ein ihm entsprechendes Ebenbild hervorgebracht worden ist, so dass das Gute im Denkbaren zum Denken und zum Gedachte sich verhält, wie die Sonne in der sinnlich sichtbaren Welt zum Gesicht und zum Gesehenen}(Pol. 508b-c)}
\zitatblock{\enquote{Dasselbe Verhältnis denke dir nun auch so in Bezug auf die Seele: Wenn sie darauf ihren Blick heftet, was das wahre und wesenhafte Sein bescheint, so vernimmt und erkennt sie es gründlich und scheint Verunft zu haben}(Pol. 508d)}
\enquote{Was den Dingen, die erkannt werden, Wahrheit verleiht und dem Erkennenden das Vermögen des Erkennens gibt, das begreife also als die Wesenheit des eigentlichen Guten}(Pol. 508e)
\zitatblock{\enquote{Und so räume denn auch nun ein, dass den durch die Vernunft erkennbaren Dingen von dem Guten nicht nur das Erkanntwerden zuteilwird, sondern dass ihnen dazu noch von jenem das Sein und die Wirklichkeit zukommt, ohne dass das höchste Gute Wirklichkeit ist, es ragt vielmehr über die Wirklichkeit an Würde und Kraft hinaus}(Pol. 509b-c)}
Bei den mathematischen Dingen im Liniengleichnis heißt es: \zitatblock{\enquote{Nicht war, auch das weißt du, dass sie sich der sichtbaren Dinge bedienen und ihre Demonstrationen auf jene beziehen, während doch nicht auf diese als solche, als sichtbare, ihre Gedanken ziehen, sondern nur auf das, wovon jene sichtbaren Dinge aus Schattenbilder sind.[\dots] Selbst die Körper, die sie bilden und zeichnen, wovon es auch SChatten und Bilder im Gewässer gibt, eben diese Körper gebrauchen sie weiter auch nur als Schattenbilder und suchen dadurch zu Schauung eben jener Ausführung zu glangen, die niemand anders schaun kann als mit dem denkenden Verstand}(Pol. 510e-511a1)}
Der letzte Abschnitt des Liniengleichnisses: 
Apelt Übersetzung: \zitatblock{\enquote{So verstehe denn auch folgendes: unter dem zweiten Abschnitt des Denkbaren meine ich das, was der denkende Verstand unmittelbar selbst erfaßt mit der Macht der Dialektik, indem er die Voraussetzungen nicht als unbedingt Erstes und Oberstes ansieht, sondern in Wahrheit als bloße Voraussetzungen, d.h. Unterlagen, gleichsam Stufen und Aufgangssütztpunkte, damit er bis zum Voraussetzungslosen vordringend an den wirklichen Anfang des Ganzen gelange, und wenn er ihn erfaßt hat, an alles sich haltend was mit ihm in Zusammenhang steht, wieder herabsteige ohne irgendwie das sinnlich Wahrnehmbare dabei mit zu verwenden, sondern nur die Begriffe selbst nach ihrem eigenen inneren Zusammenhang, und mit Begriffen auch abschließe.}(Pol. 511B-C Apelt)}\nocite{PoliteiaApelt}
Schleiermacher Übersetzung:\zitatblock{\enquote{So verstehe denn nun auch, dass ich unter dem anderen Unterabschnitte der nur durch die Vernunft erkennbaren Hälfte das verstehe, was die Vernunft durch die Macht der Dialektik erfasst und wobei sie ihre Voraussetzungen nicht als Erstes und Oberstes ausgibt, sondern als eigentliche Voraussetzungen, gleichsam nur als Einschnitts- und Anlaufungspunkte, damit sie zu dem auf keiner Voraussetzung mehr beruhenden Anfang des Ganzen gelangt, und wenn sie ihn erfasst hat, an alles sich haltend was mit ihm in Zusammenhang steht, wieder herabsteige ohne das sinnlich Wahrnehmbare dabei zu verwenden, sondern nur die Begriffe selbst nach ihrem Zusammenhang, und mit Begriffen auch abschließe}(Pol. 511b-c Schleiermacher)}
Den Ursprung der zwei Welten stammt sehr wohl vom Höhlengleichnis. Da es hier eine deutliche Zweiheit von Welten gibt, die man betreten und auch verlassen kann.
\zitatblock{\enquote{Wenn aber, fuhr ich fort, jemand ihn aus dieser Höhle mit Gewalt den rauen und steilen Aufgang aufwärts zöge und ihn nicht losließe, bis er ihn ans Licht der Sonne herausgebracht hätte, würde er wohl Schmerzen empfunden haben? Würde er über dieses Hinausziehen aufgebracht werden und, nachdem er ans Sonnenlicht gekommen ist, die Augen voller Blendung haben und also gar nichts von den Dingen sehen können, die jetzt als wirklich ausgegeben werden?}(Pol. 515e-516a Schleiermacher)}
Dieser Vorgang lässt sich an dasselbe Argument der unsterblichen Seele knüpfen. Also dass es ein Unveränderbares geben muss, an dem sich das Werdende/Veränderliche vollziehen kann. Also von in der Höhle nach Außen.
\enquote{[er] würde über [die Sonne] die Einsicht gewinnen, [\dots] dass sie alles ordnet im Bereich der sichtbare Weltund von allen jenen Erscheinungen, die er dort sah, gewissermaßn die Ursache ist.}(Pol. 516 b4 Schleiermacher)
\zitatblock{\enquote{Das Gleichnis also, mein lieber Glaukon, fuhr ich fort, ist nun in jeder Beziehung auf die vorhin ausgesprochenen Behauptungen anzuwenden. Die sich uns mittels des Gesichts offenbarende Welt vergleiche einerseits mit der Wohnung im unterirdischen Gefängnis, und das Licht des Feuers in ihr mit dem Vermögen der Sonne. Das Hinaufsteigen und das Beschauen der Gegenstände über der Erde stelle dir andererseits als den Aufschwung der Seele in das Gebiet des nur durch die Vernunft Erkennbaren vor, und du wirst dann meine Meinung hierüber haben, weil du sie doch einmal zu hören verlangst. Ein Gott mag aber wissen, ob sie richtig ist! Aber meine Ansichten hierüber sind nun einmal die: Im Bereich der Vernunfterkenntnis ist der Begriff des Guten nur zu allerletzt und mühsam wahrzunehmen. Nach seiner Ansicht muss man zur Einsicht kommen, dass er für alle Dinge die Ursache von allem Richtigen und Schönen ist, indem er in der sichtbaren Welt das Licht und die Sonne erzeugt. Sodann auch im Bereich des durch die Vernunft Erkennbaren selbst als Herrscher waltend, gewährt er sowohl die Wahrheit als auch Vernunfteinsicht. Ferner muss man zur Einsicht kommen, dass das Wesen des Guten ein jeder erkannt haben muss, der verständig handeln will, sei es in seinem eigenen Leben oder in öffentlichen Angelegenheiten.}(Pol. 516b-d Schleiermacher)}
%hierzwischen werden die Lehren der Arithemtik, der Geometrie, der Astronomie und der Akustik angeführt. Diese sind die Vorstufe zur Dialektik 
\zitatblock{\enquote{und dass nur die Dialektik imstande ist, dem, der die oben beschriebenen Lehrfächer studiert hat, dies zu zeigen und auf eine andere Weise aber es nicht möglich ist? [\dots] Und auch das wird uns weiter niemand in Abrede stellen, [\dots] wenn wir behaupten, dass kein anderes wissenschaftliches Verfahren das Sein eines jeden Dinges zu erfassen strebt, denn alles andere Können und Wissen ist entweder auf menschliche Meinungen und Begierden, oder ist auf die verschiedenen Arten des Entstehenden, auf dessen Zusammensetzung oder ihre Pflege gerichtet.}(Pol. 533b Schleiermacher)}
\zitatblock{\enquote{Die Wissenschaften, denen wir zugestehen, dass sie etwas vom Seienden erfassen, wie Geometrie und ihre verwandten, sehen wir zwar über das Sein träumen, aber im wachen Zustand ist es ihnen unmöglich, es zu schauen, solange sie sich unerwiesener Voraussetzungen bedienen und sie ganz unberührt lassen, weil sie dies nicht begründen können. Denn wobei der Anfang aus dem besteht, was man nicht weiß, und Ende und Mitte aus dem Nichtgewussten zusammengeflochten werden, wie kann auf eine solche Weise angenommen werden, dass eine Wissenschaft entsteht?}(Pol. 533c Schleiermacher)}
\zitatblock{\enquote{Es genügt, also fuhr ich fort, den ersten Abschnitt des Erkennens Wissenschaft zu nennen, den zweiten Verstandeseinsicht, den driten Glaube, den vierten Wahrerscheinen, und einerseits die beiden letzten zusammen Meinung, andererseits die ersten zusammen Vernunfteinsicht, dabei bezieht sich Meinung auf das wandelbare Werden, Vernunfteinsicht auf das unwandelbare Sein, so dass wie Sein zum Werden, so Vernunfteinsicht zu Meinung, und wie Wissenschaft zum Glaube, so Verstandeseinsicht zum Wahrscheinen sich verhält. Die entsprechenden Verhältnisse dessen, woraus sie sich beziehen, sowohl des durch Meinung Erfassbaren als auch bei dem durch Vernunft Erkennbaren, und ihre Unterteilung wollen wir jetzt, mein lieber Glaukon, beiseitesetzen, damit wir nicht in noch viel umfassendere Erörterungen geraten als vorher.}(Pol. 534a-b1 Schleiermacher)}
596a ff. Drei Seinsweisen.
Beginn mit Annahme von beliebigen Vielheiten von Tischen und Betten. Es gibt von diesen Gerätschaften nur zwei Begriffe, einen von Bett und einen von Tisch und der Werkmeister macht den Begriff (vgl. Pol. 596b Schleiermacher)
Es wird ein noch außerordentlicher Meister gegeben, der auch alle Erzeugnisse der Erde bildet, alle lebenden Wesen hervorbringt und alles übrige sowohl sich selbst. (vgl. Pol. 596c Schleiermacher)
Ein Maler ist damit gemeint, denn dieser macht auch auf gewisse Weise alles. (vgl. Pol. 596e)
Es entstehen drei verschiedene Seinsweisen: \enquote{Also Maler, Tischler und Gott sind drei Meister für drei Arten von Betten}(Pol. 597b)
Der Maler ist der Nachbildner, der Tischler der Werkmeister und der Gott der Wesensbildner (vgl. Pol. 597d-e)

\subsubsection{Parmenides 131a 141e, 142bff, 142eff, 143bff, 165e-166c}
\enquote{Wenn das Seiende vieles wäre: so müsste dieses viele unter einander auch ähnlich sein und unähnlich? Dieses aber wäre unmöglich, denn weder könnte das Unähnliche ähnlich, noch das Ähnliche unähnlich sein?}(Parm. 127e1-4) 
Das Seiende, wenn es eines sein soll, was es auch sein muss, kann nicht vieles sein, da man keine Unterschiede in dem Seienden mehr ausmachen könnte.
Der Grund für den Dialog ist ein Buch, welches gegen diejenigen gerichtet ist, die versuchen das Viele existiert. (Also pro Parmenides es gibt nur das Eine)
Das Eine und das Viele lassen sich in einem Ding zeigen. (Ein Ding ist ein Ding, kann aber mit hat ein vorne, hat ein hinten, hat eine linke Seite eine rechte Seite beschrieben werden) 
\zitatblock{\enquote{Wenn aber jemand, wie ich nur eben sagte, zuvörderst die Begriffe selbst aussonderte, die Ähnlichkeit und Unähnlichkeit, die Vielheit nd die Einheit, die Bewegung und die Ruhe, und alle von dieser Art, und dann zeigt, dass diese auch unter sich können mit einander vermischt und voneinander getretnnt werden, das o Zenon, habe er gesagt, würde mir gewaltige Freude machen.}(Parm. 129d2-e4)}
131a Eingangs In welcher Weise haben die Dinge an den angenommenen Begriffen teil?
\enquote{[\dots] es gebe gewisse Begriffe durch deren Aufnahme in sichdiese andern Dinge den Namen von ihnen erhalten [\dots].Also muss entweder den ganzen Begriff oder einen Teil davon jedes aufnehmende in sich aufnehmen?}(Parm. 130e5-a7)
\enquote{eigentlich scheint es mir sich so zu verhalten, dass nämlich diese Begriffe gleichsam als Urbilder dastehen in der Natur, die andern Dinge aber diesen gleichen und Nachbilder sind; und dass die Aufnahme der Begriffe in die andern Dinge nichts anders ist, als dass diese ihnen nachgebildet werden.}(Parm. 132c12-d5)
\enquote{Also auch nicht durch Ähnlichkeit nehmen die andern Dinge die Begriffe auf}(Parm. 133a5-6)
Die andern Dinge sind damit also wirklich getrennt von den Begriffen/Ideen?

141e Das Eine ist weder geworden, noch wurde es oder war es, noch ist es jetzt geworden oder wird oder ist, noch wird es in Zukunft geworden sein oder wird werden oder wird sein. Das Eine hat auf keine Weise gar keine Zeit an sich.
142b Das \enquote{Ist} ist etwas anders als das \enquote{Eins}
\zitatblock{\enquote{Wenn also Eins nicht ist, so wird auch nicht irgend etwas von dem Anderen weder Eins zu sein vorgestellt noch Vieles. Denn ohne Eins Vieles vorstellen ist unmöglich.}(166a5-b2)}
\enquote{Also auch zusammengefasst, wenn Eisn nicht ist so ist nichts[\dots]}(166c1-c2)
\enquote{[\dots] das Eins sei nun oder sei nicht, es selbst und das Andere insgesamt, für sich sowohl als in Beziehung auf einandern, alles auf alle Weise ist und nicht ist, und scheint sowohl als nicht scheint.}(166c2-c6)
\subsubsection{Philebos 14cff, 23eff, 27b 30ab, 28c, 30d}
\enquote{Denn dass Eines vieles ist und Vieles eines, ist doch wunderbar zu sagen, und es ist wohl leichter zu streiten mit dem, der das von beiden behauptet}(Phil. 14c Schleiermacher \nocite{PhilebosSchleiermacher})
\zitatblock{\enquote{Wenn aber jemand den Menschen als Eines setzt, [\dots] und das Schöne als Eines, und das Gute als Eines, über dies und ähnliche Einzelne wird bei fleißiger Behandlung und Auseinanderlegung leicht Streitigkeit entstehen.[\dots] Zuerst ob man wohl annehmen darf, dass es dergleiche Einzelnes gebe als wahrhaft seiend. Dann aber auch, wie doch dieses Einzelne, da jedes von ihnen immer dasselbe sei, und weder Werden noch Untergang unterworfen, dennoch zuerst zwar unwandelbar dieses Eine sei, hernach aber in dem Werdenden und Unendlichen wiederum, sei es nun als Zerteiltes Vieles geworden, zu setzen ist, oder ganz von ihnen getrennt und außerhalb ihrer selbst, [\dots] dieses selbige Eine zugleich in Einem sowohl als in Vielem erscheint.}(Philebos 15b Schleiermacher)}
\zitatblock{\enquote{Zuerst nun lasse uns von diesen Vieren die Dreie ausondern, und versuchen, da wir die Zweie von ihnen jede gar vielfach zerspalten und zerissen sehen, ob wir, wenn wir sie jedes in Eins zusammengebracht haben werden, bemerken könen, wie wohl jedes von ihnen Eins und Vieles war.[\dots] Die Zweie also, die ich vorlege, sollen die eben genannten sein, das eine das Unbegrenzte, das andere die Begrenzung. Dass nun das Unbegrenzte gewissermaßen Vieles ist, will ich versuchen dir zu erklären, die Begrenzung aber soll auf uns warten.}(Philebos 23e-24a Schleiermacher)}



\subsubsection{Phaidon 74e-75d, 99d-105c}
Aber doch an den Wahrnehmungen muss man bemerken, dass alles so in den Wahrnehmungen vorkommende jenem nachstrebt, was das gleiche ist und dass es dahinter zurückbleibt. [\dots] Ehe wir also anfingen zu sehen oder zu hören, oder die anderen Sinne zu gebrauchen, mussten wir schon irgendwoher die Erkenntnis bekommen haben des eigentlichen Gleichen, was es ist, wenn wir doch das Gleiche in den Wahrnehumgen als auf jenes beziehen sollten, dass dergleichen alles zwar strebt zu sein wie jenes, aber doch immer schlechter ist. (75a-b) 
[\dots], dass ich voraussetzte, es gebe ein Schönes an und für sich, und ein Gutes und Großes und so alles andere, woraus, wenn du mir zugibst und einräumst dass es sei, ich dann hoffe, dir die Ursache zu zeigen und nachzuweisen, dass die Seele unsterblich ist (100b-c)
Erste Voraussetzung: Wenn irgend etwas anders schön ist außer jenem, selbstschönen, es wegen nichts anderem schön sei, als weil es Teil hat an jenem Schönen. (100c)

\subsubsection{Timaios 27b-29b, 51b-52d}
Es soll über das All gesprochen werden, \enquote*{wie es entstanden ist oder auch ungeworden ist.}(Tim. 27c)\nocite{TimaiosSchleiermacher}
\zitatblock{\enquote{Was ist das stets Seiende und kein Entstehen Habende und was das stets Werdende, aber nimmerdar Seiende; das eine ist durch verstandesmäßiges Denken zu erfassen, ist stets sich selbst gleich, das andere dagegen ist durch \emph{bloßes} mit vernunftloser Sinneswahrnehmung verbundenes Meinen zu vermuten, ist werdend und vergehend, nie aber wirklich seiend.}(Tim. 27d-28a)}
Der Gestalter muss also als Vorbild das sich stets gleich Verhaltende, wenn etwas schönes gestaltet werden soll. Wenn er allerdings etwas Gewordenes als Vorbild nimmt, so wird es nicht schön.(vgl. Tim. 28a-b)
\enquote{Ist aber diese Welt schön und ihr Werkmeister gut, dann war oofenbar sein Blick auf das Unvergängliche gerichtet; ist \emph{sie} aber - was auch nur auszusprechen frevelhaft wäre, dann \emph{war sein} Blick auf das Gewordene \emph{gerichtet}. Jedem aber ist doch deutlich, dass \emph{er} auf das Unvergängliche \emph{gerichtet war}, denn sie (die Welt) ist das Schönste unter dem Gewordenen, er der Beste unter den Ursachen.}(Tim. 29a)
\enquote{Das aber zugrunde gelegt, ist es ferner durchaus notwendig, dass diese Welt von etwas ein Abbild sei.}(Tim. 29b)
\enquote{Gibt es ein Feuer an sich und für sich und alles das, wovon wir stets in dieser Weise reden, als jeweils an sich und für sich seiend, oder ist allein das, was wir sehen und sonst vermittels des Körpers wahrnehmen, da es eine solche Wahrheit (Realität) hat, und gibt es anderes außer diesen auf keine Art und Weise, sondern behaupten wir jeweils vergeblich, dass es von jeglichem eine denkbare Form gebe, und waren das nicht als \emph{leere} Worte?}(Tim. 51b-c)
\enquote{Wenn Vernunft und richtige Meinung zwei verschiedene Arten sind, dann gibt es auf alle Fälle dies Dinge an sich, Formen, die sich von uns nicht wahrnehmen lassen, sondern nur gedacht werden.[\dots] Aber jene beiden sind als zwei zu bezeichnen, da sie gesondert entstanden und von unähnlicher Beschaffenheit sind. Denn das eine entsteht in uns durch Belehrung das andere durch Überredung.}(Tim. 51d-e)
\subsubsection{Sophistes 251a-259d}
Die Ausführung der Problematik des Einen und Vielen, wie deren Einheit und Verschiedenheit und Verbindung untereinander, wird sich im Sophistes in der Dihairese besonders gewidmet, bzw. hier besonders gezeigt, wenn es darum geht den Weg der Definition zu gehen und hierbei die unterschiedlichen Ebenen voneinander trennen zu können, aber doch den Sinn einer Definition, die Abgrenzung von anderen Dingen auf der selben Ebene von einer höheren und niederen Ebene zu unterscheiden.
\subsection*{Geschichte der Philosophie Band I Altertum und Mittelalter Johannes Hirschberger 1980}
Die wichtigsten genannten Stellen der \enquote{Ideenlehre}:
\begin{itemize}
    \item {Phaidon
    \begin{itemize}
        \item{74a-75d (Erkennbarkeit des Gleichen und des Verschiedenen in Dingen, mit Wiedererinnerung)}
        \item{99d-105c (Die Alternative des Sokrates, seine Ideenlehre);}
    \end{itemize}}
    \item{Politeia
        \begin{itemize}
            \item{507d-509b (Idee des An-sich-Guten und Sonnengleichnis)}
            \item{509d-511e (Liniengleichnis)}
            \item{514a-516c (Höhlengleichnis)}
            \item{596a-597e (drei Seinsweisen, Bild Naturding Idee)}
        \end{itemize}}
    \item{Timaios
    \begin{itemize}
        \item{27b-29b (Entstehung der Welt, des Seienden und Werdenden)} 
        \item{51b-52d (Zusammenfassung)} 
        \item{Deckt sich mit dem, was beim Hans zu finden ist: Timaios 27c1-53c4 und Phaidon 100d}
    \end{itemize}}
    \item{Sophistes 251a-259d (Gemeinschaft der Ideen und die Dialektik);}
    \item{Parmenides 130e-135b (Selbstkritik); In welcher Weise haben die Dinge an den angenommenen Begriffen teil?}
\end{itemize}
\subsubsection*{Parmenides}
1. Hypothese (137c4-142a8)\\
2. Hypothese (142b1-157b5)\\
3. Hypothese (157b6-159b1)\\
4. Hypothese (159b2-160b4)\\
5. Hypothese (160b5-163b6)\\
6. Hypothese (163b7-164b4)\\
7. Hypothese (164b5-165e1)\\
8. Hypothese (165e2-166c5)
\subsubsection*{Dialektik und Prinzipientheorie Migliori}

\subsection{Methexis und Parousia von Ideen und Sinnesdingen}
Hier muss von zwei unterschiedlichen Bereichen gesprochen werden, damit man von einer Bezugnahmen diesen Ausmaßes sprechen kann. 
\section{Einwände an diese Theorie, dass es nur eine Welten gibt}
Halfwassen Seite 76 in Szlezák: Dagegen spricht eine Reihe von Zeugnissen für eine prizipielle Unterordnung der (griechischer Begriff) unter das Eine, ohne dass sie als das Prinzipat aus dem Einen abgleitet würde, womit im übrigen ihr Status als Prinzip aufgehoben wäre.
\section{Abfederung der Kritik}
\section{Zusammenfassung und wie/wann man von einer oder zwei Welten sprechen sollte}
Im Grunde ist hier schon die Zusammenfassung, bei der geklärt wird, dass es auf die Ebenen ankommt, auf denen man von zwei Welten sprechen darf und auf welchen Ebenen man eben nicht davon sprechen darf. Außerdem auch 
Wie also über die Idee des Guten, als noch jenseits des Seins gesprochen wird ist diese Zweiheit insofern zulässig, als dass hier wirklich von einer zweiten Welt sich sprechen ließe, einfach weil es nicht mehr über diese eine Welt geht, sondern gerade darüber hinaus und nicht mehr nur auf eine Welt beschränkt ist. Einzuwenden ist dann selbstverständlich inwiefern es dann möglich ist, dass die Idee des Guten dann überhaupt auf die erste Welt einfluss nehmen kann, wenn sie doch einer anderen Welt angehört.  
\section{Anwendung auf modernes Problem}
Artikel: Elon Musk: „ChatGPT lügt“ - jetzt kündigt er „TruthGPT“ an \footcite[][]{MuskTruthGPTFokus}\\
oder Can Elon Musk’s TruthGPT define and protect the truth? \footcite[][]{MuskTruthGPTReuters}\\
Elon Musk will eine neue KI auf den Markt bringen, die nach Wahrheit streben soll und \enquote{Das Wesen des Universums} ergründen soll
Wahrscheinlich wird sich in den drei Monaten, in denen diese Arbeit entstehen wird, nichts so grundlegendes verändern, dass die eingehenden Überlegungen deutlich abweichen dürften. 

Der Suche nach Wahrheit scheint ein neuer Begleiter an die Seite getreten zu sein. Die scheinbar übermächtige Kapazität und Möglichkeit von künstlicher Intelligenz (KI). Damit soll also versucht werden der Wahrheit in ihrem Wesen auf den Grund zu kommen. Der aktuelle Stand ist, dass diese Idee eine KI zu entwickeln, um das Wesen des Universums zu ergründen, lediglich bekannt gegeben. Das heißt es sind noch keine Ergebnisse oder Berichte von möglichen Antworten dieser KI bekannt gegeben worden, um das Ausmaß oder die Tragweite dieser KI einschätzen zu können. Aus dem Blickwinkel der Philosophie jedoch ist alleine schon diese Idee äußerst brisant. Enthusiasten mögen jetzt behaupten, damit sei möglicherweise das Ende aller Philosophie eingeläutet, wobei erst auf die ersten Ergebnisse gewartet werden müsste, Überlegungen allerdings durchaus angestellt werden können.
Der Ursprung der Ideenlehre war gerade die Wahrheit der Welt zu erklären. Schließt sich hiermit der Kreis nach knapp 2400 Jahren?

Oder eventuell die Entwicklung von unreal enginge 5 und die Möglichkeiten, die es bringt mit einer so unfassbar realistischen Grafik die Realität so gut abzubilden.

Oder doch eher die Apple Vision Pro?
\section{Ausblick}
\section{Fazit}
\newpage
\nocite{politeia}
\nocite{Parmenides}
\section*{Literaturverzeichnis}
\printbibliography[keyword={Primärliteratur}, title={Primärliteratur}]
\printbibliography[keyword={Sekundärliteratur}, title={Sekundärliteratur}]
\end{document}