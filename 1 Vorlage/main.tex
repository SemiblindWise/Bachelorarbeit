\documentclass[12pt]{article}
\usepackage[greek,main=german]{babel}
\usepackage[utf8]{inputenc}
\usepackage{csquotes}
\usepackage{blindtext}
\usepackage{titlesec}
\usepackage[right=2.5 cm, left=2.5 cm, top=2.5 cm, bottom=2.5 cm]{geometry}
\usepackage[onehalfspacing]{setspace}
\usepackage{ragged2e}
\usepackage[T1]{fontenc}
\usepackage{hyperref}
\usepackage{enumitem}
\usepackage{newtxtext}
\usepackage{setspace}
\setstretch{1.5}

\usepackage{blindtext}
\usepackage{fancyhdr}
\renewcommand{\headrulewidth}{0pt}
\usepackage[citestyle=verbose-ibid, bibstyle=authoryear, dashed=false, backend=biber]{biblatex}

\newcommand*{\zitatblock}[1]{%
    \begin{quote}
    \fontsize{10}{12}\selectfont
    \setlength{\parskip}{1.0em}
    #1
    \end{quote}
}


\addbibresource{literatur.bib}

\onehalfspacing
\begin{document}
\pagenumbering{gobble}
\begin{titlepage}
	\begin{center}
		\vspace*{1cm}
		
		\textbf{\LARGE Die \enquote{Zwei}-Welten in der platonischen Ideenlehre}
		
		\vspace{0.5cm}
		\large
		
		Bachelorarbeit\\
		zur Erlangung des akademischen Grades\\
		Bachelor of Arts (B.A.)\\
		im Fach Philosophie\\
		\vspace {1.5cm}
	\end{center}
	\vspace{2.5cm}
	\raggedright
		Universität Augsburg\\
		Philosophisch-Sozialwissenschaftliche Fakultät\\
		Institut für Philosophie\\
		
		\vspace{1.5cm}
		
		\begin{tabular}{@{}ll}
			\makebox[2.5cm][l]{eingereicht von:} & \hspace{2cm} Florian Wittek\\
			& \hspace{2cm} geboren am 5. März 1999\\
			& \hspace{2cm} in Augsburg\\
			& \hspace{2cm} 1617430\\
		\end{tabular}\\	
		\vspace{1.5cm}

		Erstprüferin/Erstprüfer: \hspace{2cm} Prof. Dr. Georg Gasser\\
		Zweitprüferin/Zweitprüfer: \hspace{1.4cm} Dr. Vorname Nachname

		\vfill
		Augsburg, den \hspace{2.4cm} XX. Monat 2023
		
	
\end{titlepage}
\newpage
\pagenumbering{gobble}
\subsection*{Vorwort}
Hier käme das Vorwort
\newpage
\tableofcontents
\newpage
\pagestyle{fancy}
\fancyhf{}
\fancyfoot[R]{\thepage}
\pagenumbering{arabic}
\setcounter{page}{1}
\justifying
\subsection*{Timaios Dialog}
Englisch:   http://classics.mit.edu/Plato/timaeus.html\\
Deutsch:    http://www.zeno.org/Philosophie/M/Platon/Timaios\\
Mit Zitation: http://www.opera-platonis.de/Timaios.pdf


\subsection*{Notizen}
% Der Hauptunterschied zwischen Gnoseologie und Epistemologie besteht also darin, dass Gnoseologie sich stärker auf die Struktur und die Möglichkeiten der menschlichen Erkenntnis konzentriert, während Epistemologie sich mehr auf die Natur des Wissens und die Kriterien für wahres Wissen konzentriert. Es handelt sich im Grunde um zwei verschiedene Perspektiven auf die Erkenntnistheorie, die in verschiedenen philosophischen Traditionen entwickelt wurden.
% Das Gute kann nur Eines sein. Es kann nur ein Gutes geben. Einheit in Vielheit. Wenn es mehrere Gute geben würde, wäre es sinnlos von gut zu sprechen. Erst wenn man die Arete einer Sache findet, kann man von seiner Erfüllung sprechen. Beim Guten gibt es keine Alternative mehr. Es ist und muss vollkommmen sein.
Es wird versucht die Epistemologische, bzw. gnoseologsiche Ebene des Verständnisses von zwei Welten aufzuzeigen. Dies sollte recht einfach werden. Die ontologische Ebene hingegen wird schwieriger. Außerdem wird ein eher pragmatischeres Argument geliefert, das an verschiedenen Stellen auffindbar ist 
Siehe dazu Aufstieg Halfwassen S. 251 das \enquote{intelligible Licht} hat einen Doppelcharakter als Wahrheit und als Sein, einen ontologischen und einen gnoseologischen Aspekt.
Alt-Akademische Erledigung der okzidentalen Metaphysik:
Timaios 27c1-53c4 
Philebos 25a3-b1, 31c8-c11

Wichtige Stellen im Timaios 27c1-53c4
Es gebe ein Schönes an und für sich und ein Gutes und Großes (Phaidon 100b)
Nenne es nun Anwesenheit oder Gemeinschaft [\dots], dass vermöge des Schönen alle schönen Dinge schön werden. (Phaidon 100D)
Das ist schwierig in der Hinsicht, ob hier schon von den schönen Dingen gesprochen wird, also dass bereits in ihrem Sein die Dinge schön sind und erst durch die Idee des Schönen erkannt werden können, dass sie schön sind, weil es hier in dem Satz so strukturiert wurde, dass alle schönen Dinge erst schön werden, also voher eigentlich schon als schön verstanden werden dürften. 
Also die Unterscheidung der Erkenntnis des Schönen in oder an den Dingen gegen die ontologische Deutung, dass die Dinge erst dadurch existieren.

Wenn man den dialektischen Ansprüchen Platons gerecht werden möchte, dahingehend dass man diese bis zum Ende hindurchgeht und sich auch über die der Dialektik inhärenten Dialektik, also die Anwendung der Dialektik auf sich selbst, bewusst wird und dieser eben konsequent bleibt, so bleibt am Ende nichts anderes übrig, als dass es die Aufgabe ist, dass sich die Arbeit der Philosophie gerade damit beschäftigen müsste, dass sich die Dialektik mit dem Thema der Ideen als deren Werkzeug auseinandergehalten werden muss. Dabei gilt es also, dass sich auch in Verbindung mit dem Verständnis von Einem und Vielen ein einheitliches Bild zusammenführen lassen muss.\\
Dazu mehr bei Kutschera 2002 S.186f. \footcite[vgl.][S. 186f.]{Kutschera}
\zitatblock{\enquote{Im \emph{Sophisten} wird betont, dass wir in jeder Aussage Begriffe miteinander verknüpfen und dass es Aufgabe der Dialektik ist, zu untersuchen, welche Begriffe bzw. Ideen sich miteinander verbinden lassen und welche nicht. Es geht dabei um eine Untersuchung von Begriffsverhältnissen, und dazu gehört insbesondere auch die Teilhabe einer Idee an einer anderen.}\footcite[][S. 186f.]{Kutschera}}
Getrennt kann nur werden, was vorher zusammengesetzt war (Phaed. 92E, 78C)

Wie grenzt man die Arbeit von dem Thema der Unsterblichkeit der Seele ab, so dass man zwar das Thema behandel kann, aber nicht in die Frage nach dem Tod und der Schau der Ideen vor dem Eintritt in die Erfahrungswelt abzuschweifen und dennoch das Thema fruchtbar zu machen.

\subsection*{Literatur Notizen. Nicht eingeordnet}
\subsubsection*{Natorp Platons Ideenelehre}
Wichtig bei Gottfired Martin\nocite{NatorpIdeenlehre}
\subsubsection*{Der Staat}
Interne Gliederung nicht im Inhaltsverzeichnis: S. 183 Dialektische Begründung der Lehre von den drei Seelenteilen (436-441), S. 185 Direkte Einführung der Ideenlehre (475-486), S. 188 Die Idee des Guten (502-518), S. 201 Der Erkenntnisweg zum Unbedingten (521-534) Wissenschaften und Dialektik
Es fehlt der letzte Teil 596 im Unterkapitel
\zitatblock{\enquote{Im Phaedo wurde zwar schon alles Sein zuletzt begründet in den Grundlegungen des Denkens. Aber hier sollen wir uns gar etwas denken, das über beides, das Denken und das gedachte Sein hinaus liegt. Aber doch wiederum liegt es im Bereiche, in der Gattung des Denkbaren [\dots]. Es ist das Letzte zwar unter dem Erkennbaren (517B), nur eben noch zu erblicken, aber doch erblickt man es, und muss dann zu dem Schluss kommen, dass es der Grund ist von allem Rechten und Schönen, im sichtbaren Reich der erzeugende Grund des Lichts und des Herrn, der Sonne, im Reich des Denkens selbst als Herrscher Wahrheit und Vernunft verleihend}\footcite[][S. 191]{NatorpIdeenlehre}}
\enquote{Wie also ist es gleichwohl über das Sein und über das Denken hinaus? Jedenfalls insofern es das letzte begründende Prinzip des Seins wie des Erkennens ist.}\footcite[][S. 191]{NatorpIdeenlehre}
\enquote{Es vertritt, nicht eine (besondere) Setzung des Denkens, mithin nicht ein (besonderes) Sein noch eine (besondere) Erkenntnis, sondern die Denksetzung selbst, als letztbegründendes Prinzip alles besonderen Seins, aller besonderen Erkenntnis}\footcite[][S. 192]{NatorpIdeenlehre}
\zitatblock{\enquote{Nicht \emph{ein} letztes logisches Prinzip, sondern \emph{das} Prinzip des Logischen selbst und überhaupt, in welchem alle besondere Denksetzung und damit alles besondere Sein [\dots] zuletzt zu begründen ist. [\dots] Das Gesetz ist es allgemein, welches den Gegenstand konstituiert; dieses Gesetzt selbst, dass im Gesetz der Gegenstand zu begründen, ist somit übergegenständlich, auch über alle besonderen Gesetz, nicht ein, sondern \emph{das} Gesetz; woraus zugleich klar wird, inwiefern dies letzte Prinzip sogar über die Erkenntnis der Wissenschaft hinaus ist.}\footcite[vgl.][S. 194f.]{NatorpIdeenlehre}}
\subsubsection*{Parmenides}
Wenn man auf erkannte Dinge nicht verzichten will, dann schieben sich die Ideen an die Stelle der Dinge, womit aber die Methodenbedeutung, die Natorp in den Ideen sieht, verloren geht, womit man die reinsten Begriffe nicht aus, sondern an der Erfahrung gewinnen kann.\footcite[vgl.][S. 222f.]{NatorpIdeenlehre} \enquote{Wem nun die Erfahrung aufhörte Problem zu sein, wem also auch die Idee selbst sich nicht ferner am Problem der Erfahrung als Wissenschaft entwickeln konnte, dem musste die Ideenwelt erstarren zu etwas wie einer andern Welt gegebener Dinge[\dots]}\footcite[vgl.][S. 223]{NatorpIdeenlehre}
\enquote{Die Fehlmeinung, die aus den Ideen Dinge macht [\dots]}\footcite[][S. 225]{NatorpIdeenlehre}
\subsubsection*{Reine Begriffshandbücher}
\enquote{Trennung und Teilnahme sind die zwei Hauptbegriffe der Dialektik. Getrennt kann nur werden, was vorher zusammengesetzt war (Phaed. 92 E, 78 C)}\footcite[][S. 349]{Perls}
Erklärung zu Phaed. 66B: \enquote{Die doxa ist die Verbindung einer Wahrnehmung mit einer Idee. Also ist ihre eine Hälfte nicht ohne die körperliche Wahrnehmung möglich.}\footcite[][S. 350]{Perls}
\enquote{Allerdings hat sich Platon nirgends näher darüber geäußert, wie er die Teilhabe ontologisch im einzelnen aufgefasst hat.}\footcite[][S. 172]{Gigon75}
Ausdehnung der Relation zwischen Erfahrungswelt und Ideen über den Bereich der ethischen Begriffe hinaus, auf die Totalität der Erfahrungswelt. \footcite[vgl.][S. 172]{Gigon75} \enquote{Überall wo eine Vielheit ähnlicher Dinge oder Phänomene auf ein vorgeordnete Einheit hinweist, deutet er nun diese Einheit als Idee.}\footcite[][S. 172]{Gigon75}
In den Spätdialogen \emph{Sophistes, Politikos} und \emph{Philebos} werden nicht mehr die ontologische Dimension der Ideen betrachtet, sondern nur Allgemeinbegriffe, die lediglich die Geordnetheit der Erfahrungswelt aufgezeigt wird und nicht mehr unveränderliche und urbildlich für sich selbst bestehende Wesenheiten gemeint.\footcite[vgl.][S. 174]{Gigon75}

\zitatblock{\enquote{Die Frage nach der Möglichkeit des Wissensgewinns führt bei Platon also zu einem ontologischen und anthropologischen Dualismus: Aus der Auffassung, dass die Welt des sinnenfälligen Werdens keine sichere Erkenntnis vermitteln kann, folgert er die Gegebenheit eines welttranszendenten Bereichs rein idealer Wesenswahrheiten, die nur der geistigen Erkenntnis des Denkens zugänglich sind und von der Seele immer schon apriorisch gewusst werden.}\footcite[][S. 99]{ThurnerDualismus}}
Es wird von Platon das Wort Chorismos nie in Bezug auf Ideen und Einzeldinge verwendet. Er verwendet es lediglich innerhalb der Seelenlehre in der Bestimmung des Todes als \enquote{Erlösung und Trennung der Seele vom Leib} (Phaidon 67d)\footcite[vgl.][S. 282]{ThurnerTrennung} 
Thurner beginnt mit dem Tiamios Dialog (27d-47e) als Impuls für die zwei Welten. \footcite[vgl.][S. 283]{ThurnerTrennung}
\zitatblock{\enquote{Die Sinnendinge haben den Charakter von Abbildern, weil sie vom Weltbildner (dem DEMIURG) nach dem Vorbild (paradeigma) der rein geistigen Ideen (noêta) gestaltet worden sind (Ti 29a—31 b). Dies impliziert ein selbstständiges Sein dieser Urbilder vor und jenseits der Sinnendinge. Zwischen dem Bereich der rein intelligiblen Urbilder und ihrer sinnlich-materiellen Ähnlichkeiten vermittelt die „WELTSEELE" (psychê tu pantos), die vom Demiurgen durch eine Mischung von unveränderlich Unteilbarem und körperlich Teilbarem zusammengesetzt wurde (Ti 35a).}\footcite[][S. 283]{ThurnerTrennung}}
Fraglich wird allerdings, wenn man es so formuliert, dass \zitatblock{\enquote{[d]ie Unterscheidung zwischen dem geistigen Bereich idealen Seins und dem Bereich des Sinnlichen, das sich zwischen Sein und Nicht-sein befindet}\footcite[vgl.][S. 283]{ThurnerTrennung},} dann wird sehr schnell klar, dass es unmöglich ist, dass etwas zwischen Sein und Nicht-Sein zu verorten ist und damit völlig aus jeglicher Logik fällt. Dabei hat Thurner wohl im Sinn, dass das Sonnengleichnis verschiedene Seinsebenen skizziert. \enquote{[\dots] eine Unterscheidung der Gesamtheit des Seins in zwei unterschiedlich bestimmte Seinsbereiche, die aber durch ein Urbild-Abbild-Verhältnis miteinander verbunden sind}\footcite[vgl.][S. 284]{ThurnerTrennung} Diese Unterscheidung ist nicht auf die Erkenntnisstufen bezogen!
Es gibt eine Stelle in Pol. 477b, wo die Meinung als etwas zwischen Seiendem und Nicht-Seiendem gesetzt wird. Dies kommt aber daher, dass Erkenntnis auf das Seiende sich bezieht und Unkenntnis sich auf das Nichtseiende. Diese Behandlung des Nicht-Seienden und der Möglichkeit von falschen Aussagen (also auch Meinungen) kommt erst im Sophistes zu einem Abschluss, dass das Nicht-Seiende dennoch ist, in Abhängigkeit vom Seienden. Daher ist ein \enquote{Dazwischen} nicht zulässig. 
\enquote{Diese absolute Transzendenz des Guten bestimmt Platon näher, indem er am Schluss des Sonnengleichnisses über das Gute sagt, es sei \enquote{jenseits des Seins, dieses an Würde und Kraft überragend.}(509b)}\footcite[][S. 284]{ThurnerTrennung}
\enquote{Das abschließende Höhlengleichnis veranschaulicht die anthropologische, vor allem das Leben der Philosophen betreffenden Konsequenzen der Platonischen Ontologie zweier Seinsbereiche.}\footcite[][S. 284]{ThurnerTrennung} Dieser Punkt wird nicht weiter ausgeführt, also dass es zwei Seinsbereiche im Höhlengleichnis zu geben scheint. Wird hier der Unterschied im Höhlengleichnis zwischen den Dingen im gesamten Gleichnis und den entsprechenden Erkenntnisstufen gesetzte oder das drinnen und draußen von der Höhle unterschieden?
\zitatblock{\enquote{Nicht zuletzt im Sonnengleichnis wird deutlich, dass die Platonische Tendenz zur geistigen \enquote{Abtrennung} jener bleibenden, idealen Wesenseigenschaften, die auf der Stufe der unklaren Sinneserkenntnis noch als \enquote{unabgetrennt} (achöristos; vgl. Resp 524c) wahrgenommen werden, nur richtig verstanden wird, wenn man sie als die eine komplementäre Hälfte eines dialektischen Gedankenzusammenhanges sieht. Platon bringt dadurch zum Ausdruck, dass die idealen Prinzipien des Seins ihre Begründungsfunktion nur dann erfüllen können, wenn sie dem von ihnen Begründeten ontologisch überlegen sind. Paradoxerweise sind die Ideen nur dann und deshalb als TEILHABE-Ursache in den werdenden Sinnendingen anwesend (vgl. Phlb 26e; Ti 28a), weil sie diese zugleich überragen.}\footcite[][S. 284f.]{ThurnerTrennung}}
Dies ist nicht genau genug formuliert worden, sodass nicht klar ist, was hiermit gemeint sein soll und wie es schlussendlich gemeint und zu verstehen sein soll.
Die idealen Prinzipien sind dem, was sie begründen, ontologisch überlegen. Wie soll das verstanden werden?
methexis/Teilhabe der Dinge an den Ideen


Nochmals einsehen Platonisches Philosophieren 70/CD 3067 K89

\subsubsection*{David Ross: Plato's theory of ideas (1951) 70/CD 3067 R823}
Ideenkritik im Parmenides ist eine Selbstkritik Platons (aus Martin S.151) S. 84



\subsubsection*{Rethinking Plato and Platonism 70/CD 3067 W878 R4}
Es kann nur eine Welt geben, nicht mehrere nebeneinander (Tim. 31a2-31b2) unterstützt von Vogel in der Weise, dass er festhält, \zitatblock{\enquote{As for present-day philosophy, it shows the tendency to eliminate the existence of a \enquote{transcendent} reality, in so far as this is meant to be a reality, existing \enquote{somewhere beyond} the world in which we live: There is one reality only, this one here and now. Certainly, it can be analysed into its intelligible forms, and this may be called a metaphysic of immanency}\footcite[][S. 161]{Vogel}}
35c sehr wichtig!!!
\zitatblock{\enquote{There are four grounds on which Plato is usually qualified as a dualist: (1) his position in metaphysics, usually refered to as the two worlds theory; (2) his radical antithesis of soul and body, as it is commonly understood; (3) the doctrine of two ultimate principles, which he appears to have held at least in his later years; (4) the so-called cosmic dualism, attributed to him by early Christian writers and still ascribed to him by some present-day scholars}\footcite[][S. 159]{Vogel}}
Die Forderung von Vogel ist sehr hoch angesetzt, denn \enquote{[w]hat i \emph{do} want is to give to our philosophers a correct view of what Plato held, and thus, by means of a true picture, contribute something to prevent misunderstandings and clear the way to a true metaphysic}\footcite[vgl.][S. 161]{Vogel}
\subsubsection*{Platons Philosophie II Kutschera 70/CD 3067 K97-2}
Parmenides Kapitel S.164ff
\subsubsection*{Vorhaben Kutscheras}
6.2 und 6.3 Eingehen auf ersten Teil des Parmenides. 6.4 Überleitung zum zweiten Teil. 6.5 Grundgedanken seiner Interpretation des zweiten Teils. 6.7 Probleme mit seiner Interpretation.\\
\enquote{Der \emph{Parmenides} ist neben dem \emph{Timaios} [\dots] sicher jener Dialog, der am schwierigsten zu verstehen ist.}\footcite[][S. 161]{Kutschera}
Der Einstieg in den zweiten Teil des Dialogs ist, dass der junge Sokrates behauptet, dass die Paradoxien von Zenon sich auflösen ließen, wenn man Ideen von ihren empirischen Instanzen unterscheide.\footcite[vgl.][S. 161]{Kutschera}
Es muss sich erst noch weiter in der Philosphie geübt werden, was auch an der Übung getan wird, dass Parmenides dies an dem Beispiel des Einen tut. Dabei stellt Parmenides aus der Existenz des Einen, wie aus der Nicht-Existenz des Einen eine Kette von Widerspruchen ab. Es folgt: \enquote{Es selbst (Das Eine) wie die anderen sind, sowohl für sich wie in Beziehung aufeinandner, ales auf alle Weise und sind es nicht, und scheinen es zu sein und scheinen es nicht zu sein.}\footcite[vgl.][S. 162]{Kutschera}

Kutschera setzt S1: \textbf{Sind F und G gegensätzliche Eigenschaften, so gilt nicht: F hat die Eigenschaften G}
Seite 169. Dem Gedanken des Chorismos liegt zunächst die Unterscheidung der Ideen von ihren empirischen Instanzen zugrunde (130b2-3) (Parmenides). Mit Verweis auf Eutyphron 5d1-2 
\enquote{Für Platon, für den Ideen Gegenstände waren, gilt [eine Idee ist verschieden von all ihren empirischen Instanzen], weil eine Idee das ihren empirischen Instanzen Gemeinsame repräsentiert und der Grund für deren Sosein ist.}\footcite[vgl.][S. 169]{Kutschera} Somit S2:\textbf{Eine Idee ist verschieden von all ihren empirischen Instanzen} und S3:\textbf{Die Existenz und Beschaffenheit der Ideen hängt nicht von der Existenz und Beschaffenheit empirischer Dinge ab.} Unabhängikeit der Ideen vom Entstehen und Vergehen ihrer körperlichen Instanzen.\footcite[vgl.][S. 167]{Kutschera}

Ausschlaggebend ist im Parmenides die Stelle 133c7 ff.\footcite[vgl.][S. 179]{Kutschera}
%Eine interessante Sache, die von Hirschberger hervorgebracht wird, ist die Unterscheidung von zwei Möglichkeiten der Diairesis entweder von oben nach unten, wie es im \emph{Sophistes} durchgeführt worden ist, aber auch von unten nach oben, indem man das Allgemeine aus dem Individuellen heraushebt, um schlussendlich an dem obersten Absoluten anzukommen.\footcite[vgl.][S. 106f.]{Hirschberger} 
%Es geht mit dieser Dialektik darum, dass es um die Erklärung des gesamten Seins durch Aufweis der Strukturidee der Welt geht.\footcite[vgl.][S. 107]{Hirschberger}
%\enquote{Und schließlich geht es in ihr, sofern sie das ganze Sein zusammenschaut und in ihm überall die Parousie der Idee des Guten entdeckt, um den Nachweis der Fußspur Gottes im All.}\footcite[][S. 107]{Hirschberger} 


\section{Einleitung}
Ich gehe davon aus, dass es kein Chorismos zwischen Ideen und Sinnesdingen gibt. Sofern damit gemint ist, dass es zwei Welten gibt, in denen jeweils Ideen und Dinge sich aufhalten. Es wird dabei noch klar darauf einzugehen sein, was mit Chorismos zwischen Ideen und Sinnesdingen gemeint ist und gemeint sein soll. 
Man könnte hier einen Anfang setzten, wenn man ganz einfach beginnt, also damit, dass mit den Basics angefangen werden muss.\\
Die Transzendenz in Platon(s Dialektik) 
Die Frage nach dem Einen, den Ideen\footcite[][]{Staudacher} und den das Seiende übersteigende Ideen, folglich transzendent.\footcite[][]{Bordt}
Womöglich auch damit in Verbindung die Dialektik, dass die verschiedenen Stufen der Ideen und der Sinnesdingen überwunden werden, bis hin zur obersten Stufe, welche damit dann auch noch überwunden werden kann, oder auch nicht, da die Prinzipien der Dialektik nicht mehr auf diesen Bereich anwendbar sind oder eben schon noch, aber bei der nächsten Stufe dann nicht mehr?\\
Siehe dazu\footcite[vgl.][S. 104ff]{Hirschberger}

Es kann nur eine Welt geben, nicht mehrere nebeneinander (Tim. 31a2-31b2) unterstützt von Vogel in der Weise, dass er festhält, \zitatblock{\enquote{As for present-day philosophy, it shows the tendency to eliminate the existence of a \enquote{transcendent} reality, in so far as this is meant to be a reality, existing \enquote{somewhere beyond} the world in which we live: There is one reality only, this one here and now. Certainly, it can be analysed into its intelligible forms, and this may be called a metaphysic of immanency}\footcite[][S. 161]{Vogel}}
35c sehr wichtig!!!
\zitatblock{\enquote{There are four grounds on which Plato is usually qualified as a dualist: (1) his position in metaphysics, usually refered to as the two worlds theory; (2) his radical antithesis of soul and body, as it is commonly understood; (3) the doctrine of two ultimate principles, which he appears to have held at least in his later years; (4) the so-called cosmic dualism, attributed to him by early Christian writers and still ascribed to him by some present-day scholars}\footcite[][S. 159]{Vogel}}
Die Forderung von Vogel ist sehr hoch angesetzt, denn \enquote{[w]hat i \emph{do} want is to give to our philosophers a correct view of what Plato held, and thus, by means of a true picture, contribute something to prevent misunderstandings and clear the way to a true metaphysic}\footcite[vgl.][S. 161]{Vogel}
\enquote{If by the term \enquote{two worlds} is meant: two kins of reality, then the existence of two worlds is explicitly posited in \emph{Phaedo} 79a.}\footcite[][S. 161]{Vogel}
\enquote{physical being is a kind of reality, but a kind of reality which can nether exist by itself nor be known or explained from itself.}\footcite[][S. 162]{Vogel}
\zitatblock{\enquote{\enquote{Partial being}- so it appears to be in that remarkable passage of Rep. V where Plato's Sokrates argues that the object of that lower form of cognition which is not concerned with the full \enquote{being} of \enquote{things themselves} but is a \enquote{view} of thing presenting themselves to our senses, none the less is always \emph{something}. Now \enquote{something} could not be percieved if it were not existing, - just \enquote{non-being}. It \emph{must} have a share of \enquote{being} in it, though it is not \enquote{being} in the full and total sense. It must be a mixture of \enquote{being} and \enquote{non-being}, a middle thing [\dots].}\footcite[vgl.][S. 163]{Vogel}}
\enquote{[\dots] he argues that the lower form of cognition which is not concerned with perfect being but with imperfect things, nevertheless has ome definite object: doxa, so he says, is always \enquote{having a view of \emph{something}}. It is just impossible to have a view of nothing. \emph{Ergo} \enquote{being} cannot be denied to that which is the object of \enquote{a view}. Yet it is not full being: it must be \emph{both being and non-being} [\dots].}\footcite[vgl.][S. 165]{Vogel}
Der Punkt, den Vogel hier machen will, ist, dass die beiden Arten des Seins keien gleichwertigen oder entgegengesetzte Pole sind, sondern eine ontologisches Schema der Unterordnung darstellen.\footcite[vgl.][S. 165]{Vogel}

Es wird wohl schwierig sein, den Gedanken, dass die Seele in der Transzendenz die Ideen bereits geschaut hat, zu verbinden, da hier das Moment der Transzendenz nicht aufgelöst werden kann, ohne auch noch die Seelenlehre zu betrachten.
Die zugrunde liegende Frage ist eigentlich, wie kann die Idee des Schönen, Guten, Gerechten an den schönen, guten, gerechten Dingen teilhaben?\footcite[vgl.][S. 16]{Martin73}

Die Aufteilung im Liniengleichnis muss so aufgefasst werden, dass der größte Teil dem untersten Teil zugesprochen werden muss. Dies macht daher Sinn, da sich das Liniengleichnis auf einen Zielpunkt hin konstruiert, die Idee des Guten. Bei einer Betrachtung von oben herab also muss gelten, dass sich von dem obersten Punkt aus, also der Spitze her - wie eine Art Definitionsbaum - immer mehr Varianten darunter fallen (Man siehe hierzu auch die Dihairese im Sophistes), dieses Prinzip auch für den weiteren Weg \enquote{nach unten} gelten muss, sodass am Ende eine sehr viel breitere Basis besteht und der große Teil des Liniengleichnisses auf der untersten Ebene zu verorten ist. Hiermit entsteht also eine pyramidenähnliche Form. Pyramidenähnlich daher, weil der mittlere Teil nicht ganz einer perfekten Pyramide entsprechen würde, da die zweite und dritte Stufe im Liniengleichnis die gleiche Fläche zukommen müsste. Dies wird weiterhin dadurch unterstützt, dass im Höhlengleichnis am Grunde der Höhle die Schatten an der Höhlenwand jeweils anders interpretiert werden können und damit eine unzählbare Vielheit an Möglichkeiten besteht. Außerdem wird durch jedes Wackeln Zucken des Feuers, welches das Licht auf die Gegenstände wirft, die wiederum den Schatten auf die Höhlenwand werfen, umso zahlreicher. Dabei noch nicht beachtet, dass das Wenden, Drehen und Zusammensetzten der Gegenstände von denjenigen, welche die Gegenstände hinter der Mauer her tragen, nochmals die Zahl der Möglichkeiten anhebt.\\
Diese Ansichtsweise findet ebenfalls Halt, wenn man sich den pythagoreischen Tetraktys ansieht und die im platonischen Denken stark vertretene Dialektik des Einen und Vielen, welche vom Einen beginnt und in das Viele mündet.\\
Problem bei dieser Arbeit wird sein, dass ich bereits ein eigenes Verständnis von dem habe, wie Platon zu lesen sein sollte, oder eben wie Platon zu verstehen und auszulegen ist. 

Noch bevor man die Ideenlehre angehen könnte, noch bevor man sich dem widmen kann, was mit wem in Verbindung steht oder stehen kann und was nicht, muss sich über die Verhältnisse von Einem und Vielem in aller ihrer Formen zugewandt werden.
Die eigentliche Frage, die es zu lösen gilt ist die von der Möglichkeit von unveränderlichen Dingen, die für ales Werdende den Grund angeben. Wo ist der Baum, wenn er gefällt wird, verarbeitet wird, verbrannt wird. Dies ist vermutlich eher eine aristotelische Frage, als eine platonische, wenn man dies so formuliert.
\subsubsection*{Platon Handbuch Horn Müller Söder}
Zu Transzendenz S. 347ff\\
Verschiedene Arten von Aufstiegen. Aufstieg zu etwas Hinreichendem im \emph{Phaidon} (101d5-e1), der Aufstieg zur Idee des Schönen in der Diotima-Rede des Sokrates im \emph{Symposion} (211b5-d1), der Aufstieg zum nicht-vorausgestzten Anfang im Liniengleichnis der \emph{Politeia} (VI 511b3-7); der Aufstieg zur Idee des Guten im Höhlengleichnis der \emph{Politeia} (VII 515c6-516b7); der Aufsieg zum über-himmlischen Ort im Seelen-Mythos des \emph{Phaidros}(246d6-248b5). All diese Aufstiege schließen das Transzendieren ihrer jeweiligen Anfangs- und Zwischenstationen ein. (Aus dem Lateinischen \emph{transcendere} \enquote{übersteigen}, \enquote{überschreiten}) 
Damit ging die Annahme voraus, dass Platon \enquote{Philosophie als Transzendieren} (Hafwassen 1998) porträtiere. 
Es werden in der Regel den Ideen im Allgemeinen (als Entitäten) oder der Idee des Guten im Besonderen Transzendenz zugesprochen.\\
Raum-Zweittranszendenz und im Verhältnis zu ihren sinnlich wahrnehmbaren Partizipanten. Der Idee des Guten wird speziell noch Seinstranszendenz zugeschrieben.\footcite[vgl.][S. 347]{StrobelTranszendenz}
\enquote{[\dots], dass Ideen nicht räumlich lokalisert werden können und die Prädikate, die auf sie zutreffen, von Zeitbezügen frei sind.}\footcite[vgl.][S. 347]{StrobelTranszendenz}
Es bleibt die Frage nach der Transzendenz gegenüber den sinnlich wahrnehmbaren Partizipanten. Es hängt davon ab, wie man diese These verstehen mag.\\
\emph{Nicht-Immanenz}: Eine gegebene Idee \emph{F} ist nicht in/an den Sinnendingen, die F sind.\\
\emph{Unabhängige Existenz}: Eine gegebene Idee \emph{F} kann existieren, ohne dass ein Sinnesding, das F ist, existiert, aber umgekehrt kann kein Sinnending, das F it, existieren, ohne dass die Idee \emph{F} existiert\footcite[][S.348]{StrobelTranszendenz}

Es bedarf einer eindeutigen und ausführlichen Erläuterung der Fragestellung, um diese an dem Rest des Textes erarbeiten zu können. Es geht dabei darum, dass die Ausgangsfrage im Grunde alle Aspekte der platonischen Philosophie unter sich fasst, sodass man alles - ausgeklammert sei die Ethik - behandeln müsste.

\section{Die Rede von Platons zwei Welten}
Was motiviert eine solche Rede (Problem der Einheit – Vielheit / Wesen – Akzidenz ….)?
Es macht durchaus Sinn anfangs mit den Begriffen von zwei Welten an die Texte heranzugehen, da offensichtlich von zwei Bereichen gesprochen wird, die aufgestellt und unterschieden werden müssen. Dafür bedarf es, wie später noch deutlicher formuliert, in erster Linie dieser Begrifflichkeiten, um überhaupt Formulierungen zu beginnen.
\subsection{Einheit-Vielheit Grundproblematiken bei diesem Thema der zwei Welten}
Es müsste erst einmal grundlegend die Notwendigkeit der Verbindung von der Ideenlehre und dem Einen und Vielen hergestellt werden. 
Das Grundproblem, das hier leider auch zugrunde liegt ist die Verschränkung des Einen und Vielen, wenn man von den Dingen und Ideen spricht, die sich eben in diesen Begriffen jeweils unterschiedlich ausführen lassen. Dabei kommt auch noch die Frage nach der Verschiedenheit und der Einheit desselbigen hinzu.
Es sollte hier allerdings nicht zu sehr eingegangen werden, da dies sonst viel zu viel werden würde.
Siehe hierzu Miglioris Ausführungen zu Philebos und Parmenides\footcite[vgl.][S. 110ff.]{Migliori}
Die Dinge können auf verschiedenen Ebenen als Eines und Vieles beschrieben werden, bzw. kann Einheit und Vielheit an einem Einzelding beschrieben werden.\footcite[vgl.][S. 112]{Migliori}
\subsection{Chorismos}
Wie wird der Begriff des Chorismos wörtlich gebraucht, welche Bedeutung hat das? Wie 


\subsubsection{Methexis und Parousia von Ideen und Sinnesdingen}
Hier muss von zwei unterschiedlichen Bereichen gesprochen werden, damit man von einer Bezugnahmen diesen Ausmaßes sprechen kann.
Wenn man diese zwei Bereiche so definiert, dass es eine Teilhabe und eine Anwesenheit des einen in dem anderen gibt, so entsteht das Problem, dass sich eine Art Schema zwischen diese Bereich schleicht, das wiederum zwei neue Grenzen schafft, die es zu überbrücken oder zusammenzuführen gilt. Dies lässt sich leider nicht lösen, da dies in einen infiniten Regress führt. Daher muss dieses Konzept von neu begonnen werden, um diesem Problem entgehen zu können. Dies ist wirklich nur dann zulässig, wenn ein drittes gegeben oder gesucht wird, das diese Bereiche miteinander zu verbinden, das sich auch noch zwischen diese beiden Bereiche fügt.
Dies wird von Graeser auf S. 147f. behandelt.  
\\
Zentrale, in der Platonforschung verwendete Textpassagen (Gleichnisse)\\
Überleitung zum Teil dessen, dass es im Höhlen- und Liniengleichnis zu dem Verstädnis kommt, dass man in vielen Lehrwerken von zwei Welten spricht, da es einfacher ist dies so darzustellen. Es muss sich aber hier dem Thema zugewandt werden, ob es sinnvoll ist von diesen zwei Welten sprechen zu können. Bzw. zu welchem Grad dies möglich ist oder auf welcher Ebene man von zwei Welten sprechen kann und ab welchem Punkt nicht mehr. Es gilt daher diese beiden Positionen deutlich voneinander zu trennen und darzustellen, um sich dann dem Punkt zuzuwenden, dass es nur eine Welt nach diesem Motto geben kann. Dabei soll versucht werden, das jeweilige Verständnis von dem, wie man von zwei Welten spricht und sprechen kann, darzustellen und deutlich zu machen, wie und ab wann man den Begriff der \enquote{zwei} Welten verwenden darf und wann dies nicht mehr zulässig ist.

\section{Interpretation A dieser Rede: Es gibt zwei Welten}
Wie will man von zwei Welten sprechen? Von räumlich, zeitlich, metaphysisch oder von ontologisch getrennten Welten? Es wird zu zeigen sein, wie es sich verhält je nach dem mit welcher Blickrichtung die zwei Welten zu betrachten sind. Außerdem wird sich die Frage nach der Unterscheidung zwischen dem Seinsbereich, dem Ideenbereich und dem Bereich der Idee des Guten zu stellen sein, da hier von möglichen drei Welten zu sprechen sein kann.\\
Wie werden die Welten, von denen im Folgenden gesprochen werden soll, dargestellt? Wie werden zum ersten die Sinnenwelt gegenüber der Ideenwelt konzipiert und auseinander dargestellt und wie werden sie dann im weiteren Schritt wieder versucht zusammenzubringen, da dies auch von den Autoren erkannt wird als eine absolute Notwendigkeit.\\
Es wird sich in der Behandlung der Originalstellen an die Chronologie der Schaffensperioden gehalten. Außerdem ist dabei zu beachten, dass sich die Deutlichkeit der Begriffe und der \enquote{Lehre} erst mit der Politeia verdeutlichen werden und in den frühen Dialogen noch nicht deutlich genug ausgedrückt wird. 
Also: Wie wird die Ideenlehre aus der Perspektive der zwei Welten aufgebaut? 


Der Einstieg in Platons Ideenlehre findet meist so statt, dass man die beiden Begriffe der Sinnenwelt und Ideenwelt nennt und diese dann gegeneinander und miteinander zu Verstehen geben will. Dabei finden sich Formulierungen von verschiedenen Autoren: 
\zitatblock{\enquote{[Die Zwei-Welten-Lehre] unterscheidet zwischen einer raum-zeitlichen Welt des Werdens und einer jenseits von Raum und Zeit befindlichen Welt des Seins. Die Welt des Seins, die allein dem Denken zugänglich ist, repräsentiert Unwandelbarkeit, Idealität und Normativität; die Welt des Werdens, die der Wahrnehmung zugeordnet ist, stellt sich als vergängliche Abbildung der ewigen Strukturen jenseits von Raum und Zeit dar.}\footcite[vgl.][S. 133]{GraeserPhiloGeschichte}} 
Um näher zu klären, was mit der raum-zteilichen Welt des Werdens und der jenseits von Raum und Zeit befindlichen Welt des Seins gemeint sein soll, muss ein Blick in die Originalstellen geworfen werden. Dabei wird sich eingangs an den Beispielen von Disse bedient, die von ihm für die Darstellung der Ideenlehre angeführt werden. Begonnen wird also mit dem ausführlicheren Teil der Politeia und deren Gleichnissen, um dann mit den erwägten Stellen aus dem Symposion und dem Phaidon abzuschließen. Zudem werden dann noch weitere Textstellen aus den Originalen herangezogen, um weiter zu ergänzen, was unklar oder noch weiter ausgeführt werden muss. Wichtig für das Verständnis dieser Stellen ist, dass eine kurze Einordnung gegeben wird, in welchem Kontext die jeweilige Stelle steht, d.h. welche Fragestellung der Text zu beantworten gedenkt, also auf welches Ziel die Argumentation hinausläuft.
Bevor in die Textausschnitte sich begeben werden kann, muss eine Sache noch behandelt werden. Wie Graeser soeben angedeutet hat, werden die beiden Welten auf zweierlei Weise unterschieden. Das eine geht über die Beschreibung des Seins und des Werdens, das anderer geht über die Weise der Erkenntnis. Hierzu heißt es von Gottfired Martin, dass es zwei Kriterien der Unterscheidung der Ideen und Dingen gtib. In der Weise ihrer Erfassbarkeit und dass sie in verschiedener Weise sind.\footcite[vgl.][S. 40]{Martin73}
Diese erstliche Unterscheidung wird im Folgenden von den Autoren meist zusammen genannt werden. Die chronologisch erste Stelle, die hierbei genannt wird, ist Phaidon 79a.
\subsection{Phaidon}
%Die Auflösung zwischen Körper und Seele. Sicherlich wichtig
%Der Phaidon Dialog steht unter dem Licht des bevorstehenden Todes Sokrates, wobei dieser völlige Sorglosigkeit an den Tag legt.\footcite[vgl.][S. 28]{DisseMetaphysik}
%Mit diesem Dialog scheint die Ideenlehre in ihrem leiblichen Ende zu gipfeln. Damit ist gemeint, dass mit der Behandlung des Todes in diesem Dialog die schlussendliche Auflösung oder \emph{Trennung} der Seele vom Körper thematisiert wird. Dabei wird nochmals aufgegriffen, dass nicht mittels der Sinnesorgane der Mensch zu wahrem Wissen gelangen wird - also durch die Körperlichkeit - , sondern durch das reine Denken, welches auf die Ideen gerichtet ist.\footcite[vgl.][S. 29]{DisseMetaphysik}
Die hier zu nennennde Phaidon Stelle ist so eingeleitet, dass es vierlei Dinge, wie Menschen, Pferde oder Kleider gibt, die sich im Gegensatz zur Schönheit, die den vielen Dingen zugeschrieben wird, nie gleich bleiben, also sich immer in Veränderung befinden. Die Schönheit hingegen bleibt immer gleich und ist unsichtbar und nur durch das Denken erlangbar.
So heißt es dann: \enquote{Sollen wir also, spach er, zwei Arten der Dinge\footnote{\emph{dúo eide tôn onton}  Kassner: zwei Ordnungen der Wesen\nocite{PhaidonKassner}} setzen, sichtbar die einen und die andere unsichtbar?}(Schleiermacher \nocite{PhaidonSchleiermacher}Phaidon 79a)
Martin hingegen übersetzt hier mit \enquote{Zwei Weisen des Seins}.
Diese Übersetzungen zeigen, dass sich diese verschiedenen Formen, Modi oder Arten des Seins als Schwierigkeit herausstellen, also wie diese Unterscheidung gemeint sein sollte, da Martin keine ontologische Implikationen in die Übersetzung tragen möchte.\footcite[vgl.][S. 37]{Martin73} Diese Unterscheidung der zwei Weisen des Seins gibt Martin die reine Unterscheidung zwischen den Ideen und den Dingen. 
Als Martin dann nochmals die Stelle angeht, allerdings unter ontologischen Gesichtspunkten heißt es unter Berücksichtigung, dass eindeutig ein Plural verwendet wird, wird die vorsichtige Übersetzung der \enquote{zwei Weisen des Seins} verwendet, da gerade die Dinge von den Ideen unterschieden werden sollen.\footcite[vgl.][S. 216]{Martin73} So fasst Martin zusammen, dass Wie immer man sich in der Übersetzung helfen möge, Platon zwei Weisen des Seins, das Sein der Ideen und das Sein der Dinge kennt und er sie unterscheidet und zugleich zusammenstellt\footcite[vgl.][S. 216]{Martin73}
Mit der Formulierung des \enquote{zugleich zusammenstellen} ist gemeint, dass das Sein der Dinge und das Sein der Ideen - wie später die Zuordnung dieser Beiden Bereiche heißen soll - gleichzeitig gemeint sein soll.
Vogel hingegen zieht aus der Phaidon Stelle einen eher drastischeren Schluss:
\enquote{If by the term \enquote{two worlds} is meant: two kins of reality, then the existence of two worlds is explicitly posited in \emph{Phaedo} 79a.}\footcite[][S. 161]{Vogel}
Fraglich bleibt jedoch bei diesem Dialog, inwiefern die Rede von der Seele zum Körper gedacht werden soll, da Martin es so ausdrückt:
\zitatblock{\enquote{Nach [der Anamnesislehre] ist Erkenntnis immer eine Wiedererinnerung. Das heißt doch, und Platon sagt dies auch ausdrücklich, daß die Seele die Ideen in einem früheren Leben vor der Geburt kennengelernt haben muß. Dies wiederum kann doch nur heißen, daß die Ideen nicht in dieser Welt sind.}\footcite[vgl.][S. 160]{Martin73}}
Somit gibt es dabei keine deutlichere Ausdrucksweise die Ideen als etwas nicht in dieser Welt zu verorten.
%Zitat aus Phaidon 79d: Die Seele verhält sich zu jenen Gegenständen immer in derselben Weise, da sie eben damit etwas erfasst, das selbst auch von dieser Art ist. Diesen Gegenständen komme es zu, \enquote{niemals in keiner Weise, irgendwie auch nur die geringste Veränderung zu erleiden. (Phaidon 78d)}\footcite[][S. 97]{Hirschberger}
%Hierin liegt der Grund, dass nicht scharf genug differenziert wird und die Definition nicht eingehalten wird. Also die verschiedenen Ebenen oder Stufen der Ideen, die jeweils als eigenständig also absolut unveränderlich und ewig gelten und dann in den Bereich des Denkens eintreten und somit nicht mehr die absolute Veränderlichkeit aufweisen können dadurch aber gedacht werden können und eben in den Objekten anwesend sein können.\footcite[vgl.][S. 180f.]{Kutschera}

\subsection{Symposion (Gastmahl)}
Im Symposion geht es darum zu erfassen, wie man den Weg zum obersten Schönen erreichen kann. Dabei soll mit einem einzigen Körper und dessen Schönheit begonnen werden, dann zur Schönheit jedes einzelnen Körpers weitergegangen werden, um dann über immer allgemeinere Schönheit an Körpern diese auch hinter sich lassen soll. Hier kommt der Übergang zu den geistigen Schönheit, wobei die körperliche Schönheit als niederer und geringer schätzen und auch verachten soll, um dann die Schönheit in Sitten und Gesetzen erkennen und schätzen soll, bis man auf eine einzige Erkenntnis des Schönen selber sich richten soll. (Sym. 210a-e)  
Dieses letzte Schöne wird dann wie folgt beschrieben: \zitatblock{\enquote{Zuförderst ist es ein beständig Seindes, was weder wird noch vergeht und weder zunimmt noch abnimmt, sodann nicht nach der einen Seite betrachtet schön, nach der andern unschön, noch auch bald schön und bald nicht, noch in Vergleich mit dem einen schön, mit dem andern aber hässlich, oder teilweise schön und teilweise hässlich, oder nach der Meinung einiger schön, nach der von andern aber hässlich.} (Symp. 211a)}
Hierin erkennt Disse, was für Platon die eigentliche Wriklichkeit des wahren Seins gegenüber dem unmittelbaren sinnlich wahrnehmbaren Sein ist.\footcite[vgl.][S. 27]{DisseMetaphysik} Dies ist eine graduelle Zuwendung zur Schönheit selbst, die nicht mehr in den körperlichen Dingen zu finden ist, sondern nur durch den Geist erkennbar ist. Somit folgert Disse aus dieser Stelle, dass es \enquote{[f]ür den Menschen [darauf an]kommt, dass er die vergänglich-körperliche Welt verlässt, um sich alleine der ewig-geistigen zuzuwenden.}\footcite[vgl.][S. 28]{DisseMetaphysik} Dieses Verlassen allerdings darf nur so verstanden werden, dass damit eine Abwendung von den Sinnesdingen und eine Zuwendung zu den Ideen, bzw. dann zur Schönheit selbst, sein soll. 
Der Aufstieg im Symposion ist so zu verstehen, dass es sich um ein ewiges Sein gegenüber dem vergänglichen und um ein geisitges gegenüber dem körperlichen handelt.\footcite[vgl.][S. 27f.]{DisseMetaphysik}
\subsection{Politeia}
Den größten Anteil hat nun die Politeia an der \enquote{Ideenlehre}.
Der Eingang in die Stellen aus der Politeia über das Sonnengleichnis, das Liniengleichnis und das Höhlengleichnis werden über die Frage \enquote{was das Gute selbst eigentlich ist} (Pol. 507a) gestellt, wobei von dem Guten selbst Abstand genommen wird und nur auf den Sprössling des Guten geblickt wird, da der sprechende Sokrates sich nur dazu in der Lage sieht. Somit wird rückwirkend Bezug genommen, dass es eine Vielheit von Schönem und Gutem gibt, genauso, dass es ein Schönes an sich und ein Gutes an sich gibt. (Pol. 507b-c) Hierbei wird allerdings nur der reine Begriff, das Wort, gemeint, also dass viele Dinge mit einem Begriff wie etwa schön usw. beschrieben werden können und noch keine ontologische Bedeutung des Schönen selbst mit gemeint sein soll. Des weiteren wird der Vielheit zugeschrieben, dass sie sichtbar und nicht denkbar, die Begriffe nur denkbar und nicht sichtbar sind. Hierin besteht schon die erste Unterscheidung auf eine Zweiheit hin.

Die Dinge, die man durch die Augen und allen anderen Sinneswahrnehmungen erkennen kann, sind eindeutig in diesem Falle sichtbar. Die Beschreibung über das Denkbare ist so gemeint, dass die Begriffe einzig und alleine denkbar sind, also dass man die Schönheit, welche sich in - oder an - vielen Dingen befinden kann, nicht selbst sehen kann oder eben \enquote{das Wesen der Länge} selbst nicht sichtbar ist, sondern nur die sich zeigende Länge in einem Ding, welches eine Länge aufweist, welche dann auch messbar ist.
Nun besteht allerdings die Frage, wie dieser Unterschied erkannt werden kann, also wie der sog. \enquote{Gesichtsinn} - gemeint sind in erster Linie die Augen, wobei auch alle anderen Sinnesorgane darunter fallen - etwas erkennen kann.
Dabei wird eine Dreiheit zwischen Erkennendem (Subjekt), Erkanntem (Objekt) und dem Prozess des Erkennens konzipiert, wo die Sonne das Medium darstellt und in allen der Dreiheiten zu finden ist. Also dem Auge als Erkennendem, dem Ding als Erkennendes und über das Licht als das, was das Erkennen erst möglich macht. So heißt es:
\zitatblock{\enquote{Unter dieser Sonne also [\dots] denke dir, verstehe ich den Sprössling des Guten, der von dem eigentlichen wesenhaften Guten als ein ihm entsprechendes Ebenbild hervorgebracht worden ist, so dass das Gute im Denkbaren zum Denken und zum Gedachte sich verhält, wie die Sonne in der sinnlich sichtbaren Welt zum Gesicht und zum Gesehenen}(vgl. Pol. 508b-c)} 
Auf die Rolle des wesenhaften Guten oder auch Idee des Guten wird später noch näher eingegangen. 
So wird das Verhältnis aus dem Bereich der Sinnesdinge auch auf den Bereich des Geistes gelegt:
\zitatblock{\enquote{Dasselbe Verhältnis denke dir nun auch so in Bezug auf die Seele (psyche): Wenn sie darauf ihren Blick heftet, was das wahre und wesenhafte Sein bescheint, so vernimmt und erkennt sie es gründlich und scheint Verunft zu haben}(Pol. 508d)}
\zitatblock{\enquote{Was den Dingen, die erkannt werden, Wahrheit verleiht und dem Erkennenden das Vermögen des Erkennens gibt, das begreife also als die Wesenheit des eigentlichen Guten} (Pol. 508e)}
\zitatblock{\enquote{Und so räume denn auch nun ein, dass den durch die Vernunft erkennbaren Dingen von dem Guten nicht nur das Erkanntwerden zuteilwird, sondern dass ihnen dazu noch von jenem das Sein und die Wirklichkeit zukommt, ohne dass das höchste Gute Wirklichkeit ist, es ragt vielmehr über die Wirklichkeit an Würde und Kraft hinaus}(Pol. 509b-c)}
Es lässt sich soweit festhalten, dass die eine Seite die Welt der sich in ständiger Veränderung befindlichen Dinge und die andere Seite die Welt der ewigen und immerwährenden Ideen bestimmt ist, da diese als grundlegend unterschiedlich dargelegt worden sind.
Diese Ausführung über die Unterscheidung wird im Liniengleichnis fortgesetzt (Pol. 509e-511e).
Es wird eine gedachte Linie aufgestellt, welche zuerst ungleich geteilt wird. Diese entstanden zwei Bereichen entsprechen den  eben dargelegten Bereiche der Dinge und der Ideen. Jetzt wird allerdings nocheinmal jeder Bereich nach dem ersten Trennverhältnis unterteilt, sodass eine Linie mit 4 Bereichen entsteht, welche das Verhältnis 4-2-2-1 aufweist. Mit dem untersten Bereich angefangen, wird der kleinste Bereich mit Schatten und Spiegelungen der Sinnenwelt gefüllt. Die Linie fortschreitend wird der nächste Bereich mit den Dingen selber befüllt, also die gesamte Tier- und Pflanzenwelt und aller weiterer Körper. Zudem wird diesem Bereich auf einer weiteren Ebene die Erkenntnisweise der Meinung und des Scheins beigefügt. Der Abschnitt, der durch die Vernunft Erkennbarem, wird auch unterteilt nach dem Verhältnis der ersten Teilung der Linie. Hierbei werden dem ersten Abschnitt, welcher den Dingen am nähsten steht der Bereich geometrischen und arithmetischen Begriffe zugeteilt. Jedoch gebraucht man diese Begriffe noch unter Anwendung auf die Dinge aus der Dingenebene, wodurch dieser Bereich noch nicht als \enquote{rein} bezeichnet wird. Die letzte Hälfte allerdings geht noch einen Schritt weiter, dahingegehend, dass sie mithilfe der Dialektik sich von den Sinnesdingen vollständig löst und die Prinzipien behandelt, die auf keiner Voraussetzung mehr beruhen. Der Unterschied dieser letzten beiden Bereiche wird so erklärt, dass auf dem ersten hier genannten Bereich es immer noch Voraussetzungen gibt, zwar mit dem Verstand gearbeitet wird, also ein rechnendes oder mechanisches Denken, aber immer noch einen Bezug zur Sinnenwelt besteht. Dies ändert sich auf der letzten Ebene, da es hier um eine Vernunfterkenntnis geht, die voraussetzungsfrei ist, bzw. dorthin gelangen möchte mithilfe der Dialktik, die sich der reinen Begriffe bedient. 
Daher wird an diesem Punkte auch die Unterscheidung getroffen, dass nur auf der Ebene der Vernunfterkenntnis, der höchsten Stufe wahre Erkenntnis gewonnen werden kann, da sie eben voraussetzungslos ist. Hierin wird außerdem das 4-2-2-1 Verhältnis eingeholt, dahingehend, dass derjenigen Bereichen, welche den Anteil an der Wahrheit entsprechen sollen, auch den größeren Anteil haben sollen.
Das wichtige dabei ist nun, dass der Linie nicht nur der Gegentandsbereich inne ist, sondern auf den gleichen Stufen entsprechend auch noch die gnoseologische Gegenstück hinzugenommen wird. Das heißt, dass den Schatten das Meinen und Raten (eikasia) entspricht, den Dingen das für wahr erscheinen (pistis), für die geometrischen Dinge dann das, wie schon angedeutet, diskursive/rechnende Denken (dianoia) und für die Ideen die Vernunfterkenntnis (noesis). 
Nun schließt sich das prominenteste der drei Gleichnisse an; das Höhlengleichnis (Pol. 514-516).
Hier wird in Bezug auf Bildung und Unbildung das Bildniss einer Höhle gezeichnet, in der am untersten Ende Menschen angekettet vor einer Wand sitzen und gezwungen sind auf eine Wand zu Blicken, auf der Schatten vorbeiziehen. Diese Schatten stammen von Gegenständen, die hinter den Menschen - noch hinter einer dazwischen liegenden Wand - vorbeigetragen werden und von einem Feuer so beschienen werden, dass sie eben ihre Schatten auf die Höhlenwand werfen. Nun wird allerdings einer dieser Gefesselten befreit und gewaltsam gegen das Licht gedreht und zum Feuer geführt. So wird beschrieben, dass er damit eine erste, stärkere Wahrheit erblicken würde, wenn er die Gegenstände und das Feuer sieht und sich an deren Anblick gewöhnt. Hierdurch würde er erkennen, dass er nur auf Schatten geschaut hat und nicht auf die \enquote{wahren} Dinge. Das Gleichnis wird weitergeführt, indem der Gefesselte dann weiter aus der Höhle unter äußerster Mühsal den Weg aus der Höhle geführt wird. Wenn er dann aus der Höhle herausgeführt worden ist, so würde er gar nichts erkennen können, da er von dem Licht der Sonne vollkommen geblendet ist. Mit der Zeit, die er dann außerhalb der Höhle verbringen würde, würde er zuerst die Spiegelungen und Schatten erkennen und immer so weiter über die Erscheinungen Nachts, bis er dann die Gegenstände erblicken kann und am Ende eben die Sonne selber, die als Urheberin für alles verantwortlich ist. \footnote{Es schließt sich noch die Rückkehr des Gefangenen an die Ausgangsposition an das Gleichnis an. Dies wird hier aber nicht weiter zu thematisieren sein.} Es schließt das Gleichnis mit folgenden Worten:
\zitatblock{\enquote{Die mittels des Gesichts sich uns offenbarende Welt vergleiche einerseits mit der Wohnung im unterirdischen Gefängnisse, und das Licht des Feuers in ihr mit dem Vermögen der Sonne, das Hinaufsteigen und das Beschauen der Gegenstände über der Erde andererseits stelle dir als den Aufschwung der Seele in das Gebiet des nur durch die Vernunft Erkennbaren vor. [\dots] im Bereiche der Vernunfterkenntnis sei der Bereich des Guten nur zu allerletzt mühsam wahrzunehmen, und nach seiner Anschauung müsse man zur Einsicht kommen, dass er für alle Dinge die Ursache von allem Richtigen und Schönen sei, indem er der sichtbaren Welt das Licht und die Sonne erzeugt, sodann auch im Bereich des durch die Vernunft Erkennbaren selbst als Herrscher waltend sowohl die Wahrheit als auch uns Vernunfteinsicht gewährt, ferner zur Einsicht kommen, dass das Wesen des Guten ein jeder erkannt haben müsse, der verständig handeln will [\dots].} (Pol. 517b-d)}
Damit zieht Disse den Schluss, dass die Sinnenwelt und das Reich des Denkbaren sich im Höhlengleichnis wie im Symposion gegenüberstehen.\footcite[vgl.][S. 28]{DisseMetaphysik}
Dies alles zusammengenommen kommt Disse zu dem Schluss, dass \enquote{Platon somit im Höhlengleichnis zunächst einmal grundsätzlich zwischen zwei Welten [unterscheidet] und eine Bewegung des Menschen, nämlich die Tätigkeit der Philosophie, die ihn von der ersten in die zweite, eigentliche führen soll., [bestimmt].}\footcite[vgl.][S. 23f.]{DisseMetaphysik}
Abstufung einer höchsten Idee zu den anderen Ideen.
\enquote{Das Gebiet des Tageslichts außerhalb der Höhle ist der Bereich dessen, was uns durch reines Denken zugänglich ist. Die Sonne aber wird mit dem höchsten Punkt im Bereich des Denkbaren verglichen.}\footcite[][S. 49]{DisseMetaphysik}

\subsection{Timaios}
Als Zusammenfassung wird dann Tim. 52a herangezogen wo es heißt:
\zitatblock{\enquote{[\dots] das eine sei die Form, die sich stets in sich andere von anderswoher aufnehmend noch selbst in anderes irgendwohin gehend, unsichtbar und auch sonst unwahrnehmbar, das, was der Vernunft zu betrachten zuteil wurde. Ein zweites aber sei das ihm Gleichnamige und Ähnliche, wahrnehmbar und geboren, stets hin- und hergerissen, an einer bestimmten Stelle entstehend und von da wieder verschwindend, durch Meinung in Verbindung mit sinnlicher Wahrnehmung erfassbar.} (vgl. Tim. 52a)}

\zitatblock{\enquote{Was ist das stets Seiende und kein Entstehen Habende und was das stets Werdende, aber nimmerdar Seiende; das eine ist durch verstandesmäßiges Denken zu erfassen, ist stets sich selbst gleich, das andere dagegen ist durch \emph{bloßes} mit vernunftloser Sinneswahrnehmung verbundenes Meinen zu vermuten, ist werdend und vergehend, nie aber wirklich seiend.}(Tim. 27d-28a)} Graeser hingegen deutet hier die Dinge als \enquote{niemals Seiendes}\footcite[vgl.][S. 140]{GraeserPhiloGeschichte}

% Das Vielgestaltige und Veränderliche, das sich bald so und bald anders zeigt, erfassen wir im sinnlichen Sehen; das Einheitliche und Unveränderliche, das immer dasselbe Wassein sehen läßt, zeigt sich nur im die Sinneswahrnehmung transzendierenden reinen Denken (507 B 9-10). Entsprechend unterscheidet Platon eine Welt des Intelligiblen (vonios topos, 508 C 1, 517 B 5, vgl. C 3) und eine Welt des Sinnenfälligen oder Sichtbaren (opatos Toлos, vgl. 508 C 2, 517 C 3) als zwei Arten des Seienden (do con tan övtav, Phaid. 79 A 5; Politeia 509 D 1-3: dvo αὐτὰ εἶναι . . . τὸ μὲν νοητοῦ γένος τε καὶ τόπος, τὸ δ' αὖ ὁρατοῦ). Dic sichtbare Welt aber ist das Abbild der intelligiblen Welt (vgl. Tim. 29 A- B), sie hat nur in der Teilhabe an dieser Sein, darum entsprechen sich die Strukturen beider Welten in strenger Analogie.\footcite[][S. 246]{halfwassenaufstieg2006}


\zitatblock{\enquote*{Den Gegenstandsbereich des Denkens können wir somit, im Gegensatz zur sinnlich wahrgenommenen Welt, d.h. zur Sinnenwelt, die \emph{Ideenwelt} nennen. Grundlage von Platons Metaphysik ist die so genannte Zweiweltenlehre, die Lehre des Gegensatzes zwischen Sinnen- und Ideenwelt, wobei die Ideenwelt als die höhere, weil ewige und geistige Wirklichkeit die der Sinnenwelt zugrunde liegende, sie fundierende ist, nach der die an den Körper gebundene menschliche Seele stirbt.}\footcite[][S. 29]{DisseMetaphysik}}



\enquote{Jedenfalls bedeutet Platos Zwei-Welten-Lehre eine radikale Unterscheidung zwischen Sein und Werden, zwischen Wirklichkeit und Schein, zwischen Echtheit und Unechtheit.}\footcite[][S. 134]{GraeserPhiloGeschichte}
\enquote{Platon charakterisiert die Idee als das, was wirklich ist, ihre raum-zeitlichen Instanzen jedoch als etwas, was nicht wirklich ist.}\footcite[][S. 139]{GraeserPhiloGeschichte} Mit raum-zeitlichen Instanzen sind Sinnesdinge gemeint. 

\enquote{Die Ideen bilden bei Platon einen von der Sinnenwelt eindeutig getrennten Wirklichkeitsbereich}\footcite[][S. 31]{DisseMetaphysik}

\zitatblock{\enquote{Die Frage nach der Möglichkeit des Wissensgewinns führt bei Platon also zu einem ontologischen und anthropologischen Dualismus: Aus der Auffassung, dass die Welt des sinnenfälligen Werdens keine sichere Erkenntnis vermitteln kann, folgert er die Gegebenheit eines welttranszendenten Bereichs rein idealer Wesenswahrheiten, die nur der geistigen Erkenntnis des Denkens zugänglich sind und von der Seele immer schon apriorisch gewusst werden.}\footcite[][S. 99]{ThurnerDualismus}}

Es wird von ihm das Verständnis gezeichnet, dass die Ideen und die Welt der Ideen eine \enquote{Ideale Wirklichkeit} darstellen und auch nach dem möglichen Vergehen der materiellen Welt immer noch existieren mögen, da man feststellen kann, dass die gelieferten Beispiele von mathematischen Ideen die Frage nach dem Anfang ihrer Existenz nicht beantworten können.\footcite[vgl.][S. 99]{Hirschberger} 
\zitatblock{\enquote{Auch wegen dieser unausschöpfbar reichen, zeugenden Fruchtbarkeit ist die Ideenwelt die stärkere Wirklchkeit. Darum unterscheidet also Platon die Ideenwelt [\dots] von der sichtbaren Welt [\dots] und erblickt nur in jener die wahre und eigentliche Welt, in dieser aber bloß ein Abbild, das in der Mitte steht zwischen Sein und Nichtsein.}\footcite[vgl.][S. 100]{Hirschberger}}
Die Stelle die hier vermutlich gemeint sein sollte ist Pol. 478d-e, wo es allerdings um die Meinung geht und wie diese gegenüber der Erkenntnis zu verstehen ist. Es geht darum, dass das Meinen weder als Unwissenheit noch als Erkenntnis beschrieben wird und damit weder sein noch nicht-sein aufweist. Die Frage von falschen Aussagen, also auch von Meinungen, wird im Sophistes nochmal behandelt, wo das Nicht-Seiende als dennoch Seiendes beschrieben wird. 
Es wird jedoch versucht zu unterscheiden zwischen der ersten lexikalischen Bedeutung der Trennung (chorismos) als eine räumliche Trennung, wobei Platon den Ideen in Tim 52a-b räumliche Ausdehnung abspricht. Allerdings wird daraufhin lediglich festgestellt, dass Platon keine weiter Erklärung diesbezüglich liefert, da er auch nicht mit den Begriffen der Transzendenz und Immanenz hantiert.\footcite[vgl.][S. 34f]{DisseMetaphysik} Man muss aber zu Gute halten, dass das Problem erkannt wird, wenn man von der Transzendenz der Ideen, im Gegensatz zur raumzeitlichen Welt, spricht.\footcite[vgl.][S. 35]{DisseMetaphysik} 
Es wird aber dann eingeräumt, wenn man behauptet die Ideen sind und sie alleine sind das Sein, nicht zulässig ist, da somit keine Dinge sein könnten. Daher muss es irgendeine Form geben, bei der den Dingen ein Sein zukommt, auch wenn es ein leicht abgeleitetes Sein ist.\footcite[vgl.][S. 131]{Martin73}
Graeser stellt somit auch die Frage, \enquote{warum Platon sich genötigt sieht, die tatsächlichen Gegenstände des Wissens von dem Bereich solcher immerhin bekannten Gegenstände zu trennen und als Gegenstandsbereich \emph{sui generis} jenseits der Welt der Erfahrung zu lokalisieren.}\footcite[][S. 135]{GraeserPhiloGeschichte}
Auch wird folgendes eingeräumt: \zitatblock{\enquote{Wenn man aber wie Platon davon ausgeht, dass es so etwas wie Transzendenz gibt, ist der \enquote{méthexis}-Begriff dann nicht unumgehbar? Zwar bleibt die Schwierigkeit bestehen, wie man das Verhältnis von etwas, was von dieser Welt getrennt ist, zu dieser Welt denken soll, wenn kein räumliches Verhältnis zwischen beiden gemeint sein kann, bzw. wie Dinge an etwas Unteilbarem teilhaben können.}\footcite[][S. 48]{DisseMetaphysik}}




\footnote{Es sei darauf hingewiesen, dass noch weiter Stellen aus den Originalen hinzugezogen werden könnten, was von den Autoren auch getan wird, hier jedoch auf ine kleiner Auswahl bezug genommen worden ist. Für weiter Stellen siehe Appendix.}
Wie man jetzt anhand der Interpretationen sehen kann, ist die Deutung der zwei Welten durchaus als problematisch erkannt worden. Es gilt also darzulegen, wie die Autoren diesem Problem beikommen, da dies so nicht stehen gelassen werden kann.
Bevor sich also dem anderen großen Teil dieser Arbeit gewidmet werden kann, muss zuerst behandelt werden, wie die Interpretationen diese beiden Bereiche zusammenführen, da eine völlige Trennung, wie schon von Hrischberger festgehalten nicht möglich ist, da es für Platon eine Einheit des Seins gibt\.footcite[vgl][S. 100]{Hirschberger}

\subsection{Zusammenführung der bisher auseinander gehaltenen Welten}

Es wurde bis hier das Verständnis gezeichnet, dass die Ideenlehre zwei Welten darstellt, die augenscheinlich voneinander getrennt sind, was sich auch u.A. bei Graeser S. 134, 139 und Disse S. 34 finden lies oder auch bei Burgin: \zitatblock{\enquote{In his theory, Plato assumed that the physical world was the sensible realm, as people could grasp it with their five senses, while the world of Ideas was the intelligible realm, as people could comprehend it only with their intellect.}\footcite[][S. 179]{Burgin}}
Es bedarf jetzt unter Zuhilfenahme des Bisherigen nochmal auf diese beiden Welten und wie diese zusammengebracht werden können zu blicken.
Hierbei liefert Hischberger schon eine Lösungsmöglichkeit:
\zitatblock{\enquote{Die Transzendenz der Idee ist keine totale, sondern nur eine modale. Der erkenntnis-theoretische Sinn dieser Begriffe besagt, dass alles Erkennen in der erfahrbaren, raumzeitlichen Welt ein \enquote{Analogismus}, ein Lesen der Sinneswahrnehmung durch Hinbeziehen auf einen urbildlichen Begriff ist}\footcite[][S. 94]{Hirschberger}}
\enquote{Nur ein mangelder Metaphysik- und Transzendenzbegriff [\dots] führt zu der Zweiweltentheorie eines totalen Chorismos, wo in Wirklichkeit nur ein modaler gemeint war, eine \enquote{Trennung} des Seins nach seinem Wesen in Gegründetes und Gründendes. Es ist eine Modifizierung, der es ebensosehr auf die Trennung wie auf die Einheit ankam}\footcite[][S. 96]{Hirschberger}

Aus den Symposion und Timaios Stellen, sowie dem Höhlengleichnis wird die notwendige Bewegung weg von den Sinnesdingen hinauf zu den Ideen beschrieben. Aus der Beschreibung des Höhlengleichnisses geht hervor, dass die Menschen am untersten Ende der Höhle gefesselt sind. Dies ist also die zwangsläufige Ausgangsposition, in der sich jeder befindet. Von dieser Position aus gilt es voranzuschreiten, also zu den Ideen zu gelangen und diese zu erkennen. Damit ist der geradlienige Weg vorgegeben, sodass eine notwendige Verbindung der zwei Bereiche angenommen werden muss, auch wenn es dann ein Übergang in den geistigen Bereich sein wird. 
Disse formuliert es so:  
\zitatblock{\enquote{Als Grundstruktur kann festgehalten werden, dass es Platon um den Gegensatz zweier Welten geht und um den Aufstieg des Menschen von der einen in die andere mittels der Philosophie, d.h. [\dots] wesentlich durch die Tätigkeit des von aller Sinneswahrnehmung getrennten Denkens.}\footcite[vgl.][S. 29]{DisseMetaphysik}}
Nun ist es jedoch so, dass dieser Weg eine andere geistige Richtung aufweist, als es die Beziehung zwischen den Ideen und den Sinnesdingen abbilden.
Dieser richtet sich nämlich von oben nach unten. Oben ist hier die Idee des Guten selbst (Pol. 509 Die Idee des Guten überragt noch alles andere), die die Ideen bestimmt, welche wiederum in den Sinnesdingen stecken. So heißt es von Graeser: 
\zitatblock{\enquote{Was Plato hier im Rahmen der naturphilosophischen Hypothese bezüglich der Existenz zweier Arten von Dingen mit dem Hinweis auf das Merkmal der Eingestaltigkeit zum Ausdruck bringen wollte, ist, dass die Ideen, anders als die raum-zeitlichen Dinge, nur für sich existieren und genau das sind, was die raum-zeitlichen Dinge nur in Form von Eigenschaften aufweisen.}\footcite[][S. 140]{GraeserPhiloGeschichte}} und von Disse diesbezüglich:
\zitatblock{\enquote{Somit ist bei Platon nicht die Sinneswahrnehmung als die Grundlage für unsere Erkenntnisinhalte, sondern die durch das reine Denken zugänglichen Erkenntnisinhalte, die Ideen, sind die Voraussetzung für unser Erkennen durch die Sinne. Auch wenn die Sinneswahrnehmung die Wiedererinnerung bewirkt, das Wiedererinnerte selbst, die Idee der Gleichheit, ist die Voraussetzung dafür, dass man zwei Gegenstände als gleich erkennt. Damit aber erweisen sich die Ideen als völlig unabhängig von der Sinnenwelt.}\footcite[][S. 34]{DisseMetaphysik}}
Hierbei ist scharf anzumerken, dass sich dasselbe Argument auf unterschiedliche Aspekte des Verhältnisses von Ideen und Sinnesdingen beziehen. Von Graeser wird die ontologische , von Disse hingegen die gnoseologische Ebene bedient. Diese nahe Verknüpfung wird von Platon im Liniengleichnis eindeutig dargestellt. Auf der einen Seite die Gegenstände, auf der anderen Seite dann die Erkenntnisweise. 

Diese vollkommene Loslösung von der veränderlichen Welt ist nur mit dem Tod möglich (Dafür muss eine weiter Stelle gegeben werden, aus der das Hervorgeht). Denn wie sollte der Philosoph auf seinem Weg nach oben die ursprüngliche veränderliche Welt verlassen? Dies ist gar unmöglich. Daher ist die Formulierung: \enquote{Sinnenwelt und Reich des Denkbaren stehen sich wie im \enquote{Gastmahl} gegenüber}\footcite[][S. 28]{DisseMetaphysik} nicht zulässig, sofern dies nicht epistemologisch verstanden wird, was hier nicht klargestellt wird.
Die Frage ist also, wie sehr kann davon ausgegangen werden, dass allein schon mit der Verbindung aus Körper und Seele die Körperlichkeit als etwas so Niederes beschrieben wird, es aber erst durch den Tod die wirkliche Auflösung vonstatten gehen kann? Dies wäre so, als würde man die eine Hälfte des Seins, der man unmittelbar anhängt so absprechen und verneinen, ohne die aber der erste Schritt gar nicht möglich wäre, da dies das erste ist, womit man konfrontiert ist und zu dem man einen \enquote{einfachen} Zugang besitzt, von dem man aus erst weitermachen kann. Wozu hätte Platon seine Beschreibungen dann immer mit dem Körperlichen begonnen? Die Irreführung oder den Punkt der wahren Erkenntnis damit nicht zu vernachlässigen.
Deutlich für das Problem wird es in folgender Formulierung: \zitatblock{\enquote*{Grundlage von Platons Metaphysik ist die so genannte Zweiweltenlehre, die Lehre des Gegensatzes zwischen Sinnen- und Ideenwelt, wobei die Ideenwelt als die höhere, weil ewige und geistige Wirklichkeit die der Sinnenwelt zugrunde liegende, sie fundierende ist, nach der die an den Körper gebundene menschliche Seele stirbt.}\footcite[][S. 29]{DisseMetaphysik}} Zuerst wird der Unterschied auf den epistemologischen Aspekt berufen, allerdings direkt danach als Unterschied auf einer höheren/anderen Wirklichkeit beschrieben. Ob es einen wirklichen Gegensatz darstellt muss noch aufgezeigt werden.
Die Sinnesdinge werden dann im Gegensatz zu den Ideen als eine \enquote{Exemplifikation von Ideen} bezeichnet und verfügen somit auch nicht über reines Sein, was Selbstständigkeit ausmacht.\footcite[vgl.][S. 146]{GraeserPhiloGeschichte}
\enquote{physical being is a kind of reality, but a kind of reality which can neither exist by itself nor be known or explained from itself.}\footcite[][S. 162]{Vogel}
\zitatblock{\enquote{\enquote{Partial being}- so it appears to be in that remarkable passage of Rep. V where Plato's Sokrates argues that the object of that lower form of cognition which is not concerned with the full \enquote{being} of \enquote{things themselves} but is a \enquote{view} of thing presenting themselves to our senses, none the less is always \emph{something}. Now \enquote{something} could not be percieved if it were not existing, - just \enquote{non-being}. It \emph{must} have a share of \enquote{being} in it, though it is not \enquote{being} in the full and total sense. It must be a mixture of \enquote{being} and \enquote{non-being}, a middle thing [\dots].}\footcite[vgl.][S. 163]{Vogel}}
\enquote{[\dots] he argues that the lower form of cognition which is not concerned with perfect being but with imperfect things, nevertheless has one definite object: doxa, so he says, is always \enquote{having a view of \emph{something}}. It is just impossible to have a view of nothing. \emph{Ergo} \enquote{being} cannot be denied to that which is the object of \enquote{a view}. Yet it is not full being: it must be \emph{both being and non-being} [\dots].}\footcite[vgl.][S. 165]{Vogel}
Der Punkt, den Vogel hier machen will, ist, dass die beiden Arten des Seins keinen gleichwertigen oder entgegengesetzte Pole sind, sondern eine ontologisches Schema der Unterordnung darstellen.\footcite[vgl.][S. 165]{Vogel}

%Es wird jedoch versucht zu unterscheiden zwischen der ersten lexikalischen Bedeutung der Trennung (chorismos) als eine räumliche Trennung, wobei Platon den Ideen in Tim 52a-b räumliche Ausdehnung abspricht. Allerdings wird daraufhin lediglich festgestellt, dass Platon keine weiter Erklärung diesbezüglich liefert, da er auch nicht mit den Begriffen der Transzendenz und Immanenz hantiert.\footcite[vgl.][S. 34f]{DisseMetaphysik} Man muss aber zu Gute halten, dass das Problem erkannt wird, wenn man von der Transzendenz der Ideen, im Gegensatz zur raumzeitlichen Welt, spricht.\footcite[vgl.][S. 35]{DisseMetaphysik} Also dass nicht mehr über etwas gesprochen werden kann, wenn man diese Sache außerhalb des logischen Raumes verortet, wenn man ihr die absolute Transzendenz zuspricht.\\


%Es wird von ihm das Verständnis gezeichnet, dass die Ideen und die Welt der Ideen eine \enquote{Ideale Wirklichkeit} darstellen und auch nach dem möglichen Vergehen der materiellen Welt immer noch existieren mögen, da man feststellen kann, dass die gelieferten Beispiele von mathematischen Ideen die Frage nach dem Anfang ihrer Existenz nicht beantworten können.\footcite[vgl.][S. 99]{Hirschberger} Keine Verweise auf das Original. Dabei auch schwierig ist folgende Formulierung:\enquote{Ferner bilden sie die obersten Strukturpläne der Welt, ohne ihrerseits davon abhängig zu sein. Sie sind das Sein des Seienden.}\footcite[][S. 99]{Hirschberger} Diese Zuschreibung ist eigentlich nur der Idee des Guten vorbehalten. Die Frage ist, wenn man dieser Formulierung kurz nachgehen würde, wären nur die Sinnesdinge Seiend, die Ideen hingegen nicht.
Direkt im Anschluss wird damit auch den Dingen die Möglichkeit abgesprochen den \enquote{reinen Wert und Wesen selbst} zu erreichen.\footcite[vgl.][S. 100]{Hirschberger} (Phaidon 75b)
Gänzlich unbrauchbar wird diese Ansicht, als Hirschberger es wie folgt formuliert:
\zitatblock{\enquote{Auch wegen dieser unausschöpfbar reichen, zeugenden Fruchtbarkeit ist die Ideenwelt die stärkere Wirklchkeit. Darum unterscheidet also Platon die Ideenwelt [\dots] von der sichtbaren Welt [\dots] und erblickt nur in jener die wahre und eigentliche Welt, in dieser aber bloß ein Abbild, das in der Mitte steht zwischen Sein und Nichtsein.}\footcite[vgl.][S. 100]{Hirschberger}}
Auf die Formulierung des \enquote{in der Mitte zwischen Sein und Nichtsein} wird zu einem späteren Punkt eingegangen.
%Eine interessante Sache, die von Hirschberger hervorgebracht wird, ist die Unterscheidung von zwei Möglichkeiten der Diairesis entweder von oben nach unten, wie es im \emph{Sophistes} durchgeführt worden ist, aber auch von unten nach oben, indem man das Allgemeine aus dem Individuellen heraushebt, um schlussendlich an dem obersten Absoluten anzukommen.\footcite[vgl.][S. 106f.]{Hirschberger} 
%Es geht mit dieser Dialektik darum, dass es um die Erklärung des gesamten Seins durch Aufweis der Strukturidee der Welt geht.\footcite[vgl.][S. 107]{Hirschberger}
%\enquote{Und schließlich geht es in ihr, sofern sie das ganze Sein zusammenschaut und in ihm überall die Parousie der Idee des Guten entdeckt, um den Nachweis der Fußspur Gottes im All.}\footcite[][S. 107]{Hirschberger}
%\enquote{So ist für Platon Dialektik im eigentlichen Sinn viel mehr als nur Logik, sie ist immer Metaphysik und wird als solche zugleich zur Grundlage der Ethik, Pädagogik und Politik}\footcite[][S. 108]{Hirschberger}
%Oder wie es Burgin zusammenfasst: \zitatblock{\enquote{In his theory, Plato assumed that the physical world was the sensible realm, as people could grasp it with their five senses, while the world of Ideas was the intelligible realm, as people could comprehend it only with their intellect.}\footcite[][S. 179]{Burgin}}
%Dies als eindeutige Zusammenfassung, dass sie erkenntnistheorisch voneinander getrennt sein müssen.




%\subsubsection*{Graeser Geschichte der Philosophie Band II Antike Platon}
%\enquote{[Die Zwei-Welten-Lehre] unterscheidet zwischen einer raum-zeitlichen Welt des Werdens und einer jenseits von Raum und Zeit befindlichen Welt des Seins.}\footcite[][S. 133]{GraeserPhiloGeschichte}
%\enquote{Jedenfalls bedeutet Platos Zwei-Welten-Lehre eine radikale Unterscheidung zwischen Sein und Werden, zwischen Wirklichkeit und Schein, zwischen Echtheit und Unechtheit.}\footcite[][S. 134]{GraeserPhiloGeschichte}
%So \enquote{radikal} darf diese Unterscheidung nicht gezogen werden. Leider wird diese radiakle Unterscheidung so stehen gelassen ohne weiter darauf einzugehen.
%Graeser stellt somit auch die Frage, \enquote{warum Platon sich genötigt sieht, die tatsächlichen Gegenstände des Wissens von dem Bereich solcher immerhin bekannten Gegenstände zu trennen und als Gegenstandsbereich \emph{sui generis} jenseits der Welt der Erfahrung zu lokalisieren.}\footcite[][S. 135]{GraeserPhiloGeschichte}
%\enquote{Platon charakterisiert die Idee als das, was wirklich ist, ihre raum-zeitlichen Instanzen jedoch als etwas, was nicht wirklich ist.}\footcite[][S. 139]{GraeserPhiloGeschichte}
%\enquote{die Ideen, die nur dem Denken zugänglich sind.}\footcite[][S. 139]{GraeserPhiloGeschichte} Wie unterscheidet sich diese Formulierung zu der, dass Ideen nur gedacht werden können? Dieses \emph{nur} bringt das Problem mit sich, dass die Dinge von den Ideen völlig getrennt sind. Das heißt, dass eine Verbindung von Ideen und Dingen so angedeutet wird, dass es diese nicht gibt. Ebenso wie die Formulierungen, dass die Ideen den Dingen zugrunde liegen würden. 
%\enquote{Was Plato hier im Rahmen der naturphilosophischen Hypothese bezüglich der Existenz zweier Arten von Dingen mit dem Hinweis auf das Merkmal der Eingestaltigkeit zum Ausdruck bringen wollte, ist, dass die Ideen, anders als die raum-zeitlichen Dinge, nur für sich existieren und genau das sind, was die raum-zeitlichen Dinge nur in Form von Eigenschaften aufweisen.}\footcite[][S. 140]{GraeserPhiloGeschichte}
%Im Timaios wird die Stelle 27d so von Graeser verwendet, dass er die Dinge der Wahrnehmung als \enquote{niemals Seiendes}\footcite[vgl.][S. 140]{GraeserPhiloGeschichte} beschreibt. Bei Schleiermacher heißt es im Kontext: \enquote{Was ist das stets Seiende und kein Entstehen Habende und was ist das stets Werdende, aber nimmerdar Seiende.} oder auch in einer anderen Übersetzung:\enquote{Wie haben wir uns das immer Seiende, welches kein Werden zulässt, und wie das immer Werdende zu denken, welches niemals zum Sein gelangt?}
%Aus der Grundporblematik der immerseienden Ideen und der werdenden Sinnesdingen werden von Graeser drei Interpretationswege dargelegt, dass Platon (1) eine Theorie der Existenz-Stufung vor Augen hatte, (2) den Gedanken eines Mehr an Sein auf das \emph{esse essentiae} bezogen wissen wollte oder (3) er beide Vorstellungen nicht als distinkte philosophische Optionen erkannte und sie konfundierte.\footcite[vgl.][S. 140]{GraeserPhiloGeschichte} Hierbei schließt Graeser 1 und 3 als Möglichkeiten aus.
%\enquote{So lässt sich weder zeigen, dass Plato die Ausdrücke \enquote{existiert} und \enquote{ist wirklich} synonym verwendete, noch findet sich irgendwo eine Argumentation aus der hervorgeht, dass x in höherem Grade über F verfügt als y und x deshalb mehr existiert als y.}\footcite[][S. 140]{GraeserPhiloGeschichte}
%Ein interessanter Punkt von Graeser ist, dass er dieses \enquote{wirkliche Sein} der Ideen so darstellt, dass es eher einen semantischen Charakter aufweist. Dies wird so gemeint, dass Sätze der einen Art - wenn es um Ideen geht - wahr ohne weitere Qualifikation sind, die anderen Sätze - über Sinnesdinge - nicht wahr ohne weitere Qualifikation sind. Somit haben Ideen das \enquote{wirkliche Sein}, die Dinge der raum-zeitlichen Welt hingegen nicht.\footcite[vgl.][S. 141]{GraeserPhiloGeschichte}
%Graeser sieht hier die Ideenlehre so, \enquote{dass die sog. Einzeldinge über keinen ontologisch unabhängigen Status verfügen. Sie sind sozusagen rein relationale Gebilde, d.h. Wesen, die ihren Charakter und ihre Existenz ganz und gar den Ideen verdanken.}\footcite[][S. 145]{GraeserPhiloGeschichte}
%Die Sinnesdinge werden dann im Gegensatz zu den Ideen als eine \enquote{Exemplifikation von Ideen} bezeichnet und verfügen somit auch nicht über reines Sein, was Selbstständigkeit ausmacht.\footcite[vgl.][S. 146]{GraeserPhiloGeschichte}
%Auch hier wird wieder auf Pol. 497b9-10 verwiesen, wo es eigentlich um die Meinung als etwas zwischen Erkenntnis und Unkenntnis steht also weder Sein noch nicht sein und weder beides noch keines von beiden ist.
%Direkt im Anschluss daran wird das Problem der Teilhabe Relation besprochen, dass es nur zwei Möglichkeiten gibt, wie die Ideen an den Dingen teilhaben können, diese beiden Möglichkeiten wieder in eine Aporie führen.
%\zitatblock{\enquote{(P) Wenn es so etwas wie eine Teilhabe gibt, dann entweder (Q) in der Weise, dass die Idee als Ganzes partizipiert wird, oder aber (R) in der Weise, dass nur ein Teil von ihr partizipiert wird. Nun kann es weder Teilhabe am Ganzen der Idee geben (~Q), noch Teilhabe an einem Teil der Idee(~R). Also gibt es keine Teilhabe (~P).}\footcite[][S. 146]{GraeserPhiloGeschichte}}

%\subsubsection*{Gottfried Martin: Platons Ideenlehre (1973) (70/CD 3067 M381)}
%Zwei Weisen des Seins (Phaidon 79A)\footcite[vgl.][S. 37]{Martin73}(duo eide ton) (zwei Ordnungen der Wesen\footcite[][S. 43]{PhaidonKassner} oder zwei Arten der Dinge\footcite[][S. 247]{PhaidonSchleiermacher})
%Diese Übersetzungen zeigen, dass sich diese verschiedenen Formen, Modi oder Arten des Seins als Schwierigkeit herausstellen, also wie diese Unterscheidung gemeint sein sollte, da Martin keine ontologische Implikrationen in die Übesetzung tragen möchte.\footcite[vgl.][S. 37]{Martin73} Diese Unterscheidung der zwei Weisen des Seins gibt Martin die reine Unterscheidung zwischen den Ideen und den Dingen.
%Unter ontologischen Gesichtspunkten widmet sich Martin dann später auf S. 216
%Da in Phaidon 79a eindeutig ein Plural verwendet wird, es in jeglichen Übersetzungen jedoch nicht möglich ist eine adäquate Übersetzung zu liefern, wird die vorsichtige Übersetzung der \enquote{zwei Weisen des Seins} verwendet, da gerade die Dinge von den Ideen unterschieden werden sollen.\footcite[vgl.][S. 216]{Martin73} \enquote{Wie immer man sich in der Übersetzung helfen möge, Platon kennt zwei Weisen des Seins, das Sein der Ideen und das Sein der Dinge, er unterscheidet sie und der stellt sie zugleich zusammen.}\footcite[][S. 216]{Martin73}

%\enquote{Auch hier setzt Platon mit der ausdrücklichen Formulierug ein: zwei Weisen. (Politeia 509D Liniengleichnis)}\footcite[][S. 39]{Martin73}
%Zu Beginn erst nur terminologische Herangehensweise an platonische Begriffe.
%Die Frage ist die Unterscheidung und die unterschiedliche Bedeutung von eidos und idea, welche von Martin nicht als strikt voneinander trennbar aufgefasst wird.\footcite[vgl.][S. 39]{Martin73} in Anlehnung an Ritter.\\
%Aus Phaid. 79a erfasst Martin zwei Kriterien der Unterscheidung der Ideen und Dingen in der Weise ihrer Erfassbarkeit und dass sie in verschiedener Weise sind.\footcite[vgl.][S. 40]{Martin73}
%Leider geht Martin hier nicht weiter auf die Unterscheidung der Erfassbarkeit ein, als darauf, dass die Ideen unsichtbar und nur denkbar und die Dinge sichtbar und in ständiger Veränderung sich befinden. 
%Zweite Unterscheidung vom Sein her. Ideen sind unvergänglich, Dinge sind unter ständigem Wandel begriffen. \footcite[vgl.][S. 41]{Martin73}
%42 Die Ideen sind ewig, immerwährend. Wie sind sie dann im Sophistes in Bewegung. Das geht wohl damit einher, dass man die Ideen als denkbar verstehen muss, also auch verstanden werden müssen. Dabei muss das Element der Bewegung bzw. Veränderung auftreten, da man sonst nicht von einer Erkenntnis sprechen kann. Das Ding (auch die Idee) kann erst dadurch als erkannt bezeichnet werden, wenn man eine Veränderung der Idee von nicht erkannt zu erkannt festlegen kann. (Phaidros 247e, Sophistes 240b, 246b, 248a, 249d)

%{\color{red}{Aus Pol. 479e wird gezeigt, dass \enquote{[\dots] die Existenz der Ideen auf de[m] Unterschied zwischen echtem Wissen [\dots] und bloßen Meinungen [gründet].}\footcite[vgl.][S. 51]{Martin73}}
%Dafür bedarf es aber die Existenz eines besonderen Bereichs, eben der Ideen, so dass das echte Wissen sich auf die Ideen bezieht, das bloße Meinen aber auf die Dinge.\footcite[vgl.][S. 51]{Martin73}
%Ähnliches findet sich laut Martin auch im Timaios 27d, was die Unterscheidung von Ideen und Dingen nochmals unterstreicht.
%Aus der Verschiedenheit von Einsicht (nous) und Meinung (doxa) schließt Martin die Notwendigkeit auch ihre verschiedener Gegenstandsbereiche.\footcite[vgl.][S. 52]{Martin73}}
%Die Zweiweltentheorie (bei Platon, vorher andere Vertreter) S. 89ff.
%Diese Zweiweltentheorie wird so dargestellt, dass der Transzendenzgedanke im Phaidros (Götterreise), Symposion, Phaidon (Unsterblichkeit der Seele) und Politeia an den Mythoserzählungen angelegt wird und damit diese Überweltlichkeit als eine Zweiweltlichkeit interpretiert wird.\footcite[vgl.][S. 88ff.]{Martin73}
%\enquote{Eine jede Erkenntnis kann nichts anderes sein als eine Erkenntnis durch Ideen, und jede Erkenntnis der Ideen wiederum kann nichts anderes sein als Wiedererkenntnis und also Wiedererinnerung. Unsere Seele muss also die Ideen schon vor der Gebrt in dieses Leben gekannt haben. Daraus folgt weiterhin, dass die Ideen außerhalb dieser sinnlich wahrnehmbaren Welt existieren müssen, und Platon sagt dies ausdrücklich.}\footcite[][S. 92]{Martin73}
%\enquote{Die Seele trennt sich im Tode vom Leib. Sie wird gerichtet werden. Ihr Ziel ist der Weg nach oben. Dies darf nicht in einem humanitären Entwicklungssinn als eine Höherentwicklung verstanden werden, sondern es meint in der Tat die dort oben liegende Welt des reinen Seins.}\footcite[][S. 93]{Martin73}
%Diese Zweiweltentheorie ist rein auf die Trennung von Leib und Seele und die Unsterblichkeit der Seele nach dem Tod bezogen. Also auch die Vollendung des Aufstiegs nach dem Tod zu dem \enquote{dort oben liegenden Welt des reinen Seins}\footcite[][S. 93]{Martin73}
%Es werden laut Martin die drei Ausdrücke TOPOS, GENOS und KOSMOS verwendet, um die beiden Bereich der Ideen und der Dinge zu charackterisieren.\footcite[vgl.][S. 95]{Martin73}
%Leider wird die topos Unterscheidung primär auf das Höhlengleichnis angewandt ebenso auf die Mythos Erzählungen und nicht konsequent von den anderen Bedeutungen abgerenzt und deutlich unterschieden.
%\enquote{So erweisen sich auch unter diesem Gesichtspunkt die vier großen Ideendialoge, der Phaidros, das Symposion, der Phadion und die Politeia als einheitlich. Die Welt hier unten ist nicht die wahre Wirklichkeit, die wahre Wirklichkeit ist die Welt der Ideen dort oben. Ich darf wiederholen: Ich will nicht sagen, dass dies Platons eigentliche Überzeugung und dass diese Zweiweltentheorie das letzte Wort Platons ist. Man muss sich vielmehr vor Augen halten, dass diese räumliche Darstellung des Unterschieds zwischen Ideen und Dingen eine Notwendigkeit unseres Denkens und Sprechens ist.}\footcite[][S. 96]{Martin73}
%\enquote{Lässt man sich von einer naheliegenden Bedeutung des Wortes \enquote{Metaphysik} leiten und versteht man unter Metaphysik die Lehre von dem, was hinter der Natur liegt, dann ist die Ideenlehre des Phaidon der Anfang und der Urspurng der Metaphysik. Hier werden [\dots] die Ideen als etwas verstanden, was hinter und über der Natur liegt.[\dots] Was ist dies transzendente Sein der Ideen?}\footcite[vgl.][S. 128]{Martin73}
%Nimmt man die Lehre von den transzendenten Realitäten als Ziel einer Philosophie, dann sind Phaidon und Phaidros Anfang und Ursprung. Wenn man aber die Frage: Was ist Sein? nimmt, dann wird der Sophiestes zu einem zentralen Dialog.\footcite[vgl.][S. 129]{Martin73}
%Probelmatisch wird es dann wieder, wenn es heißt \enquote{Erst nach dieser Entdeckung der Ideenlehre kann Platon das Sein in zwei Weisen des Seins differenzieren, und wir hatten die ausdrückliche Differenzierung in Phaidon (79a) und in der Politeia (509d) gefunden.}\footcite[][S. 131f]{Martin73} Im Gegensatz zu Parmenides und Heraklit, die zwar sagen konnte, alles ist Wasser, aber nicht was ist das Wasser?
%Es wird aber dann eingeräumt, wenn man behauptet die Ideen sind und sie alleine sind das Sein, nicht zulässig ist, da somit keine Dinge sein könnten. Daher muss es irgendeine Form geben, bei der den Dingen ein Sein zukommt, auch wenn es ein leicht abgeleitetes Sein ist.\footcite[vgl.][S. 131]{Martin73}
%Damit würde man notwendigerweise wieder bei dem Punkt ankommen, den man an Parmenides und Heraklit angesetzt hat, dass man nur ein Oberstes gesetzt hat und nicht mehr nach dem Obersten selber fragen kann.  
%\zitatblock{\enquote{Nach [der Anamnesislehre] ist Erkenntnis immer eine Wiedererinnerung. Das heißt doch, und Platon sagt dies auch ausdrücklich, daß die Seele die Ideen in einem früheren Leben vor der Geburt kennengelernt haben muß. Dies wiederum kann doch nur heißen, daß die Ideen nicht in dieser Welt sind}\footcite[vgl.][S. 160]{Martin73}}
%Diese Auffassung ist nicht haltbar. Dies hat damit zu tun, dass mit dieser Ansicht einige Implikationen einhergehen müssen, auf die wir keinen Zugriff oder keinen Zugang haben. Was ich damit meine ist, dass man, wenn man davon ausgeht, dass die Seele die Ideen vor der Geburt in einer \emph{anderen} Welt kennengelernt oder geschaut hat, dass wir eigentlich nichts über diese Welt aussagen können. Dabei bedarf es der Annahme, dass wir entweder sagen, man kann nur etwas über diese Welt aussagen, das haltbar bleibt, wenn wir eine nochmals übergeordnete Welt annehmen, die die Ideen und Sinneswelt verbindet, auf die allerdings wiederum zugegriffen werden kann, da sonst keine logsichen Schlüsse und Aussagen möglich wären. Oder man muss in irgendeiner Weise beweisen können, dass wir einen logischen Zugang zu diese anderen Welt aufweisen, wodurch man im eigentlichen Sinne durchaus Aussagen treffen könnte, dies aber nicht getan wird. Zumindest nur bis zu diese Punkt.
%Dies ist sehr eng an das Konzept des Einen und Vielen geknüpft, ergo der Dialektik. Der Wissenschaft des Auseinanderhaltens und Zusammenführens von Begriffen.

%Natürlich muss man diese beiden Bereiche dahingehend differenzieren und eine gedankliche Grenze setzen, da nur so von beiden in ihrer isolierten Form gesprochen werden kann. Nur dadurch ist es möglich, dass die beiden Bereiche aufeinander Auswirkung und überhaupt Wirkung haben.
%Die Frage dabei lauter also von Martin, wie das CHORIS im griechischen gebraucht wird. Entwerder eine rein räumliche Trennung, was gerade dann für das Problem sprechen würde, oder ob auch andere Trennungen oder Teilungen/Sonderungen damit gemeint sein könnten. (S. 163ff.)
%Die andere Möglichkeit ist die begriffliche Trennung von Dingen, die wohl am meisten von Platon vertreten wird\footcite[vgl.][S. 165]{Martin73}. Beispiele: Buch 10 der Politeia: 595A, Laches 195A, Eutydemos 289C
%Es liegt bei Platon faktisch, bei Aristoteles dann explizit ein räumlicher, ein zeitlicher und ein begrifflicher Chorsimos vor.\footcite[vgl.][S. 166]{Martin73}
%\enquote{In der Kritik der Ideenlehre geht Aristotles immer davon aus, dass der Chorismos bei Platon immer rein räumlich gemeint ist.}\footcite[][S. 166]{Martin73}
%Räumliche Trennung der Begriffe im Höhlengleichnis auf welcher Ebene? Es lassen sich mehrere Ebenen aufzeigen. Siehe Liniengleichnis. Die Trennlinie ist scharf, aber bis zu welchem Grad darf man die Auflösung des Schärfegrades, mit dem man das Bild betrachtet, drehen, bis man den Blick dafür verliert, dass es sich alles dennoch in einer Welt abspielt, denn wie könnte dann der Aufstieg überhaupt erfolgen.
%Es stellt sich bei der Methexis die Frage, \enquote{ob es sich um eine Teilnahme der Dinge an den Ideen oder um eine Anwesenheit der Ideen in den Dingen oder um eine Gemeinschaft der Ideen und der Dinge handelt.}\footcite[vgl.][S. 170]{Martin73}

%Martin bleibt dabei, dass Platon nicht ausreichende Texte überliefert hat, aus denen hervorgeht, was im Parmenides über den Chorismos und die Methexis (Anteilhabe) vorgetragen worden ist, um diese aufgeworfenen Probleme lösen zu können.\footcite[vgl.][S. 173f.]{Martin73}

%\subsubsection*{Martins Ousia}
%\enquote{In der Tat behauptet Platon die Existenz der Ideen und er formuliert das, indem er sie als Ousia bezeichnet und indem er sie als ON bezeichnet.}\footcite[][S. 187]{Martin73}
%Martin kommt nach einiger Analyse - auch anderer Autoren - zu dem Schluss, dass Ousia nicht nur das Sein der Ideen meint, sondern auch das Sein der Dinge. Dabei bleibt er nicht bei dieser Deutung stehen, sondern geht auch noch auf die Bedeutung von Ousia als das allgemeine Sein sowohl der Ideen als auch der Dinge ein.\footcite[vgl.][S. 195ff.]{Martin73}
%\zitatblock{\enquote{Der Text bezeichnet ausdrücklich beides, das Schreiben und das Sprechen als unmöglich. So sagt Platon etwa:\enquote{Soviel wenigstens weiß ich, dass ich, wenn ich es ausspräche oder niederschriebe[\dots]}(Pol. 341 D). Nimmt man die Philosophie so, wie Platon sie hier nimmt und wie sie immer genommen werden sollte, dann kann sie weder ausgesprochen noch niedergeschrieben werden.}\footcite[][S. 251]{Martin73}}
%Das Höhlengleichnis hier nocheinmal herangezogen wird deutlich, dass es der Aufstieg also eine Bewegung, eine Handlung, ist, die die Philosophie für Platon am Ende ausmacht. Denn rein durch das Sprechen und Hören oder Schreiben und Lesen wird niemand zur Idee des Guten schreiten können. Lediglich durch eine innere Anstrengung, die durch die Rede oder den Text entfacht wird.
%\subsubsection*{Martins Chorismos in den vier großen Ideendialoge}

%\begin{itemize}
 %   \item {Phaidros schwer bestritten werden, dass es Chorsimos gibt. Die Stellen hier sind die Beschreibung der Reise der Seele auf dem Göttewagen mit den Göttern, wo die Ideen im eigentlichen Sinne vor der Geburt geschaut werden. Das ist vermutlich einfach zu viel und würde zu weit weg führen.}
  %  \item {Symposion mit dem Aufstieg zur Idee des Schönen}
   % \item {Politeia mit dem Höhlengleichnis, aber auch das Sonnen und Liniengleichnis}
    %\item {Phaidon. Chorsimos also Trennung von Ideen und Dingen über die Anamnesislehre, also Trennung in der Erkennbarkeit und Widererinnerung}
    %\item {Parmenides (faktische Reflexion). Explizite Behandlung des Chorismos Problems. Hier wird auch das Wort \enquote{choris} verwendet (130BCD)}
    %\item {Sophistes (methodische Reflexion)}
%\end{itemize}
%\footcite[vgl.][S. 160]{Martin73}


\subsection{Welche Probleme gibt es mit dieser Ansicht}
Übergang zu diesem Unterkapitel ist wohl das Auslassen der Idee des Guten als doch noch in dem System angesiedelt aber leider von vielen Autoren weggelassen in der Ausformulierung ihrer Interpretation, obwohl die Idee des Guten das Ende und das Höchste der Lehre markiert.
Das grundlegende Problem ist wohl, dass die Ideenlehre nie dahingehend unterschieden wird, welches Ziel Platon in den angegeben Stellen hatte. Also auf welche Frage, oder welches Problem hin, die Dialoge ausglegt sind und welche Interpretationen außerdem möglich wären. Legitimität erhält dieser Einwand darin, dass die Dialoge nie als festes Werk und Lehre dargestellt worden sind und werden sollten, sondern dass Philosophie eigentlich immer im Dialog stattfinden soll. (Dies wird von Martin bestritten S. 251f.) Damit soll der Blick darauf gerichtet werden, warum die Autoren hier nie die Frage gestellt haben, ob ein gnoseologischer oder ontologischer Aspekt von Platon angesprochen worden ist, oder sogar beides.  
Wenn man von dieser \enquote{eigentlichen} Wirklichkeit oder der \enquote{eigentlichen} Welt spricht, kommt man in der Darstellungsform und der Argumentation vom Weg ab. Die Phaidon Stelle sagt zwar, dass die Sinnenwelt zwar danach strebt, wie die Ideenwelt zu sein, allerdings dahinter zurück bleibt, hier aber nicht von dem ewigen Teil des Menschen - der Seele - gesprochen wird, welcher eben auch, wie die Ideen, ewig ist und somit diesen Zugang auf erkenntnistheorischer Ebene besitzt und somit zum einen Zugang zu dieser Ebene besitzt und zum anderen auch diese mögliche Angleichung haben kann. 
Zum anderen ist es so, wenn man von dieser eigentlichen Wirklichkeit spricht, es den Anschein erweckt, dass man diese Wirklichkeit als eigentliches oder wahres Seiendes bezeichnet. Umgekehrt ließe das nur zu, dass die Sinnenwelt nicht wahres Seiendes oder uneigentliches Wirkliches ist. Somit bliebe uns nichts anders übrig als der Sinnenwelt einen Grad des Seienden abzusprechen. Problematisch ist es in der Hinsicht, dass wir uns in dieser Sinnenwelt befinden und dies unser \enquote{Startpunkt} ist, wie es im Höhlengleichnis beschrieben wird. Ebenfalls ist es so, dass wir, selbst wenn man die Höhle verlassen und die Sonne gesehen hat, die Aufgabe haben zurück in die Höhle zu gehen. Also am Ende nicht aus der Welt zu fallen oder diese gar zu verlassen.
Es ist ebenfalls so, dass es einfacher ist von unten nach oben zu gehen, also von den Sinnendingen anzufangen, um dann zu den Ideen hoch zu gehen, als von oben herunter zu konsturieren. Man siehe hier mögicherweise die Konstruktion des Staates selbst, wo mit den Bauern begonnen wird und erst darauf aufbauend der Staat nach oben konstruiert wird. Daher ist fraglich, inwiefern diese \enquote{wahre} oder \enquote{wirkliche} Ideenwelt oder -ebene besser oder hilfreicher in der Hinsicht ist, von welchem Bezugspunkt man ausgeht, der einem zur Verfügung steht. Dagegen spricht allerdings die Vorgehensweise im Sophistes der Diairesis von Begriffen, wie etwa dem Angelfischer. Hierbei wird von oben nach unten vorgegangen, um am Ende der Untersuchung beim Angelfischer zu landen. Der Weg nach oben ist äußerst mühsam und schmerzlich. Der Weg nach unten allerdings ist, wie im Sophistes gesehen, immer wieder zu einem anderen Ergenbnis gekommen, sodass immer wieder von Vorne begonnen werden musste und daher immer mit dem Problem versehen nicht sicher zu sein, ob man nun die richtige \enquote{Abzweigung} genommen hat. Dieses Problem hat man von unten nach oben nicht, da es nur einen Endpunkt gibt. 

Es wirkt fast so, als würde man versuchen die beiden Seinsbereiche, die man identifiziert hat miteinander zu verbinden, ohne ein Drittes zu setzten, das die beiden Bereiche verbindet, was zu dem altbekannten Problem des infiniten Regresses gelangt, wo man wiederum ein drittes benötigt, um das erste Dritte mit einem der ursprünglichen zwei zu verbinden.

Es wird zwar von Martin erkannt, dass das reine Sein der Ideen eine Unmöglichkeit darstellt, was an der Idee des Schönen exemplifiziert wird, dass es also somit keine schönen Einzeldinge geben könnte, aber hier wird nicht notwendig zuende gedacht, dass es das Sein der Dinge erst in dem Sein von Dingen und Ideen zusammengedacht und in der Idee des Guten überwunden werden kann. Also in eine Einheit gebracht werden kann. 

Zu Tim 52a sei folgendes gesagt: Im Original-Text steht allerdings wieder, dass es um die Unterscheidung von Vernunft und richtiger Meinung geht. Denn dies muss unterschieden werden, da wir sonst alles, was wir vermittels des Körpers wahrnehmen, als höchst zuverlässig annehmen müssen. (Tim 51d) Dies geht wieder in die Richtung des Erkennens von Wahrheit und nicht von Zwei Welten. Das wird deutlich, da bei Disse in das Zitat das Wort \emph{Gebiet} nach \emph{Zweite} eingefügt worden ist, wo eigentlich auf die richtige Meinung rekurriert wird.
Es wird daraufhin wieder Pol. 475d-480a zitiert und falsch ausgelegt dahingehend, dass wieder von Dingen, die mehr oder weniger Sein haben können, gesprochen wird. Jedoch beruft sich die zitierte Stelle ebenfalls nur auf den Unterschied von Meinungen und waher Erkenntnis. Da hier wahre Meinung als weder wirklich seiend noch nicht-seiend verstanden wird, muss hier die wahre Meinung als etwas zwischen Sein und nicht-Sein verortet werden. Hier geht es aber nicht direkt um das Sein von Sinnesdingen und Ideen.\footcite[vgl.][S. 37f.]{DisseMetaphysik} 

\enquote{Das Gebiet des Tageslichts außerhalb der Höhle ist der Bereich dessen, was uns durch reines Denken zugänglich ist. Die Sonne aber wird mit dem höchsten Punkt im Bereich des Denkbaren verglichen.}\footcite[][S. 49]{DisseMetaphysik} Das steht so nicht im Text. Die Person wird von der Höhle aus der Höhle geführt. Es steht nichts im Text, dass hier von einem Übergang in eine \emph{andere Welt} gesprochen wird. Auch nicht, dass hier der Bereich des Denkens stattfindet. Offensichtlich wird nur eine Höhle verlassen. Das Höhlengleichnis ist lediglich eine Metapher dafür, wie der Weg bestritten wird und nicht, dass aus der einen Welt in eine andere Welt gegangen werden soll. Dass die Ähnlichkeit der Sonne im Höhlengleichnis und im Sonnengleichnis nicht zu verkennen ist, ist klar. Aber auch hier ist die Idee des Guten nicht \enquote{außerhalb} der Welt. Nur die direkte Sicht in die Sonne ist nicht möglich, was die Einsicht in die reine Idee des Guten als unmöglich erachtet wird. 
Problem dabei ist die eingehende Auslegung der Metaphysik als die Frage \enquote{[\dots] nach einer wahren Wirklichkeit, nach einem Sein, das mehr als das uns unmittelbare Gegebene ist.}\footcite[vgl.][S. 17]{DisseMetaphysik} Wie soll man sich dieses \emph{Mehr} überhaupt vorstellen, wenn es uns möglicherweise gar nicht zugänglich oder gegeben ist.

%\enquote{Platon unterscheidet somit im Höhlengleichnis zunächst einmal grundsätzlich zwischen zwei Welten und bestimmt eine Bewegung des Menschen, nämlich die Tätigkeit der Philosophie, die ihn von der ersten in die zweite, eigentliche führen soll.}\footcite[][S. 23f.]{DisseMetaphysik} Wo wird das deutlich? Es gibt keinen Verweis auf den Text. Wie sind hier zwei Welten gemeint? Denn die Bewegung beginnt in der Höhle und führt zum Feuer und dann aus der Höhle hinaus. Dieses \enquote{aus der Höhle hinaus} könnte man als einen Übergang in eine andere/zweite Welt deuten, wobei man hierbei sehr aufpassen muss dies nicht als eine räumlich getrennte Welt zu verstehen. Denn wie wäre es möglich diese eine Welt zu verlassen und in diese andere Welt einzutreten, wenn diese vorher nicht schon verbunden gewesen sein müssen.
%Es wird Tim 52a zitiert als Zusammenfassung der Zweiweltenlehre. Im Original-Text steht allerdings wieder, dass es um die Unterscheidung von Vernunft und richtiger Meinung geht. Denn dies muss unterschieden werden, da wir sonst alles, was wir vermittels des Körpers wahrnehmen, als höchst zuverlässig annehmen müssen. (Tim 51d) Dies geht wieder in die Richtung des Erkennens von Wahrheit und nicht von Zwei Welten. Das wird deutlich, da bei Disse in das Zitat das Wort Gebiet nach Zweite eingefügt worden ist, wo eigentlich auf die richtige Meinung rekurriert wird.
Wie kann folgender Satz dann verstanden werden? \enquote{Die geläufigste Interpretation ist, dass es bei Platon letztlich Ideen von allem gibt, was in der Sinnenwelt existiert, mit Ausnahme von Individuen.}\footcite[][S. 31]{DisseMetaphysik} 
%Direkt im Anschluss wird die eigene Aussage nichtig gemacht, als der Demiurg aus dem Timaios hergenommen wird, der sich bei der Erschaffung der Welt sich an den Ideen als Urbilder bedient, um die Welt zu erschaffen. 
Hiernach richten sich die Ideen nach den Dingen. Wie geht das d'accord, dass sich eigentlich alles nach der Idee des Guten richtet und alles auf die Idee des Guten hin ausgerichtet ist? Diese Formulierung dreht dieses Verhältnis aber um, sodass sich die Ideenwelt nach der Sinnenwelt richten müsste. Zudem wird es nach dieser Ansicht schwer Zahlen o.Ä. als \enquote{existierend} zu nennen, wenn man diese doch gar nicht in der Sinnenwelt existent sehen kann. 
%Man könnte hier zum Liniengleichnis schauen und sich das Verhältnis ansehen, welcher Bereich der größte wäre. Hier besteht allerdings kein Konsens darüber, ob der obersten oder der untersten Stufe die größte \enquote{Fläche} zukommt.
%Es wird daraufhin wieder Pol. 475d-480a zitiert und falsch ausgelegt dahingehend, dass wieder von Dingen, die mehr oder weniger Sein haben können, gesprochen wird. Jedoch beruft sich die zitierte Stelle wieder nur auf den Unterschied von Meinungen und waher Erkenntnis. Da hier wahre Meinung als weder wirklich seiend noch nicht-seiend verstanden wird, muss hier die wahre Meinung als etwas zwischen Sein und nicht-Sein verortet werden. Hier geht es aber nicht direkt um das Sein von Sinnesdingen und Ideen.\footcite[vgl.][S. 37f.]{DisseMetaphysik} Diese Darstellung wird dann im Anschluss daran getroffen.
%Auch wird folgendes eingeräumt: \zitatblock{\enquote{Wenn man aber wie Platon davon ausgeht, dass es so etwas wie Transzendenz gibt, ist der \enquote{méthexis}-Begriff dann nicht unumgehbar? Zwar bleibt die Schwierigkeit bestehen, wie man das Verhältnis von etwas, was von dieser Welt getrennt ist, zu dieser Welt denken soll, wenn kein räumliches Verhältnis zwischen beiden gemeint sein kann, bzw. wie Dinge an etwas Unteilbarem teilhaben können.}\footcite[][S. 48]{DisseMetaphysik}}
Disse S. 50: 
Einen transzendenten Bereich nochmals zu transzendieren macht wenig Sinn. Wie soll dies gelingen? Wenn der erste transzendierte Bereich bereits über dem ersten Bereich liegt, wie soll es möglich sein diesen Bereich nochmals als etwas darüber oder übersteigendes zu beschreiben? Wenn der erste überstiegene Bereich schon außerhalb des ersten grundlegenden Bereichs liegt, wie kann hier logisch noch ein weiterer übersteigender Bereich eingeholt werden? Eine reine Staffelung von Definitionen oder von Beschreibungen auf in einem Definitionsbaum, ja, aber dabei verliert der Begriff der Transzendenz seine Bedeutung. Denn nur die Beschreibung in Pol. 509b, dass es noch über das Sein erhaben ist, oder hinaus geht, lässt sich nur der Idee des Guten zuschreiben. 
Das vermutlich ursprüngliche Problem hierbei liegt daran, dass man in der Behandlung des Themas folgendermaßen beginnt: \enquote{Den apriorischen Begriffen unseres Geistes korrespondieren entsprechende Gegenstände. Diese Gegenstandswelt interessiert Platon ebenso wie die Frage nach der Quelle der Wahrheit.}\footcite[][S. 97]{Hirschberger}
Was hier deutlich wird, ist die chronologisch verdrehte Betrachtungsweise dieses Problems. Das heißt, dass aus Sicht Kants und auch mit kantischen Begriffen herangegangen wird, um platonische Philosophie zu beschreiben und zu erklären. Diese These bedarf weitaus mehr Erklärung, die aber nicht geliefert wird. Es wird sich damit begnügt, dass Platon hier so erklärt wird, dass bei Kant nur die Formen a priori sind, bei Platon auch die Inhalte und Platon damit als reiner Rationalist bezeichnet wird.\footcite[vgl.][S.96]{Hirschberger} 
%Nochmal ansehen, was davor geschrieben wird, da es heißt, \enquote{Nur ein mangelder Metaphysik- und Transzendenzbegriff - \enquote{Metaphysik}: das schlechthin unzugängliche \enquote{Jenseitige}- führt zu der Zweiweltentheorie eines totalen Chorismos, wo in Wirklichkeit nur ein modaler gemeint war, eine \enquote{Trennung} des Seins nach seinem Wesen in Gegründetes und Gründendes. Es ist eine Modifizierung, der es ebensosehr auf die Trennung wie auf die Einheit ankam}\footcite[][S. 96]{Hirschberger}
%Hirschberger stellt vorher klar, dass:\zitatblock{\enquote{Die Transzendenz der Idee ist keine totale, sondern nur eine modale. Der erkenntnis-theoretsiche Sinn dieser Begriffe besagt, dass alles Erkennen in der erfahrbaren, raumzeitlichen Welt ein \enquote{Analogismus}, ein Lesen der Sinneswahrnehmung durch Hinbeziehen auf einen urbildlichen Begriff ist}\footcite[][S. 94]{Hirschberger}}
%Fraglich ist allerdings, wie sehr diese Deutung als rein erkenntnistheoretisch oder auch ontologisch gemeint ist. Denn es scheint, als sei diese Deutung primär auf die erkenntnistheoretische Weise eingegangen.
%Ähnliches findet sich auch bei Thurner:\zitatblock{\enquote{Die Frage nach der Möglichkeit des Wissensgewinns führt bei Platon also zu einem ontologischen und anthropologischen Dualismus: Aus der Auffassung, dass die Welt des sinnenfälligen Werdens keine sichere Erkenntnis vermitteln kann, folgert er die Gegebenheit eines welttranszendenten Bereichs rein idealer Wesenswahrheiten, die nur der geistigen Erkenntnis des Denkens zugänglich sind und von der Seele immer schon apriorisch gewusst werden.}\footcite[][S. 99]{ThurnerDualismus}}
Es geht weiter mit der Idee des Guten, die so beschrieben wird, dass sie \enquote{[\dots] gewissermaßen nochmals die bereits transzendente Ideenwelt [transzendiert]}\footcite[vgl.][S. 50]{DisseMetaphysik}
Wenn es dann weitergeht mit der Beschreibung und der Interpretation vom Timaios wird es ziemlich schwierig, wie das jetzt zu verstehen ist, weil Disse die Seele in diesen vorher dargelegten Seinsbereich zu legen, so dass die Seele als Bindegleid zwischen Kosmos und Ideenwelt verstanden wird. Dieser Kosmos ist zwar noch als \enquote{Diesseits} der Ideenwelt genannt, aber als Wohnstätte der Götter und der unsterblichen Seelen, von der Sinnenwelt abgetrennt. Das heißt der Absatz: \zitatblock{\enquote{Die Seele gehört damit im Verhältnis zu den Ideen eindeutig noch in den Bereich dieser Welt. [\dots] ihr kommt die Mittelstellung zwischen Kosmos und Ideenwelt zu. Sie gehört zwar dem Kosmos an, ist aber nicht wie die Dinge der Sinnenwelt körperlich, sondern unkörperlich und bildet aufgrund ihrer Unkörperlichkeit zugleich den Kontaktpunkt zur Ideenwelt.}\footcite[vgl.][S. 58]{DisseMetaphysik}} macht in sich wenig Sinn. Auch wie dann weiter die Darstellung der Seelen, die auf dem Rücken des Himmelsgewölbes stehen und hinausschauen, um dabei die Ideenwelt zu schauen, ist sehr fragwürdig. 
%Dies liegt wohl an dem heutigen Verständnis davon, wenn man von Welten - also auch zwei Welten - spricht. Es wird dieses Verhältnis lediglich als räumlich vorgestellt, was automatisch zu der falschen Annahme von zwei Welten neben- oder übereinander führt.

\section{Interpretation B dieser Rede: Es gibt nur eine Welt}
Das Grundproblem, das bei der Trennung der beiden Welten auftritt, spielt sich auf dem Kampfplatz der Dialektik ab. Hierbei hällt Perls fest, dass \enquote{Trennung und Teilnahme die zwei Hauptbegriffe der Dialektik [sind und nur] getrennt werden [kann], was vorher zusammengesetzt war.}\footcite[vgl.][S. 349]{Perls}
Hierzu wird die Stelle Phaidon 78c herangezogen, in der es heißt:
\zitatblock{\enquote{Und nicht wahr, dem was man zusammengesetzt hat und was seiner Natur nach zusammengesetzt ist, kommt wohl zu auf dieselbe Weise aufgelöst zu werden wie es zusammengesetzt worden ist; wenn es aber etwas unzusammengesetztes gibt, diesem wenn sonst irgend einem kommt wohl zu, dass ihm dieses nicht begegnet?} (Phaidon 78c)}
Anders formuliert, ist darunter zu verstehen, wie sich die Verhältnisse zwischen dem Ganzen und dessen Teilen verstehen lassen, da gerade die Teile, welche ein Ganzes bilden, gedanklich so wieder eingeholt werden müssen, dass diese ebenfalls wieder als ein Ganzes für sich Bestehendes verstanden werden müssen. Somit heißt es von Maurizio Migliori:
\zitatblock{\enquote{Diese Verbindung von Ganzem und Teil lässt uns das verstehen, was das Herz einer jeden Dialektik ist: die Möglichkeit, die Identität der Gegensätze zu behaupten, ohne den Satz vom Widerspruch zu negieren.}\footcite[][S. 150]{Migliori}}
Gemeint sein ist, dass eine Sache sowohl als Teil als auch als Ganzes verstanden werden können muss, da sonst das Denken nicht aufrecht erhalten werden kann und konsistent ist.
Hinzu kommt jedoch noch, dass die Dinge auf verschiedenen Ebenen als Eines und Vieles beschrieben werden können, bzw. Einheit und Vielheit an einem Einzelding beschrieben werden kann.\footcite[vgl.][S. 112]{Migliori} Dies fasst Robert Wallisch darin zusammen, dass \enquote{[die Sinnesdinge unnenbar viele sind], doch auch die Ideen, welche die Vielen durch noetische Bündelung zu Einheiten bewältigen, sind selbst wiederum \textbf{viele} Einheiten und keinesfalls eine letzte denkbare Einheit [\dots].}\footcite[vgl.][S. 12]{Wallisch}
Dies steht dem Bisherigen so gegenüber, dass es unzulässig wäre die Sinnenwelt von der Ideenwelt so abzutrennen, dass man diese beiden Bereiche erst wieder zusammenfügen müsste, da es für eine vollständige Erarbeitung, bzw. Einholung, noch weiter gehen muss, bis zu einer letzten Einheit.
Somit muss das Verhältnis zwischen Ideen und Sinnesdingen nochmal von neuem betrachtet werden, da damit im Grunde genommen eine Umkehr der Herangehensweise gefordert ist, also dass zu Beginn nicht die Trennung angestrebt wird, sondern im Eingang gerade auf die Zusammensetzung, bzw. auf die Einheit aus Beiden, geblickt werden muss, um dann eine fromale Trennung vornehmen zu können, die aber dann mit dem Vorwissen der Einheit aus beiden gedacht werden muss, was die Problematik der ersten Interpretation überwinden würde.
Um dieser Interpretation nun folgen zu können, wird im Nachfolgenden unter anderem die Auffassung von Jens Halfwassen dargelegt, die sich mit der Darstellung Platons und der Auslegung desselben auf den Einheitsgedanken der gesamten Ideenlehre beschäftigt.
%Es wird an der Stelle eigentlich bereits deutlich, wenn in der Politeia in die drei Gleichnisse eingeleitet wird. Dies geschieht einzig und alleine über die Sinne. Also wie kann die Sinnenwelt als minderwertig angesehen werden, wenn sie doch für die Erarbeitung und Erklärung für die Ideenwelt als erstes herangezogen wird.
%Wenn diese beiden Welten so voneinander getrennt sein mögen, dann wäre es doch sicherlich nicht ratsam auf diese Weise in die Lehre über die Ideen auf diese Weise einzusteigen. Es sind also die Sinnesdinge und deren Wahrnehmung, welche eigentlich nicht für die wahre Erkenntnis geeignet ist, die den Beginn und auch den einfachsten Zugang (ob es auch der einzige Zugang ist, ist noch fraglich) kennzeichnen, was nur heißen kann, dass man diese \enquote{Welt}, wie man am Ende sehen wird, nicht einfach aufgeben kann und sich somit nur noch in der Ideenwelt aufzuhalten sucht.


\subsection{Metaphysik des Einen}
{\color{red}Überleitung schreiben:
Der Kerngedanke, der hier darzulegen sein wird, ist zum einen die Darstellung des Einheitsprinzips, welches nicht als ein noch vor das Seiende und Denkbare gestelltes Prinzip ist\footcite[vgl.][S.99]{halfwassen2015spuren} und zum anderen der Einheitsgedanke, in Anlehnung an die Idee des Guten, der noch das Seiende transzendiert und als absolute Einheit verstanden wird. Dies darf nicht als eine Gegensätzlichkeit verstanden werden, sondern entspricht nur einer unterschiedlichen Wirkungsmacht der jeweils entfalteten und eingefalteten Einheit. Was dies konkret zu bedeuten hat, wird im Folgenden dargelegt.}
%Um diesen Weg der Interpretation nachvollziehbar zu machen, also noch bevor die einzelnen Originalstellen wiederholt betrachtet werden, gilt es dem Prinzip der Einheit Vielheit Relation zuzuwenden, da Halfwassen dieses Verhältnis als Ausgangspunkt nimmt, um über die platonische Metaphysik auszuführen. Darum wird es zu zeigen sein, dass die Einheit nicht als ein noch vor das Seiende gestellte Prinzip ist.\footcite[vgl.][S.99]{halfwassen2015spuren}
\subsubsection{Transzendenzgedanke von Halfwassen}
Eingangs ist für die Erarbeitung allerdings anzuführen, wie Halfwassen den Begriff der Transzendenz fasst, da dieser Begriff auch in der Tradition unterschiedlichste Bedeutungen erhalten hat. Den Begriff der Transzendenz fasst er bei Platon in eine schwache und eine starke Variante. Die schwache Variante wäre das, was im vorigen Kapitel dargelegt worden ist, welche er als \enquote{graduelle Transzendenz} bezeichnet und das Übergangsverhältnis der jeweils ursprünglichen Seinsstufen zu den von ihr abgeleiteten und ontologisch abhängigen Stufen darlegt.\footcite[vgl.][S. 29]{halfwassen2015spuren} Wichtig dabei ist, \zitatblock{\enquote{dass das Denken in der Lage ist das Transzendente und das von ihm Transzendierte zu einer Einheit zusammenzufassen, indem es das größere Ganze in den Blick nimmt, das diese \emph{und} jene Seite, Begründetes und gründenden Grund gleichermaßen umfasst.}\footcite[][S. 29]{halfwassen2015spuren}} Damit ist gemeint, dass dieses Denken, so wie es beschrieben wird, die Sinnesdinge und die Ideen in einer Einheit zusammenzudenkt. D.h. obwohl das Begründende das Begründete auf eine gewisse Weise transzendiert, das Denken dennoch in der Lage ist diese beiden Bereiche in einem Denken zu können.
%was bereits an dieser Stelle schon mehr ist, als es die vorherige Dartellung geliefert hat.
Dem gegenüber steht die starke Transzendenz, das eben \enquote{[\dots] nicht mehr mit dem, was sie transzendiert, in die gemeinsame Sphäre eines Überstiegenes und Übersteigendes gleichermaßen umfassenden Ganzen zusammengefasst werden kann.}\footcite[vgl.][S. 29]{halfwassen2015spuren} Hiermit wird später noch die Idee des Guten gemeint sein, da auch auf die Stelle Pol. 509b referiert wird, in der das Wesen der IdG weit höher ist und als Quelle von Erkenntnis und Wahrheit an Herrlichkeit über den Ideen und Sinnesdinge steht.
Diese Darstllung der starken Transzendenz, die scheinbar das Denken verlässt, da das Begründende nicht mehr in die gleiche Sphäre wie das Begründete gefasst werden kann, ist gerade das, was an der ersten Interpretation kritisiert worden ist. 
Mit dieser Setzung eines Begründenden, welches nicht mehr in dieselbe Sphäre wie das Begründete gesetzt wird, wird von Halfwassen als Zurückführung auf den Versuch eines einzigen absoluten Urgrundes beschrieben und auch als Ziel der platonischen Dialektik aufgefasst:
\zitatblock{\enquote{Platonische Dialektik ist der Versuch, die Vielfalt der grundlegenden Voraussetzungen unserer denkenden Bezugnahme auf Wirklichkeit auf einen einzigen absoluten Urgrund zurückzuführen: sie ist also die Suche nach dem Absoluten als dem unbedingten Ursprung und Urgrund des Ganzen der Wirklichkeit.}\footcite[][S. 95]{halfwassen2015spuren}}
Mit dieser Ausdrucksweise der denkenden Bezugnahme auf die Wirklichkeit ist zudem eine andere Bedeutung des Ziels der Interpretation aufgestellt. Da hier nicht explizit von der Erkenntnis der Wirklichkeit oder dem Sein der Wirklichkeit gesprochen wird, ist das Ziel bereits ein anderes als aus dem ersten Teil. Mit dieser Herangehenseise ist gerade beides darunter zu fassen, da das Sein als Sein, genauso aber auch die Erkenntnis der Wirklichkeit, in \emph{zu denkende} Terme gebracht werden muss. Damit ist also die Nähe der beiden Bereiche des ontologischen und gnoseologsichen Anspruchs deutlich gemacht.


\subsection{Die reine Einheit}
%\subsubsection*{Auf den Spuren des Einen Jens Halfwassen  75/BF 1495 H169}
%Es sei gesagt, dass hier oft von dem Prinzip des Einen oder dem Absoluten Einen gesprochen wird, diese Erarbeitung wird allerdings nachgestellt.
%Seite 91f Platons Metaphysik des Einen.\\
Halfwassens Auffassung zur platonischen Prinzipientheorie/Ideenlehre kommt daher, dass er diese über die Dialektik fasst, welche als \enquote{in ganz allgemeinem Sinn hypothesis-Forschung - also philosophische Reflexion von allgemeinen Grundlagen}\footcite[][S. 94]{halfwassen2015spuren} beschrieben wird. Damit ist gemeint, dass platonische Dialektik auf eine - wie schon angedeutet - Letztbegründung abzielt, welche schlussendlich \enquote{voraussetzungs-los oder un-bedingt - anhypothetos - ist.}\footcite[][S. 95]{halfwassen2015spuren}
Nur mit dieser Letztbegründung, welche dargelegt werden muss, kann von einer absoluten Vollendung gesprochen werden. Ließe sich diese Letztbegründung nicht liefern, oder wäre diese keine wirkliche Letztbegründung im eigentlichen Sinne, würde dieses \enquote{System} ewig weiterlaufen und würde das eigentlichen Ziel nicht vollenden. Daher heißt es: 
\zitatblock{\enquote{Das Dialektikprogramm der Politeia beschreibt deutlich den Aufstieg zu einem Unbedingten und Absoluten, das Urgrund von allem ist. Dies scheint einen irreduziblen Prinzipiendualisimus auszuschließen: denn wenn dem Einen die Vielheit als gleichursprüngliches und unabhängiges Prinzip gegenüberstünde, dann wäre das Eine nicht mehr das Prinzip von allem, und es wäre auch nicht mehr 
%{ἀνυποθετος ἀρχή}
(anupothetos arche), da seine Wirksamkeit als Ursprung dann durch sein Zusammenwirken mit dem Vielheitsprinzip bedingt wäre.}\footcite[vgl.][S. 70f.]{HalfwassenMonismusDualismus}}
%Politeia 511b, 533c
Hierbei wurden die Prinzipien der Einheit und der Vielheit genannt, die im Folgenden noch weiter dargelegt werden. An diesem Punkt erklärt sich aber, dass es nur \emph{ein} Letztbegründendes Prinzip geben kann und nicht noch ein weiteres, das gleichwertig oder auf der selben Ebene sich befinden kann.
Damit ist zwar mit Halfwassen nicht vollständig auszuschließen, dass es lediglich den Monismus in dieser Weise in der Deutung der Prinzipienlehre gibt, sondern sich gerade in Anlehnung an Krämer, dass \enquote{[d]ie monistische Lösung einem Rückgriff hinter den Gegensatz der beiden Prinzipien [entspricht], ohne ihn aufzuheben.}\footcite[vgl.][S. 333]{Krämer1964Geistmetaphysik} 
Es ist hier bereits angedeutet, dass es für die vollständige Darstellung der Ideenlehre unumgänglich ist beide dieser Prinzipien anzuführen und einzubringen, ohne dass diese sich aufheben. Jedoch ist die starke Tendenz hin zu einer einheitlichen Lehre betont, welche sich nicht auf zwei gleichwertige Prinzipien stützt, sondern eben aus einem Prinzip (der Einheit) alle weiteren Prnizipien (der Vielheit) ableiten soll. Somit heißt es von Halfwassen:
%Daraufhin verweis auf Parmenides 8 Hypothesen, in denen Einheit und Vielheit zueinander in jedem Verhältnis untersucht werden. Siehe Appendix
\enquote{Wenn unsere Deutung richtig ist, dann verbindet Platons Prinzipienlehre einen Monismus in der Reduktion zum Absoluten mit einem Dualismus in der Deduktion des Seienden.}\footcite[][S. 79]{HalfwassenMonismusDualismus} 
Hierfür liefert Halfwassen bereits die erste Anlaufstelle wo es hießt, dass die bekanntlich zwei letzten Prinzipien Platons innerakademischer Prinzipienlehre, auf deren Zusammenwirken alles Seiende zurückgeführt wird das absolute Eine (auto to en) und die unbestimmte Zweiheit (aoristos dyas) sind.\footcite[vgl.][S. 67]{HalfwassenMonismusDualismus}
%Diese Formulierung greift die eingefaltete und entfaltete Einheit auf, die damit verbunden zu sein scheint, da hier von der Reduktion zum Absoluten und der Deduktion des Seienden die Rede ist.
%Anders formuliert heißt es von Wallisch: \zitatblock{\enquote{Wenn die Ideen zu Beginn des sechsten Buches gedachte konstante Einheiten waren, welche die unstete Vielheit bewältigen - gleichsam als immergültige Bündelungen des Vielen zu noetischen Einheiten - so muss das agathon als eine noch höhere, eine letzte Instanz, welche ihrerseits die Ideen bedingt, d.h. den Ideen in analoger Weise übergeordnet ist wie die Ideen dem konkreten Vielen, als letzte denkbare Einheit angesprochen werden}\footcite[][S. 10]{Wallisch}}

Dies alles wird im Folgenden noch innerlich erarbeitet und dargelegt werden müssen.
Für die Erarbeitung werden von Halfwassen drei Prinzipien aufgestellt, welche gelten müssen.
Diese drei Prinzipien stammen von Plotin. Hierbei könnte man den Einwand erheben, ob es überhaupt sinnvoll ist, mit der von Plotin gelieferten Einheitsmetaphysik an Platon heranzutreten, da damit eine mögliche Einheitsmetaphysik von Plotin doch auf Platon übertragen werden würde. Wie es Halfwassen festhält, ist es jedoch so, dass Plotin sich lediglich als Interpret Platons versteht (Enneade V 1,8) und Platon als Begründer der henologischen Tradition zu verstehen ist,\footcite[vgl.][S. 92]{halfwassen2015spuren} also die Einheitsmetaphysik nicht bei Plotin zu verorten ist, sondern bereits bei Platon.
Somit lauten diese drei Prinzipien wie folgt:
\zitatblock{\enquote{1. Jedes Seiende existiert als dasjenige, was es jeweils ist, genau aus dem Grunde, weil es Eines ist.\\ 2. Die Gesamtheit aller einzelnen Seienden bildet die Einheit eines Ganzen. Einheit charakterisiert also nicht nur jedes einzelne Seiende, sondern ebenso die Totalität des Seins.\\ 3. Das Prinzip der Einheit des Ganzen und zugleich der Einheit jedes einzelnen Seienden ist \emph{das Eine selbst}. Als der einheit-verleihende Ursprung ist das Eine das Absolute, durch das alles Seiende Eines und Kraft seiner Einheit auch seiend ist.}\footcite[][S. 91]{halfwassen2015spuren}}
Einfacher gesprochen ist damit die Existenz auf drei verschiedenen Stufen gemeint, welche erst mit der dritten Stufe zur Vollendung gelangt. Begonnen wird hier mit dem jeweiligen Seienden, das ist, was eben nur durch seinen Einheitscharakter ist. Es gelangt also ins Sein, dadurch dass es Einheit aufweist, oder eben weil es einheitlich ist. Die zweite Stufe fasst die Gesamtheit alles Seienden zusammen, welches wiederum ein Ganzes bildet. Also die Summe aller einzelnen Seienden, die zusammengenommen Einheit aufweisen. Dies wird dann zusammengenommen auf der dritten Stufe, dass erst durch das Eine als Absolutes alles Seiende auf den darunterliegenden Stufen Einheit aufweist und damit erst seiend ist. 
%Dabei wird das Wesen des Guten für Platon in der \emph{reinen Einheit} verortet.
Die Wesensbestimmung des Absoluten als reine Einheit ist grundlegend für Platons Prinzipientheorie, die darum die Charakteristik einer Metaphysik des Einen hat. Erst von ihr aus lässt sich auch verstehen, wie das Absolute Sein und Wassein, Erkennen und Erkennbarkeit zugleich begründet.\footcite[vgl.][S. 96]{halfwassen2015spuren}
%(Fußnote zu Krämer S. 474ff, 535-551) 
Bemerkt sei hier der Blick auf die Unterscheidung der gnoseologischen und ontologischen Auslegung mit Blick auf die Stellen der Politeia. Primär sind hier das Sonnen- und Liniengleichnis zu nennen. 
Die Einheit versteht sich also als die grundlegende Bedingung für das Sein und die Denkbarkeit alles Seienden.\footcite[vgl.][S. 97]{halfwassen2015spuren} 
Dieses Absolute hat drei zu klärende Thesen 
%der platonischen Metaphysik
, welche (1) die Bestimmung als reine Einheit, (2) die absolute Transzendenz des Einen selbst und (3) die Ansetzung eines eigenen Prinzips für die Vielheit sind.\footcite[vgl.][S. 96]{halfwassen2015spuren}
%Die Ansiedelung des Prinzips der Vielheit wird noch gesondert betrachtet werden. 

%oder eben dass dadurch \enquote{[\dots] die Ideen als reale Gegebenheiten existieren, als Dinge des Denkens.}\footcite[vgl.][S. 11]{Wallisch} 

%\zitatblock{\enquote{Auf der Grundlage der skizzierten Henologie basiert Plotins Metaphysik des Geistes, [\dots] 
%die zwei große Themenkomplexe umfaßt,
%nämlich erstens die Konstitution des Geistes in seinem Hervorgang aus dem überseienden Einen und zweitens die immanente Struktur der Selbstbeziehung des absoluten Denkens, dessen Inhalte die reinen Wesenheiten des Seienden, also die Ideen sind, in deren untrennbarer Einheit sich der Geist intellektuell selbst anschaut.}\footcite[][S. 54]{HalfwassenGeistmetaphysik}}
Es bedarf also zuerst der Darstellung des Absoluten zu Beginn, dass es reine Einheit ist und nicht noch ein weiteres Prinzip oder ähnliches in diesem steckt. Darauffolgend bedarf dieses Absolute auch tatsächlich als transzendent darzustellen, wo es sich um eine starke Transzendenz handelt und dann zum Dritten, damit die Einheit aus seiner Transzendenz heraus seine Wirkmächtigkeit aufbringen kann, eines weiteren aber untergeordneten Prinzips, mit dessen Hilfe das Absolute sich entfalten kann.
%ein untergeordnetes Prinzip der Vielheit bedarf, um aus seiner Transzendenz heraus etwas zu bestimmen.
%Politeia 478B12f und Parm 144C4-5
%\zitatblock{\enquote{Auch das Gegenteil des Einen, das Viele, denken wir immer schon und notwendig als Einheit, nämlich als geeinte Vielheit und das bedeutet als einheitliches Ganzes aus vielen elementaren Einheiten, so dass der Gedanke des Vielen in doppelter Weise Einheit voraussetzt.}\footcite[][S. 97]{halfwassen2015spuren}} Parm 157C-158B
%Bedeutung von Einheit, in der Vielheit enthalten ist, wie Ganzheit, Einheitlichkeit als Einheit in der Vielheit oder Identität\footcite[vgl.][S. 97]{halfwassen2015spuren} Soph 254D Parm 139D 4-5
Mithin fügt sich dabei auch noch die Rolle des Werdens, des Nichtseins und des Nichts, was unter den Gedanken der Einen gefasst wird, an.\footcite[vgl.][S. 97]{halfwassen2015spuren} Diese Bestimmung steht dem, was in dem ersten Teil der Arbeit dargelegt worden ist, fundamental gegenüber, da dort das Werdende konsequent von dem Ewigen getrennt worden ist. Jetzt heißt es hier aber, dass das Werden und sogar Nichtsein und Nichts unter den Gedanken des Einen fallen. Daher bedarf es einer erneuten Betrachtung dieses Verhältnisses. Dies geschieht über die Auslegung des Vielen, worauf später zurückgekommen wird.



Alles Seiende und Denkbare ist also nur darum seiend und denkbar, weil es einheitlich ist, und zwar in der Weise, dass sein Charakter als Einheit die Grundlage seiner Denkbarkeit bildet. Daraus folgt zugleich, dass Einheit das Kriterium der Unterscheidung von Sein und Nichtsein ist und der Maßstab, an dem Seiendes von höheren und geringeren Seinsgrad messbar wird.\footcite[vgl.][S. 99]{halfwassen2015spuren}
%Auf diesen Grad der Seiendheit in Anbetracht der in dieser vorzufindenden Einheit, wird sich später zugewandt.
%Hier wird von verschiedenen Seinstufen gesprochen, die auf dem Kriterium der Einheit besteht. Welche Stufen sind hiermit gemeint?
%Hat diese Stufung der Seiendheit nur dafür Bedeutung, wenn man sich diese Stufung als einen Weg hinauf zum Absoluten Einen vorstellt und nur durch die Beschreibung, dass etwas noch einheitlicher wird je höher man gelangt zum höchsten Einen gelangen möchte. So dass rückblickend alles unter einer Einheit steht, um dann ein \enquote{niederes} Sein darzustellen, aber nicht in rein ontologischer Absicht.
\zitatblock{\enquote{Wenn ferner die Einheit von etwas der Grund seines Seins ist, dann ist jedes etwas auch in dem Grade seiend, indem es Eines ist. Je einheitlicher etwas ist, desto seiender (mallon on, Politeia 515 D 3) ist es dann auch. Erst sein henologischer Ansatz erlaubt Platon die Graduierung von Sein, die der Eleatismus noch nicht kennt. Einheit als Grund des Seins generiert den ontologischen Komparativ und damit die Grundlage der Ideenlehre, der zufolge die einheitliche Wesenheit von etwas seiender ist als ihre vielen individuellen Instanziierungen, und zwar genau darum, weil die eine Schönheit selbst oder die eine Gerechtigkeit selbst den vielen Fällen erscheinender Schönheit oder Gerechtigkeit als die diese Vielheit begründende Einheit zugrunde liegt. (Politeia 476 A, 479 A — 480 A, 507 B)}\footcite[][S. 99f.]{halfwassen2015spuren}}
Ob hier tatsächlich durch eine höherstufige Einheit ein höherer Grad an Sein folgt ist fraglich. Zum einen wird die Existenz der Schönheit selbst oder des Guten selbst impliziert, zum anderen ist fraglich wie man sich diesen höheren Grad an Sein vorzustellen hat, da sich die Schatten als Schatte nicht mehr oder weniger Sein zukommt, als es Bäume als Bäume zukommen dürfte. Denn dabei ist der Einheitsgedanke \emph{als etwas selbst} eindeutig gewahrt. Zwar mag hier eine deutlichere Zustandsveränderung im Werden feststellen, also dass sich Schatten und Spiegelungen sehr viel schneller verändern als es Dinge tun würden, dies ist aber keine zureichende Bestimmung für das Sein. 
Über das Eine selbst kann man nichts mehr aussagen, wie über dessen Seiendheit, da es selber nicht mehr bestimmt ist und eigentlich nur das Prinzip für die Bestimmung für alles weitere ist. Würde man über das Eine selbst so zu sprechen versuchen, nimmt man es wiederum aus seiner Transzendenz heraus, da das Denken diese Prinzipien braucht, um etwas zu denken, wie es bisher erklärt worden ist.
% Ausführung in Parmenides 137C-142A oder Aufstieg zum Einen 282ff und Kapitel XI:
\zitatblock{\enquote{Wird das Eine nur in sich selbst betrachtet, dann weist es als reine Einheit jedwede Bestimmung strikt von sich ab; es steht als solches jenseits aller Bestimmungen, weil jede denkbare Bestimmung es in die Vielheit hineinziehen würde. Man kann darum nichts von ihm aussagen, noch nicht einmal, daß es ist oder daß es Eines ist, weil es damit bereits eine Zweiheit wäre (141 E); die duale Struktur der Prädikation verfehlt prinzipiell die reine Einfachheit des Absoluten. Platon spricht dem absolut Einen darum systematisch alle Fundamentalbestimmungen ab, auch Sein, Einssein, Erkennbarkeit und Sagbarkeit.}\footcite[][S. 101]{halfwassen2015spuren}}
Diese Formulierung geht genau auf Pol. 509b zurück, dass die IdG noch an Herrlichkeit über dem Seienden steht. Damit wird auch deutlich, wie die Beschreibung gemeint sein soll, dass man nicht in die Sonne selbst schauen kann. Da die Sonne der IdG entspricht, ist es nicht möglich die IdG selbst zu erblicken. Zwar lässt sich etwas darüber etwas aussagen, jedoch lässt sich die 
%Damit ist allerdings der Schritt weg von dem Thema gemacht worden und man befindet sich hier in der reinen Metaphysik und der Transzendenz des Einen. 
Damit lässt sich aber sagen, dass das Eine die Seins- und Erkenntnistranszendenz aufweist. \enquote{Das Eine selbst ist ebensosehr jenseits des Geistes und der Erkennnis, wie es jenseits des Seins ist; es übersteigt den Zusammenhang von Denken und Sein, indem es das Prinzip dieses Zusammenhangs ist; und es begründet ihn gerade Kraft seiner Transzendenz.}\footcite[][S. 102]{halfwassen2015spuren}
\subsection{Das Prinzip der Vielheit}
Nun bleibt noch zu klären, inwiefern das Eine aus seiner Wirkungsmacht des Absoluten und der Seins- sowie Erkenntnistranszendenz Wirklichkeit bestimmt. Um diese Forderung zu erfüllen bedarf eines schon angedeuteten Prinzips der Vielheit, welches die Einheit immernoch erfüllend, aber diese entfaltend, die Seiendheit schlussendlich ausfüllen kann.\\
Für diese Erarbeitung müsste eigentlich der Parmenides Dialog in seiner Gänze behandelt werden. Dies ist hier leider nicht möglich. Daher werden im Folgenden Ausschnitte aus dem Parmenides angeführt, welche sich in den Ausführungen von Halfwassen finden lassen und dort genannt werden. Für den weiteren Anschluss ist auf die Stellen im Anhang hingewiesen.\\ 
Gedanklich befindet sich die Argumentation also an der Stelle, dass die Einheit als einziges Anfangsprinzip existiert, von der aus nun die Vielheit konzipiert werden muss, um die nicht denkbare absolute Einheit in die Wirklichkeit zu bringen. Um das Seiende in seiner Vielheit ableiten zu können, braucht es ein Prinzip nach dem Einen, wenn es spezifisch eben die Vielheit generiert, was die reine Einheit absolut von sich ausschließt. Dieses Viele ist noch kein bestimmtes, sondern ist \emph{vorseiend}, aber nicht überseiend wie das Eine.\footcite[vgl.][S. 103]{halfwassen2015spuren} Dies ist auch die \enquote{unbestimmte Zweiheit}. Unbestimmte Zweiheit deshalb, da diese Prinzipien noch mit keinerlei Erkenntnis- oder Seinsinhalt gefüllt sind, also eine Art von Gegenstand des Seins oder des Denkens in sich haben, folglich ein Objekt aufweisen.

Allgemein gesprochen entspricht der Vielheit als gedachten Vielheit das Konzept der geeinten Vielheit, welche die Einheit in doppelter Weise voraussetzt.\footcite[vgl.][S. 97]{halfwassen2015spuren}
In doppelter Weise deswegen, weil die Vielheit als eine geeinte Vielheit bestehen muss, in der Weise, dass diese sonst nicht als \emph{etwas} erkannt werden kann und weil die Vielheit aus vielen Einheiten besteht, welche wiederum Einheit bedingen, da die Elemente der Vielheit ebenfalls nicht \emph{etwas} wären. Wenn man z.B. von \emph{einem} Baum spricht, ist dies \emph{ein} Seiendes, welches als \emph{etwas} gedacht wird. Wenn man jetzt allerdings viele Bäume als \emph{etwas} zu denken vermag, kann man dies nur tun, indem man \emph{etwas} als \emph{eines} setzt, das als \emph{etwas} gedacht werden kann, da diese Vielheit nur in geeinter Vielheit gedacht werden kann. Diese geeinte Vielheit umfasst unter sich eine Mehrzahl von Seienden. Im Denken wird somit Bezug genommen auf die geeinte Vielheit, die meist durch einen weiteren Begriff geliefert wird. Damit ist noch keine Ähnlichkeitsrelation o.Ä. ausgedrückt, da es rein um das Zusammenspiel eines Seiendem und vieler Seienden in einer geeinten Vielheit geht. 
%einen Begriff findet, welcher wiederum \emph{eines}, also als \emph{etwas} gedacht werden kann. Dabei kann nur der Begriff des Waldes fallen. Wenn man hier jedoch einwenden würde, dass ein Wald doch nicht schon mit zwei oder drei Bäumen beginnt, wie kann man dann hier von einem \enquote{Überbegriff} sprechen. Nun, alleine weil man schon die Zahl der Bäume in dieser Darstellung wählt, sind die vielen Bäume unter \emph{eines} gefasst worden. Diese Herangehensweise kann man nun weiter \enquote{nach oben} oder \enquote{nach unten} weiterführen.
Würde man dieser Auffassung widersprechen, so heißt es von Halfwassen:
\zitatblock{\enquote{Denn wer meint, die Wirklichkeit könne auch aus einer einheitlosen Vielheit unverbundener Einzeldinge bestehen, der nimmt eben damit Denkbestimmungen wie Wirklichkeit, Vielheit und Einzelnes als realitätshaltig in Anspruch und setzt somit genau das voraus, was er bestreiten will, nämlich die Einheit von Denken und Sein.}\footcite[][S. 98]{halfwassen2015spuren}}
Hier liegt jetzt der übergreifende Punkt und die Überleitung zur \enquote{Ideenlehre}. Aus diesen beiden Prinzipien gehen alle ontologischen Fundamentalbestimmungen hervor, welche die Struktur des Ideenbereichs und damit die ganze Welt des Seienden konstituieren.\footcite[vgl.][S. 104]{halfwassen2015spuren} Das heißt, dass diese Prinzipien zwar noch vor die Ideenlehre, gestellt sind, die Ideenelehre aber fundamental durchziehen. Somit sind die Prinzipien die Bestimmungen für den Bereich der Ideen und den Bereich der Sinnesdinge verantworlich und durchziehen diese nicht nur, sondern fassen darin auch das Verhältnis der beiden Bereiche zueinander.

Damit diese beiden Prinzipien jetzt aber wieder zusammengebracht werden können, da sie bis hier noch als zwei seperate Prinzipien bestehen, unter anderem weil sie noch kein Objekt aufweisen, bedarf es eines einigenden Grundes, welcher selber nicht mehr bedingt sein darf:
\zitatblock{\enquote{Darüber hinaus ist auch das Zusammenwirken der beiden Prinzipien eine Form von Einheit und bedarf eines einigenden Grundes (vgl. Philebos 27 B mit 30 AB; Aristoteles, Metaphysik 1075 b 17—20), der nur das Eine selbst sein kann. Ferner konstituieren die Prinzipien das Sein als die Einheit eines Ganzen (Ev öXov, Parmenides 157 E 4, 158 A 7), als das seiende Eine, in dem alle entfaltete Vielheit einbegriffen bleibt.}\footcite[][S. 106]{halfwassen2015spuren}}
Es bleibt also noch ein übergreifenderes Prinzip, das die Einheit aus der Einheit \emph{und} der Vielheit bildet. Hier heißt es, dass es wieder das Eine selbst sein muss. Es wird damit also geschafft, dass nicht noch ein weiteres Drittes gesetzt wird, das in dieser Konstellation die beiden Dinge miteinander verbindet, sondern es wird aus dem bereits absoluten Einen dieses Prinzip abgeleitet. 
%Die aufschlussreichste Passus des Dialogwerks ist die Gleichnissequenz im 6. und 7. Buch der Politeia mit anschließenden Ausführungen über das Verhältnis von mathemtischer und Propädeutik und Dialektik.\footcite[vgl.][S. 135]{halfwassen2015spuren} Mit Verweis auf Krämer Arete bei Platon und Aristoteles 135-145, 473-480, 533ff
%\enquote{[\dots] warum das Gute als letztes Seins-, Erkenntnis- und Wertprinzip selbst noch jenseits des Seins stehen muss.}\footcite[vgl.][S. 136]{halfwassen2015spuren} Mit Verweis auf Politeia 509b, Parmenides 141e und Aufstieg zum Einen 19ff, 188ff, 221ff, 257ff, 277ff, 302ff und 392ff. 
Anders formuliert würde es also heißen:
\enquote{Denn schon die Aussage, dass das Eine \emph{ist}, enthält ja eine Zweiheit: nämlich die Zweiheit von Einheit und Sein, aus der sich alle anderen Grundbestimmungen des Seienden ableiten lassen, wie die 2. Hypothesis des \emph{Parmenides} lehrt.}\footcite[][S. 136f.]{halfwassen2015spuren}
%Hier gilt es den Blick zurück in die Originalstellen zu werfen, welche auch von Halfwassen angeführt werden.

Für diese Formulierung muss aber nun ein Ausweg gefunden werden, da es uns nur möglich ist zu sagen, dass das Eine \emph{ist}, wodurch nun aber die Einheit aus ihrer Absolutheit herausgeholt wird. Daher muss sich der Entfaltung des Einen in die Vielheit zugewandt werden, um näher verstehen zu können, wie und auf welche verschiedenen Stufen sich die Einheit in die Vielheit entfaltet, um notwendig erkannt zu werden.
%\subsubsection*{Der Aufstieg zum Einen 75/BF 2371 H169 Halfwassen}
%Es ist zwar mit Plotin verbunden, bzw. daher kommt die Frage des Buches, aber ab Seite 220 geht es mit Platons Transzendenz los, aber auch die Einleitung ist sehr gut.
%Stufen der Einheit bei Plotin S. 41
%Seinstranszendenz, Geisttranszendenz, Erkenntnistranszendenz S. 150ff
%Der Text von Plotin und schaut auf die Grundlegung der Einheitsmetaphysik bei Platon im Parmenides. 

\subsubsection{Die Entfaltung der Einheit (in) Vielheit}
\enquote{Er unterscheidet voneinander das Erste Eine, das schlechthin und absolut Eine, das Zweite, welches er \enquote{Eines Vieles} nennt, und das Dritte, \enquote{Eines und Vieles}; so stimmt er ebenfalls überein mit der Lehre von den drei Wesenheiten.}\footcite[][S. 187f.]{halfwassenaufstieg2006}\footnote{Ursprünglich Plotin V 1, 8, 23-27. Allerdings wird hier der Bezug zu den ersten drei Hypothesen des Parmenides Dialoges hergestellt.}
Es wird festgehalten, dass \enquote{[\dots] das absolute Eine als Urgrund des Seins und des Denkens sowie ihres Zusammenhangs notwendig selbst jenseits aller Bestimmungen des Seins und des Denkens [ist]. (Parm. 137c-142a)}\footcite[vgl.][S. 188f.]{halfwassenaufstieg2006}
\enquote{Erst die Seinstranszendenz des einheitsstiftenden Absoluten macht das [\dots] Ineins von Einheit und Vielheit denkbar; das Sein hebt als die Totalität aller Bestimmungen in sich jede Vielheit in die Einheit auf, ohne sie in Unterschiedslosigkeit untergehen zu lassen, erhält sie also zugleich; es ist die Einheit der Bestimmungen, auch der entgegengesetzten, die in ihm koinzidieren.(Parm. 142b-155e)}\footcite[vgl.][S. 189]{halfwassenaufstieg2006}
Und die dritte Hypothese:
\zitatblock{\enquote{Das so als in sich dialektisch bewegte Einheit und d.h. als Geist gedachte Sein aber bleibt nicht [\dots] in sich selbst verschlossen, sondern es setzt eine weitere Entfaltungsstufe der Einheit aus sich heraus: die Seele faltet die im Geist in Einheit eingefaltete Vielheit und Andersheit diskursiv und d.h. zugleich zeitlich sukzessiv aus in eine Vielheit ontisch distinkter und je in sich schon vielhältiger Einheiten, deren umgreifende, unterscheidende und zugleich verbindende Einheit sie selbst ist.(Parm. 155e5)}\footcite[][S. 189]{halfwassenaufstieg2006}}
%Dieses Prinzip ist \enquote{als Strukturprinzip der Ideenwelt entfaltet und zugleich als Einfaltung aller Ideen unentfaltet; es wird nur im die Verstandesgegensätze übersteigenden noetischen Denken erkannt.}\footcite[vgl.][S. 190]{halfwassenaufstieg2006} Hier fehlt der Zusammenhang zu den anderen Prinzipien. Dieser Auszug ist nur aus dem zweiten Prinzip
Zusammengefasst wird das Eine in seiner dreifachen Weise als \enquote{absolute, noetisch-komplikative und dianoetisch-explikative Einheit}\footcite[][S. 190]{halfwassenaufstieg2006} aufgefasst.
Gemeint damit sind - von Plotin entnommen - das absolut Eine, das in reiner Einfachheit jenseits aller Vielheit steht. Die noetsich-komplikative Einheit, welches die Totalität aller Bestimmtheit geeinte umfassende Eine des Seins meint, welches wiederum die Vielheit der reinen Bestimmung in sich einfaltet und zugleich ihre Entfaltung in Gegensatzpaaren als Strukturgesetz bestimmt. Es ist als Strukturprinzip der Ideenwelt entfaltet und als Einfaltung aller Ideen unentfaltet. Diese Struktur und dieses Verhältnis wird nur im noetischen Denken erkannt. Zum dritten ist die dianoetisch-explikative Einheit als das sich in die Mannigfaltigkeit distinkter Einzelbestimmungen ausfaltende und diese vorgängig in sich umgreifende Eine der Seele gemeint.\footcite[vgl.][S. 190]{halfwassenaufstieg2006} Diese dianoetisch-explikative Einheit bestimmt also alle individuellen Einzeldinge und organisiert diese untereinander wieder zu einem Gesamten, welche im dianoetischen Denken begriffen werden können. Dies wird meist als rechnendes Denken bestimmt, was sich hier auch zeigt, da es nicht um eine solche reflektierende Aufgabe handelt, sondern nur um eine Zusammen- und Auseinadnerstellung von Einzeldingen.
Hierin liegt das Verständnis dessen, wie man die Transzendenzstufen des Liniengleichnis und Höhlengleichnis zu deuten hat. Denn dieses Urprinzip lässt sich auf alle darunterliegenden Stufen ableiten. Das heißt, welche \enquote{Denkaufgabe} auf der jeweiligen Stufe vorfinden lässt, aber auch mit Blick darauf, dass die vorherige Stufe überwunden wird.
%abgesehen davon, dass man von der förmlichen Transzendenz absehen müsste. 
Von hier aus kann sich nocheinmal der Blick auf das Sonnengleichnis geworfen werden.
\zitatblock{\enquote{Das Sonnengleichnis beschreibt [die Seinstranszendenz des Absoluten] auf der Grundlage der eleatischen Unterscheidung von Sein und Erscheinungswelt als doppelte Transzendenz und legt damit den zweifachen Überstieg über die Erscheinung zum Seienden und über das Seiende im ganzen zum Absoluten als das Bewegungsgesetz der Platonischen und neuplatonischen Metaphysik fest.}\footcite[vgl.][S. 222]{halfwassenaufstieg2006}}
Das Überschreiten von der gegebenen welthaften Wirklichkeit zum wahrhaft oder eigentlichen Seienden wird als erste Transzendenz beschrieben. \footcite[vgl.][S. 222]{halfwassenaufstieg2006}.
Der intelligible Bereich der Ideen, der durch die Dialektik erforscht wird, ist somit ein untereinander einiges, in sich selbst vielfältig gegliedertes Ganzes (Struktur der Einheit \emph{in} Vielheit). Dieser Bereich wird allerdings noch durch eine Letztbestimmung, die noch über diesen Bereich hinausgeht in einem einzigen Prinzip, in dem Prinzip aller Einheit, der Idee des Guten, überschritten. Dies ist damit auch die zweite Transzendenz.\footcite[vgl.][S. 223f.]{halfwassenaufstieg2006}
Hierin liegt auch die eigangs gegebene Unterscheidung der starken und schwachen Transzendenz.
\enquote{Platon fasst damit die Transzendenz des Absoluten als \emph{absolute ontologische Transzendenz},}\footcite[][S. 224]{halfwassenaufstieg2006} da, wie schon gesehen, mit dieser Überschreitung das Denken die Einheit in seiner absolutheit halten muss.
Hier kommt der Sprung dahin, dass sich die Interpretation hin zu einer Welt festigt. 
\zitatblock{\enquote{So \emph{wissen} wir das Absolute, das aller Erkenntnis den Grund gibt, gerade weil Es selbst jenseits aller Erkenntnis ist, nur im \emph{Nichtwissen} - freilich in einem Nichtwissen, das sich selbst \emph{als} Nichtwissen weiß und das sich darum nur durch das Wissen des Wissbaren hindurch erreicht, indem es dieses transzendiert. Alles Denken und Sprechen über das absolut Transzendente muss sich darum ständig selbst widerrufen und ins Unsagbare aufheben 
%dies ist der Sinn der \enquote{negativen Theologie}, deren Begründer Platon ist.
.}\footcite[][S. 225]{halfwassenaufstieg2006}}
Diese Interpretation holt die Rolle der Idee des Guten so wieder ein, was die erste Interpretation nicht geschafft hat.
%Diese Beschreibung der Überschreitung der Ideen geht so weit, dass selbst die absolute Transzendenz der Idee des Guten noch mit eingeholt wird und somit dargestellt werden und verstanden werden kann, was die anderen Interpretationen von zwei Welten nicht möglich gemacht haben.

Hier wird dann auch deutlich, wie dieses höchste Erkenntnisziel gemeint sein soll.

Halfwassen diesen Aufstieg über die Gleichnisse in der Politeia versteht. Denn es heißt, dass es rein um das höchste Erkenntnisziel geht, also rein gnoseologisch.\footcite[vgl.][S. 226]{halfwassenaufstieg2006}
\enquote{Denn zum Menschen gehört es, das gemäß der Idee Gesagte zu verstehen, indem er zu dem geht, was aus vielen Wahrnehmungen durch das Denken zu einer Einheit zusammengefasst wird. Platon versteht somit menschliche Erkenntnis prinzipiell als auf Einheit gerichtete Synthesis und Synopsis.}\footcite[][S. 228]{halfwassenaufstieg2006}
Somit heißt es in Rückbezug auf die reine Einheit:
\enquote{[J]ede \emph{Einheit in der Vielheit} - also jede Idee - und jedes \emph{Beziehen von Vielheit auf Einheit} - also jede Erkenntnis - setzt die \emph{reine Einheit} als absolut Einen immer schon voraus.}\footcite[][S. 230]{halfwassenaufstieg2006}
Dabei ist aber immer zu beachten, dass die Bipolarität der Prinzipien die Absolutheit des Einen keineswegs aufhebt, wodurch das hier zweite Prinzip kein zweites Absolute ist, sondern nur die Entfaltungsbasis des Absoluten.\footcite[vgl.][S. 53]{HalfwassenGeistmetaphysik}
Somit gelangt in das Höhlengleichnis noch eine tiefere Bedeutung hinein, die auch die zweimalige Umkehr erklärt, wenn man \enquote{Aus der Höhle herausgekommen ist} und die wahren Dinge gesehen hat, man aber wieder zurück in die Höhle gehen soll. Dabei wird auch die Stelle Pol. 504 B2 klar, in der von einem \enquote{längeren Umweg} gesprochen wird. Gemeint damit ist  der Aufstieg zum Einen und die nachfolgende Ableitung aller reinen Ideenbestimmung.\footcite[vgl.][S. 231]{halfwassenaufstieg2006} Mit diesem Aufstieg und dieser Umkehr, welche durch das Wissen um das Absolute besteht, ergibt sich erst das Verständnis für die Konzeption der Ideen in ihren Kontexten und deren Entfaltung in der Dialektik.
%\enquote{erst auf dem Weg der auf- und absteigenden Dialektik bekommt man alle Ideen \enquote{so schön wie irgend möglich zu Gesicht}, weil die Elemente ihres Wesensaufbaus voneinander abgehoben und so dieser Wesensaufbau vom universalen Urgrund her vollkommen durchsichtig geworden ist.}\footcite[][S. 231]{halfwassenaufstieg2006}
%Von daher versteht sich Platons ethische Forderung \enquote{aus Vielen Einer zu werden}(Pol. 443E1); die innere Einheitlichkeit der Seele und ihres Abbildes, der Polis, ist das Werk der höchsten ethischen Arete, der Gerechtigkeit\footcite[vgl.][S. 237]{halfwassenaufstieg2006}
Somit wiederholt Halfwassen, dass die Idee des Guten nicht nur als die Prinzip der Arete, sondern auch als Prinzip des Seins und der Erkenntnis aufzufassen ist.\footcite[vgl.][S. 238]{halfwassenaufstieg2006}
Die Idee des Guten ist dafür verantwortlich, dass \enquote{den durch die Vernunft erkennbaren Dingen von dem Guten nicht nur das Erkanntwerden zuteil wird, sondern dass ihnen dazu noch von jenem das Sein und die Wirklichkeit zukommt.}(Pol. 509b-c) 
Hier werden Sein und Wirklichkeit nochmals als getrennt aufgezählt, da die Prinzipien der Einheit und der Vielheit, sowie der unbestimmten Vielheit zwar wirklich, aber nicht seiend sind, zumindest in dem Sinne, dass sie, wie Ideen Objekt des Denkens sind, sondern als Prinzipien, wie festgestellt, noch nicht sind.
\enquote{Das Eidos ist dialektisch bestimmbar, weil es eine Vielheit von eidetischen Bestimmtheiten in der Einheit eines Ganzen zusammenfasst: es hat den Charakter einer \enquote{Einheit aus Vielem}.}\footcite[][S. 240]{halfwassenaufstieg2006}
%\enquote{Darum kann die Gerechtigkeit, das Ordnungsprinzip von Polis und Seele, in analoger Weise auch die intelligible Ordnung der Ideen selbst charakterisieren.}\footcite[][S. 242]{halfwassenaufstieg2006}
% Die Entfaltung der Einheit in die Fülle des Seins ist Schönheit, die einende Ordnung der Vielen in einem einigen Ganzen ist Gerechtigkeit; beide erscheinen im richtigen Verhältnis des Ganzen und der Teile, also im Gesetz der geometrischen Gleichheit.\footcite[][S. 242]{halfwassenaufstieg2006}
% Das Vielgestaltige und Veränderliche, das sich bald so und bald anders zeigt, erfassen wir im sinnlichen Sehen; das Einheitliche und Unveränderliche, das immer dasselbe Wassein sehen läßt, zeigt sich nur im die Sinneswahrnehmung transzendierenden reinen Denken (507 B 9-10). Entsprechend unterscheidet Platon eine Welt des Intelligiblen (vonios topos, 508 C 1, 517 B 5, vgl. C 3) und eine Welt des Sinnenfälligen oder Sichtbaren (opatos Toлos, vgl. 508 C 2, 517 C 3) als zwei Arten des Seienden (do con tan övtav, Phaid. 79 A 5; Politeia 509 D 1-3: dvo αὐτὰ εἶναι . . . τὸ μὲν νοητοῦ γένος τε καὶ τόπος, τὸ δ' αὖ ὁρατοῦ). Die sichtbare Welt aber ist das Abbild der intelligiblen Welt (vgl. Tim. 29 A- B), sie hat nur in der Teilhabe an dieser Sein, darum entsprechen sich die Strukturen beider Welten in strenger Analogie.\footcite[][S. 246]{halfwassenaufstieg2006}
\subsubsection{Zwei Arten des Seienden}
Von hier aus ließe sich die Trennung der beiden Bereiche der Dinge und der Ideen nochmal aufgreifen unter der Berücksichtigung des Bisherigen.
\enquote{Die Aussagen der \emph{Politeia} sprechen darum entscheidend für eine monistische Deutung der Prinzipienlehre.}\footcite[][S. 137]{halfwassen2015spuren}
\zitatblock{\enquote{Eine monistische Deutung der Prinzipienlehre kann freilich von vornherein keine \emph{Eliminierung} der sie durchgehen bestimmenden Bipolarität bedeuten, sondern nur ihre \emph{Relativierung} insofern, als das Vielheitsprinzip dem Einen nicht gleichursprünglich und gleichmächtig gegenüberstehen kann.}\footcite[][S. 138]{halfwassen2015spuren}}
Die beiden Arten des Seienden werden so erklärt, dass das Eine, also die Idee, der Vielheit nicht immanent ist, sondern diese noch transzendieren muss, aufgrund dessen, dass sie sonst nicht mehr \emph{Eines} wäre.\footcite[vgl.][S. 246]{halfwassenaufstieg2006}
Aus Pol. 507b9-10 ergibt sich dann bei Halfassen:\zitatblock{\enquote{Entsprechend unterscheidet Platon eine Welt des Intelligiblen (noetos topos, 508 C 1, 517 B 5, vgl. C 3) und eine Welt des Sinnenfälligen oder Sichtbaren (oratos topos, vgl. 508 C 2, 517 C 3) als zwei Arten des Seienden (duo eide ton onton, Phaid. 79 A 5; Politeia 509 D 1-3: duo auto einai...to men ontou te kai topos, to d au oratou). Die sichtbare Welt aber ist das Abbild der intelligiblen Welt (vgl. Tim. 29 A- B), sie hat nur in der Teilhabe an dieser Sein, darum entsprechen sich die Strukturen beider Welten in strenger Analogie.}\footcite[][S. 246]{halfwassenaufstieg2006}}
Damit würde also der im Eingang gelieferten Definition Nicht-Immanenz und der unabhängigen Existenz die Möglichkeit der unabhängigen Existenz widersprochen werden müssen, da hier die Dinge nicht losgelöst von den Ideen existieren könnten.
%\enquote{Die Sonne ist aber Bild und Analogon des Agathon: Die Sonne ist in der Dimension des Sichtbaren im Verhältnis zum Sehen und zum Gesehenen das, was das Gute selbst in der Sphäre des Intelligiblen in Bezug zum Geist und zum Gedachten ist (508)}\footcite[][S. 250]{halfwassenaufstieg2006}
Der Einheitscharakter des intelligiblen Lichtes, welches mit dem Licht der Sonne zu vergleichen ist, aber sich eben auf die geistige Einsicht bezieht, verleiht den Ideen ihren Einheitscharakter auf zweierlei Weise. Die Ideen heben sich als Einzelne voneinander ab und werden gleichzeitig zu einer Einheit des kosmos noetos verbunden, was sie erst intelligibel macht.\footcite[vgl.][S. 252]{halfwassenaufstieg2006} 
Dabei spielt auch die Anmerkung, dass \enquote{Erkennen die Zurückführung der Vielheit auf die Einheit ihres Grundes [bedeutet]}(Phaidr. 249B6-C1),\footcite[vgl.][S. 252]{halfwassenaufstieg2006} eine wichtige Bedeutung.
Des weiteren kommen dem Einheitsgrund zusammenfassend drei Prinzipien zu.
\zitatblock{\enquote{Der Einheitsgrund der Ideen des Denkens, das Gute als das absolute Eine, ist mithin 
\begin{itemize}
    \item (1) das Prinzip der Erkennbarkeit der Ideen;
    \item (2) das Prinzip der Erkenntniskraft des Nous;
    \item (3) das Prinzip des aktualen Wissens, d.h. aber der Einheit von Denken und Sein im Erkenntnisakt
\end{itemize}}\footcite[][S. 253]{halfwassenaufstieg2006}}
Rückbezogen auf das Sonnengleichnis heißt es dann:
\enquote{Wahrheit und Wissen oder Einsicht aber sind Weisen der Einheit in der Vielheit, in denen sich das absolute Eine manifestiert, wie die Sonne in dem von ihr ausstrahlendem Licht.}\footcite[][S. 252]{halfwassenaufstieg2006}

\enquote{Wahrheit, Wissen und Nous haben also ihre Bestimmtheit in der \emph{Einheit} von Denken und Sein, deren Urgrund das Gute selbst ist.}\footcite[][S. 257]{halfwassenaufstieg2006}


\zitatblock{\enquote{Sowenig aber das Licht oder das Sehen selbst die Sonne ist, sowenig ist die Wahrheit oder das Wissen selbst das Gute und das Eine selbst; beide verdanken ihre vereinigende Kraft vielmehr einem jenseitigen Ursprung. Denn Wissen und Wahrheit haben zwar Einheitscharakter [\dots] jedoch so, daß sie ihre Einheit gerade in der Vielheit und Unterschiedenheit dessen haben, was in ihnen geeint ist
% wie ja auch das Licht die Einheit und Selbigkeit der Helle in der Vielheit des in ihr Aufscheinenden ist oder wie das Sehen die Einheit des Sehaktes in der Zweiheit von Sehendem und Gesehenem ist. Ebenso setzt die Einheit des Wissens die Zweiheit von Wissendem und Gewußtem voraus, deren Einheit als der Bezug Unterschiedener das Wissen ist.
[\dots]. Und die Wahrheit gründet als Unverborgenheit in der eidetischen Differenziertheit des reinen Seins, zuletzt also in der zahlenhaften Struktur des Ideenkosmos. Als Einheit in der Vielheit aber vermag weder das Wissen noch die Wahrheit die Einheit von Sein und Denken, in der sie beide ihr Wesen haben, von sich her zu begründen. Das Gute selbst, das als Prinzip der Einheit von Sein und Denken Wahrheit und Wissen allererst ermöglicht, ist darum nicht mit ihnen identisch, sondern liegt notwendig als reine absolute Einheit über sie beide hinaus.}\footcite[vgl.][S. 257]{halfwassenaufstieg2006}}
Hieran schließt sich wieder der Zusatz, dass das absolute Eine nicht nur Erkenntnis und Erkennbarkeit liefert, sonder auch das Prinzip des Seins und Wesensfülle aller Seienden schafft.\footcite[vgl.][S. 258]{halfwassenaufstieg2006}
%Fußnote: Vgl. Parm. 157 B- 158 D, 165 E-166 C, Soph. 245 AB; dazu die Referate Arist. Metaph. A 6 987 b 21, 988 a 11; Alexander, In Metaph. 56, 30 r. H.; Sextus Emp., Adv. Math. X 260-261, 277. Speusipp bei Proklos, In Parm. VII 40, 1 ff. 501, 62 ff. Steel, Simplikies, In Phys. 454, 15 (Porphyrics) und 455, 6 f. (Alexander)
%\enquote{Das Agathon - das absolute Eine selbst - ist als \emph{Urgrund} des Seins und der Seiendheit selbst \emph{jenseits} des Seins und der Seiendheit; durch sein über das Sein hinausliegendes Übermaß an Mächtigkeit stiftet es Sein, Seiendheit und Erkennbarkeit in eins und zumal, indem es den Ideen - und durch sie allem Seienden - Einheitscharakter und damit zugleich Identität, feste Bestimmtheit, Abgrenzung von Anderem und Fürsichsein verleiht.}\footcite[][S. 258f.]{halfwassenaufstieg2006}
Damit natürlich auch einhergehend stiftet das absolute Eine zugleich Identität, feste Bestimmtheit, Abgrenzung von Anderem und Fürsichsein der Ideen und allem weiteren Seienden.\footcite[vgl.][S. 258f.]{halfwassenaufstieg2006} Hiermit inbegriffen ist auch Nicht-Sein.
%Da Sein wesenhafte Bestimmtheit bedeutet und damit schon eine Zweiheit ist: es lässt sich auseinanderlegen in etwas und das, was es ist, also in Bestimmtes und Bestimmendes, liegt das absolute Eine selbst über alle Bestimmtheit hinaus und ist notwendig \emph{jenseits des Seins und der Seiendheit}\footcite[vgl.][S. 259f.]{halfwassenaufstieg2006}
Für das Denken und dessen Aufstieg zum Absoluten muss also gelten, dass alle Bestimmungen des reinen Seins transzendiert werden müssen. Mit anderen Worten muss eine Aufhebung aller Voraussetzungen stattfinden.(vgl. Pol. 533c8-9)\footcite[vgl.][S. 263]{halfwassenaufstieg2006}

%\subsubsection*{Szlezák Platonsiches Philosophieren}
%\subsubsection*{Der Ursprung der Geistmetaphysik Halfwassen}
%Rekapitualtion des Buches von Krämer mit demselben Titel. Was hat sich in den 35 Jahren zwischen diesen beiden Büchern getan? \footcite[vgl.][S. 50]{HalfwassenGeistmetaphysik}
%Halfwassen führt an, dass die Interpretation Plotins sich bereits, durch Krämer nachgewiesen, in der \enquote{alten Akademie} nachweisbar sind, wie etwa die absolute Transzendenz des Einen über das Sein und den Geist.\footcite[vgl.][S. 52]{HalfwassenGeistmetaphysik}\footnote{Hierzu werden Krämer Der Ursprung der Geistmetaphysik 1964 S. 338-369, ebd. 1959 S. 535-551 und 1969 aufgezählt.} Endlich hebt auch für Platon und Speusipp die Bipolarität der Prinzipien die Absolutheit des Einen keineswegs auf, ist auch hier das zweite Prinzip kein zweites Absolutes, sondern nur die Entfaltungsbasis des Absoluten.\footcite[][S. 53]{HalfwassenGeistmetaphysik}

%\zitatblock{\enquote{Auf der Grundlage der skizzierten Henologie basiert Plotins Metaphysik des Geistes, die zwei große Themenkomplexe umfaßt, nämlich erstens die Konstitution des Geistes in seinem Hervorgang aus dem überseienden Einen und zweitens die immanente Struktur der Selbstbeziehung des absoluten Denkens, dessen Inhalte die reinen Wesenheiten des Seienden, also die Ideen sind, in deren untrennbarer Einheit sich der Geist intellektuell selbst anschaut}\footcite[][S. 54]{HalfwassenGeistmetaphysik}}
%die dritte (in Plotins Zählung die vierte) Hypothesis des >>Parmenides<<, in der Platon zeigt, wie die von sich selbst her unbestimmte und seinslose Vielheit durch die Teilhabe an dem überseienden absoluten Einen zum in sich selbst bestimmten und vollendeten Einen und Ganzen (en holon teleion) des seienden Ideenkosmos wird.\footcite[vgl.][S. 54f.]{HalfwassenGeistmetaphysik}

%\zitatblock{\enquote{Denn halten wir fest: der Platonismus interpretiert das Sein aufgrund seiner Vermittlungsstruktur von Einheit und Vielheit als Geist, er analysiert die Struktur der Selbstbeziehung des Denkens, und er faßt den Zusammenhang der Ideen in der Einheit des Denkens als das Urbild und den aktiv bestimmenden Grund der geordneten Welt, konzipiert also das denkende Selbstverhältnis zugleich als Begründung der Welt}\footcite[][S. 60]{HalfwassenGeistmetaphysik}}
%\subsubsection*{Szlezák Halfwassen Monismus Dualismus in Platons Prinzipienlehre}

%Halfwassen identifiziert die bekanntlich zwei letzten Prinzipien Platons innerakademischer Prinzipienlehre, auf deren Zusammenwirken alles Seiende zurückgeführt wird: das absolute Eine (auto to en) und die unbestimmte Zweiheit (aoristos dyas)\footcite[vgl.][S. 67]{HalfwassenMonismusDualismus}

%Er hält auch fest, dass Krämer in seinem Buch \enquote{Der Ursprung der Geistmetaphysik} (1964) \enquote{[\dots] nicht mit einer Einheit der Gegensätze bei Platon, sondern im Sinne der neuplatonischen Platondeutung mit einer Zurückführung des Vielheitsprinzips auf das Eine als allbegrüdendes Ur-Prinzips.}\footcite[][S. 68]{HalfwassenMonismusDualismus} abzielt. (Krämer Seite 332-334; 1964) 
%\zitatblock{\enquote{Das Dialektikprogramm der Politeia beschreibt deutlich den Aufstieg zu Einem Unbedingten und Absoluten, das Urgrund von allem ist. Dies scheint einen irreduziblen Prinzipiendualisimus auszuschließen: denn wenn dem Einen die Veilehit als gleichursprüngliches und unabhängiges Prinzip gegenüberstünde, dann wäre das Eine nicht mehr das Prinzip von allem, und es wäre auch nicht mehr 
%{ἀνυποθετος ἀρχή}
%(anupothetos arche), da seine Wirksamkeit als Ursprung dann durch sein Zusammenwirken mit dem Vielheitsprinzip bedingt wäre.}\footcite[vgl.][S. 70f.]{HalfwassenMonismusDualismus}}
%Politeia 511b, 533c
%Damit ist zwar mit Halfwassen nicht vollständig auszuschließen, dass es lediglich den Monismus in dieser Weise in der Deutung der Prinzipienlehre gibt, sondern sich gerade in Anlehnung an Krämer, dass \enquote{[d]ie monistische Lösung einem Rückgriff hinter den Gegensatz der beiden Prinzipien [entspricht], ohne ihn aufzuheben}\footcite[vgl.][S. 333]{Krämer1964Geistmetaphysik}
%Daraufhin verweis auf Parmenides 8 Hypothesen, in denen Einheit und Vielheit zueinander in jedem Verhältnis untersucht werden. Siehe Appendix
%\enquote{Wenn unsere Deutung richtig ist, dann verbindet Platons Prinzipienlehre einen Monismus in der Reduktion zum Absoluten mit einem Dualismus in der Deduktion des Seienden.}\footcite[][S. 79]{HalfwassenMonismusDualismus}


%Das hier ist Halfwassen\\
%Lé Timée de Platon contributions à l'histoire de sa réception = Platos Timaios- Das hier ist das Buch, aus dem Hans zitiert hat. Also \enquote{Der Demiurg von J. Halfwassen}
%\subsubsection*{Hans Joachim Krämer. Arete bei Platon und Aristoteles in Platonisches Philosophieren 70/CD 3067 K89}
%peras (Ende)\\
%meros (Teil)
%mega-mikron (Das Groß-Kleine, Das zweite Prinzip neben dem en/apeiron (Einen))
%253f.: Das Prinzip der Zahlen ist noch vor dem der Ideen, da dieses Prinzip des Einen und Vielen erst die Ideen bedingen. 


\subsubsection*{Die letzte denkbare Einheit (agathon) Robert Wallisch}
%Es wird auf Paul Natorps Platons Ideenlehre zitiert, die ich in der Sophistes Arbeit zitiert habe. 
%Erste Arbeitshypothese: \zitatblock{\enquote{Wenn die Ideen zu Beginn des sechsten Buches gedachte konstante Einheiten waren, welche die unstete Vielheit bewältigen - gleichsam als immergültige Bündelungen des Vielen zu noetischen Einheiten - so muss das agathon als eine noch höhere, eine letzte Instanz, welche ihrerseits die Ideen bedingt, d.h. den Ideen in analoger Weise übergeordnet ist wie die Ideen dem konreten Vielen, als letzte denkbare Einheit angesprochen werden}\footcite[][S. 10]{Wallisch}}

%Das agathon ist im noetischen Bereich des Erkennens wie die Sonne, die als ein Drittes die Möglichkeit des Erkennens und des Denkens erst schafft.\footcite[vgl.][S. 10]{Wallisch} 
%Dabei wird eine Grenzüberschreitung angesprochen, die das Gute (agathon) jenseits der Seite der Dinge verortet. Damit sind Ideen und Dinge auf der einen Seite, das Absolute (agathon), die Transzendenz des Guten auf der anderen Seite\footcite[vgl.][S. 11]{Wallisch} gemeint.
\enquote{Nur durch das agathon existieren die Ideen als reale Gegebenheiten, als Dinge des Denkens}\footcite[][S. 11]{Wallisch} und erzeugt somit die Ideen erst. Dabei wird eine Grenzüberschreitung angesprochen, die das Gute (agathon) jenseits der Seite der Dinge verortet. Damit sind Ideen und Dinge auf der einen Seite, das agathon, die Transzendenz des Guten auf der anderen Seite\footcite[vgl.][S. 11]{Wallisch} gemeint.
%Alleine hiermit ist dem ersten Teil der Arbeit widersprochen. Von wirklicher Transzendenz, oder starker Transzendenz ist nur zwischen der IdG und den Ideen \emph{zusammen mit} den Dingen gedacht.
%Entscheidend dabei ist der Begriff der ousia
%\enquote{Unnennbar viele sind die Sinnesdinge; doch auch die Ideen, welche die Vielen durch noetische Bündelung zu Einheiten bewältigen, sind selbst wiederum \textbf{viele} Einheiten und keinesfalls eine letzte denkbare Einheit [\dots]}\footcite[vgl.][S. 12]{Wallisch}
Im Gegensatz zu Natorp nennt Wallisch das agathon nicht eine absolute Einheit oder eine Grundprädikation, ein letztes Objekt der Grundwissenschaft, also ein keinesfalls losgelöstes, ein auf die Welt anwendbares Gesetz des Denkens [\dots], sondern vielmehr die jedes Denken präzedierende ontologische Gegebenheit, die durch Erweiterung auf das Ganze zu erlangen ist.\footcite[vgl.][S. 14]{Wallisch} Also wieder ein noch an Herrlichkeit dem Sein transzendent gestellt.
%Er hält außerdem fest, dass \enquote{[\dots] nicht eine Teilung in zwei Welten, sondern die Unterscheidung verschiedener gnoselogischer Zugänge zu derselben Welt intendiert ist.}\footcite[vgl.][S. 15]{Wallisch}
\zitatblock{\enquote{Wenn wir nun die Ideen als nicht (stark) transzendent aufgefasst haben, so darf darunter keinesfalls eine Leugnung ihrer Sonderung von den Sinnesdingen verstanden werden. Platons Texte lassen keinen Zweifel daran zu, dass die Ideen als real und getrennt von den konkreten Dingen existierend zu denken sind. Tatsächlich ist die platonische Idee eine gnoseologische Gegebenheit und somit als reale Entität anzusprechen, die von den Sinnesdingen gesondert existiert und auf die Sinnesdinge Wirkung hat.}\footcite[vgl.][S. 17]{Wallisch}}
Es wird mit Platons Texten auf Rep. 579a ff, Phaidon 74a-76e, 78a, Philebos 15a,b und Timaios 51c-e, 52a hingewiesen. Getrennt also nur in der Hinsicht, dass eine Unterscheidung gemacht werden muss, wenn man von Bedingten und Bedingenden spricht, aber nur in der Weise, dass es um Erkenntnis geht, nicht um direkte Seinsbedingungen. Daher heißt es weiter: 
\enquote{Weil die Ideen, noetische Einheiten, immer noch viele sind, befinden sie sich diesseits des ontologischen Chorismos, der die transzendente totale Einheit von den Vielen trennt.}\footcite[][S. 17]{Wallisch} Damit ist also gesagt, dass Dinge genauso wie die Ideen auf derselben ontologischen \enquote{Seite} sich befinden, eben nur im Gegensatz zum agathon (Idee des Guten, Absoluten Einen) stehen.
Tim 51b7-c5 Stelle zeigt \emph{unzweideutig}, dass die Sonderung der Ideen erkenntnistheorisch und nicht ontologisch aufzufassen ist.\footcite[vgl.][S. 19]{Wallisch}
%Damit ist also deutlich gemacht, dass eine gnoseologische Trennung zwischen Dingen und Ideen gemeint sein soll und keine ontologische.


Ein weitere Punkt wird von Wallisch so ausgearbeitet, dass er
% - genauso wie Natorp - 
die Ideen mehr als Methode oder Mittel begreift, als als Ziel.\footcite[vgl.][S. 26]{Wallisch} Herangezogen wird dafür die Stelle Phaidon 99e4-100a5, in der die Ideen als eine \enquote{Hypothese} verstanden werden, die als gedanklich eingesetzte Stütze oder Instrument und somit nicht als Objekte von Erkenntnis verstanden werden.\footcite[vgl.][S. 26]{Wallisch} Es wird dabei betont, dass es auf die ontologische Wahrheit konzipiert worden ist und nicht in Hinblick auf die Erkennbarkeit der konkreten Sinnesdingen, also mit Blick darauf, dass mithilfe der Ideen zu einer höheren Erkenntnisstufe des Seins gelangt werden soll.\footcite[vgl.][S. 27]{Wallisch}
Dieser Punkt ist besonders dahingehend interessant, wenn man diesen bis zuende verfolgt. Damit würde man die Ideen lediglich als Leiter verwenden, welche man nach ihrem gebrauch hinter sich lassen kann, da man auf der nächsten höheren Ebene angelangt ist, jedoch soll von diesem \enquote{hinabblicken} auf die Sinnesdinge abgesehen werden. 




\subsubsection*{Zusammenfassung dieser Interpretation}
Zusammenfassend kann man somit festhalten, dass es eine zweifach doppelte Ausführung der Einheit Vielheit Relation in diesem Gebilde gibt. 
Begonnen mit den Sinnesdingen, die als Vielheit identifiziert worden sind, werden diese ontologisch von den Ideen bedingt, welche im Gegensatz zu den Sinnesdingen hier als Einheit gedacht werden müssen. Da die Ideen für sich betrachtet noch unter sich Vielheit aufweisen, müssen diese von einer weiteren Einheit bestimmt werden, welche aber ontologisch diesen Bereich transzendiert, in der Form des agathon (IdG). Selbiges Verhältnis fungiert ebenso auf der gnoseologischen Ebene, wo die Vielheit der Sinnesdinge von den Ideen bedingt werden, welche wiederum vom agathon bedingt sind. Man könnte hier noch Kleinteiliger werden und die im Liniengleichnis gelieferten \enquote{Unterkategorien} der Sinnesdinge und der Ideen mit einbeziehen, jedoch würde sich hierbei die Zahl der Einheit Vielheit Relation lediglich um vier vergrößern, was der grundlegenden Darstellung keine weitere Bedeutung beifügt. Kurz angesprochen kämen die Übergänge und Verbindungen von Schatten zu den Dingen und von den geometrischen Dingen zu den reinen Ideen hinzu. 
%\subsection*{Naheliegende Stellen bei Platon %ursprünglich von Halfwassen)
%}
%Darstellung an denjenigen Stellen, die von den Autoren immer wieder genannt werden und somit am häufigsten verwendet werden. Es müssen erst noch die Stellen gesichtet werden. Stellen von Halfwassen genannt:
%\begin{itemize}
 %   \item {Politeia 506de, 509c-d 510b7, 511b6, 511b7, 517c2-d, 524ff, 533c, 534b}
  %  \item {Parmenides 141e, 142bff, 142eff, 143bff, 165e-166c}
   % \item {Philebos 14cff, 23eff, 27b 30ab, 28c, 30d}
    %\item {2. Brief 312e}
%\end{itemize}

%Zur absoluten Transzendenz des Einen und Guten im einzelnen Halfwassen (1992) 19ff, 188ff, 221ff, 257ff, 302ff und Krämer (1969)
%Es wird viel auf Aristoteles Metaphysik verwiesen.
%\subsection{Rückbezug auf Originalstellen}
%Hier kann sich jetzt nochmal auf die Origianstellen berufen werden und nochmal betrachtet werden, inwiefern die andere Deutung halt findet.


Es sei also gesagt, dass diese Interpretation die \enquote{Trennung} von Bereichen nicht auf der zu tief angesetzten Ebene von Sinnesdingen zu den Ideen vornimmt, sondern die Ideenlehre in ihrer Gesamtheit zu begreifen möchte, d.h. die Grenze so verschiebt, dass diese erst durch die starke Transzendenz beschränkt wird am Übergang zur IdG als einheitstiftendes Prinzip, welches nicht mehr überschritten werden kann und auch nicht mehr selbst gedacht werden kann. Aus diesem Grunde sind die Ideen als noetische Einheiten immer noch viele und befinden sich mit den Sinnesdingen diesseits des ontologischen Chorismos, der erst mit dem Übergang zur IdG/dem Absoluten anzulegen ist.\footcite[][S. 17]{Wallisch}

\zitatblock{\enquote{Denn halten wir fest: der Platonismus interpretiert das Sein aufgrund seiner Vermittlungsstruktur von Einheit und Vielheit als Geist, er analysiert die Struktur der Selbstbeziehung des Denkens, und er faßt den Zusammenhang der Ideen in der Einheit des Denkens als das Urbild und den aktiv bestimmenden Grund der geordneten Welt, konzipiert also das denkende Selbstverhältnis zugleich als Begründung der Welt.}\footcite[][S. 60]{HalfwassenGeistmetaphysik}}
Auch von Wallisch heißt es, dass \enquote{[\dots] nicht eine Teilung in zwei Welten, sondern die Unterscheidung verschiedener gnoselogischer Zugänge zu derselben Welt intendiert ist.}\footcite[vgl.][S. 15]{Wallisch}
Wichtig hierbei ist der Begriff \enquote{derselben Welt}. Der Übergang des ontologischen Chorismos wurde erst erreicht, als der Übertritt hin zum absoluten Einen erfolgen sollte. Da dieses Absolute aber als einheitsstiftender Grund außerhalb dieses ontologischen Bereichs liegen muss, ist erst damit der Fall der starken Transzendenz erreicht. Allerdings wird dies nicht so stehen gelassen, sondern insofern wieder eingeholt, dass dieses Absolute Eine Seins- und Denkbegründung alles Seienden ist, ohne selbst - nach den uns zugänglichen Denkbegriffen - zu sein.
%\subsubsection*{Symposion}
%210ff. ist der Aufstieg zum Schönen, wo das Schöne an den schönen Leibern zuerst gesucht werden soll und dann die Suche immer weiter aufsteigt und das untere als Minderwertiges zurückgelassen werden soll.
%Der wahre Phiosoph und was er macht und ihn auszeichnet. Symp. 211e-212a

%\subsubsection*{Politeia (Gleichnisse)}
%509c-d 510b7, 511b6, 511b7, 517c2-d, 524ff, 533c, 534b, 596a-597e (drei Seinsweisen, Bild Naturding Idee)
%506de: Sokrates kann an dieser Stelle nicht über das Gute selbst sprechen, sondern versucht es in diesem Anlauf mit einem Sprössling des Guten, da Sokrates es selbst zu diesem Zeitpunkt nicht schaffen könnte.
%\zitatblock{\enquote{Dass es eine Vielheit von Schönem, sagte ich, eine Vielheit von Gutem und so überhaupt von allen Dingen gäbe, räumen wir ein und bezeichnen es auch näher in der Rede. Auch bekanntlich ein Schönes an sich, ein Gutes an sich, und so überhaupt in Bezug auf alles, was wir erst eine Vielheit von jedem hinstellten, das stellen wir dann wieder, um in einem einzigen Begiff hin, als wenn die Vielheit einer Einheit wäre, und nennen es das Wesen von jedem.}} (Pol. 507b-c)

%\enquote{Was den Dingen, die erkannt werden, Wahrheit verleiht und dem Erkennenden das Vermögen des Erkennens gibt, das begreife also als die Wesenheit des eigentlichen Guten}(Pol. 508e)
%\zitatblock{\enquote{Und so räume denn auch nun ein, dass den durch die Vernunft erkennbaren Dingen von dem Guten nicht nur das Erkanntwerden zuteilwird, sondern dass ihnen dazu noch von jenem das Sein und die Wirklichkeit zukommt, ohne dass das höchste Gute Wirklichkeit ist, es ragt vielmehr über die Wirklichkeit an Würde und Kraft hinaus}(Pol. 509b-c)}
%(Warum trennt er hier das Erkanntwerden von dem Sein und der Wirklichkeit in einer dreifachen Ausführung?)
%Bei den mathematischen Dingen im Liniengleichnis heißt es: \zitatblock{\enquote{Nicht war, auch das weißt du, dass sie sich der sichtbaren Dinge bedienen und ihre Demonstrationen auf jene beziehen, während doch nicht auf diese als solche, als sichtbare, ihre Gedanken ziehen, sondern nur auf das, wovon jene sichtbaren Dinge aus Schattenbilder sind.[\dots] Selbst die Körper, die sie bilden und zeichnen, wovon es auch SChatten und Bilder im Gewässer gibt, eben diese Körper gebrauchen sie weiter auch nur als Schattenbilder und suchen dadurch zu Schauung eben jener Ausführung zu glangen, die niemand anders schaun kann als mit dem denkenden Verstand}(Pol. 510e-511a1)}
%Der letzte Abschnitt des Liniengleichnisses: 
%Apelt Übersetzung: \zitatblock{\enquote{So verstehe denn auch folgendes: unter dem zweiten Abschnitt des Denkbaren meine ich das, was der denkende Verstand unmittelbar selbst erfaßt mit der Macht der Dialektik, indem er die Voraussetzungen nicht als unbedingt Erstes und Oberstes ansieht, sondern in Wahrheit als bloße Voraussetzungen, d.h. Unterlagen, gleichsam Stufen und Aufgangssütztpunkte, damit er bis zum Voraussetzungslosen vordringend an den wirklichen Anfang des Ganzen gelange, und wenn er ihn erfaßt hat, an alles sich haltend was mit ihm in Zusammenhang steht, wieder herabsteige ohne irgendwie das sinnlich Wahrnehmbare dabei mit zu verwenden, sondern nur die Begriffe selbst nach ihrem eigenen inneren Zusammenhang, und mit Begriffen auch abschließe.}(Pol. 511B-C Apelt)}\nocite{PoliteiaApelt}
%Schleiermacher Übersetzung:\zitatblock{\enquote{So verstehe denn nun auch, dass ich unter dem anderen Unterabschnitte der nur durch die Vernunft erkennbaren Hälfte das verstehe, was die Vernunft durch die Macht der Dialektik erfasst und wobei sie ihre Voraussetzungen nicht als Erstes und Oberstes ausgibt, sondern als eigentliche Voraussetzungen, gleichsam nur als Einschnitts- und Anlaufungspunkte, damit sie zu dem auf keiner Voraussetzung mehr beruhenden Anfang des Ganzen gelangt, und wenn sie ihn erfasst hat, an alles sich haltend was mit ihm in Zusammenhang steht, wieder herabsteige ohne das sinnlich Wahrnehmbare dabei zu verwenden, sondern nur die Begriffe selbst nach ihrem Zusammenhang, und mit Begriffen auch abschließe}(Pol. 511b-c Schleiermacher)}
%Den Ursprung der zwei Welten stammt sehr wohl vom Höhlengleichnis. Da es hier eine deutliche Zweiheit von Welten gibt, die man betreten und auch verlassen kann.
%\zitatblock{\enquote{Wenn aber, fuhr ich fort, jemand ihn aus dieser Höhle mit Gewalt den rauen und steilen Aufgang aufwärts zöge und ihn nicht losließe, bis er ihn ans Licht der Sonne herausgebracht hätte, würde er wohl Schmerzen empfunden haben? Würde er über dieses Hinausziehen aufgebracht werden und, nachdem er ans Sonnenlicht gekommen ist, die Augen voller Blendung haben und also gar nichts von den Dingen sehen können, die jetzt als wirklich ausgegeben werden?}(Pol. 515e-516a Schleiermacher)}
%Dieser Vorgang lässt sich an dasselbe Argument der unsterblichen Seele knüpfen. Also dass es ein Unveränderbares geben muss, an dem sich das Werdende/Veränderliche vollziehen kann. Also von in der Höhle nach Außen.
%\enquote{[er] würde über [die Sonne] die Einsicht gewinnen, [\dots] dass sie alles ordnet im Bereich der sichtbare Weltund von allen jenen Erscheinungen, die er dort sah, gewissermaßn die Ursache ist.}(Pol. 516 b4 Schleiermacher)
%\zitatblock{\enquote{Das Gleichnis also, mein lieber Glaukon, fuhr ich fort, ist nun in jeder Beziehung auf die vorhin ausgesprochenen Behauptungen anzuwenden. Die sich uns mittels des Gesichts offenbarende Welt vergleiche einerseits mit der Wohnung im unterirdischen Gefängnis, und das Licht des Feuers in ihr mit dem Vermögen der Sonne. Das Hinaufsteigen und das Beschauen der Gegenstände über der Erde stelle dir andererseits als den Aufschwung der Seele in das Gebiet des nur durch die Vernunft Erkennbaren vor, und du wirst dann meine Meinung hierüber haben, weil du sie doch einmal zu hören verlangst. Ein Gott mag aber wissen, ob sie richtig ist! Aber meine Ansichten hierüber sind nun einmal die: Im Bereich der Vernunfterkenntnis ist der Begriff des Guten nur zu allerletzt und mühsam wahrzunehmen. Nach seiner Ansicht muss man zur Einsicht kommen, dass er für alle Dinge die Ursache von allem Richtigen und Schönen ist, indem er in der sichtbaren Welt das Licht und die Sonne erzeugt. Sodann auch im Bereich des durch die Vernunft Erkennbaren selbst als Herrscher waltend, gewährt er sowohl die Wahrheit als auch Vernunfteinsicht. Ferner muss man zur Einsicht kommen, dass das Wesen des Guten ein jeder erkannt haben muss, der verständig handeln will, sei es in seinem eigenen Leben oder in öffentlichen Angelegenheiten.}(Pol. 516b-d Schleiermacher)}
%hierzwischen werden die Lehren der Arithemtik, der Geometrie, der Astronomie und der Akustik angeführt. Diese sind die Vorstufe zur Dialektik 
%wichtige Stelle vorher noch, wo das Höhlenlgeichnis nochmal mit dem Sonnengleichnis und der Dialektik zusammengebracht wird:
%\zitatblock{\enquote{Dagegen, sagte ich, die vorhergehende Lösung vonden Banden, und die Wendung von den Schatten zu den Bildwerken und zum Licht, und das Emporklimmen aus den unterirdischen Kerker zur Sonne, und das dort im Sonnenlichte, infolge des nch vorhandenen Unvermögens, sogleich die Tiere, Pflanzen und den Sonnenglanz anschauen zu können, zuerst gerichtete Schatten auf die im Wasser sichtbareb Spiegelungen und auf die Schatten der wirklichen Gegenstände, das aber hier zum Anschauen von Schatten des Seieden, nicht der Bilder Schatten, im Vergleich mit der Sonne ähnlichesLicht hervorgerufen werden, diese Kraft hat die gesamte Schulung in den von uns aufgestellten Lehrfächern, und dieser Weg heißt die Hinauführung des besten Seelenvemrögens zu der Anschauung des Wesens in dn Dingen, eine ganz ähnliche Hinaufführung, wie die oben erwähnte des Auges zu Anschauung des hellsten Gegenstandes in der sichtbaren Welt.}(Pol. 532b-d)}
%Damit ist eigentlich deutlich, dass die Welt \enquote{außerhalb} der Höhle keine andere Welt darstellen soll, sondern die Art und Weise verdeutlichen soll, wie der Aufstieg hin zur Sonne, zum hellsten Gegenstand, gemeint sein soll, die auf der Ebene der Ideen die Idee des Guten darstellen soll.
%\zitatblock{\enquote{und dass nur die Dialektik imstande ist, dem, der die oben beschriebenen Lehrfächer studiert hat, dies zu zeigen und auf eine andere Weise aber es nicht möglich ist? [\dots] Und auch das wird uns weiter niemand in Abrede stellen, [\dots] wenn wir behaupten, dass kein anderes wissenschaftliches Verfahren das Sein eines jeden Dinges zu erfassen strebt, denn alles andere Können und Wissen ist entweder auf menschliche Meinungen und Begierden, oder ist auf die verschiedenen Arten des Entstehenden, auf dessen Zusammensetzung oder ihre Pflege gerichtet.}(Pol. 533b Schleiermacher)}
%Diese Form der Dialektik, des Abstraktionsvermögens, ist notwendig, um die Abstrakte Ebene zu verstehen, die im Höhlengleichnis gemeint sein soll. Es wird zwar erklärt, dass die Dinge außerhalb der Höhle erkannt werden können, es aber nicht möglich ist direkt in die Sonne zu schauen. Dass hier die Sonne als etwas beschrieben wird, das man zwar nicht direkt ansehen kann, aber insofern beschreiben kann, dass es der Grund für alles weitere ist, macht nur deutlich, dass diese Abstration davon, dass man in den \enquote{beschienen} Dingen die Sonne ausmachen kann, die Sonne durchaus auf eine gewisse Art erkennen kann, ohne sie direkt zu sehen.
%\zitatblock{\enquote{Die Wissenschaften, denen wir zugestehen, dass sie etwas vom Seienden erfassen, wie Geometrie und ihre verwandten, sehen wir zwar über das Sein träumen, aber im wachen Zustand ist es ihnen unmöglich, es zu schauen, solange sie sich unerwiesener Voraussetzungen bedienen und sie ganz unberührt lassen, weil sie dies nicht begründen können. Denn wobei der Anfang aus dem besteht, was man nicht weiß, und Ende und Mitte aus dem Nichtgewussten zusammengeflochten werden, wie kann auf eine solche Weise angenommen werden, dass eine Wissenschaft entsteht?}(Pol. 533c Schleiermacher)}
%\zitatblock{\enquote{Es genügt, also fuhr ich fort, den ersten Abschnitt des Erkennens Wissenschaft zu nennen, den zweiten Verstandeseinsicht, den driten Glaube, den vierten Wahrerscheinen, und einerseits die beiden letzten zusammen Meinung, andererseits die ersten zusammen Vernunfteinsicht, dabei bezieht sich Meinung auf das wandelbare Werden, Vernunfteinsicht auf das unwandelbare Sein, so dass wie Sein zum Werden, so Vernunfteinsicht zu Meinung, und wie Wissenschaft zum Glaube, so Verstandeseinsicht zum Wahrscheinen sich verhält. Die entsprechenden Verhältnisse dessen, woraus sie sich beziehen, sowohl des durch Meinung Erfassbaren als auch bei dem durch Vernunft Erkennbaren, und ihre Unterteilung wollen wir jetzt, mein lieber Glaukon, beiseitesetzen, damit wir nicht in noch viel umfassendere Erörterungen geraten als vorher.}(Pol. 534a-b1 Schleiermacher)}
%596a ff. Drei Seinsweisen.
%Beginn mit Annahme von beliebigen Vielheiten von Tischen und Betten. Es gibt von diesen Gerätschaften nur zwei Begriffe, einen von Bett und einen von Tisch und der Werkmeister macht den Begriff (vgl. Pol. 596b Schleiermacher)
%Es wird ein noch außerordentlicher Meister gegeben, der auch alle Erzeugnisse der Erde bildet, alle lebenden Wesen hervorbringt und alles übrige sowohl sich selbst. (vgl. Pol. 596c Schleiermacher)
%Ein Maler ist damit gemeint, denn dieser macht auch auf gewisse Weise alles. (vgl. Pol. 596e)
%Es entstehen drei verschiedene Seinsweisen: \enquote{Also Maler, Tischler und Gott sind drei Meister für drei Arten von Betten}(Pol. 597b)
%Der Maler ist der Nachbildner, der Tischler der Werkmeister und der Gott der Wesensbildner (vgl. Pol. 597d-e)


%\subsubsection{Phaidon 74e-75d, 99d-105c}
%Aber doch an den Wahrnehmungen muss man bemerken, dass alles so in den Wahrnehmungen vorkommende jenem nachstrebt, was das gleiche ist und dass es dahinter zurückbleibt. [\dots] Ehe wir also anfingen zu sehen oder zu hören, oder die anderen Sinne zu gebrauchen, mussten wir schon irgendwoher die Erkenntnis bekommen haben des eigentlichen Gleichen, was es ist, wenn wir doch das Gleiche in den Wahrnehmunen als auf jenes beziehen sollten, dass dergleichen alles zwar strebt zu sein wie jenes, aber doch immer schlechter ist. (75a-b) 
%[\dots], dass ich voraussetzte, es gebe ein Schönes an und für sich, und ein Gutes und Großes und so alles andere, woraus, wenn du mir zugibst und einräumst dass es sei, ich dann hoffe, dir die Ursache zu zeigen und nachzuweisen, dass die Seele unsterblich ist (100b-c)
%Erste Voraussetzung: Wenn irgend etwas anders schön ist außer jenem, selbstschönen, es wegen nichts anderem schön sei, als weil es Teil hat an jenem Schönen. (100c)

%\subsubsection*{Timaios 27b-29b, 51b-52d}
%Es soll über das All gesprochen werden, \enquote*{wie es entstanden ist oder auch ungeworden ist.}(Tim. 27c)\nocite{TimaiosSchleiermacher}
%\zitatblock{\enquote{Was ist das stets Seiende und kein Entstehen Habende und was das stets Werdende, aber nimmerdar Seiende; das eine ist durch verstandesmäßiges Denken zu erfassen, ist stets sich selbst gleich, das andere dagegen ist durch \emph{bloßes} mit vernunftloser Sinneswahrnehmung verbundenes Meinen zu vermuten, ist werdend und vergehend, nie aber wirklich seiend.}(Tim. 27d-28a)}
%Der Gestalter muss also als Vorbild das sich stets gleich Verhaltende, wenn etwas schönes gestaltet werden soll. Wenn er allerdings etwas Gewordenes als Vorbild nimmt, so wird es nicht schön.(vgl. Tim. 28a-b)
%\enquote{Ist aber diese Welt schön und ihr Werkmeister gut, dann war oofenbar sein Blick auf das Unvergängliche gerichtet; ist \emph{sie} aber - was auch nur auszusprechen frevelhaft wäre, dann \emph{war sein} Blick auf das Gewordene \emph{gerichtet}. Jedem aber ist doch deutlich, dass \emph{er} auf das Unvergängliche \emph{gerichtet war}, denn sie (die Welt) ist das Schönste unter dem Gewordenen, er der Beste unter den Ursachen.}(Tim. 29a)
%\enquote{Das aber zugrunde gelegt, ist es ferner durchaus notwendig, dass diese Welt von etwas ein Abbild sei.}(Tim. 29b)
%\enquote{Gibt es ein Feuer an sich und für sich und alles das, wovon wir stets in dieser Weise reden, als jeweils an sich und für sich seiend, oder ist allein das, was wir sehen und sonst vermittels des Körpers wahrnehmen, da es eine solche Wahrheit (Realität) hat, und gibt es anderes außer diesen auf keine Art und Weise, sondern behaupten wir jeweils vergeblich, dass es von jeglichem eine denkbare Form gebe, und waren das nicht als \emph{leere} Worte?}(Tim. 51b-c)
%\enquote{Wenn Vernunft und richtige Meinung zwei verschiedene Arten sind, dann gibt es auf alle Fälle dies Dinge an sich, Formen, die sich von uns nicht wahrnehmen lassen, sondern nur gedacht werden.[\dots] Aber jene beiden sind als zwei zu bezeichnen, da sie gesondert entstanden und von unähnlicher Beschaffenheit sind. Denn das eine entsteht in uns durch Belehrung das andere durch Überredung.}(Tim. 51d-e)
%\subsection{Nomoi}
%ab 896 wird die Seele beschrieben, wie sie sich selbst und alles weiter bewegt, \enquote{es ist auf das Vollständige geziegt, dass sie der Anfang aller Bewegung und eben damit auch, dass sie das Ursprüngliche aller Dinge ist.}
%\enquote{Und wenn nun ferner die Seele alles durchwaltet und allem innewohnt, was überall sich bewegt, muss man da nicht auch dem ganzen Weltall eine solche es durchwaltende Seele zuschreiben? - Sicher. - Eine oder meherere? Mehrere antwortete ich für euch. Mindestens müssen wir ihrer zwei annehmen, eine wohltätige und eine, welche das Gegenteil vollbringen kann.}
%\subsubsection{Phaidros 249}
%\nocite{phaidros}
%Denn der Mensch muss in Begriffen Ausgedrücktes begreifen, was aus einer Vielheit innlicher Wahrnehmungen sich ergebend, durch den Verstand zur Einheit zusammengefasst wird.(Phaidr. 249b6-c1 Schleiermacher)
%\subsubsection{Sophistes 251a-259d}
%Die Ausführung der Problematik des Einen und Vielen, wie deren Einheit und Verschiedenheit und Verbindung untereinander, wird sich im Sophistes in der Dihairese besonders gewidmet, bzw. hier besonders gezeigt, wenn es darum geht den Weg der Definition zu gehen und hierbei die unterschiedlichen Ebenen voneinander trennen zu können, aber doch den Sinn einer Definition, die Abgrenzung von anderen Dingen auf der selben Ebene von einer höheren und niederen Ebene zu unterscheiden.
%\subsubsection*{Geschichte der Philosophie Band I Altertum und Mittelalter Johannes Hirschberger 1980}
%Die wichtigsten genannten Stellen der \enquote{Ideenlehre}:
%\begin{itemize}
 %   \item {Phaidon
  %  \begin{itemize}
   %     \item{74a-75d (Erkennbarkeit des Gleichen und des Verschiedenen in Dingen, mit Wiedererinnerung)}
    %    \item{99d-105c (Die Alternative des Sokrates, seine Ideenlehre);}
    %\end{itemize}}
    %\item{Politeia
     %   \begin{itemize}
      %      \item{507d-509b (Idee des An-sich-Guten und Sonnengleichnis)}
       %     \item{509d-511e (Liniengleichnis)}
        %    \item{514a-516c (Höhlengleichnis)}
         %   \item{596a-597e (drei Seinsweisen, Bild Naturding Idee)}
        %\end{itemize}}
    %\item{Timaios
    %\begin{itemize}
     %   \item{27b-29b (Entstehung der Welt, des Seienden und Werdenden)} 
      %  \item{51b-52d (Zusammenfassung)} 
       % \item{Deckt sich mit dem, was beim Hans zu finden ist: Timaios 27c1-53c4 und Phaidon 100d}
    %\end{itemize}}
    %\item{Sophistes 251a-259d (Gemeinschaft der Ideen und die Dialektik);}
    %\item{Parmenides 130e-135b (Selbstkritik); In welcher Weise haben die Dinge an den angenommenen Begriffen teil?}
%\end{itemize}
%\subsubsection*{Parmenides}
%1. Hypothese (137c4-142a8)\\
%2. Hypothese (142b1-157b5)\\
%3. Hypothese (157b6-159b1)\\
%4. Hypothese (159b2-160b4)\\
%5. Hypothese (160b5-163b6)\\
%6. Hypothese (163b7-164b4)\\
%7. Hypothese (164b5-165e1)\\
%8. Hypothese (165e2-166c5)

 

\section{Welcher Interpretation ist der Vorzug zu geben?}
\enquote{Nur ein mangelnder Metaphysik- und Transzendenzbegriff - \enquote{Metaphysik}: das schlechthin unzugängliche \enquote{Jenseitige}- führt zu der Zweiweltentheorie eines totalen Chorismos, wo in Wirklichkeit nur ein modaler gemeint war, eine \enquote{Trennung} des Seins nach seinem Wesen in Gegründetes und Gründendes. Es ist eine Modifizierung, der es ebensosehr auf die Trennung wie auf die Einheit ankam}\footcite[][S. 96]{Hirschberger}
Es muss also die Verschiebung des Gegründeten und Gründenden nicht auf der Ebene zwischen Dingen und Ideen angewendet werden, sondern erst auf der Ebene aus Dingen \emph{und} Ideen im Gegensatz zur IdG, welche damit den ontologischen Übergang markiert.
\subsection{Wichtige Pro-Argumente}
%Gerade die Unmöglichkeit der direkten Unterscheidung der Begrifflichkeiten, die Platon verwendet, wenn es um Ideen (eidos idea) und Dinge geht, lässt zuerst auf die Möglichkeit der verschiedenen Interpretationen schließen.
Die Frage besteht, ob eine Gleichwertigkeit zwischen der ontologischen und gnoseologischen Aspekten der Deutung liegt. Selbiges gilt ebenfalls für die \enquote{Ideenlehre} selbst. Ist die Frage zulässig, ob man einer der beiden Bereiche eine wichtigere oder zuförderst darzulegende Rolle beimisst? Bedarf es zuerst der Ontologie, welche begründet werden muss, oder doch erst der Gnoselogie?

Damit sind alle Bestimmungen, welche im ersten Teil dieser Arbeit angeführt worden sind deutlich unter dem Prinzip des Einen gefasst worden, sodass eine Auseinanderdifferenzierung von Ideen und Sinnesdingen auf diese Weise keine Grundlage mehr besitzt. 
\subsection{Wichtige Contra-Argumente}
\enquote{Die gesamte Abwertung der sinnlich-körperlichen Dimension des Seins bei Platon (und der platonischen Tradition) mag in der Unfähigkeit und Angst davor ihren Grund gehabt haben, den leiblichen Tod und das Vergehen als solches anzunehmen.}\footcite[][S. 99f.]{ThurnerDualismus}
Halfwassen Seite 76 in Szlezák: Dagegen spricht eine Reihe von Zeugnissen für eine prizipielle Unterordnung der (griechischer Begriff) unter das Eine, ohne dass sie als das Prinzipat aus dem Einen abgleitet würde, womit im übrigen ihr Status als Prinzip aufgehoben wäre.
Der Hauptpunkt, der schon in der Erarbeitung klar geworden zu sein scheint, ist, dass gerade mit dem Moment der Transzendenz der Idee des Guten als der Bedingung der Möglichkeit alles Seins und Seienden eine \enquote{zweite Welt} geschaffen worden ist, die Eingangs kritisiert worden ist.

Schwierig ist der Aspekt des Seinsgrades je nachdem wie sehr der Einheitsgedanke vorzufinden ist. \footcite[vgl.][S. 99f.]{halfwassen2015spuren}
Alles ist denkbar und seiend, weil es einheitlich ist. 
Ist das nur die Möglichkeit es zu denken oder auch der Grund dafür?

%\subsection*{Abfederung der Kritik}
%Soll das noch gemacht werden? 
%Die Erkenntbarkeit der Idee des Guten kann nur in Abhängigkeit alles Erkenntnis aller Ideen geschehen. D.h. nur mithilfe der Ideen und deren Prinziphaftigkeit kann die Idee des Guten eingeholt werden, ohne diese direkt schauen zu können. 


\section{Zusammenfassung und wie/wann man von einer oder zwei Welten sprechen sollte}
Im Grunde ist hier schon die Zusammenfassung, bei der geklärt wird, dass es auf die Ebenen ankommt, auf denen man von zwei Welten sprechen darf und auf welchen Ebenen man eben nicht davon sprechen darf. Außerdem auch 
Wie also über die Idee des Guten, als noch jenseits des Seins gesprochen wird ist diese Zweiheit insofern zulässig, als dass hier wirklich von einer zweiten Welt sich sprechen ließe, einfach weil es nicht mehr über diese eine Welt geht, sondern gerade darüber hinaus und nicht mehr nur auf eine Welt beschränkt ist. Einzuwenden ist dann selbstverständlich inwiefern es dann möglich ist, dass die Idee des Guten dann überhaupt auf die erste Welt einfluss nehmen kann, wenn sie doch einer anderen Welt angehört.  
%\section{Anwendung auf modernes Problem}
%Artikel: Elon Musk: „ChatGPT lügt“ - jetzt kündigt er „TruthGPT“ an \footcite[][]{MuskTruthGPTFokus}\\
%oder Can Elon Musk’s TruthGPT define and protect the truth? \footcite[][]{MuskTruthGPTReuters}\\
%Elon Musk will eine neue KI auf den Markt bringen, die nach Wahrheit streben soll und \enquote{Das Wesen des Universums} ergründen soll
%Wahrscheinlich wird sich in den drei Monaten, in denen diese Arbeit entstehen wird, nichts so grundlegendes verändern, dass die eingehenden Überlegungen deutlich abweichen dürften. 

%Der Suche nach Wahrheit scheint ein neuer Begleiter an die Seite getreten zu sein. Die scheinbar übermächtige Kapazität und Möglichkeit von künstlicher Intelligenz (KI). Damit soll also versucht werden der Wahrheit in ihrem Wesen auf den Grund zu kommen. Der aktuelle Stand ist, dass diese Idee eine KI zu entwickeln, um das Wesen des Universums zu ergründen, lediglich bekannt gegeben. Das heißt es sind noch keine Ergebnisse oder Berichte von möglichen Antworten dieser KI bekannt gegeben worden, um das Ausmaß oder die Tragweite dieser KI einschätzen zu können. Aus dem Blickwinkel der Philosophie jedoch ist alleine schon diese Idee äußerst brisant. Enthusiasten mögen jetzt behaupten, damit sei möglicherweise das Ende aller Philosophie eingeläutet, wobei erst auf die ersten Ergebnisse gewartet werden müsste, Überlegungen allerdings durchaus angestellt werden können.
%Der Ursprung der Ideenlehre war gerade die Wahrheit der Welt zu erklären. Schließt sich hiermit der Kreis nach knapp 2400 Jahren?

%Oder eventuell die Entwicklung von unreal enginge 5 und die Möglichkeiten, die es bringt mit einer so unfassbar realistischen Grafik die Realität so gut abzubilden.

%Oder doch eher die Apple Vision Pro?
\section{Ausblick}
Zu behaupten, dass die Ideenlehre Platons nicht idealistisch sei, ist in Anbetracht der Forderung, dass Philosophen Könige oder die Könige Philosophen werden sollten (Pol. 473c) eher irrsinnig.\\
Auch die Ansicht, dass Platons Philosophie nur eine Philosophie sei und es viele Philosophien gibt, in Anlehnung an Hegel, ist widersinnig. Einfach aus dem Grund, dass der Anspruch der Wahrheit in jeder Philosophie liegt und man diesen nicht in jede Philosophie legen kann, ohne den Sinn von Wahrheit dabei zu verlieren. Das heißt, wenn man viele Wahrheiten setzt, mit welchem Wahrheitsanspruch lässt sich diese Annahme wiederum als wahr aussagen. Nur dadurch, dass man annimmt, dass diese Aussage wahr ist, womit man sich in der Eingangs angenommenen These widersprechen müsste.
Bei Tim 27d, der von manchen als ausschlaggebend für die platonische Ideenlehre angesehen wird, räumt Platon selber ein, dass \enquote{meiner Meinung nach folgendes zu unterscheiden ist}. Dies macht nur nochmal deutlich, dass es Platon darum ging sich nicht nur von den Dialogen \enquote{überreden} zu lassen, sondern, da vielfach geziegt worden ist, dass Meinungen kein wirkliches Wissen darstellt, man selber diese Vorgebrachten Dinge für sich durchdenken muss, um sie nicht nur zu verstehen, sondern auch um sie zu überprüfen, da sich das Verständnis gerade nur in der noesis vollstreckt.   
\section{Fazit}
Auch der Ausdruck \enquote{Bürger zweier Welten} ist somit unzulässig. Stammt eigentlich von Kant
Inwiefern ist Erkenntnis notwendig für alles weiter? Gerechtigkeit, Moral, Gutes und Schlechtes, Richtiges und Falsches? 
\newpage
\nocite{politeia}
\nocite{Parmenides}
\section{Appendix}
Ausgesonderte Informationen, die den Haupttext überladen würden:\\
%Parmenides und Phaidros Dialog.
%Phaidros von Martin als einer der vier großen Ideenlehre Dialoge angsprochen 
%\subsubsection*{Parmenides 131a 141e, 142bff, 142eff, 143bff, 165e-166c}
%\enquote{Wenn das Seiende vieles wäre: so müsste dieses viele unter einander auch ähnlich sein und unähnlich? Dieses aber wäre unmöglich, denn weder könnte das Unähnliche ähnlich, noch das Ähnliche unähnlich sein?}(Parm. 127e1-4) 
%Das Seiende, wenn es eines sein soll, was es auch sein muss, kann nicht vieles sein, da man keine Unterschiede in dem Seienden mehr ausmachen könnte.
%Der Grund für den Dialog ist ein Buch, welches gegen diejenigen gerichtet ist, die versuchen das Viele existiert. (Also pro Parmenides es gibt nur das Eine)
%Das Eine und das Viele lassen sich in einem Ding zeigen. (Ein Ding ist ein Ding, kann aber mit hat ein vorne, hat ein hinten, hat eine linke Seite eine rechte Seite beschrieben werden) 
%\zitatblock{\enquote{Wenn aber jemand, wie ich nur eben sagte, zuvörderst die Begriffe selbst aussonderte, die Ähnlichkeit und Unähnlichkeit, die Vielheit nd die Einheit, die Bewegung und die Ruhe, und alle von dieser Art, und dann zeigt, dass diese auch unter sich können mit einander vermischt und voneinander getretnnt werden, das o Zenon, habe er gesagt, würde mir gewaltige Freude machen.}(Parm. 129d2-e4)}
%131a Eingangs In welcher Weise haben die Dinge an den angenommenen Begriffen teil?
%\enquote{[\dots] es gebe gewisse Begriffe durch deren Aufnahme in sich diese andern Dinge den Namen von ihnen erhalten [\dots].Also muss entweder den ganzen Begriff oder einen Teil davon jedes aufnehmende in sich aufnehmen?}(Parm. 130e5-a7)
%\enquote{eigentlich scheint es mir sich so zu verhalten, dass nämlich diese Begriffe gleichsam als Urbilder dastehen in der Natur, die andern Dinge aber diesen gleichen und Nachbilder sind; und dass die Aufnahme der Begriffe in die andern Dinge nichts anders ist, als dass diese ihnen nachgebildet werden.}(Parm. 132c12-d5)
%\enquote{Also auch nicht durch Ähnlichkeit nehmen die andern Dinge die Begriffe auf}(Parm. 133a5-6)
%Die andern Dinge sind damit also wirklich getrennt von den Begriffen/Ideen?

%141e Das Eine ist weder geworden, noch wurde es oder war es, noch ist es jetzt geworden oder wird oder ist, noch wird es in Zukunft geworden sein oder wird werden oder wird sein. Das Eine hat auf keine Weise gar keine Zeit an sich.
%142b Das \enquote{Ist} ist etwas anders als das \enquote{Eins}
%\zitatblock{\enquote{Wenn also Eins nicht ist, so wird auch nicht irgend etwas von dem Anderen weder Eins zu sein vorgestellt noch Vieles. Denn ohne Eins Vieles vorstellen ist unmöglich.}(166a5-b2)}
%\enquote{Also auch zusammengefasst, wenn Eisn nicht ist so ist nichts[\dots]}(166c1-c2)
%\enquote{[\dots] das Eins sei nun oder sei nicht, es selbst und das Andere insgesamt, für sich sowohl als in Beziehung auf einandern, alles auf alle Weise ist und nicht ist, und scheint sowohl als nicht scheint.}(166c2-c6)
%\subsubsection*{Philebos 14cff, 23eff, 27b 30ab, 28c, 30d}
%\enquote{Denn dass Eines vieles ist und Vieles eines, ist doch wunderbar zu sagen, und es ist wohl leichter zu streiten mit dem, der das von beiden behauptet}(Phil. 14c Schleiermacher \nocite{PhilebosSchleiermacher})
%\zitatblock{\enquote{Wenn aber jemand den Menschen als Eines setzt, [\dots] und das Schöne als Eines, und das Gute als Eines, über dies und ähnliche Einzelne wird bei fleißiger Behandlung und Auseinanderlegung leicht Streitigkeit entstehen.[\dots] Zuerst ob man wohl annehmen darf, dass es dergleiche Einzelnes gebe als wahrhaft seiend. Dann aber auch, wie doch dieses Einzelne, da jedes von ihnen immer dasselbe sei, und weder Werden noch Untergang unterworfen, dennoch zuerst zwar unwandelbar dieses Eine sei, hernach aber in dem Werdenden und Unendlichen wiederum, sei es nun als Zerteiltes Vieles geworden, zu setzen ist, oder ganz von ihnen getrennt und außerhalb ihrer selbst, [\dots] dieses selbige Eine zugleich in Einem sowohl als in Vielem erscheint.}5(Philebos 15b Schleiermacher)}
%\zitatblock{\enquote{Zuerst nun lasse uns von diesen Vieren die Dreie ausondern, und versuchen, da wir die Zweie von ihnen jede gar vielfach zerspalten und zerissen sehen, ob wir, wenn wir sie jedes in Eins zusammengebracht haben werden, bemerken könen, wie wohl jedes von ihnen Eins und Vieles war.[\dots] Die Zweie also, die ich vorlege, sollen die eben genannten sein, das eine das Unbegrenzte, das andere die Begrenzung. Dass nun das Unbegrenzte gewissermaßen Vieles ist, will ich versuchen dir zu erklären, die Begrenzung aber soll auf uns warten.}(Philebos 23e-24a Schleiermacher)}
\newpage
\section*{Literaturverzeichnis}
\printbibliography[keyword={Primärliteratur}, title={Primärliteratur}]
\printbibliography[keyword={Sekundärliteratur}, title={Sekundärliteratur}]
%\printbibliography[keyword={Zeitungsartikel}, title={Zeitungsartikel}]
\end{document}