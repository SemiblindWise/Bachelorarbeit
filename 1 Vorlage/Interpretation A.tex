\section{Interpretation A dieser Rede: Es gibt zwei Welten}
Wie will man von zwei Welten sprechen? Von räumlich, zeitlich, metaphysisch oder von ontologisch getrennten Welten? Es wird zu zeigen sein, wie es sich verhält je nach dem mit welcher Blickrichtung die zwei Welten zu betrachten sind. Außerdem wird sich die Frage nach der Unterscheidung zwischen dem Seinsbereich, dem Ideenbereich und dem Bereich der Idee des Guten zu stellen sein, da hier von möglichen drei Welten gesprochen werden kann. Mit der Idee des Guten ließe sich hier bereits die Annahme machen, dass es sich anstatt um zwei Welten um drei Welten handeln könnte. Es wird sich aber zeigen, dass die Idee des Guten nicht mehr zum selben Seinsbereich gehört und daher nicht mehr logisch eingeholt werden kann, worauf der zweite Tei dieser Arbeit noch weiter eingeht. Hier sei jedoch gesagt, dass sich die Autoren nur auf den Unterschied der beiden Seinsbereiche der Ideen und der Sinnesdinge verstehen und die Idee des Guten nur noch hinzugefügt wird, auch wenn das Verhältnis zwischen allen drei Bereichen von großer Bedeutung ist, um die gesamte \enquote{Ideenlehre} zu begeifen.\\
Wie werden die Welten nun, von denen im Folgenden gesprochen werden soll, dargestellt? Wie wird zum ersten die Sinnenwelt gegenüber der Ideenwelt konzipiert und auseinander dargestellt und wie werden sie dann im weiteren Schritt wieder versucht zusammenzubringen, da dies auch von den Autoren erkannt wird als eine absolute Notwendigkeit.\\
Bei der weiteren Erarbeitung der Platonstelle wird sich an die Chronologie der Texte gehalten. Außerdem ist dabei zu beachten, dass sich die Deutlichkeit der Begriffe und der \enquote{Lehre} erst mit der Politeia verdeutlichen werden und in den frühen Dialogen noch nicht deutlich genug ausgedrückt wird. 
Also: Wie wird die Ideenlehre aus der Perspektive der zwei Welten aufgebaut?\\
Der Einstieg in Platons Ideenlehre findet meist so statt, dass man die beiden Begriffe der Sinnenwelt und Ideenwelt nennt und diese dann gegeneinander und miteinander zu verstehen geben will. Dabei finden sich Formulierungen von verschiedenen Autoren: 
\zitatblock{\enquote{[Die Zwei-Welten-Lehre] unterscheidet zwischen einer raum-zeitlichen Welt des Werdens und einer jenseits von Raum und Zeit befindlichen Welt des Seins. Die Welt des Seins, die allein dem Denken zugänglich ist, repräsentiert Unwandelbarkeit, Idealität und Normativität; die Welt des Werdens, die der Wahrnehmung zugeordnet ist, stellt sich als vergängliche Abbildung der ewigen Strukturen jenseits von Raum und Zeit dar.}\footcite[vgl.][S. 133]{GraeserPhiloGeschichte}} 
Um näher zu klären, was mit der raum-zeitlichen Welt des Werdens und der jenseits von Raum und Zeit befindlichen Welt des Seins gemeint sein soll, muss ein Blick in die Originalstellen geworfen werden. Dabei wird sich eingangs an den Beispielen von Jörg Disse bedient, die von ihm für die Darstellung der Ideenlehre angeführt werden. Begonnen wird also mit Ausschnitten aus dem Phaidon und dem Symposion Dialog um dann mit dem ausführlicheren Teil der Politeia und deren Gleichnissen das Hauptaugenmerk zu legen, um dann mit ergänzenden Stellen aus dem Timaios abzuschließen. 
Im Anhang befinden sich noch Stellen aus dem Parmenides, dem Phaidros und dem 2. Brief. Diese Stellen sind für die weitere Erarbeitung nicht unerheblich, werden aber in dieser Arbeit nicht behandelt werden können.
Wichtig für das Verständnis dieser Stellen ist, dass eine kurze Einordnung gegeben wird, in welchem Kontext die jeweilige Stelle steht, d.h. welche Fragestellung der Text zu beantworten gedenkt, also auf welches Ziel die Argumentation hinausläuft.
Einer Sache muss sich noch zugewandt werden, bevor die Textausschnitte behandelt werden können. Wie Graeser oben angedeutet hat, werden die beiden Welten auf zweierlei Weise unterschieden. Das eine geht über die Beschreibung des Seins und des Werdens, das anderer geht über die Weise der Erkenntnis. Hierzu heißt es folglich, dass es zwei Kriterien der Unterscheidung der Ideen und Dingen gibt. In der Weise ihrer Erfassbarkeit und dass sie in verschiedener Weise sind.\footcite[vgl.][S. 40]{Martin73}
Diese erstliche Unterscheidung wird im Folgenden von den Autoren meist zusammen genannt werden, muss jedoch beachtet werden, da dies für die Deutung der Originalstellen essenziell ist. Hierauf wird zu einem späteren Zeitpunkt dieser Arbeit nochmals zurückgekommen. Die chronologisch erste Stelle, die sich angesehen wird, ist Phaidon 79a.
\subsection{Phaidon}
Die hier genannte Phaidon Stelle ist so eingeleitet, dass es vierlei Dinge wie Menschen, Pferde oder Kleider gibt, die sich im Gegensatz zur Schönheit, die nur den vielen Dingen zugeschrieben wird, nie gleich bleiben, also sich immer in Veränderung befinden. Die Schönheit selbst hingegen bleibt immer gleich und ist unsichtbar und nur durch das Denken erlangbar.
So heißt es dann: \enquote{Sollen wir also, spach er, zwei Arten der Dinge (\emph{dúo eide tôn onton})  \footnote{Kassner: zwei Ordnungen der Wesen\nocite{PhaidonKassner}} setzen, sichtbar die einen und die andere unsichtbar?} (Schleiermacher \nocite{PhaidonSchleiermacher}Phaidon 79a)\\
Gottfried Martin hingegen übersetzt hier mit \enquote{Zwei Weisen des Seins}.
%\footnote{An diesen zwei Arten der Dinge wird sich in der folgenden Interpretaion immer wieder gehalten, da die Phaidon Stelle in der Chronologie an den Anfang der platonsichen Ideenlehre gesetzt wird.}
Diese Übersetzungen und die damit einhergehende Deutung durch die Übersetzung zeigen, dass sich diese verschiedenen Formen, Modi oder Arten des Seins als Schwierigkeit herausstellen, also wie diese Unterscheidung gemeint sein sollte. Hierbei ist es Martin erstlich wichtig, dass er keine ontologische Implikationen in die Übersetzung tragen möchte.\footcite[vgl.][S. 37]{Martin73} Diese Unterscheidung der zwei Weisen des Seins gibt Martin die reine Unterscheidung zwischen den Ideen und den Dingen als eine rein logische. Also eine reine Unterscheidung darin, dass es sich dabei um zwei Verschiedene handelt, aber noch nicht um etwa zwei Entgegengesetzte.
Als Martin dann nochmals die Stelle angeht, allerdings unter ontologischen Gesichtspunkten, heißt es, unter Berücksichtigung eines im Original dort eindeutig verwendeten Plurals, dass die vorsichtige Übersetzung der \enquote{zwei Weisen des Seins} gegeben wird, da gerade die Dinge von den Ideen unterschieden werden sollen.\footcite[vgl.][S. 216]{Martin73} Also auch auf der ontologischen Ebene eine Unterscheidung gefordert wird. So fasst Martin zusammen, dass wie immer man sich in der Übersetzung helfen möge, Platon zwei Weisen des Seins, das Sein der Ideen und das Sein der Dinge - wie später die Zuordnung dieser beiden Bereiche heißen soll - kennt und er sie unterscheidet und zugleich zusammenstellt.\footcite[vgl.][S. 216]{Martin73}
Mit der Formulierung des \enquote{zugleich zusammenstellen} ist gemeint, dass das Sein der Dinge und das Sein der Ideen gleichzeitig gemeint sein soll.
Eine weitere Interpretation von C.J. de Vogel dieser Textstelle zieht hingegen aus der Phaidon Stelle einen eher drastischeren Schluss:
\enquote{If by the term \enquote{two worlds} is meant: two kinds of reality, then the existence of two worlds is explicitly posited in \emph{Phaedo} 79a.}\footcite[][S. 161]{Vogel}
Damit ist eine eindeutige Unterscheidung zweier Bereiche festgehalten, die dem Sein also der ontologischen Ebene zugeschrieben wird und sogar mit dem Begriff zweier Welten beschrieben wird. Wie dieser Unterschied weiter zu verstehen ist, soll sich im weiteren Verlauf zeigen. Festzuhalten sei aber hier, dass sich zwei Arten des Seins finden lassen, die voneinander getrennt betrachtet werden müssen, da hier schon angedeutet ist, dass es sich um zwei unterschiedliche Bereiche von Realität oder Wirklichkeit handelt.
%Zitat aus Phaidon 79d: Die Seele verhält sich zu jenen Gegenständen immer in derselben Weise, da sie eben damit etwas erfasst, das selbst auch von dieser Art ist. Diesen Gegenständen komme es zu, \enquote{niemals in keiner Weise, irgendwie auch nur die geringste Veränderung zu erleiden. (Phaidon 78d)}\footcite[][S. 97]{Hirschberger}
%Hierin liegt der Grund, dass nicht scharf genug differenziert wird und die Definition nicht eingehalten wird. Also die verschiedenen Ebenen oder Stufen der Ideen, die jeweils als eigenständig also absolut unveränderlich und ewig gelten und dann in den Bereich des Denkens eintreten und somit nicht mehr die absolute Veränderlichkeit aufweisen können dadurch aber gedacht werden können und eben in den Objekten anwesend sein können.\footcite[vgl.][S. 180f.]{Kutschera}

\subsection{Symposion (Gastmahl)}
Im für diese Arbeit relevanten Teil des Symposions geht es darum zu erfassen, wie man den Weg zu einem oder dann dem obersten Schönen erreichen kann. Dabei soll mit einem einzigen Körper und dessen Schönheit begonnen werden, dann zur Schönheit jedes einzelnen Körpers weitergegangen werden, um dann über immer allgemeinere Schönheit an Körpern deren Schönheit zu erkennen, um diese schlussendlich auch hinter sich lassen . Hier kommt der Übergang zu der geistigen Schönheit, wobei die körperliche Schönheit als niederer und geringer geschätzt und auch verachten werden soll, um dann die Schönheit in Sitten und Gesetzen zu erkennen und zu schätzen, bis man auf eine einzige Erkenntnis des Schönen selber sich richten soll. (Sym. 210a-e)  
Dieses letzte Schöne wird dann wie folgt beschrieben: \zitatblock{\enquote{Zuförderst ist es ein beständig Seindes, was weder wird noch vergeht und weder zunimmt noch abnimmt, sodann nicht nach der einen Seite betrachtet schön, nach der andern unschön, noch auch bald schön und bald nicht, noch in Vergleich mit dem einen schön, mit dem andern aber hässlich, oder teilweise schön und teilweise hässlich, oder nach der Meinung einiger schön, nach der von andern aber hässlich.} (Symp. 211a)}
Hierin erkennt Disse, was für Platon die eigentliche Wirklichkeit des wahren Seins gegenüber dem unmittelbaren sinnlich wahrnehmbaren Sein ist.\footcite[vgl.][S. 27]{DisseMetaphysik} D.h. dass sich die eigentliche Wirklichkeit erst im ewigen Sein der Ideen und der Idee des Schönen selber finden lässt. Also einer Höherstellung des Seins der Ideen im Gegensatz zu den Sinnesdingen. Somit folgert Disse aus dieser Stelle, dass es \enquote{[f]ür den Menschen [darauf an]kommt, dass er die vergänglich-körperliche Welt verlässt, um sich alleine der ewig-geistigen zuzuwenden.}\footcite[vgl.][S. 28]{DisseMetaphysik} Dieses Verlassen allerdings darf nur so verstanden werden, dass damit eine Abwendung von den Sinnesdingen und eine Zuwendung zu den Ideen, bzw. dann zur Schönheit selbst, gemeint sein soll. 
Der Aufstieg im Symposion ist so zu verstehen, dass es sich um ein ewiges Sein gegenüber dem vergänglichen also um ein geisitges (Sein) gegenüber dem körperlichen (Sein) handelt.\footcite[vgl.][S. 27f.]{DisseMetaphysik}
Damit lässt sich wieder ein Unterschied der beiden Seinsbereiche finden, wo jedoch die Kennzeichnung eines höheren und niederen Seinsbereiches angeführt wird, in der Formulierung des \enquote{eigentlichen Wirklichkeitsbereichs} und des \enquote{wahren Seins}, welches dem Bereich der Ideen vorbehalten ist, diese stehen sich als \enquote{Sinnenwelt und Reich des Denkbaren [gegenüberstehen].}\footcite[vgl.][S. 28]{DisseMetaphysik} Welche Probleme diese Darstellung allerdings aufweist wird sich erst in Kapitel 3.6 zeigen.
\subsection{Politeia}
Den größten Anteil hat nun die Politeia an der \enquote{Ideenlehre}, in der sich die drei großen und bekannten Gleichnisse finden lassen, die die Einordnung der beiden Seinsbereiche ein weiteres Mal ausführen.
Der Eingang in die Stellen aus der Politeia über das Sonnengleichnis, das Liniengleichnis und das Höhlengleichnis werden über die Frage \enquote{was das Gute selbst eigentlich ist} (Pol. 507a) gestellt, wobei von dem Guten selbst Abstand genommen wird und nur auf den Sprössling des Guten geblickt wird, da der sprechende Sokrates sich nur dazu in der Lage sieht. Somit wird rückwirkend Bezug genommen, dass es eine Vielheit von Schönem und Gutem gibt, genauso, dass es ein Schönes an sich und ein Gutes an sich gibt. (Pol. 507b-c) Hierbei wird allerdings nur der reine Begriff, das Wort, gemeint, also dass viele Dinge mit einem Begriff wie etwa schön usw. beschrieben werden können und noch keine ontologische Bedeutung des Schönen/Guten selbst gemeint sein soll. 
Wiederholt wird hier, dass die Vielheit sichtbar und nicht denkbar und die Begriffe nur denkbar und nicht sichtbar sind.\\
Die Dinge, die man durch die Augen und allen anderen Sinneswahrnehmungen erkennen kann, sind eindeutig in diesem Falle sichtbar. Die Beschreibung über das Denkbare ist so gemeint, dass die Begriffe einzig und alleine denkbar sind, also dass man die Schönheit oder das Gute, welche sich in - oder an - vielen Dingen befinden kann, nicht selbst sehen kann, oder eben z.B. \enquote{das Wesen der Länge} selbst nicht sichtbar ist, sondern nur die sich zeigende Länge in einem Ding, welches eine Länge aufweist, welche dann auch messbar ist, erkennen kann.
Nun besteht allerdings die Frage, wie dieser Unterschied erkannt werden kann, also wie der sog. \enquote{Gesichtsinn} - gemeint sind in erster Linie die Augen, wobei auch alle anderen Sinnesorgane darunter fallen - etwas erkennen kann.
Dabei wird im Sonnengleichnis eine Dreiheit zwischen Erkennendem (Subjekt), Erkanntem (Objekt) und dem Prozess des Erkennens, bzw. der Verbindung zwischen Subjekt und Objekt, konzipiert. Also dem Auge als Erkennendem, dem Ding als das zu Erkennende und über das Licht als das, was das Erkennen erst möglich macht, da man ohne dieses nur in der Dunkelheit wäre, also man durch das Weglassen des Lichts nicht in der Lage wäre zu sehen, welches offensichtlich durch die Sonne verkörpert wird. So heißt es:
\zitatblock{\enquote{Unter dieser Sonne also [\dots] denke dir, verstehe ich den Sprössling des Guten, der von dem eigentlichen wesenhaften Guten als ein ihm entsprechendes Ebenbild hervorgebracht worden ist, so dass das Gute im Denkbaren zum Denken und zum Gedachten sich verhält, wie die Sonne in der sinnlich sichtbaren Welt zum Gesicht und zum Gesehenen}(vgl. Pol. 508b-c)} 
Damit wurde also bereits darüber die Zweiheit aufgemacht, dass zuerst der Bereich des Sichtbaren über die Sonne vorgebracht wurde, um es dann auf den Bereich des Denkens zu übertragen.  
Auf die Rolle des wesenhaften Guten oder auch Idee des Guten wird später noch näher eingegangen. 
Dies wird dann noch weiter erläutert:
\zitatblock{\enquote{Dasselbe Verhältnis denke dir nun auch so in Bezug auf die Seele (psyche): Wenn sie darauf ihren Blick heftet, was das wahre und wesenhafte Sein bescheint, so vernimmt und erkennt sie es gründlich und scheint Verunft zu haben. [\dots] Was den Dingen, die erkannt werden, Wahrheit verleiht und dem Erkennenden das Vermögen des Erkennens gibt, das begreife also als die Wesenheit des eigentlichen Guten} (vgl. Pol. 508d-e)}
Bis hier war die hauptsächliche Rolle des Guten die Möglichkeit der Erkenntnis zu liefern, also \emph{dass} Erkenntnis überhaupt möglich ist, das jedoch auf den Erkennenden und auf das zu Erkennende angewandt worden ist. Jetzt kommt jedoch noch eine weitere wichtige Rolle dem Guten zu:
\zitatblock{\enquote{Und so räume denn auch nun ein, dass den durch die Vernunft erkennbaren Dingen von dem Guten nicht nur das Erkanntwerden zuteilwird, sondern dass ihnen dazu noch von jenem das Sein und die Wirklichkeit zukommt, ohne dass das höchste Gute Wirklichkeit ist, es ragt vielmehr über die Wirklichkeit an Würde und Kraft hinaus}(Pol. 509b-c)}
Somit kommt aus der Idee des Guten nicht nur die Möglichkeit Dinge - damit ist beides gemeint Sinnesdinge und Ideen - zu erkennen, sondern auch noch, dass sie das Sein der Ideen und Sinnesdinge bestimmt. Dabei ist sie selber aber nicht einsehbar und auch nicht seiend, da sie noch über die Wirklichkeit hinaus gesetzt ist. 
%Es lässt sich soweit festhalten, dass die eine Seite die Welt der sich in ständiger Veränderung befindlichen Dinge und die andere Seite die Welt der ewigen und immerwährenden Ideen bestimmt ist, da diese als grundlegend unterschiedlich dargelegt worden sind.
Es wird also der Bereich des Sichtbaren, welcher durch das Gesicht, die Gegenstände und die Sonne als Lichtquelle ausgezeichnet wird, von dem Bereich des Denkbaren, welches dementsprechend analog zum Gesicht das Denken und den Gegenständen das Gedachte dargestellt wird, unterschieden. 
Diese Ausführung über die Unterscheidung der Erkenntnis wird im Liniengleichnis fortgesetzt (Pol. 509e-511e).
Es wird eine gedachte Linie aufgestellt, welche zuerst ungleich geteilt wird. Diese entstandenen zwei Bereichen entsprechen den  eben dargelegten Bereiche der Dinge (sichtbaren Gegenständen) und der Ideen (Denkbare). Jetzt wird allerdings nocheinmal jeder Bereich nach dem ersten Trennverhältnis unterteilt, sodass eine Linie mit 4 Bereichen entsteht, welche das Verhältnis 4:2:2:1 aufweist. Mit dem untersten Bereich angefangen, wird der kleinste Bereich mit Schatten und Spiegelungen der Sinnenwelt gefüllt. Die Linie fortschreitend wird der nächste Bereich mit den Dingen selber befüllt, also die gesamte Tier- und Pflanzenwelt und aller weiterer Körper. Der andere Abschnitt, der durch die Vernunft erkannt wird, wird auch unterteilt nach dem Verhältnis der ersten Teilung der Linie. Hierbei werden dem ersten Abschnitt, welcher den Dingen noch am nähsten steht der Bereich geometrischer und arithmetischer Begriffe zugeteilt. Jedoch gebraucht man diese Begriffe noch unter Anwendung auf die Dinge aus der Dingenebene, wodurch dieser Bereich noch nicht als \enquote{rein} bezeichnet wird. Die letzte Hälfte allerdings geht noch einen Schritt weiter, dahingegehend, dass sie mithilfe der Dialektik sich von den Sinnesdingen vollständig löst und die Prinzipien behandelt, die auf keiner Voraussetzung mehr beruhen. Der Unterschied dieser letzten beiden Bereiche wird so erklärt, dass auf dem ersten hier genannten Bereich es immer noch Voraussetzungen gibt, zwar mit dem Verstand gearbeitet wird, also ein rechnendes oder mechanisches Denken gemeint ist, aber immer noch einen Bezug zur Sinnenwelt besteht. Dies ändert sich auf der letzten Ebene, da es hier um eine Vernunfterkenntnis geht, die voraussetzungsfrei ist, bzw. dorthin gelangen möchte mithilfe der Dialektik, die sich der reinen Begriffe bedient. 
Daher wird an diesem Punkte auch die Unterscheidung getroffen, dass nur auf der Ebene der Vernunfterkenntnis, der höchsten Stufe, wahre Erkenntnis gewonnen werden kann, da sie eben voraussetzungslos ist. Hierin wird außerdem das 4:2:2:1 Verhältnis eingeholt, dahingehend, dass derjenigen Bereichen, welche den Anteil an der Wahrheit entsprechen sollen, auch den größeren Anteil haben sollen.
Das Wichtige dabei ist nun, dass die Linie nicht nur den Gegenstandsbereich inne hat, sondern auf den gleichen Stufen entsprechend auch noch die gnoseologischen Gegenstücke hinzugenommen werden. Das heißt, dass den Schatten das Meinen und Raten (eikasia) entspricht, den Dingen das für wahr Erscheinen (pistis). Den geometrischen Dingen kommt dann das, wie schon angedeutet, diskursive/rechnende Denken (dianoia) und den Ideen die Vernunfterkenntnis (noesis) zu. Dabei wird eine graduelle Steigerung des \enquote{Wahrheitswertes} ausgedrückt, wodurch man nur durch die Ideen und in den Ideen als Gegenstand der Erkenntnis unter Vernunfterkenntnis die voraussetzungsfreie und unumstößliche Wahrheit erkennt. 
Die Aufteilung im Liniengleichnis muss so aufgefasst werden, dass der größte Teil dem untersten Teil zugesprochen werden muss. Dies macht daher Sinn, da sich das Liniengleichnis auf einen Zielpunkt hin konstruiert, die Idee des Guten. Bei einer Betrachtung von oben herab also muss gelten, dass sich von dem obersten Punkt aus, also der Spitze her - wie eine Art Definitionsbaum - immer mehr Varianten darunter fallen (Man siehe hierzu auch die Dihairese im Sophistes), dieses Prinzip auch für den weiteren Weg \enquote{nach unten} gelten muss, sodass am Ende eine sehr viel breitere Basis besteht und der große Teil des Liniengleichnisses auf der untersten Ebene zu verorten ist. Hiermit entsteht also eine pyramidenähnliche Form. Pyramidenähnlich daher, weil der mittlere Teil nicht ganz einer perfekten Pyramide entsprechen würde, da die zweite und dritte Stufe im Liniengleichnis die gleiche Fläche zukommen müsste. Dies wird weiterhin dadurch unterstützt, dass im Höhlengleichnis am Grunde der Höhle die Schatten an der Höhlenwand jeweils anders interpretiert werden können und damit eine unzählbare Vielheit an Möglichkeiten besteht. Außerdem wird durch jedes Zucken und jede Veränderung des Feuers, welches das Licht auf die Gegenstände wirft, die wiederum den Schatten auf die Höhlenwand werfen, umso zahlreicher. Dabei ist außerdem noch nicht beachtet, dass das Wenden, Drehen und Zusammensetzten der Gegenstände von denjenigen, welche die Gegenstände hinter der Mauer her tragen, nochmals die Zahl der Möglichkeiten anhebt.\\ Diese Ansichtsweise findet ebenfalls Halt, wenn man sich den pythagoreischen Tetraktys ansieht und die im platonischen Denken stark vertretene Dialektik des Einen und Vielen, welche vom Einen beginnt und in das Viele mündet.
Eine schlussendliche Wegbeschreibung findet sich in dem hieran anschließenden und prominentesten Gleichnis, dem Höhlengleichnis (Pol. 514-516).
Hier wird in Bezug auf Bildung und Unbildung das Bildniss einer Höhle gezeichnet, in der am untersten Ende Menschen angekettet vor einer Wand sitzen und gezwungen sind auf eine Wand zu Blicken, auf der Schatten vorbeiziehen. Diese Schatten stammen von Gegenständen, die hinter den Menschen - noch hinter einer dazwischen liegenden Wand - vorbeigetragen werden und von einem Feuer so beschienen werden, dass sie eben ihre Schatten auf die Höhlenwand werfen. Nun wird allerdings einer dieser Gefesselten befreit und gewaltsam zum Licht gedreht und zum Feuer geführt. So wird beschrieben, dass er damit eine erste, stärkere Wahrheit erblicken würde, wenn er die Gegenstände und das Feuer sieht und sich an deren Anblick gewöhnt. Hierdurch würde er erkennen, dass er nur auf Schatten geschaut hat und nicht auf die \enquote{wahren} Dinge. Das Gleichnis wird weitergeführt, indem der Gefesselte dann aus der Höhle unter äußerster Mühsal den , geführt - oder gezerrt - wird. Wenn er dann aus der Höhle herausgeführt worden ist, so würde er gar nichts erkennen können, da er von dem Licht der Sonne vollkommen geblendet ist. Mit der Zeit, die er dann außerhalb der Höhle verbringen würde, würde er in der Nacht beginennd etwas erkennen um dann am Tage zuerst die Spiegelungen und Schatten erkennen, bis er dann die Gegenstände erblicken kann und am Ende eben die Sonne selber, die als Urheberin für alles verantwortlich ist. \footnote{Es schließt sich noch die Rückkehr des Gefangenen an die Ausgangsposition an das Gleichnis an. Dies wird hier aber nicht weiter zu thematisieren sein.} Es schließt das Gleichnis mit folgenden Worten:
\zitatblock{\enquote{Die mittels des Gesichts sich uns offenbarende Welt vergleiche einerseits mit der Wohnung im unterirdischen Gefängnisse, und das Licht des Feuers in ihr mit dem Vermögen der Sonne, das Hinaufsteigen und das Beschauen der Gegenstände über der Erde andererseits stelle dir als den Aufschwung der Seele in das Gebiet des nur durch die Vernunft Erkennbaren vor. [\dots] im Bereiche der Vernunfterkenntnis sei der Bereich des Guten nur zu allerletzt mühsam wahrzunehmen, und nach seiner Anschauung müsse man zur Einsicht kommen, dass er für alle Dinge die Ursache von allem Richtigen und Schönen sei, indem er der sichtbaren Welt das Licht und die Sonne erzeugt, sodann auch im Bereich des durch die Vernunft Erkennbaren selbst als Herrscher waltend sowohl die Wahrheit als auch uns Vernunfteinsicht gewährt, ferner zur Einsicht kommen, dass das Wesen des Guten ein jeder erkannt haben müsse, der verständig handeln will [\dots].} (Pol. 517b-d)}
Damit zieht Disse den Schluss, dass die Sinnenwelt und das Reich des Denkbaren sich im Höhlengleichnis wie im Symposion gegenüberstehen.\footcite[vgl.][S. 28]{DisseMetaphysik}
Dies alles zusammengenommen kommt Disse zu dem Schluss, dass \enquote{Platon somit im Höhlengleichnis zunächst einmal grundsätzlich zwischen zwei Welten [unterscheidet] und eine Bewegung des Menschen, nämlich die Tätigkeit der Philosophie, die ihn von der ersten in die zweite, eigentliche führen soll.}\footcite[vgl.][S. 23f.]{DisseMetaphysik}
Abstufung einer höchsten Idee zu den anderen Ideen.
\enquote{Das Gebiet des Tageslichts außerhalb der Höhle ist der Bereich dessen, was uns durch reines Denken zugänglich ist. Die Sonne aber wird mit dem höchsten Punkt im Bereich des Denkbaren verglichen.}\footcite[][S. 49]{DisseMetaphysik}

\subsection{Timaios}
In dem letzten Dialog, der hier angeführt werden soll, wird die Weltenstehung behandelt, wo die immerwährenden Ideen und die sich stets verändernden Sinnesdingen nochmals aufgegriffen werden. 
\zitatblock{\enquote{Was ist das stets Seiende und kein Entstehen Habende und was das stets Werdende, aber nimmerdar Seiende; das eine ist durch verstandesmäßiges Denken zu erfassen, ist stets sich selbst gleich, das andere dagegen ist durch \emph{bloßes} mit vernunftloser Sinneswahrnehmung verbundenes Meinen zu vermuten, ist werdend und vergehend, nie aber wirklich seiend.}(Tim. 27d-28a)} Graeser hingegen deutet hier die Dinge anstatt des \enquote{nie aber wirklich seiend} als \enquote{niemals Seiendes}.\footcite[vgl.][S. 140]{GraeserPhiloGeschichte} Da er hiermit die Dinge als \enquote{nicht-seiend} beschreibt, werden die Dinge auf einer eindeutigen Weise auf einer ontologischen Ebene von den Ideen unterschieden und abgetrennt.
Als Zusammenfassung für das Bisherige wird von Disse dann Tim. 52a herangezogen
\footnote{Schleiermacher Übersetzung: \enquote{[\dots] das eine sei die Form, die sich stets in sich andere von anderswoher aufnehmend noch selbst in anderes irgendwohin gehend, unsichtbar und auch sonst unwahrnehmbar, das, was der Vernunft zu betrachten zuteil wurde. Ein zweites aber sei das ihm Gleichnamige und Ähnliche, wahrnehmbar und geboren, stets hin- und hergerissen, an einer bestimmten Stelle entstehend und von da wieder verschwindend, durch Meinung in Verbindung mit sinnlicher Wahrnehmung erfassbar.} (vgl. Tim. 52a)} 
wo es heißt:
\zitatblock{\enquote{[\dots] das Gebiet der unwandelbaren Ideen, die ungeworden und unzerstörbar sind [\dots], dem Auge verborgen und auch den anderen Sinnen nicht wahrnehmbar, [sie sind] genau also das, dessen Betrachtung Sache des reinen Denkens ist. Das Zweite (Gebiet) ist das, was mit jenem gleichbenannt und ihm ähnlich ist, sinnlich wahrnehmbar, erzeugt, in immer währender Bewegung, an einem bestimmten Orte entstehend und von da wieder verschwindend, [\dots] mit der Sinneswahrnehmung erfassbar.}\footcite[][S. 30]{DisseMetaphysik}}
Auch hier wird wieder das Augenmerk darauf gelegt, dass den Ideen und den Dingen zwei \enquote{Gebiete} entsprechen, jedoch unter dem Zusatz der Erkenntnismöglichkeit über das Denken und die Sinneswahrnehmung. Auch wenn mit dem Begriff des Gebiets nur eine implizite räumliche Trennung zu zeigen ist, da auch eine Deutung in Form von \enquote{Bereich} gemeint sein könnte, ist nicht abzuweisen, dass die Trennung hiermit final gegeben ist.


% Das Vielgestaltige und Veränderliche, das sich bald so und bald anders zeigt, erfassen wir im sinnlichen Sehen; das Einheitliche und Unveränderliche, das immer dasselbe Wassein sehen läßt, zeigt sich nur im die Sinneswahrnehmung transzendierenden reinen Denken (507 B 9-10). Entsprechend unterscheidet Platon eine Welt des Intelligiblen (vonios topos, 508 C 1, 517 B 5, vgl. C 3) und eine Welt des Sinnenfälligen oder Sichtbaren (opatos Toлos, vgl. 508 C 2, 517 C 3) als zwei Arten des Seienden (do con tan övtav, Phaid. 79 A 5; Politeia 509 D 1-3: dvo αὐτὰ εἶναι . . . τὸ μὲν νοητοῦ γένος τε καὶ τόπος, τὸ δ' αὖ ὁρατοῦ). Dic sichtbare Welt aber ist das Abbild der intelligiblen Welt (vgl. Tim. 29 A- B), sie hat nur in der Teilhabe an dieser Sein, darum entsprechen sich die Strukturen beider Welten in strenger Analogie.\footcite[][S. 246]{halfwassenaufstieg2006}

\subsection{Zusammenführung der bisher auseinander gehaltenen Welten}

Es wurde bis hier das Verständnis gezeichnet, dass die Ideenlehre zwei Welten darstellt, die augenscheinlich voneinander getrennt sind, was sich eindeuti u.A. bei Graeser S. 134, 139, Vogel S. 161-165 und Disse S. 34 finden lies oder wie es auch bei Burgin heißt: \zitatblock{\enquote{In his theory, Plato assumed that the physical world was the sensible realm, as people could grasp it with their five senses, while the world of Ideas was the intelligible realm, as people could comprehend it only with their intellect.}\footcite[][S. 179]{Burgin}}
Es bedarf jetzt unter Zuhilfenahme des Bisherigen nochmal auf diese beiden Welten und wie diese zusammengebracht werden können zu blicken, da die Verbidung, bzw. die Bezugnahme und die Transzendenz der Bereiche noch weiter ausgeführt werden muss.
Hierbei liefert Hischberger eine Lösungsmöglichkeit:
\zitatblock{\enquote{Die Transzendenz der Idee ist keine totale, sondern nur eine modale. Der erkenntnis-theoretische Sinn dieser Begriffe besagt, dass alles Erkennen in der erfahrbaren, raumzeitlichen Welt ein \enquote{Analogismus}, ein Lesen der Sinneswahrnehmung durch Hinbeziehen auf einen urbildlichen Begriff ist}\footcite[][S. 94]{Hirschberger}}
\enquote{Nur ein mangelder Metaphysik- und Transzendenzbegriff [\dots] führt zu der Zweiweltentheorie eines totalen Chorismos, wo in Wirklichkeit nur ein modaler gemeint war, eine \enquote{Trennung} des Seins nach seinem Wesen in Gegründetes und Gründendes. Es ist eine Modifizierung, der es ebensosehr auf die Trennung wie auf die Einheit ankam}\footcite[][S. 96]{Hirschberger}\\
Es muss sich also zeigen, inwiefern die Sinnesdinge als Begründetes und die Ideen als Gründendes zu verstehen sind, was dann auch zu den Fromulierung über die höhere Seinsstufen führen wird.
%Aus den Symposion und Timaios Stellen, sowie dem Höhlengleichnis wird die notwendige Bewegung weg von den Sinnesdingen hinauf zu den Ideen beschrieben. Aus der Beschreibung des Höhlengleichnisses geht hervor, dass die Menschen am untersten Ende der Höhle gefesselt sind. Dies ist also die zwangsläufige Ausgangsposition, in der sich jeder befindet. Von dieser Position aus gilt es voranzuschreiten, also zu den Ideen zu gelangen und diese zu erkennen. Damit ist der geradlienige Weg vorgegeben, sodass eine notwendige Verbindung der zwei Bereiche angenommen werden muss, auch wenn es dann ein Übergang in den geistigen Bereich sein wird. 
%Disse formuliert es so:  
%\zitatblock{\enquote{Als Grundstruktur kann festgehalten werden, dass es Platon um den Gegensatz zweier Welten geht und um den Aufstieg des Menschen von der einen in die andere mittels der Philosophie, d.h. [\dots] wesentlich durch die Tätigkeit des von aller Sinneswahrnehmung getrennten Denkens.}\footcite[vgl.][S. 29]{DisseMetaphysik}}
%Dieser Aufstieg allerdings impliziert automatisch, dass es eine Verbindung zwische beiden Welten geben muss, wenn die Philosophie als der Wegbereiter zwischen den Welten gelten soll.
%Nun ist es jedoch so, dass dieser Weg eine andere geistige Richtung aufweist, als es die Beziehung zwischen den Ideen und den Sinnesdingen abbilden.
%Dieser richtet sich nämlich von oben nach unten. Oben ist hier die Idee des Guten selbst (Pol. 509 Die Idee des Guten überragt noch alles andere), die die Ideen bestimmt, welche wiederum in den Sinnesdingen stecken. So heißt es von Graeser: 
Gründendes und Begründetes können sich in zwei verschiedenen Weisen zueinander verhalten. Zum einen können sie eine erkenntnistheoretische Verbindung aufweisen. Zum anderen eine ontologische Verbindung. Bei der ontologischen Verbindung ist gemeint, dass es sich um Eigenschaften des seins handelt, welche von den Ideen für sich existieren, für die Sinnesdinge jedoch nur in einer Form des Abbildes ausführen. Somit heißt es von Graeser:
\zitatblock{\enquote{Was Plato hier im Rahmen der naturphilosophischen Hypothese bezüglich der Existenz zweier Arten von Dingen mit dem Hinweis auf das Merkmal der Eingestaltigkeit zum Ausdruck bringen wollte, ist, dass die Ideen, anders als die raum-zeitlichen Dinge, nur für sich existieren und genau das sind, was die raum-zeitlichen Dinge nur in Form von Eigenschaften aufweisen.}\footcite[][S. 140]{GraeserPhiloGeschichte}} 
Für die erkenntnistheoretische Verbindung ist es mit den Worten von Disse wie folgt zu verstehen:
\zitatblock{\enquote{Somit ist bei Platon nicht die Sinneswahrnehmung als die Grundlage für unsere Erkenntnisinhalte, sondern die durch das reine Denken zugänglichen Erkenntnisinhalte, die Ideen, sind die Voraussetzung für unser Erkennen durch die Sinne. Auch wenn die Sinneswahrnehmung die Wiedererinnerung bewirkt, das Wiedererinnerte selbst, die Idee der Gleichheit, ist die Voraussetzung dafür, dass man zwei Gegenstände als gleich erkennt. Damit aber erweisen sich die Ideen als völlig unabhängig von der Sinnenwelt.}\footcite[][S. 34]{DisseMetaphysik}}
Der Grund hierfür ist, wie schon gesehen, dass die Sinneswahrnehmung, da sie eben den Gegenstand der ständigen Veränderung hat, keine gesicherte Wahrheit \enquote{produzieren} kann. Daher werden also für die sichere Erkenntnis die Ideen herangezogen. 
Diese Verhältnismäßigkeit findet sich auch im Liniengleichnis. Hier werden von beiden Autoren die jeweils eine Seite des Liniengleichnisses bedient: Die der Gegenstände und die der Erkenntnisinhalte. Hinzu kommt hier aber, dass es sich bei den Ideen um einen höheren Seinsgrad handelt, als den der Sinnesdinge, da nur dadurch die Konzeption des begründens den Ideen zuteil wird und diese dadurch auszeichnet.
%Hierbei ist scharf anzumerken, dass sich dasselbe Argument auf unterschiedliche Aspekte des Verhältnisses von Ideen und Sinnesdingen beziehen. Von Graeser wird die ontologische, von Disse hingegen die gnoseologische Ebene bedient. Diese nahe Verknüpfung wird von Platon im Liniengleichnis eindeutig dargestellt. Auf der einen Seite die Gegenstände, auf der anderen Seite dann die Erkenntnisweise.\\
%Diese vollkommene Loslösung von der veränderlichen Welt ist nur mit dem Tod möglich (Dafür muss eine weiter Stelle gegeben werden, aus der das Hervorgeht). Denn wie sollte der Philosoph auf seinem Weg nach oben die ursprüngliche veränderliche Welt verlassen? Dies ist gar unmöglich. Daher ist die Formulierung: \enquote{Sinnenwelt und Reich des Denkbaren stehen sich wie im \enquote{Gastmahl} gegenüber}\footcite[][S. 28]{DisseMetaphysik} nicht zulässig, sofern dies nicht epistemologisch verstanden wird, was hier nicht klargestellt wird.
%Die Frage ist also, wie sehr kann davon ausgegangen werden, dass allein schon mit der Verbindung aus Körper und Seele die Körperlichkeit als etwas so Niederes beschrieben wird, es aber erst durch den Tod die wirkliche Auflösung vonstatten gehen kann? Dies wäre so, als würde man die eine Hälfte des Seins, der man unmittelbar anhängt so absprechen und verneinen, ohne die aber der erste Schritt gar nicht möglich wäre, da dies das erste ist, womit man konfrontiert ist und zu dem man einen \enquote{einfachen} Zugang besitzt, von dem man aus erst weitermachen kann. Wozu hätte Platon seine Beschreibungen dann immer mit dem Körperlichen begonnen? Die Irreführung oder den Punkt der wahren Erkenntnis damit nicht zu vernachlässigen.
%\zitatblock{\enquote{Grundlage von Platons Metaphysik ist die so genannte Zweiweltenlehre, die Lehre des Gegensatzes zwischen Sinnen- und Ideenwelt, wobei die Ideenwelt als die höhere, weil ewige und geistige Wirklichkeit die der Sinnenwelt zugrunde liegende, sie fundierende ist}\footcite[][S. 29]{DisseMetaphysik}} %Zuerst wird der Unterschied auf den epistemologischen Aspekt berufen, allerdings direkt danach als Unterschied auf einer höheren/anderen Wirklichkeit beschrieben. Ob es einen wirklichen Gegensatz darstellt muss noch aufgezeigt werden.
Die Sinnesdinge werden dann im Gegensatz zu den Ideen als eine \enquote{Exemplifikation von Ideen} dargestellt und verfügen somit auch nicht über reines Sein, was Selbstständigkeit ausmacht.\footcite[vgl.][S. 146]{GraeserPhiloGeschichte}
Diese Exemplifikation der Ideen wird von Vogel ausgeweitet, dahingehend dass die Ideenwelt nicht \enquote{alleinig} aus sich heraus bestehen und gedacht werden kann,
%\enquote{physical being is a kind of reality, but a kind of reality which can neither exist by itself nor be known or explained from itself.}
\footcite[vgl.][S. 162]{Vogel} um dann weiterführend eine Unterschiedlichkeit in den Seinsstufen festzustellen:
\zitatblock{\enquote{\enquote{Partial being}- so it appears to be in that remarkable passage of Rep. V where Plato's Sokrates argues that the object of that lower form of cognition which is not concerned with the full \enquote{being} of \enquote{things themselves} but is a \enquote{view} of thing presenting themselves to our senses, none the less is always \emph{something}. Now \enquote{something} could not be percieved if it were not existing, - just \enquote{non-being}. It \emph{must} have a share of \enquote{being} in it, though it is not \enquote{being} in the full and total sense. It must be a mixture of \enquote{being} and \enquote{non-being}, a middle thing [\dots].}\footcite[vgl.][S. 163]{Vogel}}
Es wird hier also eine Verbidung darüber geschaffen, dass die Ideen als wahres Sein bestehen und die Sinnesdinge als \emph{etwas}, was zwar auch nicht \emph{nicht seiendes} ist, aber auch nicht das vollwertige Sein aufweist. Daher wird es beschrieben als teilweise seinend und teilweise nicht-seiend, also als etwas zwischen diesen beiden Polen. Selbiges Argument findet sich dann weiter auch auf der erkenntnistheoretischen Ebene: 
\zitatblock{\enquote{[\dots] he (Plato) argues that the lower form of cognition which is not concerned with perfect being but with imperfect things, nevertheless has one definite object: doxa, so he says, is always \enquote{having a view of \emph{something}}. It is just impossible to have a view of nothing. \emph{Ergo} \enquote{being} cannot be denied to that which is the object of \enquote{a view}. Yet it is not full being: it must be \emph{both being and non-being} [\dots].}\footcite[vgl.][S. 165]{Vogel}}
Der Punkt, den Vogel hier machen will, ist, dass die beiden Arten des Seins keinen gleichwertigen oder entgegengesetzte Pole sind, sondern eine ontologisches Schema der Unterordnung darstellen.\footcite[vgl.][S. 165]{Vogel}
Dieser Ansicht folgt auch Disse dahingehend, dass etwas mehr oder weniger Sein haben kann. Dabei wird das eigentliche Sein im Höhlengleichnis außerhalb der Höhle gefunden, im Symposion das Schöne an sich und im Phaidon die Gegenstände des reinen Denkens.\footcite[vgl.][S. 37]{DisseMetaphysik}\\ Dabei haben die Sinnesdinge nur an dem vollkommenen Sein der Ideen teil und sind daher von einer niedrigeren Seinsstufe, aufgrund dessen, dass sie unter ständiger Veränderung begriffen werden und damit auch keine wahre Erkenntnis der Schönheit oder des Guten in sich haben können.
Auch von Hirschberger wird aus Phaidon 75b festgehalten, dass die Sinnenwelt danach trachtet das Reich der Ideen abzubilden, aber immer nur annäherungsweise an diese herankommt und nicht zum reinen Wert und Wesen selbst gelangt.\footcite[vgl.][S. 100]{Hirschberger} So hießt es dann:
%Gänzlich unbrauchbar wird diese Ansicht, als Hirschberger es wie folgt formuliert:
\zitatblock{\enquote{Auch wegen dieser unausschöpfbar reichen, zeugenden Fruchtbarkeit ist die Ideenwelt die stärkere Wirklchkeit. Darum unterscheidet also Platon die Ideenwelt [\dots] von der sichtbaren Welt [\dots] und erblickt nur in jener die wahre und eigentliche Welt, in dieser aber bloß ein Abbild, das in der Mitte steht zwischen Sein und Nichtsein.}\footcite[vgl.][S. 100]{Hirschberger}}
Auf die Formulierung des \enquote{in der Mitte zwischen Sein und Nichtsein} wird zu einem späteren Punkt eingegangen.
Aus dieser Darstellung wird der Aufstieg hin zu den Ideen nochmals aufgegriffen und dahingegehend ausgelegt, dass es in der Grundstruktur darum geht, \enquote{dass es Platon um den Gegensatz zweier Welten geht und um den Aufstieg des Menschen von der einen in die andere mittels der Philosophie, d.h. [\dots] wesentlich durch die Tätigkeit des von aller Sinneswahrnehmung getrennten Denkens.}\footcite[vgl.][S. 29]{DisseMetaphysik}
Aus den Symposion und Timaios Stellen, sowie dem Höhlengleichnis wird die notwendige Bewegung weg von den Sinnesdingen hinauf zu den Ideen beschrieben. Aus der Beschreibung des Höhlengleichnisses geht hervor, dass die Menschen am untersten Ende der Höhle gefesselt sind. Dies ist also die zwangsläufige Ausgangsposition, in der sich jeder befindet. Von dieser Position aus gilt es voranzuschreiten, also zu den Ideen zu gelangen und diese zu erkennen. Damit ist der geradlienige Weg vorgegeben, sodass eine notwendige Verbindung der zwei Bereiche angenommen werden muss, auch wenn es dann ein Übergang in den geistigen Bereich sein wird. 
%Dieser Aufstieg allerdings impliziert automatisch, dass es eine Verbindung zwische beiden Welten geben muss, wenn die Philosophie als der Wegbereiter zwischen den Welten gelten soll.
Anzumerken ist, dass dieser Weg eine andere geistige Richtung aufweist, als es die Beziehung zwischen den Ideen und den Sinnesdingen abbilden.
Dieser richtet sich nämlich von oben nach unten. Oben ist hier die Idee des Guten selbst (Pol. 509 Die Idee des Guten überragt noch alles andere), die die Ideen bestimmt, welche wiederum in den Sinnesdingen stecken.
%Es wird jedoch versucht zu unterscheiden zwischen der ersten lexikalischen Bedeutung der Trennung (chorismos) als eine räumliche Trennung, wobei Platon den Ideen in Tim 52a-b räumliche Ausdehnung abspricht. Allerdings wird daraufhin lediglich festgestellt, dass Platon keine weiter Erklärung diesbezüglich liefert, da er auch nicht mit den Begriffen der Transzendenz und Immanenz hantiert.\footcite[vgl.][S. 34f]{DisseMetaphysik} Man muss aber zu Gute halten, dass das Problem erkannt wird, wenn man von der Transzendenz der Ideen, im Gegensatz zur raumzeitlichen Welt, spricht.\footcite[vgl.][S. 35]{DisseMetaphysik} Also dass nicht mehr über etwas gesprochen werden kann, wenn man diese Sache außerhalb des logischen Raumes verortet, wenn man ihr die absolute Transzendenz zuspricht.\\
%Es wird von ihm das Verständnis gezeichnet, dass die Ideen und die Welt der Ideen eine \enquote{Ideale Wirklichkeit} darstellen und auch nach dem möglichen Vergehen der materiellen Welt immer noch existieren mögen, da man feststellen kann, dass die gelieferten Beispiele von mathematischen Ideen die Frage nach dem Anfang ihrer Existenz nicht beantworten können.\footcite[vgl.][S. 99]{Hirschberger} Keine Verweise auf das Original. Dabei auch schwierig ist folgende Formulierung:\enquote{Ferner bilden sie die obersten Strukturpläne der Welt, ohne ihrerseits davon abhängig zu sein. Sie sind das Sein des Seienden.}\footcite[][S. 99]{Hirschberger} Diese Zuschreibung ist eigentlich nur der Idee des Guten vorbehalten. Die Frage ist, wenn man dieser Formulierung kurz nachgehen würde, wären nur die Sinnesdinge Seiend, die Ideen hingegen nicht.

%Eine interessante Sache, die von Hirschberger hervorgebracht wird, ist die Unterscheidung von zwei Möglichkeiten der Diairesis entweder von oben nach unten, wie es im \emph{Sophistes} durchgeführt worden ist, aber auch von unten nach oben, indem man das Allgemeine aus dem Individuellen heraushebt, um schlussendlich an dem obersten Absoluten anzukommen.\footcite[vgl.][S. 106f.]{Hirschberger} 
%Es geht mit dieser Dialektik darum, dass es um die Erklärung des gesamten Seins durch Aufweis der Strukturidee der Welt geht.\footcite[vgl.][S. 107]{Hirschberger}
%\enquote{Und schließlich geht es in ihr, sofern sie das ganze Sein zusammenschaut und in ihm überall die Parousie der Idee des Guten entdeckt, um den Nachweis der Fußspur Gottes im All.}\footcite[][S. 107]{Hirschberger}
%\enquote{So ist für Platon Dialektik im eigentlichen Sinn viel mehr als nur Logik, sie ist immer Metaphysik und wird als solche zugleich zur Grundlage der Ethik, Pädagogik und Politik}\footcite[][S. 108]{Hirschberger}
%Oder wie es Burgin zusammenfasst: \zitatblock{\enquote{In his theory, Plato assumed that the physical world was the sensible realm, as people could grasp it with their five senses, while the world of Ideas was the intelligible realm, as people could comprehend it only with their intellect.}\footcite[][S. 179]{Burgin}}
%Dies als eindeutige Zusammenfassung, dass sie erkenntnistheorisch voneinander getrennt sein müssen.




%\subsubsection*{Graeser Geschichte der Philosophie Band II Antike Platon}
%\enquote{[Die Zwei-Welten-Lehre] unterscheidet zwischen einer raum-zeitlichen Welt des Werdens und einer jenseits von Raum und Zeit befindlichen Welt des Seins.}\footcite[][S. 133]{GraeserPhiloGeschichte}
%\enquote{Jedenfalls bedeutet Platos Zwei-Welten-Lehre eine radikale Unterscheidung zwischen Sein und Werden, zwischen Wirklichkeit und Schein, zwischen Echtheit und Unechtheit.}\footcite[][S. 134]{GraeserPhiloGeschichte}
%So \enquote{radikal} darf diese Unterscheidung nicht gezogen werden. Leider wird diese radiakle Unterscheidung so stehen gelassen ohne weiter darauf einzugehen.
%Graeser stellt somit auch die Frage, \enquote{warum Platon sich genötigt sieht, die tatsächlichen Gegenstände des Wissens von dem Bereich solcher immerhin bekannten Gegenstände zu trennen und als Gegenstandsbereich \emph{sui generis} jenseits der Welt der Erfahrung zu lokalisieren.}\footcite[][S. 135]{GraeserPhiloGeschichte}
%\enquote{Platon charakterisiert die Idee als das, was wirklich ist, ihre raum-zeitlichen Instanzen jedoch als etwas, was nicht wirklich ist.}\footcite[][S. 139]{GraeserPhiloGeschichte}
%\enquote{die Ideen, die nur dem Denken zugänglich sind.}\footcite[][S. 139]{GraeserPhiloGeschichte} Wie unterscheidet sich diese Formulierung zu der, dass Ideen nur gedacht werden können? Dieses \emph{nur} bringt das Problem mit sich, dass die Dinge von den Ideen völlig getrennt sind. Das heißt, dass eine Verbindung von Ideen und Dingen so angedeutet wird, dass es diese nicht gibt. Ebenso wie die Formulierungen, dass die Ideen den Dingen zugrunde liegen würden. 
%\enquote{Was Plato hier im Rahmen der naturphilosophischen Hypothese bezüglich der Existenz zweier Arten von Dingen mit dem Hinweis auf das Merkmal der Eingestaltigkeit zum Ausdruck bringen wollte, ist, dass die Ideen, anders als die raum-zeitlichen Dinge, nur für sich existieren und genau das sind, was die raum-zeitlichen Dinge nur in Form von Eigenschaften aufweisen.}\footcite[][S. 140]{GraeserPhiloGeschichte}
%Im Timaios wird die Stelle 27d so von Graeser verwendet, dass er die Dinge der Wahrnehmung als \enquote{niemals Seiendes}\footcite[vgl.][S. 140]{GraeserPhiloGeschichte} beschreibt. Bei Schleiermacher heißt es im Kontext: \enquote{Was ist das stets Seiende und kein Entstehen Habende und was ist das stets Werdende, aber nimmerdar Seiende.} oder auch in einer anderen Übersetzung:\enquote{Wie haben wir uns das immer Seiende, welches kein Werden zulässt, und wie das immer Werdende zu denken, welches niemals zum Sein gelangt?}
%Aus der Grundporblematik der immerseienden Ideen und der werdenden Sinnesdingen werden von Graeser drei Interpretationswege dargelegt, dass Platon (1) eine Theorie der Existenz-Stufung vor Augen hatte, (2) den Gedanken eines Mehr an Sein auf das \emph{esse essentiae} bezogen wissen wollte oder (3) er beide Vorstellungen nicht als distinkte philosophische Optionen erkannte und sie konfundierte.\footcite[vgl.][S. 140]{GraeserPhiloGeschichte} Hierbei schließt Graeser 1 und 3 als Möglichkeiten aus.
%\enquote{So lässt sich weder zeigen, dass Plato die Ausdrücke \enquote{existiert} und \enquote{ist wirklich} synonym verwendete, noch findet sich irgendwo eine Argumentation aus der hervorgeht, dass x in höherem Grade über F verfügt als y und x deshalb mehr existiert als y.}\footcite[][S. 140]{GraeserPhiloGeschichte}
%Ein interessanter Punkt von Graeser ist, dass er dieses \enquote{wirkliche Sein} der Ideen so darstellt, dass es eher einen semantischen Charakter aufweist. Dies wird so gemeint, dass Sätze der einen Art - wenn es um Ideen geht - wahr ohne weitere Qualifikation sind, die anderen Sätze - über Sinnesdinge - nicht wahr ohne weitere Qualifikation sind. Somit haben Ideen das \enquote{wirkliche Sein}, die Dinge der raum-zeitlichen Welt hingegen nicht.\footcite[vgl.][S. 141]{GraeserPhiloGeschichte}
%Graeser sieht hier die Ideenlehre so, \enquote{dass die sog. Einzeldinge über keinen ontologisch unabhängigen Status verfügen. Sie sind sozusagen rein relationale Gebilde, d.h. Wesen, die ihren Charakter und ihre Existenz ganz und gar den Ideen verdanken.}\footcite[][S. 145]{GraeserPhiloGeschichte}
%Die Sinnesdinge werden dann im Gegensatz zu den Ideen als eine \enquote{Exemplifikation von Ideen} bezeichnet und verfügen somit auch nicht über reines Sein, was Selbstständigkeit ausmacht.\footcite[vgl.][S. 146]{GraeserPhiloGeschichte}
%Auch hier wird wieder auf Pol. 497b9-10 verwiesen, wo es eigentlich um die Meinung als etwas zwischen Erkenntnis und Unkenntnis steht also weder Sein noch nicht sein und weder beides noch keines von beiden ist.
%Direkt im Anschluss daran wird das Problem der Teilhabe Relation besprochen, dass es nur zwei Möglichkeiten gibt, wie die Ideen an den Dingen teilhaben können, diese beiden Möglichkeiten wieder in eine Aporie führen.
%\zitatblock{\enquote{(P) Wenn es so etwas wie eine Teilhabe gibt, dann entweder (Q) in der Weise, dass die Idee als Ganzes partizipiert wird, oder aber (R) in der Weise, dass nur ein Teil von ihr partizipiert wird. Nun kann es weder Teilhabe am Ganzen der Idee geben (~Q), noch Teilhabe an einem Teil der Idee(~R). Also gibt es keine Teilhabe (~P).}\footcite[][S. 146]{GraeserPhiloGeschichte}}

%\subsubsection*{Gottfried Martin: Platons Ideenlehre (1973) (70/CD 3067 M381)}
%Zwei Weisen des Seins (Phaidon 79A)\footcite[vgl.][S. 37]{Martin73}(duo eide ton) (zwei Ordnungen der Wesen\footcite[][S. 43]{PhaidonKassner} oder zwei Arten der Dinge\footcite[][S. 247]{PhaidonSchleiermacher})
%Diese Übersetzungen zeigen, dass sich diese verschiedenen Formen, Modi oder Arten des Seins als Schwierigkeit herausstellen, also wie diese Unterscheidung gemeint sein sollte, da Martin keine ontologische Implikrationen in die Übesetzung tragen möchte.\footcite[vgl.][S. 37]{Martin73} Diese Unterscheidung der zwei Weisen des Seins gibt Martin die reine Unterscheidung zwischen den Ideen und den Dingen.
%Unter ontologischen Gesichtspunkten widmet sich Martin dann später auf S. 216
%Da in Phaidon 79a eindeutig ein Plural verwendet wird, es in jeglichen Übersetzungen jedoch nicht möglich ist eine adäquate Übersetzung zu liefern, wird die vorsichtige Übersetzung der \enquote{zwei Weisen des Seins} verwendet, da gerade die Dinge von den Ideen unterschieden werden sollen.\footcite[vgl.][S. 216]{Martin73} \enquote{Wie immer man sich in der Übersetzung helfen möge, Platon kennt zwei Weisen des Seins, das Sein der Ideen und das Sein der Dinge, er unterscheidet sie und der stellt sie zugleich zusammen.}\footcite[][S. 216]{Martin73}

%\enquote{Auch hier setzt Platon mit der ausdrücklichen Formulierug ein: zwei Weisen. (Politeia 509D Liniengleichnis)}\footcite[][S. 39]{Martin73}
%Zu Beginn erst nur terminologische Herangehensweise an platonische Begriffe.
%Die Frage ist die Unterscheidung und die unterschiedliche Bedeutung von eidos und idea, welche von Martin nicht als strikt voneinander trennbar aufgefasst wird.\footcite[vgl.][S. 39]{Martin73} in Anlehnung an Ritter.\\
%Aus Phaid. 79a erfasst Martin zwei Kriterien der Unterscheidung der Ideen und Dingen in der Weise ihrer Erfassbarkeit und dass sie in verschiedener Weise sind.\footcite[vgl.][S. 40]{Martin73}
%Leider geht Martin hier nicht weiter auf die Unterscheidung der Erfassbarkeit ein, als darauf, dass die Ideen unsichtbar und nur denkbar und die Dinge sichtbar und in ständiger Veränderung sich befinden. 
%Zweite Unterscheidung vom Sein her. Ideen sind unvergänglich, Dinge sind unter ständigem Wandel begriffen. \footcite[vgl.][S. 41]{Martin73}
%42 Die Ideen sind ewig, immerwährend. Wie sind sie dann im Sophistes in Bewegung. Das geht wohl damit einher, dass man die Ideen als denkbar verstehen muss, also auch verstanden werden müssen. Dabei muss das Element der Bewegung bzw. Veränderung auftreten, da man sonst nicht von einer Erkenntnis sprechen kann. Das Ding (auch die Idee) kann erst dadurch als erkannt bezeichnet werden, wenn man eine Veränderung der Idee von nicht erkannt zu erkannt festlegen kann. (Phaidros 247e, Sophistes 240b, 246b, 248a, 249d)

%{\color{red}{Aus Pol. 479e wird gezeigt, dass \enquote{[\dots] die Existenz der Ideen auf de[m] Unterschied zwischen echtem Wissen [\dots] und bloßen Meinungen [gründet].}\footcite[vgl.][S. 51]{Martin73}}
%Dafür bedarf es aber die Existenz eines besonderen Bereichs, eben der Ideen, so dass das echte Wissen sich auf die Ideen bezieht, das bloße Meinen aber auf die Dinge.\footcite[vgl.][S. 51]{Martin73}
%Ähnliches findet sich laut Martin auch im Timaios 27d, was die Unterscheidung von Ideen und Dingen nochmals unterstreicht.
%Aus der Verschiedenheit von Einsicht (nous) und Meinung (doxa) schließt Martin die Notwendigkeit auch ihre verschiedener Gegenstandsbereiche.\footcite[vgl.][S. 52]{Martin73}}
%Die Zweiweltentheorie (bei Platon, vorher andere Vertreter) S. 89ff.
%Diese Zweiweltentheorie wird so dargestellt, dass der Transzendenzgedanke im Phaidros (Götterreise), Symposion, Phaidon (Unsterblichkeit der Seele) und Politeia an den Mythoserzählungen angelegt wird und damit diese Überweltlichkeit als eine Zweiweltlichkeit interpretiert wird.\footcite[vgl.][S. 88ff.]{Martin73}
%\enquote{Eine jede Erkenntnis kann nichts anderes sein als eine Erkenntnis durch Ideen, und jede Erkenntnis der Ideen wiederum kann nichts anderes sein als Wiedererkenntnis und also Wiedererinnerung. Unsere Seele muss also die Ideen schon vor der Gebrt in dieses Leben gekannt haben. Daraus folgt weiterhin, dass die Ideen außerhalb dieser sinnlich wahrnehmbaren Welt existieren müssen, und Platon sagt dies ausdrücklich.}\footcite[][S. 92]{Martin73}
%\enquote{Die Seele trennt sich im Tode vom Leib. Sie wird gerichtet werden. Ihr Ziel ist der Weg nach oben. Dies darf nicht in einem humanitären Entwicklungssinn als eine Höherentwicklung verstanden werden, sondern es meint in der Tat die dort oben liegende Welt des reinen Seins.}\footcite[][S. 93]{Martin73}
%Diese Zweiweltentheorie ist rein auf die Trennung von Leib und Seele und die Unsterblichkeit der Seele nach dem Tod bezogen. Also auch die Vollendung des Aufstiegs nach dem Tod zu dem \enquote{dort oben liegenden Welt des reinen Seins}\footcite[][S. 93]{Martin73}
%Es werden laut Martin die drei Ausdrücke TOPOS, GENOS und KOSMOS verwendet, um die beiden Bereich der Ideen und der Dinge zu charackterisieren.\footcite[vgl.][S. 95]{Martin73}
%Leider wird die topos Unterscheidung primär auf das Höhlengleichnis angewandt ebenso auf die Mythos Erzählungen und nicht konsequent von den anderen Bedeutungen abgerenzt und deutlich unterschieden.
%\enquote{So erweisen sich auch unter diesem Gesichtspunkt die vier großen Ideendialoge, der Phaidros, das Symposion, der Phadion und die Politeia als einheitlich. Die Welt hier unten ist nicht die wahre Wirklichkeit, die wahre Wirklichkeit ist die Welt der Ideen dort oben. Ich darf wiederholen: Ich will nicht sagen, dass dies Platons eigentliche Überzeugung und dass diese Zweiweltentheorie das letzte Wort Platons ist. Man muss sich vielmehr vor Augen halten, dass diese räumliche Darstellung des Unterschieds zwischen Ideen und Dingen eine Notwendigkeit unseres Denkens und Sprechens ist.}\footcite[][S. 96]{Martin73}
%\enquote{Lässt man sich von einer naheliegenden Bedeutung des Wortes \enquote{Metaphysik} leiten und versteht man unter Metaphysik die Lehre von dem, was hinter der Natur liegt, dann ist die Ideenlehre des Phaidon der Anfang und der Urspurng der Metaphysik. Hier werden [\dots] die Ideen als etwas verstanden, was hinter und über der Natur liegt.[\dots] Was ist dies transzendente Sein der Ideen?}\footcite[vgl.][S. 128]{Martin73}
%Nimmt man die Lehre von den transzendenten Realitäten als Ziel einer Philosophie, dann sind Phaidon und Phaidros Anfang und Ursprung. Wenn man aber die Frage: Was ist Sein? nimmt, dann wird der Sophiestes zu einem zentralen Dialog.\footcite[vgl.][S. 129]{Martin73}
%Probelmatisch wird es dann wieder, wenn es heißt \enquote{Erst nach dieser Entdeckung der Ideenlehre kann Platon das Sein in zwei Weisen des Seins differenzieren, und wir hatten die ausdrückliche Differenzierung in Phaidon (79a) und in der Politeia (509d) gefunden.}\footcite[][S. 131f]{Martin73} Im Gegensatz zu Parmenides und Heraklit, die zwar sagen konnte, alles ist Wasser, aber nicht was ist das Wasser?
%Es wird aber dann eingeräumt, wenn man behauptet die Ideen sind und sie alleine sind das Sein, nicht zulässig ist, da somit keine Dinge sein könnten. Daher muss es irgendeine Form geben, bei der den Dingen ein Sein zukommt, auch wenn es ein leicht abgeleitetes Sein ist.\footcite[vgl.][S. 131]{Martin73}
%Damit würde man notwendigerweise wieder bei dem Punkt ankommen, den man an Parmenides und Heraklit angesetzt hat, dass man nur ein Oberstes gesetzt hat und nicht mehr nach dem Obersten selber fragen kann.  
%\zitatblock{\enquote{Nach [der Anamnesislehre] ist Erkenntnis immer eine Wiedererinnerung. Das heißt doch, und Platon sagt dies auch ausdrücklich, daß die Seele die Ideen in einem früheren Leben vor der Geburt kennengelernt haben muß. Dies wiederum kann doch nur heißen, daß die Ideen nicht in dieser Welt sind}\footcite[vgl.][S. 160]{Martin73}}
%Diese Auffassung ist nicht haltbar. Dies hat damit zu tun, dass mit dieser Ansicht einige Implikationen einhergehen müssen, auf die wir keinen Zugriff oder keinen Zugang haben. Was ich damit meine ist, dass man, wenn man davon ausgeht, dass die Seele die Ideen vor der Geburt in einer \emph{anderen} Welt kennengelernt oder geschaut hat, dass wir eigentlich nichts über diese Welt aussagen können. Dabei bedarf es der Annahme, dass wir entweder sagen, man kann nur etwas über diese Welt aussagen, das haltbar bleibt, wenn wir eine nochmals übergeordnete Welt annehmen, die die Ideen und Sinneswelt verbindet, auf die allerdings wiederum zugegriffen werden kann, da sonst keine logsichen Schlüsse und Aussagen möglich wären. Oder man muss in irgendeiner Weise beweisen können, dass wir einen logischen Zugang zu diese anderen Welt aufweisen, wodurch man im eigentlichen Sinne durchaus Aussagen treffen könnte, dies aber nicht getan wird. Zumindest nur bis zu diese Punkt.
%Dies ist sehr eng an das Konzept des Einen und Vielen geknüpft, ergo der Dialektik. Der Wissenschaft des Auseinanderhaltens und Zusammenführens von Begriffen.

%Natürlich muss man diese beiden Bereiche dahingehend differenzieren und eine gedankliche Grenze setzen, da nur so von beiden in ihrer isolierten Form gesprochen werden kann. Nur dadurch ist es möglich, dass die beiden Bereiche aufeinander Auswirkung und überhaupt Wirkung haben.
%Die Frage dabei lauter also von Martin, wie das CHORIS im griechischen gebraucht wird. Entwerder eine rein räumliche Trennung, was gerade dann für das Problem sprechen würde, oder ob auch andere Trennungen oder Teilungen/Sonderungen damit gemeint sein könnten. (S. 163ff.)
%Die andere Möglichkeit ist die begriffliche Trennung von Dingen, die wohl am meisten von Platon vertreten wird\footcite[vgl.][S. 165]{Martin73}. Beispiele: Buch 10 der Politeia: 595A, Laches 195A, Eutydemos 289C
%Es liegt bei Platon faktisch, bei Aristoteles dann explizit ein räumlicher, ein zeitlicher und ein begrifflicher Chorsimos vor.\footcite[vgl.][S. 166]{Martin73}
%\enquote{In der Kritik der Ideenlehre geht Aristotles immer davon aus, dass der Chorismos bei Platon immer rein räumlich gemeint ist.}\footcite[][S. 166]{Martin73}
%Räumliche Trennung der Begriffe im Höhlengleichnis auf welcher Ebene? Es lassen sich mehrere Ebenen aufzeigen. Siehe Liniengleichnis. Die Trennlinie ist scharf, aber bis zu welchem Grad darf man die Auflösung des Schärfegrades, mit dem man das Bild betrachtet, drehen, bis man den Blick dafür verliert, dass es sich alles dennoch in einer Welt abspielt, denn wie könnte dann der Aufstieg überhaupt erfolgen.
%Es stellt sich bei der Methexis die Frage, \enquote{ob es sich um eine Teilnahme der Dinge an den Ideen oder um eine Anwesenheit der Ideen in den Dingen oder um eine Gemeinschaft der Ideen und der Dinge handelt.}\footcite[vgl.][S. 170]{Martin73}

%Martin bleibt dabei, dass Platon nicht ausreichende Texte überliefert hat, aus denen hervorgeht, was im Parmenides über den Chorismos und die Methexis (Anteilhabe) vorgetragen worden ist, um diese aufgeworfenen Probleme lösen zu können.\footcite[vgl.][S. 173f.]{Martin73}

%\subsubsection*{Martins Ousia}
%\enquote{In der Tat behauptet Platon die Existenz der Ideen und er formuliert das, indem er sie als Ousia bezeichnet und indem er sie als ON bezeichnet.}\footcite[][S. 187]{Martin73}
%Martin kommt nach einiger Analyse - auch anderer Autoren - zu dem Schluss, dass Ousia nicht nur das Sein der Ideen meint, sondern auch das Sein der Dinge. Dabei bleibt er nicht bei dieser Deutung stehen, sondern geht auch noch auf die Bedeutung von Ousia als das allgemeine Sein sowohl der Ideen als auch der Dinge ein.\footcite[vgl.][S. 195ff.]{Martin73}
%\zitatblock{\enquote{Der Text bezeichnet ausdrücklich beides, das Schreiben und das Sprechen als unmöglich. So sagt Platon etwa:\enquote{Soviel wenigstens weiß ich, dass ich, wenn ich es ausspräche oder niederschriebe[\dots]}(Pol. 341 D). Nimmt man die Philosophie so, wie Platon sie hier nimmt und wie sie immer genommen werden sollte, dann kann sie weder ausgesprochen noch niedergeschrieben werden.}\footcite[][S. 251]{Martin73}}
%Das Höhlengleichnis hier nocheinmal herangezogen wird deutlich, dass es der Aufstieg also eine Bewegung, eine Handlung, ist, die die Philosophie für Platon am Ende ausmacht. Denn rein durch das Sprechen und Hören oder Schreiben und Lesen wird niemand zur Idee des Guten schreiten können. Lediglich durch eine innere Anstrengung, die durch die Rede oder den Text entfacht wird.
%\subsubsection*{Martins Chorismos in den vier großen Ideendialoge}

%\begin{itemize}
 %   \item {Phaidros schwer bestritten werden, dass es Chorsimos gibt. Die Stellen hier sind die Beschreibung der Reise der Seele auf dem Göttewagen mit den Göttern, wo die Ideen im eigentlichen Sinne vor der Geburt geschaut werden. Das ist vermutlich einfach zu viel und würde zu weit weg führen.}
  %  \item {Symposion mit dem Aufstieg zur Idee des Schönen}
   % \item {Politeia mit dem Höhlengleichnis, aber auch das Sonnen und Liniengleichnis}
    %\item {Phaidon. Chorsimos also Trennung von Ideen und Dingen über die Anamnesislehre, also Trennung in der Erkennbarkeit und Widererinnerung}
    %\item {Parmenides (faktische Reflexion). Explizite Behandlung des Chorismos Problems. Hier wird auch das Wort \enquote{choris} verwendet (130BCD)}
    %\item {Sophistes (methodische Reflexion)}
%\end{itemize}
%\footcite[vgl.][S. 160]{Martin73}
\subsubsection{Die Idee des Guten}
Der Idee des Guten wird in der platonischen Lehre eine ganz besondere Rolle zugeschrieben, die bisher nicht angemessen dargelegt worden ist. Sie ist  für Platon der wichtigste Gegenstand philosophischer Erfahrung überhaupt.\footcite[vgl.][S. 158]{GraeserPhiloGeschichte}
Wie bereits im Liniengleichnis angedeutet kommt der IdG eine Sonderrolle zu, welche, wie Graeser es festhält, \zitatblock{\enquote{(1) den Gegenständen der Erkenntnis Wahrheit und Erkennbarkeit verleiht (508e-509b), (2) dem erkennenden Subjekt die Möglichkeit der Erkenntnis dieser Gegenstände gewährt (508e), (3) den Gegenständen der Erkenntnis Sein und Seiendheit verleiht (509b) und (4) selbst nicht mit Seiendheit identisch ist, sondern an Würde und Bedeutung noch über Seiendheit hinausragt (509b).}\footcite[][S. 158]{GraeserPhiloGeschichte}}
Disse beschreibt es so, dass sie dem Sein der Ideen gegenüber etwas Überseiendes ist und dass die IdG gewissermaßen nochmals die bereits transzendierte Ideenwelte als deren Ursache (aitia) transzendiert.\footcite[vgl.][S. 50]{DisseMetaphysik}
%Diese Auffassung stimmt mit den Punkten 3 und 4 von Graeser zusammen, allerdings steht diese Auffassung den Punkten 1 und 2 entgegen. 

\subsection{Zusammenfassung der Interpretation der zwei Welten}
Wie kann das bisherige nun zusammengefasst werden?
\zitatblock{\enquote*{Den Gegenstandsbereich des Denkens können wir [\dots], im Gegensatz zur sinnlich wahrgenommenen Welt, d.h. zur Sinnenwelt, die \emph{Ideenwelt} nennen. Grundlage von Platons Metaphysik ist die so genannte Zweiweltenlehre, die Lehre des Gegensatzes zwischen Sinnen- und Ideenwelt, wobei die Ideenwelt als die höhere, weil ewige und geistige Wirklichkeit die der Sinnenwelt zugrunde liegende, sie fundierende ist, 
%nach der die an den Körper gebundene menschliche Seele stirbt.
}\footcite[vgl.][S. 29]{DisseMetaphysik}}

\enquote{Jedenfalls bedeutet Platos Zwei-Welten-Lehre eine radikale Unterscheidung zwischen Sein und Werden, zwischen Wirklichkeit und Schein, zwischen Echtheit und Unechtheit.}\footcite[][S. 134]{GraeserPhiloGeschichte}
Somit heißt es, dass Platon die Idee als das charakterisiert, was wirklich ist, ihre raum-zeitlichen Instanzen jedoch als etwas, was nicht wirklich ist.\footcite[vgl.][S. 139]{GraeserPhiloGeschichte} Dabei sind mit raum-zeitlichen Instanzen Sinnesdinge gemeint.\\
Somit heißt es von Disse, dass die Ideen bei Platon einen von der Sinnenwelt eindeutig getrennten Wirklichkeitsbereich bilden.\footcite[][S. 31]{DisseMetaphysik}

\zitatblock{\enquote{Die Frage nach der Möglichkeit des Wissensgewinns führt bei Platon also zu einem ontologischen und anthropologischen Dualismus: Aus der Auffassung, dass die Welt des sinnenfälligen Werdens keine sichere Erkenntnis vermitteln kann, folgert er die Gegebenheit eines welttranszendenten Bereichs rein idealer Wesenswahrheiten, die nur der geistigen Erkenntnis des Denkens zugänglich sind und von der Seele immer schon apriorisch gewusst werden.}\footcite[][S. 99]{ThurnerDualismus}}

Es wird von ihm das Verständnis gezeichnet, dass die Ideen und die Welt der Ideen eine \enquote{Ideale Wirklichkeit} darstellen und auch nach dem möglichen Vergehen der materiellen Welt immer noch existieren mögen, da man feststellen kann, dass die gelieferten Beispiele von mathematischen Ideen die Frage nach dem Anfang ihrer Existenz nicht beantworten können.\footcite[vgl.][S. 99]{Hirschberger} 
%Weiter heißt es von Johannes Hirschberger:
%\zitatblock{\enquote{Auch wegen dieser unausschöpfbar reichen, zeugenden Fruchtbarkeit ist die Ideenwelt die stärkere Wirklchkeit. Darum unterscheidet also Platon die Ideenwelt [\dots] von der sichtbaren Welt [\dots] und erblickt nur in jener die wahre und eigentliche Welt, in dieser aber bloß ein Abbild, das in der Mitte steht zwischen Sein und Nichtsein.}\footcite[vgl.][S. 100]{Hirschberger}}
%Die Stelle die hier vermutlich gemeint sein sollte ist Pol. 478d-e, wo es allerdings um die Meinung geht und wie diese gegenüber der Erkenntnis zu verstehen ist. Es geht darum, dass das Meinen weder als Unwissenheit noch als Erkenntnis beschrieben wird und damit weder sein noch nicht-sein aufweist. Die Frage von falschen Aussagen, also auch von Meinungen, wird im Sophistes nochmal behandelt, wo das Nicht-Seiende als dennoch Seiendes beschrieben wird. 
Es wird jedoch versucht zu unterscheiden zwischen der ersten lexikalischen Bedeutung der Trennung (chorismos) als eine räumliche Trennung, wobei Platon den Ideen in Tim 52a-b räumliche Ausdehnung abspricht. Allerdings wird daraufhin lediglich festgestellt, dass Platon keine weiter Erklärung diesbezüglich liefert, da er auch nicht mit den Begriffen der Transzendenz und Immanenz hantiert.\footcite[vgl.][S. 34f]{DisseMetaphysik} Dabei werden zwei Kriterien von Transzendenz aufgestellt, welche dann bei Platon als gegeben angenommen werden. Diese bestehen darin, dass eine Wirklichkeit zu verstehen ist, \enquote{1) die jenseits der sinnlichen, der raumzeitlichen Wirklichkeit ist und die 2) eine vollkommene Wirklichkeit bildet, in der erst die unvollkommene, raumzeitliche ihren eigentichen Grund findet.}\footcite[][S. 35]{DisseMetaphysik} Man muss aber zu Gute halten, dass das Problem erkannt wird, wenn man von der Transzendenz der Ideen, im Gegensatz zur raumzeitlichen Welt, spricht.\footcite[vgl.][S. 35]{DisseMetaphysik} 
Es wird aber dann eingeräumt, wenn man behauptet die Ideen sind und sie alleine sind das Sein, nicht zulässig ist, da somit keine Dinge sein könnten. Daher muss es irgendeine Form geben, bei der den Dingen ein Sein zukommt, auch wenn es ein leicht abgeleitetes Sein ist.\footcite[vgl.][S. 131]{Martin73}
Graeser stellt somit auch die Frage, \enquote{warum Platon sich genötigt sieht, die tatsächlichen Gegenstände des Wissens von dem Bereich solcher immerhin bekannten Gegenstände zu trennen und als Gegenstandsbereich \emph{sui generis} jenseits der Welt der Erfahrung zu lokalisieren.}\footcite[][S. 135]{GraeserPhiloGeschichte}
Auch wird folgendes eingeräumt: \zitatblock{\enquote{Wenn man aber wie Platon davon ausgeht, dass es so etwas wie Transzendenz gibt, ist der \enquote{méthexis}-Begriff dann nicht unumgehbar? Zwar bleibt die Schwierigkeit bestehen, wie man das Verhältnis von etwas, was von dieser Welt getrennt ist, zu dieser Welt denken soll, wenn kein räumliches Verhältnis zwischen beiden gemeint sein kann, bzw. wie Dinge an etwas Unteilbarem teilhaben können.}\footcite[][S. 48]{DisseMetaphysik}\footnote{Es sei darauf hingewiesen, dass noch weiter Stellen aus den Originalen hinzugezogen werden könnten, was von den Autoren auch getan wird, hier jedoch auf eine kleiner Auswahl bezug genommen worden ist. Für weiter Stellen siehe Appendix.}}
Wie man jetzt anhand der Interpretationen sehen kann, ist die Deutung der zwei Welten durchaus als problematisch erkannt worden. Es gilt also darzulegen, wie die Autoren diesem Problem beikommen, da dies so nicht stehen gelassen werden kann. 
Bevor sich also dem anderen großen Teil dieser Arbeit gewidmet werden kann, muss zuerst behandelt werden, wie die Interpretationen diese beiden Bereiche zusammenführen, da eine völlige Trennung, wie schon von Hrischberger festgehalten nicht möglich ist, da es für Platon eine Einheit des Seins gibt.\footcite[vgl][S. 100]{Hirschberger}

\subsection{Welche Probleme gibt es mit dieser Ansicht}
\subsubsection{Gnoseologisch oder Ontologisch}
Die Einheit des Seins, wie es von Hirschberger hier festgehalten wird, ist nur bedingt so zulässig stehen gelassen zu werden, da Hirschberger selbst auf die gnoseologische Trennung von Dingen und Ideen hinweist. 
Dies führt dazu, dass aus der gnoseologischen Auslegung und aus der Nähe zur ontologischen Auslegung Schlüsse gezogen werden in Form von \enquote{Gegründetes und Gründendes}, was nicht nach den exakt gleichen Prinzipien der gnoseologischen Auslegung auf die ontologische Auslegung angewandt werden darf, einfach weil es sich hierbei wieder um die Problematik des infiten Regresses handelt, der auch nicht in den Interpretationen von der IdG vollständig aufgehoben wird.  
Das grundlegende Problem ist wohl, dass die Interpretation der Ideenlehre nie dahingehend unterschieden wird, welches Ziel Platon in den angegeben Stellen hatte. Also auf welche Frage, oder welches Problem hin, die Dialoge ausglegt sind und welche Interpretationen außerdem möglich wären. Legitimität erhält dieser Einwand darin, dass die Dialoge nie als festes Werk und Lehre dargestellt worden sind und werden sollten, sondern dass Philosophie eigentlich immer im Dialog stattfinden soll. (Dies wird von Martin bestritten S. 251f.) Damit soll der Blick darauf gerichtet werden, warum die Autoren hier nie die Frage gestellt haben, ob ein gnoseologischer oder ontologischer Aspekt von Platon angesprochen worden ist, oder sogar beides.  
Wenn man von dieser \enquote{eigentlichen} Wirklichkeit oder der \enquote{eigentlichen} Welt spricht, kommt man in der Darstellungsform und der Argumentation vom Weg ab. In Phaidon 75a heißt es zwar, dass die Sinnenwelt zwar danach strebt, wie die Ideenwelt zu sein, allerdings dahinter zurück bleibt, hier aber nicht von dem ewigen Teil des Menschen - der Seele - gesprochen wird, welcher eben auch, wie die Ideen, ewig ist und somit diesen Zugang auf erkenntnistheorischer Ebene besitzt und somit zum einen Zugang zu dieser Ebene besitzt und zum anderen auch diese mögliche Angleichung haben kann. Dies steht auch damit in Konflikt, wenn man im Höhlengleichnis die Möglichkeit hat aus der Höhle in den vermeintlichen Bereich des Denken zu gelangen, um dann, wie schon gesehen, die Sonne als die Idee des Guten erblicken zu können. Die Frage ist hierbei ebenfalls, ob es sich um eine ontologische Deutung handelt, welche nicht sinnvoll wäre, oder eben um eine gnoseologische, also ob alles nach wahrer Erkenntnis strebt. Im Original heißt es nämlich, dass das alles danach strebe zu sein wie das Gleiche, aber dahinter zurückbleibt. Aus gnoseolgischer Perspektive ergäbe dies insofern Sinn, dass die Unwahrheit oder auch Meinung danach strebt Wahrheit zu sein, aber immer dahinter zurückbleibt, als dass sie doch Meinung ist und nicht Wahrheit. 
Zum anderen ist es so, was in der bisherigen Auslegung der Interpretation die Annahme verhärtet hat lassen, wenn man von dieser eigentlichen Wirklichkeit spricht, es den Anschein erweckt, dass man diese Wirklichkeit als eigentliches oder wahres Seiendes bezeichnet. Dabei ließe umgekehrt nur der Schluss zu, dass die Sinnenwelt nicht wahres Seiendes oder uneigentliches Wirkliches ist. Auf diesen Aspekt wurde zwar hingewiesen, allerdings kann uns dabei nichts anders übrig bleiben als der Sinnenwelt einen Grad des Seienden abzusprechen. Dabei wird eben nicht darauf geachtet, dass die angeführten Stelle sich um die Frage drehen, wie Meinung im Gegensatz zur Wahrheit zu verstehen ist und nicht auf einen Unterschied in Form von Seinsstufen abgezielt wird.\\ Hierführ muss nocheinmal ein Blick in die angeführten Stellen geworfen werden. 
Zu Tim 52a sei folgendes gesagt: Im Original-Text steht, dass es um die Unterscheidung von Vernunft und richtiger Meinung geht. Denn dies muss unterschieden werden, da wir sonst alles, was wir vermittels des Körpers wahrnehmen, als höchst zuverlässig annehmen müssen. (Tim 51d) Dabei geht in die Richtung des Erkennens von Wahrheit und nicht von Zwei Welten. Das wird deutlich, da bei Disse in das Zitat das Wort \emph{Gebiet} nach \emph{Zweite} eingefügt worden ist, wo eigentlich auf die richtige Meinung rekurriert wird. Dabei geht es also nur um eine gnoseologische Betrachtung und nicht darum eine ontologische Ebene aufzumachen.
Ähnliches findet sich auch in Pol. 475d-480a zitiert und falsch ausgelegt dahingehend, dass wieder von Dingen, die mehr oder weniger Sein haben können, gesprochen wird. Jedoch beruft sich die zitierte Stelle ebenfalls nur auf den Unterschied von Meinungen und wahrer Erkenntnis. Da hier wahre Meinung als weder wirklich seiend noch nicht-seiend verstanden wird, wodurch die Meinung als etwas zwischen Sein und nicht-Sein verortet werden muss. Hier geht es also nicht darum die Dinge der Sinnenwelt als etwas zwischen Sein und Nicht-Sein zu setzten, wie es Disse anführt.\footcite[vgl.][S. 38]{DisseMetaphysik} 
Denn nach dieser Beschreibung würden die Sinnesdinge als zwischen Sein und Nicht-Sein stehend weder Seiend noch Nicht-Seiend sein, was logisch gar unmöglich ist.   
Das, worauf diese Stelle hinausläuft, ist, dass Erkenntnis und Meinung nicht dasselbe sind und auch einen unterschiedlichen Gegenstand betrachten und einsehen. Die Erkenntnis hat als ihren Gegenstand das Seiende. Die Meinung hingegen muss etwas anderes habe, jedoch kann es als Gegenstand weder Seiendes noch Nicht-Seiendes sein, da dies auch unmöglich wäre. Somit muss der Gegenstand als etwas zwischen Sein und Nicht-Sein sein. 
\enquote{Das Gebiet des Tageslichts außerhalb der Höhle ist der Bereich dessen, was uns durch reines Denken zugänglich ist. Die Sonne aber wird mit dem höchsten Punkt im Bereich des Denkbaren verglichen.}\footcite[][S. 49]{DisseMetaphysik} Das steht so nicht im Text. Die Person wird von der Höhle aus der Höhle geführt. Es steht nichts im Text, dass hier von einem Übergang in eine \emph{andere Welt} gesprochen wird. Ebenfalls wird die Interpretation nicht im Text angestoßen, wobei die Nähe zum Sonnengleichnis wie auch zum Liniengleichnis nicht verkannt werden kann. Diese Interpretation ließe sich dahingehend nicht so scharf zeichnen, als dass im Text auf dieselbe Weise, laos durch die Augen die Dinge innerhalb wie außerhalb gesehen werden. Auch wird nicht angeführt, dass außerhalb der Höhle der Bereich des Denkens stattfindet. Offensichtlich wird nur eine Höhle verlassen. Die einzigen Hinweise, die sich hierfür finden lassen, sind, dass von \enquote{der Schau der Dinge über der Erde} (Pol. 516a) die Rede ist. 
Das Höhlengleichnis ist lediglich eine Metapher dafür, wie der Weg bestritten wird und nicht, dass aus der einen Welt in eine andere Welt gegangen werden soll. So bleibt der Bezugspunkt, von dem ausgegangen wird und zu dem auch wieder zurückgekehrt wird, immer die Höhle. Also wird nur der Gang von in der Höhle aus der Höhle heraus beschrieben. Genauso werden außerhalb der Höhle ebenso \emph{Dinge} gesehen und erkannt. Aber auch hier ist die Idee des Guten nicht \enquote{außerhalb} der Welt. Auch der Blick in die Sonne wird als möglich beschrieben, wodurch die Reinheit und ihre Beschaffenheit erblickt werden kann und somit auch die Sonne als Urheberin erkannt wird, die die Ursache für die Ordnung im Bereich der sichtbaren Welt ist. (Pol. 516b)
Problematisch ist es in der Hinsicht, dass wir uns in dieser Sinnenwelt befinden und dies unser \enquote{Startpunkt} ist, wie es im Höhlengleichnis beschrieben wird. Ebenfalls ist es so, dass wir, selbst wenn man die Höhle verlassen und die Sonne gesehen hat, die Aufgabe haben zurück in die Höhle zu gehen. Also am Ende nicht aus der Welt zu fallen oder diese gar zu verlassen. Zudem ist das Verlassen der Höhle nur unter einer Metapher zu fassen, um den förmlichen Aufstieg nochmals aus dem Liniengleichnis zu verdeutlichen.
Aus prinzipientheoretischer Sicht ist es so, dass es einfacher ist von unten nach oben zu gehen, also von den Sinnendingen anzufangen, um dann zu den Ideen hoch zu gehen, als von oben herunter zu konsturieren. Man siehe hier mögicherweise die Konstruktion des Staates selbst, wo mit den Bauern begonnen wird und erst darauf aufbauend der Staat nach oben konstruiert wird. Daher ist fraglich, inwiefern diese \enquote{wahre} oder \enquote{wirkliche} Ideenwelt oder -ebene besser oder hilfreicher in der Hinsicht ist, von welchem Bezugspunkt man ausgeht, der einem zur Verfügung steht. 
Das Gegenbeispiel findet sich im Sophistes, in dem versucht wird den Angelfischer zu definieren. Dabei wird eben von oben nach unten vorgegangen, also von einem obersten Begriff, welchen man dann weiter aufzuteilen versucht, um am Ende mit sechs unterschiedlichen Definitionen des Sophisten bedient worden zu sein. Dadurch ist nicht gewährleistet, ob man mit dieser Methode wirklich bei einer vollendeten Definition angekommen wäre oder nicht. Mit dieser \enquote{vollendeten} Definition sei hier gemeint, dass es sich bei dem Beispiel aus dem Sophistes immer noch um die Möglichkeit einer fälschlichen Definition des Angelfischer handeln könnte, auch wenn diese mögliche Definition als vollendet erscheint. Dies geschieht daher, dass man erst von einer \enquote{vollendeten} Definition sprechen könnte, wenn man sich aller möglichen Wege bedient hat, die von oben herab gehen würden. Damit hätte man zwar am Ende eine vollendete Definition, jedoch nur, wenn man ausschließen könnte, dass nicht noch ein Weg übrig geblieben wäre. Dieser Problematik entgeht man, wenn man von unten nach oben hin konzipiert.\\
%Es wirkt fast so, als würde man versuchen die beiden Seinsbereiche, die man identifiziert hat miteinander zu verbinden, ohne ein Drittes zu setzten, das die beiden Bereiche verbindet, was zu dem altbekannten Problem des infiniten Regresses gelangt, wo man wiederum ein drittes benötigt, um das erste Dritte mit einem der ursprünglichen zwei zu verbinden.
Es wird zwar von Martin erkannt, dass das reine Sein der Ideen eine Unmöglichkeit darstellt, was an der Idee des Schönen exemplifiziert wird, dass es also somit keine schönen Einzeldinge geben könnte, wenn es doch nur das Sein der Schönheit selber als Sein existiert, aber hier wird nicht notwendig zuende gedacht, dass es das Sein der Dinge erst in dem Sein von Dingen und Ideen zusammengedacht und in der Idee des Guten überwunden werden kann. Also in eine Einheit gebracht werden kann. 
\enquote{Die Ideen sind im Phaidon das Seiende (77a), sie sind im Phaidros das seiend Seiende (247e), sie sind im Sophistes das wahrhaft Seiende.}\footcite[vgl.][S. 131]{Martin73}\\
Problem dabei ist die eingehende Auslegung der Metaphysik als die Frage \enquote{[\dots] nach einer wahren Wirklichkeit, nach einem Sein, das mehr als das uns unmittelbare Gegebene ist.}\footcite[vgl.][S. 17]{DisseMetaphysik} Wie soll man sich dieses \emph{Mehr} überhaupt vorstellen, wenn es uns möglicherweise gar nicht zugänglich oder gegeben ist.
In Bezug auf den Phaidon Dialog und die darin ausgeführte Anamnesislehre heißt es noch wie folgt:
%Fraglich bleibt jedoch bei diesem Dialog, inwiefern die Rede von der Seele zum Körper gedacht werden soll, da dies unter anderem der größere Rahmen des Dialogs ist, da Martin es so ausdrückt:
\zitatblock{\enquote{Nach [der Anamnesislehre] ist Erkenntnis immer eine Wiedererinnerung. Das heißt doch, und Platon sagt dies auch ausdrücklich, daß die Seele die Ideen in einem früheren Leben vor der Geburt kennengelernt haben muß. Dies wiederum kann doch nur heißen, daß die Ideen nicht in dieser Welt sind.}\footcite[vgl.][S. 160]{Martin73}}
Somit gibt es dabei keine deutlichere Ausdrucksweise die Ideen als etwas nicht in dieser Welt zu verorten und damit eine zweite Welt anzunehmen, welche von der sichtbaren Welt unterschieden werden muss. 
%\enquote{Platon unterscheidet somit im Höhlengleichnis zunächst einmal grundsätzlich zwischen zwei Welten und bestimmt eine Bewegung des Menschen, nämlich die Tätigkeit der Philosophie, die ihn von der ersten in die zweite, eigentliche führen soll.}\footcite[][S. 23f.]{DisseMetaphysik} Wo wird das deutlich? Es gibt keinen Verweis auf den Text. Wie sind hier zwei Welten gemeint? Denn die Bewegung beginnt in der Höhle und führt zum Feuer und dann aus der Höhle hinaus. Dieses \enquote{aus der Höhle hinaus} könnte man als einen Übergang in eine andere/zweite Welt deuten, wobei man hierbei sehr aufpassen muss dies nicht als eine räumlich getrennte Welt zu verstehen. Denn wie wäre es möglich diese eine Welt zu verlassen und in diese andere Welt einzutreten, wenn diese vorher nicht schon verbunden gewesen sein müssen.
%Es wird Tim 52a zitiert als Zusammenfassung der Zweiweltenlehre. Im Original-Text steht allerdings wieder, dass es um die Unterscheidung von Vernunft und richtiger Meinung geht. Denn dies muss unterschieden werden, da wir sonst alles, was wir vermittels des Körpers wahrnehmen, als höchst zuverlässig annehmen müssen. (Tim 51d) Dies geht wieder in die Richtung des Erkennens von Wahrheit und nicht von Zwei Welten. Das wird deutlich, da bei Disse in das Zitat das Wort Gebiet nach Zweite eingefügt worden ist, wo eigentlich auf die richtige Meinung rekurriert wird.
\subsubsection{Methexis}
Wie kann folgender Satz dann verstanden werden? \enquote{Die geläufigste Interpretation ist, dass es bei Platon letztlich Ideen von allem gibt, was in der Sinnenwelt existiert, mit Ausnahme von Individuen.}\footcite[][S. 31]{DisseMetaphysik} 
%Direkt im Anschluss wird die eigene Aussage nichtig gemacht, als der Demiurg aus dem Timaios hergenommen wird, der sich bei der Erschaffung der Welt sich an den Ideen als Urbilder bedient, um die Welt zu erschaffen. 
Hiernach richten sich die Ideen nach den Dingen. Wie geht das d'accord, dass sich eigentlich alles nach der Idee des Guten richtet und alles auf die Idee des Guten hin ausgerichtet ist? Diese Formulierung dreht dieses Verhältnis um, sodass sich die Ideenwelt nach der Sinnenwelt richten müsste. Zudem wird es nach dieser Ansicht schwer Zahlen o.Ä. als \enquote{existierend} zu nennen, wenn man diese doch gar nicht in der Sinnenwelt existent sehen kann. 
Fraglich ist in diesem Zuge außerdem, wie sich die Rolle der IdG dahingehend rechtfertigen lässt, dass doch die Ideen als eindeutig ewig bezeichnet werden, aber die IdG doch als deren Grund gelten soll. Dies ist in der Hinsicht nicht zulässig, wenn man bedenkt, dass nur unter der Voraussetzung einer Kausalkette eine vorhergehende Ursache zulässig ist. Da die Ideen jedoch ewig sind, ist eine Kausalität hier nicht zulässig.
Schließlich kommt in dieser Formulierung noch einher, dass von existierenden Sinnesdingen gesprochen wird, wobei doch den Ideen wahres Sein also auch wahre Existenz zugesprochen worden ist. Bzw. wie können Sinnesdinge existieren, wenn die eigentliche Verbindung von den Ideen ausgegangen ist und nicht umgekehrt, also die Existenz der Dinge ist bedingt durch die Ideen nicht anders herum.
%Man könnte hier zum Liniengleichnis schauen und sich das Verhältnis ansehen, welcher Bereich der größte wäre. Hier besteht allerdings kein Konsens darüber, ob der obersten oder der untersten Stufe die größte \enquote{Fläche} zukommt.
%Es wird daraufhin wieder Pol. 475d-480a zitiert und falsch ausgelegt dahingehend, dass wieder von Dingen, die mehr oder weniger Sein haben können, gesprochen wird. Jedoch beruft sich die zitierte Stelle wieder nur auf den Unterschied von Meinungen und waher Erkenntnis. Da hier wahre Meinung als weder wirklich seiend noch nicht-seiend verstanden wird, muss hier die wahre Meinung als etwas zwischen Sein und nicht-Sein verortet werden. Hier geht es aber nicht direkt um das Sein von Sinnesdingen und Ideen.\footcite[vgl.][S. 37f.]{DisseMetaphysik} Diese Darstellung wird dann im Anschluss daran getroffen.
Des weiteren vergleiche hierzu oben Disse S. 48.
%Auch wird folgendes eingeräumt: \zitatblock{\enquote{Wenn man aber wie Platon davon ausgeht, dass es so etwas wie Transzendenz gibt, ist der \enquote{méthexis}-Begriff dann nicht unumgehbar? Zwar bleibt die Schwierigkeit bestehen, wie man das Verhältnis von etwas, was von dieser Welt getrennt ist, zu dieser Welt denken soll, wenn kein räumliches Verhältnis zwischen beiden gemeint sein kann, bzw. wie Dinge an etwas Unteilbarem teilhaben können.}\footcite[][S. 48]{DisseMetaphysik}}
Zudem wird die Fromulierung, dass \enquote{die Wirklichkeit, die wir unmittelbar wahrnehmen, ist aber gar nicht die eigentliche Wriklichkeit, sondern bloß eine Art Schatten.[\dots] Den Dingen, wie sie uns erscheinen, leigt eine ganz andere Wriklichkeit zugrunde, der wir eigentlich zustreben sollten.}\footcite[][S. 23]{DisseMetaphysik}
Diese Fromulierungen von zwei \enquote{Wirklichkeiten} welcher eine die grundlegendere ist, sind dahingehend unsinnig, dass es unmöglich ist zwei Wirklichkeiten aufrechtzuerhalten.

\subsubsection{Transzendenz zweiter Stufe}
Es geht weiter mit der Idee des Guten, die so beschrieben wird, dass sie \enquote{[\dots] gewissermaßen nochmals die bereits transzendente Ideenwelt [transzendiert]}\footcite[vgl.][S. 50]{DisseMetaphysik}
Einen transzendenten Bereich nochmals zu transzendieren macht wenig Sinn. Wie soll dies gelingen? 
Man könnte das Spiel fortführen, indem man die Weise der Transzendenz immer weiterführt und immer eine nächste höhere Ebene einführt, wodruch man nur in einen infiniten Regress geraten würde, auch wenn dies rein begrfflich durchaus möglich wäre, nur ontologisch keine Wertigkeit mehr aufweisen wird. Daher müsste man die IdG irgendwie so konzipieren, dass sie in der Lage ist, ohne selbst zu sein, das Sein und Seiendheit hervorbringt. Dieser ausschlaggebende Punkt wird jedoch nicht geliefert. Dabei bleibt also die Frage, wie die IdG die bereits erste transzendierte Ebene von den Sinnesdingen hin zu den Ideen nochmals überschreiten soll, wenn mit dem Übertritt jegliches Denken nicht mehr auf dieser höchsten Ebene stattfinden wird. 

%Wenn es dann weitergeht mit der Beschreibung und der Interpretation vom Timaios wird es ziemlich schwierig, wie das jetzt zu verstehen ist, weil Disse die Seele in diesen vorher dargelegten Seinsbereich zu legen, so dass die Seele als Bindegleid zwischen Kosmos und Ideenwelt verstanden wird. Dieser Kosmos ist zwar noch als \enquote{Diesseits} der Ideenwelt genannt, aber als Wohnstätte der Götter und der unsterblichen Seelen, von der Sinnenwelt abgetrennt. Das heißt der Absatz: \zitatblock{\enquote{Die Seele gehört damit im Verhältnis zu den Ideen eindeutig noch in den Bereich dieser Welt. [\dots] ihr kommt die Mittelstellung zwischen Kosmos und Ideenwelt zu. Sie gehört zwar dem Kosmos an, ist aber nicht wie die Dinge der Sinnenwelt körperlich, sondern unkörperlich und bildet aufgrund ihrer Unkörperlichkeit zugleich den Kontaktpunkt zur Ideenwelt.}\footcite[vgl.][S. 58]{DisseMetaphysik}} macht in sich wenig Sinn. Auch wie dann weiter die Darstellung der Seelen, die auf dem Rücken des Himmelsgewölbes stehen und hinausschauen, um dabei die Ideenwelt zu schauen, ist sehr fragwürdig. 

\subsubsection{Kantische Begriffe}
Ein vermutlich weiteres Problem hierbei liegt daran, dass man in der Behandlung des Themas folgendermaßen beginnt: \enquote{Den apriorischen Begriffen unseres Geistes korrespondieren entsprechende Gegenstände. Diese Gegenstandswelt interessiert Platon ebenso wie die Frage nach der Quelle der Wahrheit.}\footcite[][S. 97]{Hirschberger}
Was hier deutlich wird, ist, dass kantische Begrifflichkeiten auf platonische Philosophie angewandt wird. Dementsprechend ist fraglich, ob die Verwendung solcher Begriffe auf gleiche Weise überhaupt möglich sind, da alleine schon der Begriff des apriori nicht gänzlich äquivalent verwendet wird. Alleine schon die Beschreibung der Seelenwanderung widerspricht bereits der Auffassung des apriori. Auf diese Problematik sei hier nur hingewiesen, da eine ausführliche Analyse im Rahmen dieser Arbeit nicht möglich ist.\\Es wird sich jedoch damit begnügt, dass Platon so erklärt wird, dass bei Kant nur die Formen a priori sind, bei Platon hingegen auch die Inhalte, wodurch Platon als reiner Rationalist bezeichnet wird.\footcite[vgl.][S.96]{Hirschberger} 
%Nochmal ansehen, was davor geschrieben wird, da es heißt, \enquote{Nur ein mangelder Metaphysik- und Transzendenzbegriff - \enquote{Metaphysik}: das schlechthin unzugängliche \enquote{Jenseitige}- führt zu der Zweiweltentheorie eines totalen Chorismos, wo in Wirklichkeit nur ein modaler gemeint war, eine \enquote{Trennung} des Seins nach seinem Wesen in Gegründetes und Gründendes. Es ist eine Modifizierung, der es ebensosehr auf die Trennung wie auf die Einheit ankam}\footcite[][S. 96]{Hirschberger}
%Hirschberger stellt vorher klar, dass:\zitatblock{\enquote{Die Transzendenz der Idee ist keine totale, sondern nur eine modale. Der erkenntnis-theoretsiche Sinn dieser Begriffe besagt, dass alles Erkennen in der erfahrbaren, raumzeitlichen Welt ein \enquote{Analogismus}, ein Lesen der Sinneswahrnehmung durch Hinbeziehen auf einen urbildlichen Begriff ist}\footcite[][S. 94]{Hirschberger}}
%Fraglich ist allerdings, wie sehr diese Deutung als rein erkenntnistheoretisch oder auch ontologisch gemeint ist. Denn es scheint, als sei diese Deutung primär auf die erkenntnistheoretische Weise eingegangen.
%Ähnliches findet sich auch bei Thurner:\zitatblock{\enquote{Die Frage nach der Möglichkeit des Wissensgewinns führt bei Platon also zu einem ontologischen und anthropologischen Dualismus: Aus der Auffassung, dass die Welt des sinnenfälligen Werdens keine sichere Erkenntnis vermitteln kann, folgert er die Gegebenheit eines welttranszendenten Bereichs rein idealer Wesenswahrheiten, die nur der geistigen Erkenntnis des Denkens zugänglich sind und von der Seele immer schon apriorisch gewusst werden.}\footcite[][S. 99]{ThurnerDualismus}}

%Dies liegt wohl an dem heutigen Verständnis davon, wenn man von Welten - also auch zwei Welten - spricht. Es wird dieses Verhältnis lediglich als räumlich vorgestellt, was automatisch zu der falschen Annahme von zwei Welten neben- oder übereinander führt.
