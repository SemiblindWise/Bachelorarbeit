\section{Vorzug Interpretation B}
\zitatblock{\enquote{Nur ein mangelnder Metaphysik- und Transzendenzbegriff - \enquote{Metaphysik}: das schlechthin unzugängliche \enquote{Jenseitige}- führt zu der Zweiweltentheorie eines totalen Chorismos, wo in Wirklichkeit nur ein modaler gemeint war, eine \enquote{Trennung} des Seins nach seinem Wesen in Gegründetes und Gründendes. Es ist eine Modifizierung, der es ebensosehr auf die Trennung wie auf die Einheit ankam}\footcite[][S. 96]{Hirschberger}}
Mit dieser Formulierung ist deutlich gemacht, wie die Auseinanderdifferenzierung von Ideen und Sinnesdingen aufzufassen ist.
%\subsection{Wichtige Pro-Argumente}
%Gerade die Unmöglichkeit der direkten Unterscheidung der Begrifflichkeiten, die Platon verwendet, wenn es um Ideen (eidos idea) und Dinge geht, lässt zuerst auf die Möglichkeit der verschiedenen Interpretationen schließen.
%Die Frage besteht, ob eine Gleichwertigkeit zwischen der ontologischen und gnoseologischen Aspekten der Deutung liegt. Selbiges gilt ebenfalls für die \enquote{Ideenlehre} selbst. Ist die Frage zulässig, ob man einer der beiden Bereiche eine wichtigere oder zuförderst darzulegende Rolle beimisst? Bedarf es zuerst der Ontologie, welche begründet werden muss, oder doch erst der Gnoselogie?
%Damit sind alle Bestimmungen, welche im ersten Teil dieser Arbeit angeführt worden sind deutlich unter dem Prinzip des Einen gefasst worden, sodass eine Auseinanderdifferenzierung von Ideen und Sinnesdingen auf diese Weise keine Grundlage mehr besitzt.
Es wurde durch die im zweiten Teil der Arbeit enthaltene Interpretation deutlich, dass die Rede von zwei Welten in der Weise einer Abwertung der Sinnesdinge, wie es die erste Interpretation geliefert hat, nicht zulässig ist. Die Überwindung dieser zwei Welten wurde dadurch geliefert, dass das System aus Ideen im Verhältnis zu Sinnesdingen erst über die Idee des Guten und ihrer Rolle des absoluten Einen eingeholt werden konnte. Durch die Idee des Guten ist es möglich die \enquote{Seinsbereiche}, die den Ideen und Sinnesdingen zugesprochen worden ist dahingehend nicht als zwei Welten auszulegen, sondern gerade als einen gemeinsamen Bereich zu verstehen, der nur durch die Idee des Guten transzendiert wird. Dabei wiederholt sich allerdings nicht das gleiche Schema, sondern wird dadurch aufgehoben, dass die Idee des Guten nicht mehr selber Seiend ist und somit nicht mehr demselben \enquote{Seinsbereich} anghört, wie es Sinnesdinge und Ideen tun. Daher wurden alle Bestimmungen, die im ersten Teil angeführt worden sind, unter dem Prinzip des Einen zusammengefasst und zur Vollendung gebracht.\\
Es muss also die Verhätlnisbestimmugn aus Gegründetem und Gründenden ausgeweitet werden und nicht nur auf die Ebene der Sinnesdinge und Ideen bezogen werden, sondern erst auf der Ebene aus Dingen \emph{und} Ideen im Gegensatz zur Idee des Guten, welche damit den ontologischen Übergang markiert, eingeholt werden.
Dieser vermeintlichen Abwertung des Bereichs der Sinnesdinge kommt Thurner so bei, dass er festhält, dass 
%\subsection{Wichtige Contra-Argumente}
\zitatblock{\enquote{[d]ie gesamte Abwertung der sinnlich-körperlichen Dimension des Seins bei Platon (und der platonischen Tradition) in der Unfähigkeit und Angst davor ihren Grund gehabt haben [mag], den leiblichen Tod und das Vergehen als solches anzunehmen.}\footcite[vgl.][S. 99f.]{ThurnerDualismus}}
Dieser Abwertung ist in dieser Arbeit so begegnet worden, dass die \enquote{Ideenlehre} dahingehend aufzufassen ist, dass die \enquote{sinnlich-körperliche Dimension} zusammengenommen mit der geistigen Dimension gegenüber der Idee des Guten steht. Diese Abwertung stammt daher, dass das Missverständnis darin besteht, dass in den Stellen der vermeintlichen Abwertung eigentlich von gnoseologischen Ebenen gesprochen wird, also von dem Unterschied von Meinungen und Wahrheit und keine ontologische Bedeutung hineingetragen wird, was eine Abwertung des einen Bereiches mit sich führen würde.
%Halfwassen Seite 76 in Szlezák: Dagegen spricht eine Reihe von Zeugnissen für eine prizipielle Unterordnung der (griechischer Begriff) unter das Eine, ohne dass sie als das Prinzipat aus dem Einen abgleitet würde, womit im übrigen ihr Status als Prinzip aufgehoben wäre.
%Der Hauptpunkt, der schon in der Erarbeitung klar geworden zu sein scheint, ist, dass gerade mit dem Moment der Transzendenz der Idee des Guten als der Bedingung der Möglichkeit alles Seins und Seienden eine \enquote{zweite Welt} geschaffen worden ist, die Eingangs kritisiert worden ist.
%Schwierig ist der Aspekt des Seinsgrades je nachdem wie sehr der Einheitsgedanke vorzufinden ist. \footcite[vgl.][S. 99f.]{halfwassen2015spuren}
%Alles ist denkbar und seiend, weil es einheitlich ist. 
%Ist das nur die Möglichkeit es zu denken oder auch der Grund dafür?
%\subsection*{Abfederung der Kritik}
%Soll das noch gemacht werden? 
%Die Erkenntbarkeit der Idee des Guten kann nur in Abhängigkeit alles Erkenntnis aller Ideen geschehen. D.h. nur mithilfe der Ideen und deren Prinziphaftigkeit kann die Idee des Guten eingeholt werden, ohne diese direkt schauen zu können. 
