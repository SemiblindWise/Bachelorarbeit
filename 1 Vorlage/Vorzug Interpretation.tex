\section{Welcher Interpretation ist der Vorzug zu geben?}
\enquote{Nur ein mangelder Metaphysik- und Transzendenzbegriff - \enquote{Metaphysik}: das schlechthin unzugängliche \enquote{Jenseitige}- führt zu der Zweiweltentheorie eines totalen Chorismos, wo in Wirklichkeit nur ein modaler gemeint war, eine \enquote{Trennung} des Seins nach seinem Wesen in Gegründetes und Gründendes. Es ist eine Modifizierung, der es ebensosehr auf die Trennung wie auf die Einheit ankam}\footcite[][S. 96]{Hirschberger}
\subsection{Wichtige Pro-Argumente}
Gerade die Unmöglichkeit der direkten Unterscheidung der Begrifflichkeiten, die Platon verwendet, wenn es um Ideen (eidos idea) und Dinge geht, lässt zuerst auf die Möglichkeit der verschiedenen Interpretationen schließen.
Die Frage besteht, ob eine Gleichwertigkeit zwischen der ontologischen und gnoseologischen Aspekten der Deutung liegt. Selbiges gilt ebenfalls für die \enquote{Ideenlehre} selbst. Ist die Frage zulässig, ob man einer der beiden Bereiche eine wichtigere oder zuförderst darzulegende Rolle beimisst? Bedarf es zuerst der Ontologie, welche begründet werden muss, oder doch erst der G
\subsection{Wichtige Contra-Argumente}
\enquote{Die gesamte Abwertung der sinnlich-körperlichen Dimension des Seins bei Platon (und der platonischen Tradition) mag in der Unfähigkeit und Angst davor ihren Grund gehabt haben, den leiblichen Tod und das Vergehen als solches anzunehmen.}\footcite[][S. 99f.]{ThurnerDualismus}
Halfwassen Seite 76 in Szlezák: Dagegen spricht eine Reihe von Zeugnissen für eine prizipielle Unterordnung der (griechischer Begriff) unter das Eine, ohne dass sie als das Prinzipat aus dem Einen abgleitet würde, womit im übrigen ihr Status als Prinzip aufgehoben wäre.
Der Hauptpunkt, der schon in der Erarbeitung klar geworden zu sein scheint, ist, dass gerade mit dem Moment der Transzendenz der Idee des Guten als der Bedingung der Möglichkeit alles Seins und Seienden eine \enquote{zweite Welt} geschaffen worden ist, die Eingangs kritisiert worden ist.
\subsection*{Abfederung der Kritik}
Die Erkenntbarkeit der Idee des Guten kann nur in Abhängigkeit alles Erkenntnis aller Ideen geschehen. D.h. nur mithilfe der Ideen und deren Prinziphaftigkeit kann die Idee des Guten eingeholt werden, ohne diese direkt schauen zu können. 
