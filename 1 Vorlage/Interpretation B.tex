\section{Interpretation B dieser Rede: Es git bur eine Welt}
Das Grundproblem, das bei der Trennung der beiden Welten auftritt, spielt sich auf dem Kampfplatz der Dialektik ab. Hierbei hällt Perls fest, dass \enquote{Trennung und Teilnahme die zwei Hauptbegriffe der Dialektik [sind und nur] Getrennt werden [kann], was vorher zusammengesetzt war.}\footcite[vgl.][S. 349]{Perls}
Hierzu wird die Stelle Phaidon 78 c herangezogen, in denen es heißt:
\enquote{Und nicht wahr, dem was man zusammengesetzt hat und was seiner Natur nach zusammengesetzt ist, kommt wohl zu auf dieselbe Weise aufgelöst zu werden wie es zusammengesetzt worden ist; wenn es aber etwas unzusammengesetztes gibt, diesem wenn sonst irgend einem kommt wohl zu, dass ihm dieses nicht begegnet?} (Phaidon 78 c)
Diese Stelle steht dem Bisherigen so gegenüber, dass es unzulässig wäre die Sinnenwelt von der Ideenwelt so abzutrennen, dass man diese beiden Bereiche erst wieder zusammenfügen dürfe.
Somit muss das Verhältnis zwischen Ideen und Sinnesdingen nochmal von neuem betrachtet werden, da damit im Grunde genommen eine Umkehr der Herangehensweise gefordert ist, also dass zu Beginn nicht die Trennung angestrebt wird, sondern im Eingang gerade auf die Zusammensetzung, bzw. auf die Einheit aus Beiden, geblickt werden muss, um dann eine Trennung vornehmen zu können.
Um dieser Interpretation nun folgen zu können, wird im Nachfolgenden die Auffassung von Jens Halfwassen dargestellt, der sich mit der Darstellung Platons und der Auslegung desselben auf den Einheitsgedanken der gesamten Ideenlehre beschäftigt hat.
%Es wird an der Stelle eigentlich bereits deutlich, wenn in der Politeia in die drei Gleichnisse eingeleitet wird. Dies geschieht einzig und alleine über die Sinne. Also wie kann die Sinnenwelt als minderwertig angesehen werden, wenn sie doch für die Erarbeitung und Erklärung für die Ideenwelt als erstes herangezogen wird.
%Wenn diese beiden Welten so voneinander getrennt sein mögen, dann wäre es doch sicherlich nicht ratsam auf diese Weise in die Lehre über die Ideen auf diese Weise einzusteigen. Es sind also die Sinnesdinge und deren Wahrnehmung, welche eigentlich nicht für die wahre Erkenntnis geeignet ist, die den Beginn und auch den einfachsten Zugang (ob es auch der einzige Zugang ist, ist noch fraglich) kennzeichnen, was nur heißen kann, dass man diese \enquote{Welt}, wie man am Ende sehen wird, nicht einfach aufgeben kann und sich somit nur noch in der Ideenwelt aufzuhalten sucht.


\section*{Wie wird das von Autoren formuliert}
Um diesen Weg der Interpretation nachvollziehbar zu machen, also noch bevor die einzelnen Originalstellen wiederholt betrachtet werden, gilt es dem Prinzip der Einheit Vielheit Relation zuzuwenden, da Halfwassen dieses Verhältnis als Ausgangspunkt nimmt, um über die platonische Metaphysik auszuführen. Darum wird es zu zeigen sein, dass die Einheit nicht als ein noch vor das Seiende gestllte Prinzip ist.\footcite[vgl.][S.99]{halfwassen2015spuren}

Wichtig für die Erarbeitung ist allerdings anzuführen, wie Halfwassen den Begriff der Transzendenz fasst. Diesen Begriff fasst er bei Platon in eine schwache und eine starke Variante. Die schwache wäre das, was im vorigen Kapitel dargelegt worden ist, was er als \enquote{graduelle Transzendenz} bezeichnet und das Übergangsverhältnis der jeweils ursprünglichen Seinsstufen zu den von ihr abgeleiteten und ontologisch abhängigen Stufen meint.\footcite[vgl.][S. 29]{halfwassen2015spuren} Wichtig dabei ist, \zitatblock{\enquote{dass das Denken in der Lage ist das Transzendente und das von ihm Transzendierte zu einer Einheit zusammenzufassen, indem es das größere Ganze in den Blick nimmt, das diese \emph{und} jene Seite, Begründetes und gründenden Grund gleichermaßen umfasst.}\footcite[][S. 29]{halfwassen2015spuren}} Damit ist gemeint, dass dieses Denken, so wie es beschrieben wird, die Sinnesdinge und die Ideen in einer Einheit zusammenzudenken. D.h. obwohl die Ideen die Sinnesdinge auf eine gewisse Weise transzendieren, das Denken dennoch in der Lage ist, diese beiden Bereiche in einem Denken zu können, was bereits an dieser Stelle schon mehr ist, als es die vorherige Dartellung geliefert hat.
Dem gegenüber steht die starke Transzendenz, die eben \enquote{[\dots] nicht mehr mit dem, was sie transzendeirt, in die gemeinsame Sphäre eines Überstiegenes und Übersteigendes gleichermaßen umfassenden Ganzen zusammengefasst werden kann.}\footcite[vgl.][S. 29]{halfwassen2015spuren} Hiermit wird später noch die Idee des Guten gemeint sein, da auch auf die Stelle Pol. 509b referiert wird, in der das Wesen der IdG weit höher ist und als Quelle von Erkenntnis und Wahrheit an Herrlichkeit über den Ideen und Sinnesdinge steht.

\subsubsection*{Auf den Spuren des Einen Jens Halfwassen  75/BF 1495 H169}
Seite 91f Platons Metaphysik des Einen.\\
Siehe \footcite[][]{halfwassen2015spuren}
Drei Prinzipien in \enquote{Aufstieg zum Einen Kapitel I und II erläutert:} \zitatblock{1. Jedes seiende existiert als dasjenige, was es jeweils ist, genau aus dem Grunde, weil es Eines ist.\\2. Die Gesamtheit aller einzelnen Seienden bildet die Einheit eines Ganzen. Einheit charakterisiert also nicht nur jedes einzelne Seiende, sondern ebenso die Totalität des Seins.\\3. Das Prinzip der Einheit des Ganzen und zugleich der Einheit jedes einzelnen Seienden ist \emph{das Eine selbst}. Als der Einheit-verleihende Ursprung ist das Eine das Absolute, durch das alles Seiende Eines und kraft seiner Einheit auch seiend ist.\footcite[][S. 91]{halfwassen2015spuren}}

\zitatblock{\enquote{Platonische Dialektik ist der Versuch, die Vielfalt der grundlegenden Voraussetzungen unserer denkenden Bezugnahme auf Wirklichkeit auf einen einzigen absoluten Urgrund zurückzuführen: sie ist also die Suche nach dem Absoluten als dem unbedingten Ursprung und Urgrund des Ganzen der Wirklichkeit.}\footcite[][S. 95]{halfwassen2015spuren}}
Das Wesen des Guten ist für Platon \emph{die reine Einheit}. Die Wesensbestimmung des Absoluten als reine Einheit ist grundlegend für Platons Prinzipientheorie, die darum den Charackter einer Metaphysik des Einen hat. Erst von ihr aus lässt sich auch verstehen, wie das Absolute Sein und Wassein, Erkennen und Erkennbarkeit zugleich begründet.\footcite[vgl.][S. 96]{halfwassen2015spuren}(Fußnote zu Krämer S. 474ff, 535-551)
\enquote{Die Bestimmung des Absoluten als reine Einheit, die absolute Transzendenz des Einen selbst und die Ansetzung eines eigenen Prinzips für die Vielheit sind die drei grundlegenden Thesen der Platonischen Metaphysik des Einen.}\footcite[][S. 96]{halfwassen2015spuren}

\enquote{[\dots] [D]ie Einheit ist die grundlegende Bedingung für das Sein und die Denkbarkeit alles Seienden.}\footcite[vgl.][S. 97]{halfwassen2015spuren} Politeia 478B12f und Parm 144C4-5
\zitatblock{\enquote{Auch das Gegenteil des Einen, das Viele, denken wir immer schon und notwendig als Einheit, nämlich als geeinte Vielheit und das bedeutet als einheitliches Ganzes aus vielen elementaren Einheiten, so dass der Gedanke des Vielen in doppelter Weise Einheit voraussetzt.}\footcite[][S. 97]{halfwassen2015spuren}} Parm 157C-158B
Bedeutung von Einheit, in der Vielheit enthalten ist, wie Ganzheit, Einheitlichkeit als Einheit in der Vielheit oder Identität\footcite[vgl.][S. 97]{halfwassen2015spuren} Soph 254D Parm 139D 4-5
Auch der Gedanke von Werden, Nichtsein und Nichts wird unter dem Gedanken des Einen gefasst \footcite[vgl.][S. 97]{halfwassen2015spuren}
Damit sind alle Bestimmungen, welche im ersten Teil dieser Arbeit angeführt worden sind deutlich unter dem Prinzip des Einen gefasst worden, sodass eine Auseinanderdifferenzierung von Ideen und Sinnesdingen auf diese Weise keine Grundlage mehr besitzt. 
\zitatblock{\enquote{Denn wer meint, die Wirklichkeit könne auch aus einer einheitlosen Vielheit unverbundener Einzeldinge bestehen, der nimmt eben damit Denkbestimmungen wie Wirklichkeit, Vielheit und Einzelnes als realitätshaltig in Anspruch und setzt somit genau das voraus, was er bestreiten will, nämlich die Einheit von Denken und Sein}\footcite[][S. 98]{halfwassen2015spuren}}

\zitatblock{\enquote{Alles Seiende und Denkbare ist also nur darum seiend und denkbar, weil es einheitlich ist, und zwar in der Weise, daß sein Charakter als Einheit die Grundlage seiner Denkbarkeit [\dots] bildet. Daraus folgt zugleich, dass Einheit das Kriterium der Unterscheidung von Sein und Nichtsein ist und der Maßstab, an dem Seiendes von höheren und geringeren Seinsgrad messbar wird.}\footcite[vgl.][S. 99]{halfwassen2015spuren}}
Hier wird von verschiedenen Seinstufen gesprochen, die auf dem Kriterium der Einheit besteht. Welche Stufen sind hiermit gemeint?
\zitatblock{\enquote{Wenn ferner die Einheit von etwas der Grund seines Seins ist, dann ist jedes etwas auch in dem Grade seiend, indem es Eines ist. Je einheitlicher etwas ist, desto seiender (mallon on, Politeia 515 D 3) ist es dann auch. Erst sein henologischer Ansatz erlaubt Platon die Graduierung von Sein, die der Eleatismus noch nicht kennt. Einheit als Grund des Seins generiert den ontologischen Komparativ und damit die Grundlage der Ideenlehre, der zufolge die einheitliche Wesenheit von etwas seiender ist als ihre vielen individuellen Instanziierungen, und zwar genau darum, weil die eine Schönheit selbst oder die eine Gerechtigkeit selbst den vielen Fällen erscheinender Schönheit oder Gerechtigkeit als die diese Vielheit begründende Einheit zugrunde liegt (Politeia 476 A, 479 A — 480 A, 507 B).}\footcite[][S. 99f.]{halfwassen2015spuren}}
Über das Eine selbst kann man nichts mehr aussagen, wie über alles weitere Seiende, da es selber nicht mehr bestimmt ist und eigentlich nur das Prinzip für die Bestimmung für alles weitere ist. Würde man über das Eine selbst so zu sprechen versuchen nimmt man es wiederum aus seiner selbst heraus, da das Denken diese Prinzipien braucht, um etwas zu denken, wie es bisher erklärt worden ist. Ausführung in Parmenides 137C-142A oder Aufstieg zum Einen 282ff und Kapitel XI:
\zitatblock{\enquote{Wird das Eine nur in sich selbst betrachtet, dann weist es als reine Einheit jedwede Bestimmung strikt von sich ab; es steht als solches jenseits aller Bestimmungen, weil jede denkbare Bestimmung es in die Vielheit hineinziehen würde. Man kann darum nichts von ihm aussagen, noch nicht einmal, daß es ist oder daß es Eines ist, weil es damit bereits eine Zweiheit wäre (141 E); die duale Struktur der Prädikation verfehlt prinzipiell die reine Einfachheit des Absoluten. Platon spricht dem absolut Einen darum systematisch alle Fundamentalbestimmungen ab, auch Sein, Einssein, Erkennbarkeit und Sagbarkeit}\footcite[][S. 101]{halfwassen2015spuren}}
Diese Formulierung geht genau auf Pol. 509b zurück, dass die IdG noch an Herrlichkeit über dem Seienden steht. 
%Damit ist allerdings der Schritt weg von dem Thema gemacht worden und man befindet sich hier in der reinen Metaphysik und der Transzendenz des Einen. 
Damit lässt sich aber sagen, dass das Eine die Seins- und Erkenntnistranszendenz aufweist. \enquote{Das Eine selbst ist ebensosehr jenseits des Geistes und der Erkennnis, wie es jenseits des Seins ist; es übersteigt den Zusammenhang von Denken und Sein, indem es das Prinzip dieses Zusammenhangs ist; und es begründet ihn gerade Kraft seiner Transzendenz.}\footcite[][S. 102]{halfwassen2015spuren}\\
Es folgt das zweite Prinzip, das Prinzip der Vielheit

Um das Seiende in seiner Vielheit ableiten zu können, braucht es ein Prinzip nach dem Einen, welches erfüllt, wenn es spezifisch eben die Vielheit generiert, was die reine Einheit absolut von sich ausschließt. \footcite[vgl.][S. 103]{halfwassen2015spuren} Dieses Viele ist noch kein bestimmtes, sondern ist \emph{vorseiend}, aber nicht überseiend wie das Eine.\footcite[vgl.][S. 103]{halfwassen2015spuren} Dies ist auch die \enquote{unbestimmte Zweiheit}.
Hier liegt jetzt der übergreifende Punkt und die Überleitung zur \enquote{Ideenlehre}. Aus diesen beiden Prinzipien gehen alle ontologischen Fundamentalbestimmungen hervor, welche die Struktur des Ideenbereichs und damit die ganze Welt des Seienden konstituieren.\footcite[vgl.][S. 104]{halfwassen2015spuren}

\enquote{Darüber hinaus ist auch das Zusammenwirken der beiden Prinzipien eine Form von Einheit und bedarf eines einigenden Grundes (vgl. Philebos 27 B mit 30 AB; Aristoteles, Metaphysik 1075 b 17—20), der nur das Eine selbst sein kann. Ferner konstituieren die Prinzipien das Sein als die Einheit eines Ganzen (Ev öXov, Parmenides 157 E 4, 158 A 7), als das seiende Eine, in dem alle entfaltete Vielheit einbegriffen bleibt}\footcite[][S. 106]{halfwassen2015spuren}
Es bleibt also noch ein übergreifenderes Prnizip, das die Einheit aus der Einheit und der Vielheit bildet. Hier heißt es, dass es wieder das Eine selbst sein muss.
{\color{red}{Übergang zu Politeia}}
Die aufschlussreichste Passus des Dialogwerks ist die Gleichnissequenz im 6. und 7. Buch der Politeia mit anschließenden Ausführungen über das Verhältnis von mathemtischer und Propädeutik und Dialektik.\footcite[vgl.][S. 135]{halfwassen2015spuren} Mit Verweis auf Krämer Arete bei Platon und Aristoteles 135-145, 473-480, 533ff
\enquote{[\dots] warum das Gute als letztes Seins-, Erkenntnis- und Wertprinzip selbst noch jenseits des Seins stehen muss.}\footcite[vgl.][S. 136]{halfwassen2015spuren} Mit Verweis auf Politeia 509b, Parmenides 141e und Aufstieg zum Einen 19ff, 188ff, 221ff, 257ff, 277ff, 302ff und 392ff. 
\enquote{Denn schon die Aussage, dass das Eine \emph{ist}, enthält ja eine Zweiheit: nämlich die Zweiheit von Einheit und Sein, aus der sich alle anderen Grundbestimmungen des Seienden ableiten lassen, wie die 2. Hypothesis des \emph{Parmenides} lehrt.}\footcite[][S. 136f.]{halfwassen2015spuren}

\enquote{Die Aussagen der \emph{Politeia} sprechen darum entscheidend für eine monistische Deutung der Prinzipienlehre}\footcite[][S. 137]{halfwassen2015spuren}
\zitatblock{\enquote{Eine monistische Deutung der Prinzipienlehre kann freilich von vornherein keine \emph{Eliminierung} der sie durchgehen bestimmenden Bipolarität bedeuten, sondern nur ihre \emph{Relativierung} insofern, als das Vielheitsprinzip dem Einen nicht gleichursprünglich und gleichmächtig gegenüberstehen kann.}\footcite[][S. 138]{halfwassen2015spuren}}


\subsubsection*{Der Aufstieg zum Einen 75/BF 2371 H169 Halfwassen}
Es ist zwar mit Plotin verbunden, bzw. daher kommt die Frage des Buches, aber ab Seite 220 geht es mit Platons Transzendenz los, aber auch die Einleitung ist sehr gut.
Stufen der Einheit bei Plotin S. 41
Seinstranszendenz, Geisttranszendenz, Erkenntnistranszendenz S. 150ff
Der Text von Plotin und schaut auf die Grundlegung der Einheitsmetaphysik bei Platon im Parmenides. 
\enquote{Er unterscheidet voneinander das Erste Eine, das schlechthin und absolut Eine, das Zweite, welches er \enquote{Eines Vieles} nennt, und das Dritte, \enquote{Eines und Vieles}; so stimmt er ebenfalls überein mit der Lehre von den drei Wesenheiten.}\footcite[][S. 187f.]{halfwassenaufstieg2006}
Dabei wird auf die ersten drei Hypothesen des Parmenides verwiesen. Es wird festgehalten, dass \enquote{[\dots] das absolute Eine als Urgrund des Seins und des Denkens sowie ihres Zusammenhangs notwendig selbst jenseits aller Bestimmungen des Seins und des Denkens [ist]. (Parm. 137c-142a)}\footcite[vgl.][S. 188f.]{halfwassenaufstieg2006}
\zitatblock{\enquote{Erst die Seinstranszendenz des einheitsstiftenden Absoluten macht das [\dots] Ineins von Einheit und Vielheit denkbar; das Sein hebt als die Totalität aller Bestimmungen in sich jede Vielheit in die Einheit auf, ohne sie in Unterschiedslosigkeit untergehen zu lassen, erhält sie also zugleich; es ist die Einheit der Bestimmungen, auch der entgegengesetzten, die in ihm koinzidieren.(Parm. 142b-155e)}\footcite[vgl.][S. 189]{halfwassenaufstieg2006}}
Und die dritte Hypothese:
\zitatblock{\enquote{Das als in sich dialektisch bewegte Einheit und d.h. als Geist gedachte Sein aber bleibt nicht [\dots] in sich selbst verschlossen, sondern es setzt eine weitere Entfaltungsstufe der Eiheit aus sich heraus: die Seele faltet die im Geist in Einheit eingefaltete Vielheit und Andersheit diskursiv und d.h. zugleich zeitlich sukzessiv aus in eine Vielheit ontisch distinkter und je in sich schon vielhältiger Einheiten, deren umgreifende, unterscheidende und zugleich verbindende Einheit sie selbst ist.(Parm. 155e5)}\footcite[][S. 189]{halfwassenaufstieg2006}}

Dieses Prinzip ist \enquote{als Strukturprinzip der Ideenwelt entfaltet und zugleich als Einfaltung aller Ideen unentfaltet; es wird nur im die Verstandesgegensätze übersteigenden noetischen Denken erkannt.}\footcite[vgl.][S. 190]{halfwassenaufstieg2006}
Zusammengefasst wird das Eine in seiner dreifachen Weise als \enquote{absolute, noetisch-komplikative und dianoetisch-explikative Einheit}\footcite[][S. 190]{halfwassenaufstieg2006}
Hierin liegt das Verständnis dessen, wie man die Transzendenzstufen des Liniengleichnis und Höhlengleichnis zu deuten hat. Denn dieses Urprinzip lässt sich auf alle darunterliegenden Stufen ableiten, abgesehen davon, dass man von der förmlichen Transzendenz absehen müsste. 
Daher wird auch mit dem Aufstieg und der Transzendenz der Idee des Guten weitergemacht.
\zitatblock{\enquote{Das Sonnengleichnis beschreibt [die Seinstranszendenz des Absoluten] auf der Grundlage der eleatischen Unterscheidung von Sein und Erscheinungswelt als doppelte Transzendenz und legt damit den zweifachen Überstieg über die Erscheinung zum Seienden und über das Seiende im ganzen zum Absoluten als das Bewegungsgesetz der Platonischen und neuplatonischen Metaphysik fest.}\footcite[vgl.][S. 222]{halfwassenaufstieg2006}}
Das Überschreiten von der gegebenen welthaften Wirklichkeit zum wahrhaft oder eigentlichen Seienden wird als erste Transzendenz beschrieben. \footcite[vgl.][S. 222]{halfwassenaufstieg2006}.
Der intelligible Bereich der Ideen, der durch die Dialektik erforscht wird, ist somit ein untereinander einiges, in sich selbst vielfältig gegliedertes Ganzes (Struktur der Einheit \emph{in} Vielheit)\footcite[vgl.][S. 223]{halfwassenaufstieg2006} Dieser Bereich wird allerdings noch durch eine Letztbesitmmung, die noch über diesen Bereich hinausgeht in einem einzigen Prinzip, in dem Prinzip aller Einheit, der Idee des Guten, überschritten. Dies ist damit auch die zweite Transzendenz.\footcite[vgl.][S. 223f.]{halfwassenaufstieg2006}
Siehe hierzu auch die eingangs gegebenen Unterschied der Transzendenz
\enquote{Platon fast damit die Transzendenz des Absoluten als \emph{absolute ontologische Transzendenz}}\footcite[][S. 224]{halfwassenaufstieg2006}

Hier kommt der Sprung dahin, dass sich die Interpretation hin zu einer Welt festigt. 
\zitatblock{\enquote{So \emph{wissen} wir das Absolute, das aller Erkenntnis den Grund gibt, gerade weil Es selbst jenseits aller Erkenntnis ist, nur im \emph{Nichtwissen} - freilich in einem Nichtwissen, das sich selbst \emph{als} Nichtwissen weiß und das sich darum nur durch das Wissen des Wissbaren hindurch erreicht, indem es dieses transzendiert. Alles Denken und Sprechen über das absolut Transzendente muss sich darum ständig selbst widerrufen und ins Unsagbare aufheben 
%dies ist der Sinn der \enquote{negativen Theologie}, deren Begründer Platon ist.
.}\footcite[][S. 225]{halfwassenaufstieg2006}}

Diese Beschreibung der Überschreitung der Ideen geht so weit, dass selbst die absolute Transzendenz der Idee des Guten noch mit eingeholt wird und somit dargestellt werden und verstanden werden kann, was die anderen Interpretationen von zwei Welten nicht möglich gemacht haben.

Hier wird dann auch deutlich, wie Halfwassen diesen Aufstieg versteht. Denn er deutet es so, dass es rein um das höchste Erkenntnisziel geht, also rein gnoseologsich\footcite[vgl.][S. 226]{halfwassenaufstieg2006}
\enquote{Denn zum Menschen gehört es, das gemäß der Idee Gesagte zu verstehen, indem er zu dem geht, was aus vielen Wahrnehmungen durch das Denken zu einer Einheit zusammengefasst wird. Platon versteht somit menschliche Erkenntnis prinzipiell als auf Einheit gerichtete Synthesis und Synopsis}\footcite[][S. 228]{halfwassenaufstieg2006}
\enquote{Denn jede \emph{Einheit in der Vielheit} - also jede Idee - und jedes \emph{Beziehen von Vielheit auf Einheit} - also jede Erkenntnis - setzt die \emph{reine Einheit} als absolut Einen immer schon voraus.}\footcite[][S. 230]{halfwassenaufstieg2006}
\enquote{erst auf dem Weg der auf- und absteigenden Dialektik bekommt man alle Ideen \enquote{so schön wie irgend möglich zu Gesicht}, weil die Elemente ihres Wesensaufbaus voneinander abgehoben und so dieser Wesensaufbau vom universalen Urgrund her vollkommen durchsichtig geworden ist.}\footcite[][S. 231]{halfwassenaufstieg2006}
Somit gelangt in das Höhlengleichnis noch eine tiefere Bedeutung hinein, die auch die zweimalige Umkehr erklärt, wenn man \enquote{Aus der Höhle herausgekommen ist} und die wahren Dinge gesehen hat, man aber wieder zurück in die Höhle gehen soll. Warum hier allerdings von Umweg gesprochen wird ist fragelich. Womöglich ist mit Umweg nicht eine längere Strecke gemeint, die auf eine Art Irrweg deutet, sondern mehr eine Umwendung, die man erst erkennt, wenn man den ersten Weg nach oben gegangen ist, aber, um diesen Weg zu vollenden, erst mit dem erneuten Abstieg den Weg vollendet hat. 
%Von daher versteht sich Platons ethische Forderung \enquote{aus Vielen Einer zu werden}(Pol. 443E1); die innere Einheitlichkeit der Seele und ihres Abbildes, der Polis, ist das Werk der höchsten ethischen Arete, der Gerechtigkeit\footcite[vgl.][S. 237]{halfwassenaufstieg2006}
Die Idee des Guten nicht nur als die Prinzip der Arete, sondern auch als Prinzip des Seins und der Erkenntnis\footcite[vgl.][S. 238]{halfwassenaufstieg2006}
Die Idee des Guten ist dafür verantwortlich, dass \enquote{den durch die Vernunft erkennbaren Dingen von dem Guten nicht nur das Erkanntwerden zuteilwird, sondern dass ihnen dazu noch von jenem das Sein und die Wirklichkeit zukommt.}(Pol. 509b-c) (Warum trennt er hier neben dem Erkanntwerden auch noch Sein von Wirklichkeit?)
\enquote{Das Eidos ist dialektisch bestimmbar, weil es eine Vielheit von eidetischen Bestimmtheiten in der Einheit eines Ganzen zusammenfasst: es hat den Charakter einer \enquote{Einheit aus Vielem}}\footcite[][S. 240]{halfwassenaufstieg2006}
\enquote{Darum kann die Gerechtigkeit, das Ordnungsprinzip von Polis und Seele, in analoger Weise auch die intelligible Ordnung der Ideen selbst charakterisieren.}\footcite[][S. 242]{halfwassenaufstieg2006}
% Die Entfaltung der Einheit in die Fülle des Seins ist Schönheit, die einende Ordnung der Vielen in einem einigen Ganzen ist Gerechtigkeit; beide erscheinen im richtigen Verhältnis des Ganzen und der Teile, also im Gesetz der geometrischen Gleichheit\footcite[][S. 242]{halfwassenaufstieg2006}
% Das Vielgestaltige und Veränderliche, das sich bald so und bald anders zeigt, erfassen wir im sinnlichen Sehen; das Einheitliche und Unveränderliche, das immer dasselbe Wassein sehen läßt, zeigt sich nur im die Sinneswahrnehmung transzendierenden reinen Denken (507 B 9-10). Entsprechend unterscheidet Platon eine Welt des Intelligiblen (vonios topos, 508 C 1, 517 B 5, vgl. C 3) und eine Welt des Sinnenfälligen oder Sichtbaren (opatos Toлos, vgl. 508 C 2, 517 C 3) als zwei Arten des Seienden (do con tan övtav, Phaid. 79 A 5; Politeia 509 D 1-3: dvo αὐτὰ εἶναι . . . τὸ μὲν νοητοῦ γένος τε καὶ τόπος, τὸ δ' αὖ ὁρατοῦ). Dic sichtbare Welt aber ist das Abbild der intelligiblen Welt (vgl. Tim. 29 A- B), sie hat nur in der Teilhabe an dieser Sein, darum entsprechen sich die Strukturen beider Welten in strenger Analogie.\footcite[][S. 246]{halfwassenaufstieg2006}

\enquote{Entsprechend unterscheidet Platon eine Welt des Intelligiblen und eine Welt des Sinnenfälligen oder Sichtbaren als zwei Arten des Seienden. Die sichtbare Welt aber ist das Abbild der intelligiblen Welt, sie hat nur in der Teilhabe an dieser Sein, darum entsprechend sich die Strukturen beider Welten strenger Analogie.}\footcite[][S. 246]{halfwassenaufstieg2006}
\enquote{Die Sonne ist aber Bild und Analogon des Agathon: Die Sonne ist in der Dimension des Sichtbaren im Verhältnis zum Sehen und zum Gesehenen das, was das Gute selbst in der Sphäre des Intelligiblen in Bezug zum Geist und zum Gedachten ist (508)}\footcite[][S. 250]{halfwassenaufstieg2006}
\enquote{Denn indem das \enquote{intelligible Licht} den Ideen Einheitscharakter verleiht, hebt es sie in ihrer je eigenen Bestimmtheit voneinander ab und verbindet sie zugleich zur Einheit des kosmos noetos und macht sie dadurch erst intelligibel}\footcite[][S. 252]{halfwassenaufstieg2006}
\enquote{Erkennen bedeutet ferner die Zurückführung der Vielheit in die Einheit ihres Grundes (Phaidr. 249B6-C1)}\footcite[][S. 252]{halfwassenaufstieg2006}
\zitatblock{\enquote{Der Einheitsgrund der Ideen des Denkens, das Gute als das absolute Eine, ist mithin 
\begin{itemize}
    \item (1) das Prinzip der Erkennbarkeit der Ideen;
    \item (2) das Prinzip der Erkenntniskraft des Nous;
    \item (3) das Prinzip des aktualen Wissens, d.h. aber der Einheit von Denken und Sein im Erkenntnisakt
\end{itemize}}\footcite[][S. 253]{halfwassenaufstieg2006}}
\enquote{Wahrheit und Wissen oder Einsicht aber sind Weisen der Einheit in der Vieleit, in denen sich das absolute Eine manifestiert wie die Sonne in dem von ihr ausstrahlenden Licht.}\footcite[][S. 252]{halfwassenaufstieg2006}
% Der argumentative Übergang zu dem Schluss hier ist mir nicht ganz klar geworden aus dem Text.
\enquote{Wahrheit, Wissen und Nous haben also ihre Bestimmtheit in der \emph{Einheit} von Denken und Sein, deren Urgrund das Gute selbst ist.}\footcite[][S. 257]{halfwassenaufstieg2006}


\zitatblock{\enquote{Sowenig aber das Licht oder das Sehen selbst die Sonne ist, sowenig ist die Wahrheit oder das Wissen selbst das Gute und das Eine selbst; beide verdanken ihre vereinigende Kraft vielmehr einem jenseitigen Ursprung. Denn Wissen und Wahrheit haben zwar Einheitscharakter- und sind darin \enquote{von der Art des Agathon} -, jedoch so, daß sie ihre Einheit gerade in der Vielheit und Unterschiedenheit dessen haben, was in ihnen geeint ist wie ja auch das Licht die Einheit und Selbigkeit der Helle in der Vielheit des in ihr Aufscheinenden ist oder wie das Sehen die Einheit des Sehaktes in der Zweiheit von Sehendem und Gesehenem ist. Ebenso setzt die Einheit des Wissens die Zweiheit von Wissendem und Gewußtem voraus, deren Einheit als der Bezug Unterschiedener das Wissen ist. Und die Wahrheit gründet als Unverborgenheit in der eidetischen Differenziertheit des reinen Seins, zuletzt also in der zahlenhaften Struktur des Ideenkosmos. Als Einheit in der Vielheit aber vermag weder das Wissen noch die Wahrheit die Einheit von Sein und Denken, in der sie beide ihr Wesen haben, von sich her zu begründen. Das Gute selbst, das als Prinzip der Einheit von Sein und Denken Wahrheit und Wissen allererst ermöglicht, ist darum nicht mit ihnen identisch, sondern liegt notwendig als reine absolute Einheit über sie beide hinaus}\footcite[][S. 257]{halfwassenaufstieg2006}}
\enquote{Indem es allem Seienden Einheitscharakter verleiht, ist das absolute Eine aber nicht nur das Prinzip der Erkenntnis und der Erkennbarkeit, sondern auch das Prinzip des Seins und Wesensfülle aller Seienden}\footcite[][S. 258]{halfwassenaufstieg2006}
Fußnote: Vgl. Parm. 157 B- 158 D, 165 E-166 C, Soph. 245 AB; dazu die Referate Arist. Metaph. A 6 987 b 21, 988 a 11; Alexander, In Metaph. 56, 30 r. H.; Sextus Emp., Adv. Math. X 260-261, 277. Speusipp bei Proklos, In Parm. VII 40, 1 ff. 501, 62 ff. Steel, Simplikies, In Phys. 454, 15 (Porphyrics) und 455, 6 f. (Alexander)
\enquote{Das Agathon - das absolute Eine selbst - ist als \emph{Urgrund} des Seins und der Seiendheit selbst \emph{jenseits} des Seins und der Seiendeit; durch sein über das Sein hinausliegendes Übermaß an Mächtigkeit stiftet es Sein, Seiendheit und Erkennbarkeit in eins und zumal, indem es den Ideen - und durch sie allem Seienden - Einheitscharakter und damit zugleich Identität, feste Bestimmtheit, Abgrenzung von Anderem und Fürsichsein verleiht.}\footcite[][S. 258f.]{halfwassenaufstieg2006}
Da Sein wesenhafte Bestimmtheit bedeutet und damit schon eine Zweiheit ist: es lässt sich auseinanderlegen in etwas und das, was es ist, also in Bestimmtes und Bestimmendes, liegt das absolute Eine selbst über alle Bestimmtheit hinaus und ist notwendig \emph{jenseits des Seins und der Seiendheit}\footcite[vgl.][S. 259f.]{halfwassenaufstieg2006}
\enquote{Das Denken muss darum im Aufstieg zum Absoluten alle Bestimmungen des reinen Seins transzendieren; das Ziel des ganzen dialektischen Aufstiegs: die \enquote{Aufhebung der Voraussetzungen} (vgl. Pol. 533c8-9)}\footcite[][S. 263]{halfwassenaufstieg2006}

\subsubsection*{Szlezák Platonsiches Philosophieren}
\subsubsection*{Der Urspurng der Geistmetaphysik Halfwassen}
Rekapitualtion des Buches von Krämer mit demselben Titel. Was hat sich in den 35 Jahren zwischen diesen beiden Büchern getan? \footcite[vgl.][S. 50]{HalfwassenGeistmetaphysik}

Endlich hebt auch für Platon und Speusipp die Bipolarität der Prinzipien die Absolutheit des Einen keineswegs auf, ist auch hier das zweite Prinzip kein zweites Absolutes, sondern nur die Entfaltungsbasis des Absoluten.\footcite[][S. 53]{HalfwassenGeistmetaphysik}

\zitatblock{\enquote{Auf der Grundlage der skizzierten Henologie basiert Plotins Metaphysik des Geistes, die zwei große Themenkomplexe umfaßt, nämlich erstens die Konstitution des Geistes in seinem Hervorgang aus dem überseienden Einen und zweitens die immanente Struktur der Selbstbeziehung des absoluten Denkens, dessen Inhalte die reinen Wesenheiten des Seienden, also die Ideen sind, in deren untrennbarer Einheit sich der Geist intellektuell selbst anschaut}\footcite[][S. 54]{HalfwassenGeistmetaphysik}}

die dritte (in Plotins Zählung die vierte) Hypothesis des >>Parmenides<<, in der Platon zeigt, wie die von sich selbst her unbestimmte und seinslose Vielheit durch die Teilhabe an dem überseienden absoluten Einen zum in sich selbst bestimmten und vollendeten Einen und Ganzen (Ev öxov téXetov) des seienden Ideenkosmos wird\footcite[vgl.][S. 54f.]{HalfwassenGeistmetaphysik}

\zitatblock{\enquote{Denn halten wir fest: der Platonismus interpretiert das Sein aufgrund seiner Vermittlungsstruktur von Einheit und Vielheit als Geist, er analysiert die Struktur der Selbstbeziehung des Denkens, und er faßt den Zusammenhang der Ideen in der Einheit des Denkens als das Urbild und den aktiv bestimmenden Grund der geordneten Welt, konzipiert also das denkende Selbstverhältnis zugleich als Begründung der Welt}\footcite[][S. 60]{HalfwassenGeistmetaphysik}}

\subsubsection*{Szlezák Halfwassen Monismus Dualismus in Platons Prinzipienlehre}

Halfwassen identifiziert die bekanntlich zwei letzten Prinzipien Platons innerakademischer Prinzipienlehre, auf deren Zusammenwirken alles Seiende zurückgeführt wird: das absolute Eine (auto to en) und die unbestimmte Zweiheit (aoristos dyas)\footcite[vgl.][S. 67]{HalfwassenMonismusDualismus}

Er hält auch fest, dass Krämer in seinem Buch \enquote{Der Ursprung der Geistmetapyhsik} (1964) \enquote{[\dots] nicht mit einer Einheit der Gegensätze bei Platon, sondern im Sinne der neuplatonischen Platondeutung mit einer Zurückführung des Vielheitsprinzips auf das Eine als allbegrüdendes Ur-Prinzips.}\footcite[][S. 68]{HalfwassenMonismusDualismus} (Krämer Seite 332-334; 1964) 
\zitatblock{\enquote{Das Dialektikprogramm der Politeia beschreibt deutlich den Aufstieg zu Einem Unbedingten und Absoluten, das Urgrund von allem ist. Dies scheint einen irreduziblen Prinzipiendualsimus auszuschließen: denn wenn dem Einen die Veilehit als gleichursprüngliches und unabhängiges Prinzip gegenüberstünde}\footcite[][S. 70f.]{HalfwassenMonismusDualismus}} Politeia 511b, 533c
Damit ist zwar mit Halfwassen nicht vollständig auszuschließen, dass es lediglich den Monismus in dieser Weise in der Deutung der Prinzipienlehre gibt, sondern sich gerade in Anlehnung an Krämer, dass \enquote{[d]ie monistische Lösung entspricht einem Rückgriff hinter den Gegensatz der beiden Prinzipien, ohne ihn aufzuheben}\footcite[vgl.][S. 333]{Krämer1964Geistmetaphysik}
Daraufhin verweis auf Parmenides 8 Hypothesen, in denen Einheit und Vielheit zueinander in jedem Verhältnis untersucht werden.\\
8 Hypothesen des Parmenides nach Halfwassen:
\begin{itemize}
    \item {(2. Hypothese) Begründung des Seins vom ersten Prnizip her als die Entfaltung der Eiheit in die Vielheit durch die entzweiende und aufschließende Kraft der Zweiheit}\footcite[vgl.][S. 72]{HalfwassenMonismusDualismus}
    \item {(3.Hypothese) vom zweiten Prinzip her als die Begrenzung des Unbegrenzten durch die Einheit-setzende Macht des Einen}\footcite[vgl.][S. 72]{HalfwassenMonismusDualismus}
    \item {1. und 4. Hypothese thematisieren das Eine und die Vielheit an sich selbst, in ihrer Unbezüglichkeit. Den Prinzipien müssen an ihnen selbst alle grundlegenden Seinsbestimmungen, die Prädikate des seienden Einen sind, und schließlich auch das Sein selbst sowie Erkennbarkeit und Sagbarkeit abgesprochen werden (Parmenides 141)}\footcite[vgl.][S. 72]{HalfwassenMonismusDualismus}
    \item {(8. Hypothese) Vielheit getrennt von dem Einenauch nicht mehr Vielheit ist, sondern gar nichts.\footcite[vgl.][S. 73]{HalfwassenMonismusDualismus}}
\end{itemize}
\enquote{Wenn unsere Deutung richtig ist, dann verbindet Platons Prinzipienlehre einen Monismus in der Reduktino zum Absoluten mit einem Dualismus in der Deduktion es Seienden.}\footcite[][S. 79]{HalfwassenMonismusDualismus}


Das hier ist Halfwassen\\
Lé Timée de Platon contributions à l'histoire de sa réception = Platos Timaios- Das hier ist das Buch, aus dem Hans zitiert hat. Also \enquote{Der Demiurg von J. Halfwassen}
\subsubsection*{Hans Joachim Krämer. Arete bei Platon und Aristoteles in Platonisches Philosophieren 70/CD 3067 K89}
peras (Ende)\\
meros (Teil)
mega-mikron (Das Groß-Kleine, Das zweite Prinzip neben dem en/apeiron (Einen))
253f.: Das Prinzip der Zahlen ist noch vor dem der Ideen, da dieses Prinzip des Einen und Vielen erst die Ideen bedingen. 

\subsubsection*{(Hg.) Szlezák Maurizio Migliori Dialektik und Prinzipientheorie in Platons Parmenides und Philebos}
Die Dinge können auf verschiedenen Ebenen als Eines und Vieles beschrieben werden, bzw. kann Einheit und Vielheit an einem Einzelding beschrieben werden.\footcite[vgl.][S. 112]{Migliori}
\zitatblock{\enquote{Diese Verbindung von Ganzem und Teil lässt uns das verstehen, was das Herz einer jeden Dialektik ist: die Möglichkeit, die Identität der Gegensätze zu behaupten, ohne den Satz vom Widerspruch zu negieren.}\footcite[][S. 150]{Migliori}}

\subsubsection*{Die letzte denkbare Einheit (agathon) Robert Wallisch}
Es wird auf Paul Natorps Platons Ideenlehre zitiert, die ich in der Sophistes Arbeit zitiert habe. 
Erste Arbeitshypothese: \zitatblock{\enquote{Wenn die Ideen zu Beginn des sechsten Buches gedachte konstante Einheiten waren, welche die unstete Vielheit bewältigen - gleichsam als immergültige Bündelungen des Vielen zu noetischen Einheiten - so muss das agathon als eine noch höhere, eine letzte Instanz, welche ihrerseits die Ideen bedingt, d.h. den Ideen in analoger Weise übergeordnet ist wie die Ideen dem konreten Vielen, als letzte denkbare Einheit angesprochen werden}\footcite[][S. 10]{Wallisch}}

Das agathon ist im noetischen Bereich des Erkennens wie die Sonne, die als ein Drittes die Möglichkeit des Erkennens und des Denkens erst schafft.\footcite[vgl.][S. 10]{Wallisch} 
\enquote{Nur durch das agathon existieren die Ideen als reale Gegebenheiten, als Dinge des Denkens}\footcite[][S. 11]{Wallisch} und erzeugt somit die Ideen erst. Dabei wird eine Grenzüberschreitung angesprochen, die das Gute (agathon) jenseits der Seite der Dinge verortet. Damit sind Ideen und Dinge auf der einen Seite, das agathon, die Transzendenz des Guten auf der anderen Seite \footcite[vgl.][S. 11]{Wallisch}
Entscheidend dabei ist der Begriff der ousia
\enquote{Unnennbar viele sind die Sinnesdinge; doch auch die Ideen, welche die Vielen durch noetische Bündelung zu Einheiten bewältigen, sind selbst wiederum \textbf{viele} Einheiten und keinesfalls eine letzte denkbare Einheit [\dots]}\footcite[vgl.][S. 12]{Wallisch}
Im Gegensatz zu Natorp nennt Wallisch das agathon nicht eine absolute Einheit oder eine Grundprädikation, einletztes Objekt der Grundwissenschaft, also ein keinesfalls losgelöstes, ein auf die Welt anwenbares Gesetz des Denkens [\dots], sondern vilemehr die jedes Denken präzedierende ontologische Gegebenheit, die durch Erweiterung auf das Ganze zu erlangen ist.\footcite[vgl.][S. 14]{Wallisch}
Er hält auoerdem fest, dass \enquote{[\dots] nicht eine Teilung in zwei Welten, sondern die Unterscheidung verschiedener gnoselogischer Zugänge zu derselben Welt intendiert ist.}\footcite[vgl.][S. 15]{Wallisch}
\zitatblock{\enquote{Wenn wir nun die Ideen als nicht transzendent aufgefasst haben, so darf darunter keinesfalls eine Leugnung ihrer Sonderung von den Sinnesdingen verstanden werden. Platons Texte lassen keinen Zweifel daran zu, dass die Ideen als real und getrennt von den konkreten Dingen existierend zu denken sind. Tatsächlich ist die platonische Idee eine gnoseologische Gegebenheit und somit als reale Entität anzusprechen, die von den Sinesdingen gesondert existiert und auf die Sinnesdinge Wirkung hat.}\footcite[][S. 17]{Wallisch}}
Es wird mit Platons Texten auf Rep. 579a ff, Phaidon 74a-76e, 78a, Philebos 15a,b und Timaios 51c-e, 52a hingewiesen
\enquote{Weil die Ideen, noetische Einheiten, immer noch viele sind, befinden sie sich diesseits des ontologischen Chorismos, der die transzendente totale Einheit von den Vielen trennt.}\footcite[][S. 17]{Wallisch}
Tim 51b7-c5 Stelle zeigt \enquote{unzweideutig}, dass die Sonderung der Ideen erkenntnistheorisch und nicht ontologisch aufzufassen ist.\footcite[vgl.][S. 19]{Wallisch}
Ein weitere Punkt wird von Wallisch so ausgearbeitet, dass er - genauso wie Natorp - die Ideen mehr als Methode erscheint, als als Ziel\footcite[vgl.][S. 26]{Wallisch} Herangezogen wird dafür die Stelle Phaidon 99e4-100a5, wo die Ideen als eine \enquote{Hypothese} verstanden wird, die als gedanklich eingesetzte Stütze verstanden wird und somit nicht als Objekt von Erkenntnis verstanden werden dürfen.\footcite[vgl.][S. 26]{Wallisch} Es wird dabei betont, dass es auf die ontologische Wahrheit konzipiert worden ist und nicht in Hinblick auf die Erkennbarkeit der konkreten Sinnesdingen\footcite[vgl.][S. 27]{Wallisch}

\subsection*{Naheliegende Stellen bei Platon %ursprünglich von Halfwassen)
}
Darstellung an denjenigen Stellen, die von den Autoren immer wieder genannt werden und somit am häufigsten verwendet werden. Es müssen erst noch die Stellen gesichtet werden. Stellen von Halfwassen genannt:
\begin{itemize}
    \item {Politeia 506de, 509c-d 510b7, 511b6, 511b7, 517c2-d, 524ff, 533c, 534b}
    \item {Parmenides 141e, 142bff, 142eff, 143bff, 165e-166c}
    \item {Philebos 14cff, 23eff, 27b 30ab, 28c, 30d}
    \item {2. Brief 312e}
\end{itemize}

Zur absoluten Transzendenz des Einen und Guten im einzelnen Halfwassen (1992) 19ff, 188ff, 221ff, 257ff, 302ff und Krämer (1969)
%Es wird viel auf Aristoteles Metaphysik verwiesen.
\subsection{Rückbezug auf Originalstellen}
Hier kann sich jetzt nochmal auf die Origianstellen berufen werden und nochmal betrachtet werden, inwiefern die andere Deutung halt findet.


Es sei also gesagt, dass diese Interpretation die \enquote{Trennung} von Bereichen nicht auf der zu tief angesetzten Ebene von Sinnesdingen zu den Ideen vornimmt, sondern die Ideenelehre in ihrer Gesamtheit zu begreifen möchte, d.h. die Grenze so verschiebt, dass diese erst durch die starke Transzendenz beschränkt wird am Übergang zur IdG als einheitstiftendes Prinzip, welches nicht mehr überschritten werden kann und auch nicht mehr selbst gedacht werden kann.
%\subsubsection*{Symposion}
%210ff. ist der Aufstieg zum Schönen, wo das Schöne an den schönen Leibern zuerst gesucht werden soll und dann die Suche immer weiter aufsteigt und das untere als Minderwertiges zurückgelassen werden soll.
%Der wahre Phiosoph und was er macht und ihn auszeichnet. Symp. 211e-212a

%\subsubsection*{Politeia (Gleichnisse)}
%509c-d 510b7, 511b6, 511b7, 517c2-d, 524ff, 533c, 534b, 596a-597e (drei Seinsweisen, Bild Naturding Idee)
%506de: Sokrates kann an dieser Stelle nicht über das Gute selbst sprechen, sondern versucht es in diesem Anlauf mit einem Sprössling des Guten, da Sokrates es selbst zu diesem Zeitpunkt nicht schaffen könnte.
%\zitatblock{\enquote{Dass es eine Vielheit von Schönem, sagte ich, eine Vielheit von Gutem und so überhaupt von allen Dingen gäbe, räumen wir ein und bezeichnen es auch näher in der Rede. Auch bekanntlich ein Schönes an sich, ein Gutes an sich, und so überhaupt in Bezug auf alles, was wir erst eine Vielheit von jedem hinstellten, das stellen wir dann wieder, um in einem einzigen Begiff hin, als wenn die Vielheit einer Einheit wäre, und nennen es das Wesen von jedem.}} (Pol. 507b-c)

%\enquote{Was den Dingen, die erkannt werden, Wahrheit verleiht und dem Erkennenden das Vermögen des Erkennens gibt, das begreife also als die Wesenheit des eigentlichen Guten}(Pol. 508e)
%\zitatblock{\enquote{Und so räume denn auch nun ein, dass den durch die Vernunft erkennbaren Dingen von dem Guten nicht nur das Erkanntwerden zuteilwird, sondern dass ihnen dazu noch von jenem das Sein und die Wirklichkeit zukommt, ohne dass das höchste Gute Wirklichkeit ist, es ragt vielmehr über die Wirklichkeit an Würde und Kraft hinaus}(Pol. 509b-c)}
%(Warum trennt er hier das Erkanntwerden von dem Sein und der Wirklichkeit in einer dreifachen Ausführung?)
%Bei den mathematischen Dingen im Liniengleichnis heißt es: \zitatblock{\enquote{Nicht war, auch das weißt du, dass sie sich der sichtbaren Dinge bedienen und ihre Demonstrationen auf jene beziehen, während doch nicht auf diese als solche, als sichtbare, ihre Gedanken ziehen, sondern nur auf das, wovon jene sichtbaren Dinge aus Schattenbilder sind.[\dots] Selbst die Körper, die sie bilden und zeichnen, wovon es auch SChatten und Bilder im Gewässer gibt, eben diese Körper gebrauchen sie weiter auch nur als Schattenbilder und suchen dadurch zu Schauung eben jener Ausführung zu glangen, die niemand anders schaun kann als mit dem denkenden Verstand}(Pol. 510e-511a1)}
%Der letzte Abschnitt des Liniengleichnisses: 
%Apelt Übersetzung: \zitatblock{\enquote{So verstehe denn auch folgendes: unter dem zweiten Abschnitt des Denkbaren meine ich das, was der denkende Verstand unmittelbar selbst erfaßt mit der Macht der Dialektik, indem er die Voraussetzungen nicht als unbedingt Erstes und Oberstes ansieht, sondern in Wahrheit als bloße Voraussetzungen, d.h. Unterlagen, gleichsam Stufen und Aufgangssütztpunkte, damit er bis zum Voraussetzungslosen vordringend an den wirklichen Anfang des Ganzen gelange, und wenn er ihn erfaßt hat, an alles sich haltend was mit ihm in Zusammenhang steht, wieder herabsteige ohne irgendwie das sinnlich Wahrnehmbare dabei mit zu verwenden, sondern nur die Begriffe selbst nach ihrem eigenen inneren Zusammenhang, und mit Begriffen auch abschließe.}(Pol. 511B-C Apelt)}\nocite{PoliteiaApelt}
%Schleiermacher Übersetzung:\zitatblock{\enquote{So verstehe denn nun auch, dass ich unter dem anderen Unterabschnitte der nur durch die Vernunft erkennbaren Hälfte das verstehe, was die Vernunft durch die Macht der Dialektik erfasst und wobei sie ihre Voraussetzungen nicht als Erstes und Oberstes ausgibt, sondern als eigentliche Voraussetzungen, gleichsam nur als Einschnitts- und Anlaufungspunkte, damit sie zu dem auf keiner Voraussetzung mehr beruhenden Anfang des Ganzen gelangt, und wenn sie ihn erfasst hat, an alles sich haltend was mit ihm in Zusammenhang steht, wieder herabsteige ohne das sinnlich Wahrnehmbare dabei zu verwenden, sondern nur die Begriffe selbst nach ihrem Zusammenhang, und mit Begriffen auch abschließe}(Pol. 511b-c Schleiermacher)}
%Den Ursprung der zwei Welten stammt sehr wohl vom Höhlengleichnis. Da es hier eine deutliche Zweiheit von Welten gibt, die man betreten und auch verlassen kann.
%\zitatblock{\enquote{Wenn aber, fuhr ich fort, jemand ihn aus dieser Höhle mit Gewalt den rauen und steilen Aufgang aufwärts zöge und ihn nicht losließe, bis er ihn ans Licht der Sonne herausgebracht hätte, würde er wohl Schmerzen empfunden haben? Würde er über dieses Hinausziehen aufgebracht werden und, nachdem er ans Sonnenlicht gekommen ist, die Augen voller Blendung haben und also gar nichts von den Dingen sehen können, die jetzt als wirklich ausgegeben werden?}(Pol. 515e-516a Schleiermacher)}
%Dieser Vorgang lässt sich an dasselbe Argument der unsterblichen Seele knüpfen. Also dass es ein Unveränderbares geben muss, an dem sich das Werdende/Veränderliche vollziehen kann. Also von in der Höhle nach Außen.
%\enquote{[er] würde über [die Sonne] die Einsicht gewinnen, [\dots] dass sie alles ordnet im Bereich der sichtbare Weltund von allen jenen Erscheinungen, die er dort sah, gewissermaßn die Ursache ist.}(Pol. 516 b4 Schleiermacher)
%\zitatblock{\enquote{Das Gleichnis also, mein lieber Glaukon, fuhr ich fort, ist nun in jeder Beziehung auf die vorhin ausgesprochenen Behauptungen anzuwenden. Die sich uns mittels des Gesichts offenbarende Welt vergleiche einerseits mit der Wohnung im unterirdischen Gefängnis, und das Licht des Feuers in ihr mit dem Vermögen der Sonne. Das Hinaufsteigen und das Beschauen der Gegenstände über der Erde stelle dir andererseits als den Aufschwung der Seele in das Gebiet des nur durch die Vernunft Erkennbaren vor, und du wirst dann meine Meinung hierüber haben, weil du sie doch einmal zu hören verlangst. Ein Gott mag aber wissen, ob sie richtig ist! Aber meine Ansichten hierüber sind nun einmal die: Im Bereich der Vernunfterkenntnis ist der Begriff des Guten nur zu allerletzt und mühsam wahrzunehmen. Nach seiner Ansicht muss man zur Einsicht kommen, dass er für alle Dinge die Ursache von allem Richtigen und Schönen ist, indem er in der sichtbaren Welt das Licht und die Sonne erzeugt. Sodann auch im Bereich des durch die Vernunft Erkennbaren selbst als Herrscher waltend, gewährt er sowohl die Wahrheit als auch Vernunfteinsicht. Ferner muss man zur Einsicht kommen, dass das Wesen des Guten ein jeder erkannt haben muss, der verständig handeln will, sei es in seinem eigenen Leben oder in öffentlichen Angelegenheiten.}(Pol. 516b-d Schleiermacher)}
%hierzwischen werden die Lehren der Arithemtik, der Geometrie, der Astronomie und der Akustik angeführt. Diese sind die Vorstufe zur Dialektik 
%wichtige Stelle vorher noch, wo das Höhlenlgeichnis nochmal mit dem Sonnengleichnis und der Dialektik zusammengebracht wird:
%\zitatblock{\enquote{Dagegen, sagte ich, die vorhergehende Lösung vonden Banden, und die Wendung von den Schatten zu den Bildwerken und zum Licht, und das Emporklimmen aus den unterirdischen Kerker zur Sonne, und das dort im Sonnenlichte, infolge des nch vorhandenen Unvermögens, sogleich die Tiere, Pflanzen und den Sonnenglanz anschauen zu können, zuerst gerichtete Schatten auf die im Wasser sichtbareb Spiegelungen und auf die Schatten der wirklichen Gegenstände, das aber hier zum Anschauen von Schatten des Seieden, nicht der Bilder Schatten, im Vergleich mit der Sonne ähnlichesLicht hervorgerufen werden, diese Kraft hat die gesamte Schulung in den von uns aufgestellten Lehrfächern, und dieser Weg heißt die Hinauführung des besten Seelenvemrögens zu der Anschauung des Wesens in dn Dingen, eine ganz ähnliche Hinaufführung, wie die oben erwähnte des Auges zu Anschauung des hellsten Gegenstandes in der sichtbaren Welt.}(Pol. 532b-d)}
%Damit ist eigentlich deutlich, dass die Welt \enquote{außerhalb} der Höhle keine andere Welt darstellen soll, sondern die Art und Weise verdeutlichen soll, wie der Aufstieg hin zur Sonne, zum hellsten Gegenstand, gemeint sein soll, die auf der Ebene der Ideen die Idee des Guten darstellen soll.
%\zitatblock{\enquote{und dass nur die Dialektik imstande ist, dem, der die oben beschriebenen Lehrfächer studiert hat, dies zu zeigen und auf eine andere Weise aber es nicht möglich ist? [\dots] Und auch das wird uns weiter niemand in Abrede stellen, [\dots] wenn wir behaupten, dass kein anderes wissenschaftliches Verfahren das Sein eines jeden Dinges zu erfassen strebt, denn alles andere Können und Wissen ist entweder auf menschliche Meinungen und Begierden, oder ist auf die verschiedenen Arten des Entstehenden, auf dessen Zusammensetzung oder ihre Pflege gerichtet.}(Pol. 533b Schleiermacher)}
%Diese Form der Dialektik, des Abstraktionsvermögens, ist notwendig, um die Abstrakte Ebene zu verstehen, die im Höhlengleichnis gemeint sein soll. Es wird zwar erklärt, dass die Dinge außerhalb der Höhle erkannt werden können, es aber nicht möglich ist direkt in die Sonne zu schauen. Dass hier die Sonne als etwas beschrieben wird, das man zwar nicht direkt ansehen kann, aber insofern beschreiben kann, dass es der Grund für alles weitere ist, macht nur deutlich, dass diese Abstration davon, dass man in den \enquote{beschienen} Dingen die Sonne ausmachen kann, die Sonne durchaus auf eine gewisse Art erkennen kann, ohne sie direkt zu sehen.
%\zitatblock{\enquote{Die Wissenschaften, denen wir zugestehen, dass sie etwas vom Seienden erfassen, wie Geometrie und ihre verwandten, sehen wir zwar über das Sein träumen, aber im wachen Zustand ist es ihnen unmöglich, es zu schauen, solange sie sich unerwiesener Voraussetzungen bedienen und sie ganz unberührt lassen, weil sie dies nicht begründen können. Denn wobei der Anfang aus dem besteht, was man nicht weiß, und Ende und Mitte aus dem Nichtgewussten zusammengeflochten werden, wie kann auf eine solche Weise angenommen werden, dass eine Wissenschaft entsteht?}(Pol. 533c Schleiermacher)}
%\zitatblock{\enquote{Es genügt, also fuhr ich fort, den ersten Abschnitt des Erkennens Wissenschaft zu nennen, den zweiten Verstandeseinsicht, den driten Glaube, den vierten Wahrerscheinen, und einerseits die beiden letzten zusammen Meinung, andererseits die ersten zusammen Vernunfteinsicht, dabei bezieht sich Meinung auf das wandelbare Werden, Vernunfteinsicht auf das unwandelbare Sein, so dass wie Sein zum Werden, so Vernunfteinsicht zu Meinung, und wie Wissenschaft zum Glaube, so Verstandeseinsicht zum Wahrscheinen sich verhält. Die entsprechenden Verhältnisse dessen, woraus sie sich beziehen, sowohl des durch Meinung Erfassbaren als auch bei dem durch Vernunft Erkennbaren, und ihre Unterteilung wollen wir jetzt, mein lieber Glaukon, beiseitesetzen, damit wir nicht in noch viel umfassendere Erörterungen geraten als vorher.}(Pol. 534a-b1 Schleiermacher)}
%596a ff. Drei Seinsweisen.
%Beginn mit Annahme von beliebigen Vielheiten von Tischen und Betten. Es gibt von diesen Gerätschaften nur zwei Begriffe, einen von Bett und einen von Tisch und der Werkmeister macht den Begriff (vgl. Pol. 596b Schleiermacher)
%Es wird ein noch außerordentlicher Meister gegeben, der auch alle Erzeugnisse der Erde bildet, alle lebenden Wesen hervorbringt und alles übrige sowohl sich selbst. (vgl. Pol. 596c Schleiermacher)
%Ein Maler ist damit gemeint, denn dieser macht auch auf gewisse Weise alles. (vgl. Pol. 596e)
%Es entstehen drei verschiedene Seinsweisen: \enquote{Also Maler, Tischler und Gott sind drei Meister für drei Arten von Betten}(Pol. 597b)
%Der Maler ist der Nachbildner, der Tischler der Werkmeister und der Gott der Wesensbildner (vgl. Pol. 597d-e)


%\subsubsection{Phaidon 74e-75d, 99d-105c}
%Aber doch an den Wahrnehmungen muss man bemerken, dass alles so in den Wahrnehmungen vorkommende jenem nachstrebt, was das gleiche ist und dass es dahinter zurückbleibt. [\dots] Ehe wir also anfingen zu sehen oder zu hören, oder die anderen Sinne zu gebrauchen, mussten wir schon irgendwoher die Erkenntnis bekommen haben des eigentlichen Gleichen, was es ist, wenn wir doch das Gleiche in den Wahrnehmunen als auf jenes beziehen sollten, dass dergleichen alles zwar strebt zu sein wie jenes, aber doch immer schlechter ist. (75a-b) 
%[\dots], dass ich voraussetzte, es gebe ein Schönes an und für sich, und ein Gutes und Großes und so alles andere, woraus, wenn du mir zugibst und einräumst dass es sei, ich dann hoffe, dir die Ursache zu zeigen und nachzuweisen, dass die Seele unsterblich ist (100b-c)
%Erste Voraussetzung: Wenn irgend etwas anders schön ist außer jenem, selbstschönen, es wegen nichts anderem schön sei, als weil es Teil hat an jenem Schönen. (100c)

%\subsubsection*{Timaios 27b-29b, 51b-52d}
%Es soll über das All gesprochen werden, \enquote*{wie es entstanden ist oder auch ungeworden ist.}(Tim. 27c)\nocite{TimaiosSchleiermacher}
%\zitatblock{\enquote{Was ist das stets Seiende und kein Entstehen Habende und was das stets Werdende, aber nimmerdar Seiende; das eine ist durch verstandesmäßiges Denken zu erfassen, ist stets sich selbst gleich, das andere dagegen ist durch \emph{bloßes} mit vernunftloser Sinneswahrnehmung verbundenes Meinen zu vermuten, ist werdend und vergehend, nie aber wirklich seiend.}(Tim. 27d-28a)}
%Der Gestalter muss also als Vorbild das sich stets gleich Verhaltende, wenn etwas schönes gestaltet werden soll. Wenn er allerdings etwas Gewordenes als Vorbild nimmt, so wird es nicht schön.(vgl. Tim. 28a-b)
%\enquote{Ist aber diese Welt schön und ihr Werkmeister gut, dann war oofenbar sein Blick auf das Unvergängliche gerichtet; ist \emph{sie} aber - was auch nur auszusprechen frevelhaft wäre, dann \emph{war sein} Blick auf das Gewordene \emph{gerichtet}. Jedem aber ist doch deutlich, dass \emph{er} auf das Unvergängliche \emph{gerichtet war}, denn sie (die Welt) ist das Schönste unter dem Gewordenen, er der Beste unter den Ursachen.}(Tim. 29a)
%\enquote{Das aber zugrunde gelegt, ist es ferner durchaus notwendig, dass diese Welt von etwas ein Abbild sei.}(Tim. 29b)
%\enquote{Gibt es ein Feuer an sich und für sich und alles das, wovon wir stets in dieser Weise reden, als jeweils an sich und für sich seiend, oder ist allein das, was wir sehen und sonst vermittels des Körpers wahrnehmen, da es eine solche Wahrheit (Realität) hat, und gibt es anderes außer diesen auf keine Art und Weise, sondern behaupten wir jeweils vergeblich, dass es von jeglichem eine denkbare Form gebe, und waren das nicht als \emph{leere} Worte?}(Tim. 51b-c)
%\enquote{Wenn Vernunft und richtige Meinung zwei verschiedene Arten sind, dann gibt es auf alle Fälle dies Dinge an sich, Formen, die sich von uns nicht wahrnehmen lassen, sondern nur gedacht werden.[\dots] Aber jene beiden sind als zwei zu bezeichnen, da sie gesondert entstanden und von unähnlicher Beschaffenheit sind. Denn das eine entsteht in uns durch Belehrung das andere durch Überredung.}(Tim. 51d-e)
%\subsection{Nomoi}
%ab 896 wird die Seele beschrieben, wie sie sich selbst und alles weiter bewegt, \enquote{es ist auf das Vollständige geziegt, dass sie der Anfang aller Bewegung und eben damit auch, dass sie das Ursprüngliche aller Dinge ist.}
%\enquote{Und wenn nun ferner die Seele alles durchwaltet und allem innewohnt, was überall sich bewegt, muss man da nicht auch dem ganzen Weltall eine solche es durchwaltende Seele zuschreiben? - Sicher. - Eine oder meherere? Mehrere antwortete ich für euch. Mindestens müssen wir ihrer zwei annehmen, eine wohltätige und eine, welche das Gegenteil vollbringen kann.}
%\subsubsection{Phaidros 249}
%\nocite{phaidros}
%Denn der Mensch muss in Begriffen Ausgedrücktes begreifen, was aus einer Vielheit innlicher Wahrnehmungen sich ergebend, durch den Verstand zur Einheit zusammengefasst wird.(Phaidr. 249b6-c1 Schleiermacher)
%\subsubsection{Sophistes 251a-259d}
%Die Ausführung der Problematik des Einen und Vielen, wie deren Einheit und Verschiedenheit und Verbindung untereinander, wird sich im Sophistes in der Dihairese besonders gewidmet, bzw. hier besonders gezeigt, wenn es darum geht den Weg der Definition zu gehen und hierbei die unterschiedlichen Ebenen voneinander trennen zu können, aber doch den Sinn einer Definition, die Abgrenzung von anderen Dingen auf der selben Ebene von einer höheren und niederen Ebene zu unterscheiden.
%\subsubsection*{Geschichte der Philosophie Band I Altertum und Mittelalter Johannes Hirschberger 1980}
%Die wichtigsten genannten Stellen der \enquote{Ideenlehre}:
%\begin{itemize}
 %   \item {Phaidon
  %  \begin{itemize}
   %     \item{74a-75d (Erkennbarkeit des Gleichen und des Verschiedenen in Dingen, mit Wiedererinnerung)}
    %    \item{99d-105c (Die Alternative des Sokrates, seine Ideenlehre);}
    %\end{itemize}}
    %\item{Politeia
     %   \begin{itemize}
      %      \item{507d-509b (Idee des An-sich-Guten und Sonnengleichnis)}
       %     \item{509d-511e (Liniengleichnis)}
        %    \item{514a-516c (Höhlengleichnis)}
         %   \item{596a-597e (drei Seinsweisen, Bild Naturding Idee)}
        %\end{itemize}}
    %\item{Timaios
    %\begin{itemize}
     %   \item{27b-29b (Entstehung der Welt, des Seienden und Werdenden)} 
      %  \item{51b-52d (Zusammenfassung)} 
       % \item{Deckt sich mit dem, was beim Hans zu finden ist: Timaios 27c1-53c4 und Phaidon 100d}
    %\end{itemize}}
    %\item{Sophistes 251a-259d (Gemeinschaft der Ideen und die Dialektik);}
    %\item{Parmenides 130e-135b (Selbstkritik); In welcher Weise haben die Dinge an den angenommenen Begriffen teil?}
%\end{itemize}
%\subsubsection*{Parmenides}
%1. Hypothese (137c4-142a8)\\
%2. Hypothese (142b1-157b5)\\
%3. Hypothese (157b6-159b1)\\
%4. Hypothese (159b2-160b4)\\
%5. Hypothese (160b5-163b6)\\
%6. Hypothese (163b7-164b4)\\
%7. Hypothese (164b5-165e1)\\
%8. Hypothese (165e2-166c5)

