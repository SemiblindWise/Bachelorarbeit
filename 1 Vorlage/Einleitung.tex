\section{Einleitung}

Die Frage \enquote{Was ist X?} ist eine Frage, die sich durch die gesamte Philosophiegeschichte hindurch bewegt. Die grundlegendere Frage hierbei ist eigentlich, dass es darum geht zu fragen, wie etwas definiert wird oder definiert werden kann. Also nicht etwa \enquote{Was ist ein Hund} oder \enquote{was ist eine Geschichte}, sondern darauf abzielend, dass die Notwendigkeit besteht danach zu fragen, wie es überhaupt möglich ist mit Gewissheit eine sichere Antwort auf die \enquote{Was ist X} Frage zu geben. Für Platon ist es so, dass die Antwort auf die Frage \enquote{Was ist ein Hund} nicht so beantwortet werden kann, dass ein Beispiel für einen Hund, der allgemein als Hund anerkannt werden würde, gegeben werden kann. Dies hängt damit zusammen, dass diese Bestimmung des Hundes nur für diesen einen Hund gelten würde und nicht etwa auch für den \enquote{hundlich} erscheinenden Hund, welcher neben dem ersten Hund sitzt. Zudem ist die Frage, ob der erste Hund noch als Hund erkannt wird, wenn dieser etwa ein Bein verlieren sollte oder gar sterben würde. Selbst wenn man alle lebenden und auch schon gestorbenen Hunde heranziehen würde, um eine Definition zu bilden kommt man in das neuzeitliche Problem des Induktionsproblems. Damit soll gezeigt werden, dass anhand der Einzelbeispiele es unmöglich ist eine Definition zu geben, welche konsistent ist, da jedes Einzelbeispiel - auch aus allen anderen \enquote{Was ist X} Fragen - ständiger Veränderung unterstellt ist. Daher ist es nötig die Definition, z.B. des Hundes, anders zu gestalten, was sich nicht verändert. Diese Sache kann also nicht in den Dingen gefunden werden. Hierfür wird von Platon die Idee eingeführt, die nicht mit den Dingen gleichzusetzen ist. 
Dieses Prinzip ist eng mit der Dialektik verknüpft, welche für Platon einen zentralen Stellenwert einnimmt. Denn nur mithilfe der Dialektik ist es möglich Wissen zu erlangen, welches unbedingt ist, also nicht nochmals bestimmt durch etwas anderes, da es sonst seine eigenen Ansprüche entwerten würde.\footcite[vgl.][S.83]{Staudacher}
Diese aufgebaute Zweiheit (oder auch Dualismus) wird auf vierfache Weise an Platons Werke herangetragen:
\zitatblock{\enquote{There are four grounds on which Plato is usually qualified as a dualist: (1) his position in metaphysics, usually refered to as the two worlds theory; (2) his radical antithesis of soul and body, as it is commonly understood; (3) the doctrine of two ultimate principles, which he appears to have held at least in his later years; (4) the so-called cosmic dualism, attributed to him by early Christian writers and still ascribed to him by some present-day scholars}\footcite[][S. 159]{Vogel}}
Für die vorliegende Arbeit wird lediglich die erste der vier Ansichten von Zweiheit in der Form der \enquote{two worlds theory} (Zwei-Welten-Theorie) behandelt. Als Anknüpfung zu dieser Arbeit ließe sich der zweite Punkt sehr gut anschließen, was sich auch in diesem Text an manchen Stellen vermissen lässt. Jedoch ist für den Umfang der Arbeit hier nicht der Platz dafür. Dem dritten Punkt wird erst durch den zweiten Teil dieser Arbeit die Gewichtung gegeben, die von Vogel gemeint ist und dem vierten Punkt wird hier nicht nachgegangen.\\
Anzumerken sei hier, dass im Folgenden von Dingen oder Sinnesdingen und von Ideen die Rede sein wird, da in der englisch sprachigen Literatur von \enquote{forms} gesprochen wird, es hier aber nicht zu Missverständnissen mit dem deutschen Wort der \enquote{Form(en)} kommen soll. 
In dieser Arbeit wird daher so vorgegangen, dass in zwei Interpretationen der platonischen Ideenlehre eingeführt wird. Dabei werden die in den platonischen Texten auftretenden Bereiche der Sinnesdinge und der Ideen in der ersten Interpretation als zwei (getrennte) Welten dargestellt. Dabei wird sich der in der Interpretation gelieferten Original-Textstellen bedient und auch genannt. Dabei wird mit Stellen aus dem Phaidon begonnen, dann weiter über Stellen aus dem Symposion und der Politeia, um mit einigen Timaios Stellen abzuschließen. Anschließend werden die bis dahin noch scheinbar getrennt gehaltenen Bereiche dahingehend klargestellt, dass diese zusammegeführt werden, mit Blick auf die Rolle der Idee des Guten. Sodann wird auf die Problematiken eingegangen, die aus der Interpretaion hervorgehen werden. Im Anschluss daran wird die zweite Interpretation behandelt, die die zwei Bereiche als einen einzigen Bereich versteht, welche verstärkt auf den Einheitsgedanken dieser \enquote{Welt} eingeht. Zum Abschluss soll abgewogen werden, welcher der beiden Interpretationen der Vorzug zu geben ist.\\
Zu Beginn muss jedoch eine kurze Einführung über die Bedeutung und die Tragweite der Transzendenzauslegung geliefert werden, wenn es heißt, dass die Dinge und Ideen verschieden sind. Dabei wird eben auch direkt die Bedeutung des Aufstiegs von den Dingen zu Ideen thematisiert, da hier bereits davon gesprochen worden ist, dass man nicht durch die Dinge zu Definitionen gelangen kann, sondern zu den Ideen gelangen muss.
%Ich gehe davon aus, dass es kein Chorismos zwischen Ideen und Sinnesdingen gibt. Sofern damit gemint ist, dass es zwei Welten gibt, in denen jeweils Ideen und Dinge sich aufhalten. Es wird dabei noch klar darauf einzugehen sein, was mit Chorismos zwischen Ideen und Sinnesdingen gemeint ist und gemeint sein soll. 
%Man könnte hier einen Anfang setzten, wenn man ganz einfach beginnt, also damit, dass mit den Basics angefangen werden muss.\\
%Die Transzendenz in Platon(s Dialektik) 
%Die Frage nach dem Einen, den Ideen\footcite[][]{Staudacher} und den das Seiende übersteigende Ideen, folglich transzendent.\footcite[][]{Bordt}
%Womöglich auch damit in Verbindung die Dialektik, dass die verschiedenen Stufen der Ideen und der Sinnesdingen überwunden werden, bis hin zur obersten Stufe, welche damit dann auch noch überwunden werden kann, oder auch nicht, da die Prinzipien der Dialektik nicht mehr auf diesen Bereich anwendbar sind oder eben schon noch, aber bei der nächsten Stufe dann nicht mehr?\\
%Siehe dazu\footcite[vgl.][S. 104ff]{Hirschberger}

%Es kann nur eine Welt geben, nicht mehrere nebeneinander (Tim. 31a2-31b2) unterstützt von Vogel in der Weise, dass er festhält, \zitatblock{\enquote{As for present-day philosophy, it shows the tendency to eliminate the existence of a \enquote{transcendent} reality, in so far as this is meant to be a reality, existing \enquote{somewhere beyond} the world in which we live: There is one reality only, this one here and now. Certainly, it can be analysed into its intelligible forms, and this may be called a metaphysic of immanency}\footcite[][S. 161]{Vogel}}
%35c sehr wichtig!!!



%Es wird wohl schwierig sein, den Gedanken, dass die Seele in der Transzendenz die Ideen bereits geschaut hat, zu verbinden, da hier das Moment der Transzendenz nicht aufgelöst werden kann, ohne auch noch die Seelenlehre zu betrachten.
%Die zugrunde liegende Frage ist eigentlich, wie kann die Idee des Schönen, Guten, Gerechten an den schönen, guten, gerechten Dingen teilhaben?\footcite[vgl.][S. 16]{Martin73}


%Problem bei dieser Arbeit wird sein, dass ich bereits ein eigenes Verständnis von dem habe, wie Platon zu lesen sein sollte, oder eben wie Platon zu verstehen und auszulegen ist. 

%Noch bevor man die Ideenlehre angehen könnte, noch bevor man sich dem widmen kann, was mit wem in Verbindung steht oder stehen kann und was nicht, muss sich über die Verhältnisse von Einem und Vielem in aller ihrer Formen zugewandt werden.
%Die eigentliche Frage, die es zu lösen gilt ist die von der Möglichkeit von unveränderlichen Dingen, die für alles Werdende den Grund angeben. Wo ist der Baum, wenn er gefällt wird, verarbeitet wird, verbrannt wird. Dies ist vermutlich eher eine aristotelische Frage, als eine platonische, wenn man dies so formuliert.

%Es bedarf einer eindeutigen und ausführlichen Erläuterung der Fragestellung, um diese an dem Rest des Textes erarbeiten zu können. Es geht dabei darum, dass die Ausgangsfrage im Grunde alle Aspekte der platonischen Philosophie unter sich fasst, sodass man alles - ausgeklammert sei die Ethik - behandeln müsste.

\section{Die Rede von Platons zwei Welten}
%Was motiviert eine solche Rede (Problem der Einheit – Vielheit / Wesen – Akzidenz ….)?

Wie es sich bereits in der Einleitung angedeutet hat, festigt sich die Suche mithilfe der Ideen nach einer gesicherten Wahrheit in einer Sehnsucht, welche stark bei Platon verankert ist, um damit die sich uns offenbarende Wirklichkeit zu beschreiben und zu verstehen. Diese Sehnsucht manifestiert sich somit in einer (geistigen) Bewegung weg von den Dingen und hin zu den Ideen, da sich nur hier die gesicherte Wahrheit finden lässt. Warum dies ausschließlich bei den Ideen möglich ist, wird sich im Verlaufe der Arbeit noch zeigen. Da sich nun mit dieser Suche ein Art des Überschritts oder Überstiegs festellen und  im Folgenden zudem als Aufstieg verstehen lässt, muss sich dieser Auffassung des Aufstiegs, der Transzendenz\footnote{lat.: transcendere = übersteigen, überschreiten}, zugewandt werden.
%Es macht durchaus Sinn anfangs mit den Begriffen von zwei Welten an die Texte heranzugehen, da offensichtlich von zwei Bereichen gesprochen wird, die aufgestellt und unterschieden werden müssen. Dafür bedarf es, wie später noch deutlicher formuliert, in erster Linie dieser Begrifflichkeiten, um überhaupt Formulierungen zu beginnen.
%\subsubsection*{Platon Handbuch Horn Müller Söder}
Es werden in den Texten verschiedene Arten von Aufstiegen\footnote{Aufstieg zu etwas Hinreichendem im \emph{Phaidon} (101d5-e1), der Aufstieg zur Idee des Schönen in der Diotima-Rede des Sokrates im \emph{Symposion} (211b5-d1), der Aufstieg zum nicht-vorausgestzten Anfang im Liniengleichnis der \emph{Politeia} (VI 511b3-7); der Aufstieg zur Idee des Guten im Höhlengleichnis der \emph{Politeia} (VII 515c6-516b7); der Aufsieg zum über-himmlischen Ort im Seelen-Mythos des \emph{Phaidros}(246d6-248b5)} zu höheren Erkenntnis- und/oder Seinsstufen bestimmt, bei denen das Transzendieren ihre jeweiligen Anfangs- und Zwischenstationen mit eingeschlossen werden.\footcite[vgl.][S. 347]{StrobelTranszendenz}
%Damit ging die Annahme voraus, dass Platon \enquote{Philosophie als Transzendieren} (Hafwassen 1998) porträtiere. 
Es werden in der Regel den Ideen im Allgemeinen (als Entitäten) Raum- und Zeittranszendenz sowie im Verhältnis zu ihren wahrnehmbaren Partizipanten Transzendenz zugeschrieben. Der Idee des Guten im Besonderen wird noch Seinstranszendenz zugesprochen.\footcite[vgl.][S. 347]{StrobelTranszendenz}\\
%\enquote{[\dots], dass Ideen nicht räumlich lokalisert werden können und die Prädikate, die auf sie zutreffen, von Zeitbezügen frei sind.}\footcite[vgl.][S. 347]{StrobelTranszendenz}
Bei dem Verständnis der Transzendenz der Ideen gegenüber den Dingen lassen sich zwei Ansätze festhalten:\\
Die \emph{Nicht-Immanenz} (seperate existence): Eine gegebene Idee \emph{F} ist nicht in/an den Sinnendingen, die F sind und die \emph{Unabhängige Existenz}(ontological independence): Eine gegebene Idee \emph{F} kann existieren, ohne dass ein Sinnesding, das F ist, existiert, aber umgekehrt kann kein Sinnending, das F ist, existieren, ohne dass die Idee \emph{F} existiert.\footcite[][S.348]{StrobelTranszendenz}
Der Unterschied liegt hierbei darin, dass bei der unabhänigen Existenz jedes Sinnesding, das F ist eine Idee F ontologisch bedingt, also eine Idee F existieren muss, damit ein Sinnesding, das F ist, existieren kann. Bei der Nicht-Immanenz wird diese Unterscheidung nicht so gewählt, sondern in der Weise, dass damit ein Drittes gesetzt werden muss, damit Sinnesdinge überhaupt F-sein aufweisen können. Daher sind die Ideen nicht (direkt) in/an den Sinnesdingen. Ob damit auch behauptet werden kann, dass die Sinnesdinge seperat neben oder unter den Ideen existieren können, wird sich noch zeigen müssen.
Eine weiter Frage, die sich aufwirft, ist, dass es sich bei der Darstellung zwischen Dingen und Ideen um eine mehrfache Verschränkung der Einheit-Vielheit Relation handelt. Wie schon gesehen fallen viele Einzeldinge - wie \emph{ein} Hund - unter die Definition (Idee) des Hundes, also \emph{viele} Hunde unter \emph{einer} Idee des Hundes. Weiter gedacht gibt es dann wiederum \emph{viele} Ideen, die nebeneinander existieren, jedoch ist dabei fraglich, wie diese wiederum durch etwas einendes begriffen oder definiert werden können. Hierbei ist auch noch nicht berücksichtigt, dass die Einzeldinge gerade auch als \emph{eines} verstanden werden müssen. Zudem erföffnet sich die Frage danach, ob es sich bei dieser Konzeption nicht um einen infiniten Regress handelt, da sich hier noch keine, wie in der Dialektik geforderten, Letztbegründung handelt, welche nicht mehr bedingt sein darf. Dies sei für die weitere Erarbeitung vorausgeschickt.
%\subsection{Einheit-Vielheit Grundproblematiken bei diesem Thema der zwei Welten}
%Es müsste erst einmal grundlegend die Notwendigkeit der Verbindung von der Ideenlehre und dem Einen und Vielen hergestellt werden. 
%Das Grundproblem, das hier leider auch zugrunde liegt ist die Verschränkung des Einen und Vielen, wenn man von den Dingen und Ideen spricht, die sich eben in diesen Begriffen jeweils unterschiedlich ausführen lassen. Dabei kommt auch noch die Frage nach der Verschiedenheit und der Einheit desselbigen hinzu.
%Es sollte hier allerdings nicht zu sehr eingegangen werden, da dies sonst viel zu viel werden würde.
%Siehe hierzu Miglioris Ausführungen zu Philebos und Parmenides\footcite[vgl.][S. 110ff.]{Migliori}
%Die Dinge können auf verschiedenen Ebenen als Eines und Vieles beschrieben werden, bzw. kann Einheit und Vielheit an einem Einzelding beschrieben werden.\footcite[vgl.][S. 112]{Migliori}
%\subsection{Chorismos}
%Wie wird der Begriff des Chorismos wörtlich gebraucht, welche Bedeutung hat das? Wie 


%\subsubsection{Methexis und Parousia von Ideen und Sinnesdingen}
%Hier muss von zwei unterschiedlichen Bereichen gesprochen werden, damit man von einer Bezugnahmen diesen Ausmaßes sprechen kann.
%Wenn man diese zwei Bereiche so definiert, dass es eine Teilhabe und eine Anwesenheit des einen in dem anderen gibt, so entsteht das Problem, dass sich eine Art Schema zwischen diese Bereich schleicht, das wiederum zwei neue Grenzen schafft, die es zu überbrücken oder zusammenzuführen gilt. Dies lässt sich leider nicht lösen, da dies in einen infiniten Regress führt. Daher muss dieses Konzept von neu begonnen werden, um diesem Problem entgehen zu können. Dies ist wirklich nur dann zulässig, wenn ein drittes gegeben oder gesucht wird, das diese Bereiche miteinander zu verbinden, das sich auch noch zwischen diese beiden Bereiche fügt.
%Dies wird von Graeser auf S. 147f. behandelt.
%Zentrale, in der Platonforschung verwendete Textpassagen (Gleichnisse)\\
%Überleitung zum Teil dessen, dass es im Höhlen- und Liniengleichnis zu dem Verstädnis kommt, dass man in vielen Lehrwerken von zwei Welten spricht, da es einfacher ist dies so darzustellen. Es muss sich aber hier dem Thema zugewandt werden, ob es sinnvoll ist von diesen zwei Welten sprechen zu können. Bzw. zu welchem Grad dies möglich ist oder auf welcher Ebene man von zwei Welten sprechen kann und ab welchem Punkt nicht mehr. Es gilt daher diese beiden Positionen deutlich voneinander zu trennen und darzustellen, um sich dann dem Punkt zuzuwenden, dass es nur eine Welt nach diesem Motto geben kann. Dabei soll versucht werden, das jeweilige Verständnis von dem, wie man von zwei Welten spricht und sprechen kann, darzustellen und deutlich zu machen, wie und ab wann man den Begriff der \enquote{zwei} Welten verwenden darf und wann dies nicht mehr zulässig ist.

%Für den Verlauf der Arbeit werden die Originalstellen einmal zitiert, um sie an der gegebene Stelle im vorliegenden Text zu behandeln. Wenn dieselbe Textstelle ein weiteres mal auftritt, wird, sofern nicht eine besondere Betonung auf einer Wiederholung der wichtigen Stelle liegt, lediglich die Notation angegeben und damit darauf verwiesen.