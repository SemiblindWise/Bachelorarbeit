\section{Einleitung}
Ich gehe davon aus, dass es kein Chorismos zwischen Ideen und Sinnesdingen gibt. Sofern damit gemint ist, dass es zwei Welten gibt, in denen jeweils Ideen und Dinge sich aufhalten. Es wird dabei noch klar darauf einzugehen sein, was mit Chorismos zwischen Ideen und Sinnesdingen gemeint ist und gemeint sein soll. 
Man könnte hier einen Anfang setzten, wenn man ganz einfach beginnt, also damit, dass mit den Basics angefangen werden muss.\\
Die Transzendenz in Platon(s Dialektik) 
Die Frage nach dem Einen, den Ideen\footcite[][]{Staudacher} und den das Seiende übersteigende Ideen, folglich transzendent.\footcite[][]{Bordt}
Womöglich auch damit in Verbindung die Dialektik, dass die verschiedenen Stufen der Ideen und der Sinnesdingen überwunden werden, bis hin zur obersten Stufe, welche damit dann auch noch überwunden werden kann, oder auch nicht, da die Prinzipien der Dialektik nicht mehr auf diesen Bereich anwendbar sind oder eben schon noch, aber bei der nächsten Stufe dann nicht mehr?\\
Siehe dazu\footcite[vgl.][S. 104ff]{Hirschberger}

Es kann nur eine Welt geben, nicht mehrere nebeneinander (Tim. 31a2-31b2) unterstützt von Vogel in der Weise, dass er festhält, \zitatblock{\enquote{As for present-day philosophy, it shows the tendency to eliminate the existence of a \enquote{transcendent} reality, in so far as this is meant to be a reality, existing \enquote{somewhere beyond} the world in which we live: There is one reality only, this one here and now. Certainly, it can be analysed into its intelligible forms, and this may be called a metaphysic of immanency}\footcite[][S. 161]{Vogel}}
35c sehr wichtig!!!
\zitatblock{\enquote{There are four grounds on which Plato is usually qualified as a dualist: (1) his position in metaphysics, usually refered to as the two worlds theory; (2) his radical antithesis of soul and body, as it is commonly understood; (3) the doctrine of two ultimate principles, which he appears to have held at least in his later years; (4) the so-called cosmic dualism, attributed to him by early Christian writers and still ascribed to him by some present-day scholars}\footcite[][S. 159]{Vogel}}


Es wird wohl schwierig sein, den Gedanken, dass die Seele in der Transzendenz die Ideen bereits geschaut hat, zu verbinden, da hier das Moment der Transzendenz nicht aufgelöst werden kann, ohne auch noch die Seelenlehre zu betrachten.
Die zugrunde liegende Frage ist eigentlich, wie kann die Idee des Schönen, Guten, Gerechten an den schönen, guten, gerechten Dingen teilhaben?\footcite[vgl.][S. 16]{Martin73}

Die Aufteilung im Liniengleichnis muss so aufgefasst werden, dass der größte Teil dem untersten Teil zugesprochen werden muss. Dies macht daher Sinn, da sich das Liniengleichnis auf einen Zielpunkt hin konstruiert, die Idee des Guten. Bei einer Betrachtung von oben herab also muss gelten, dass sich von dem obersten Punkt aus, also der Spitze her - wie eine Art Definitionsbaum - immer mehr Varianten darunter fallen (Man siehe hierzu auch die Dihairese im Sophistes), dieses Prinzip auch für den weiteren Weg \enquote{nach unten} gelten muss, sodass am Ende eine sehr viel breitere Basis besteht und der große Teil des Liniengleichnisses auf der untersten Ebene zu verorten ist. Hiermit entsteht also eine pyramidenähnliche Form. Pyramidenähnlich daher, weil der mittlere Teil nicht ganz einer perfekten Pyramide entsprechen würde, da die zweite und dritte Stufe im Liniengleichnis die gleiche Fläche zukommen müsste. Dies wird weiterhin dadurch unterstützt, dass im Höhlengleichnis am Grunde der Höhle die Schatten an der Höhlenwand jeweils anders interpretiert werden können und damit eine unzählbare Vielheit an Möglichkeiten besteht. Außerdem wird durch jedes Wackeln Zucken des Feuers, welches das Licht auf die Gegenstände wirft, die wiederum den Schatten auf die Höhlenwand werfen, umso zahlreicher. Dabei noch nicht beachtet, dass das Wenden, Drehen und Zusammensetzten der Gegenstände von denjenigen, welche die Gegenstände hinter der Mauer her tragen, nochmals die Zahl der Möglichkeiten anhebt.\\ Diese Ansichtsweise findet ebenfalls Halt, wenn man sich den pythagoreischen Tetraktys ansieht und die im platonischen Denken stark vertretene Dialektik des Einen und Vielen, welche vom Einen beginnt und in das Viele mündet.
%Problem bei dieser Arbeit wird sein, dass ich bereits ein eigenes Verständnis von dem habe, wie Platon zu lesen sein sollte, oder eben wie Platon zu verstehen und auszulegen ist. 

%Noch bevor man die Ideenlehre angehen könnte, noch bevor man sich dem widmen kann, was mit wem in Verbindung steht oder stehen kann und was nicht, muss sich über die Verhältnisse von Einem und Vielem in aller ihrer Formen zugewandt werden.
Die eigentliche Frage, die es zu lösen gilt ist die von der Möglichkeit von unveränderlichen Dingen, die für alles Werdende den Grund angeben. Wo ist der Baum, wenn er gefällt wird, verarbeitet wird, verbrannt wird. Dies ist vermutlich eher eine aristotelische Frage, als eine platonische, wenn man dies so formuliert.
\subsubsection*{Platon Handbuch Horn Müller Söder}
Zu Transzendenz S. 347ff\\
Verschiedene Arten von Aufstiegen. Aufstieg zu etwas Hinreichendem im \emph{Phaidon} (101d5-e1), der Aufstieg zur Idee des Schönen in der Diotima-Rede des Sokrates im \emph{Symposion} (211b5-d1), der Aufstieg zum nicht-vorausgestzten Anfang im Liniengleichnis der \emph{Politeia} (VI 511b3-7); der Aufstieg zur Idee des Guten im Höhlengleichnis der \emph{Politeia} (VII 515c6-516b7); der Aufsieg zum über-himmlischen Ort im Seelen-Mythos des \emph{Phaidros}(246d6-248b5). All diese Aufstiege schließen das Transzendieren ihrer jeweiligen Anfangs- und Zwischenstationen ein. (Aus dem Lateinischen \emph{transcendere} \enquote{übersteigen}, \enquote{überschreiten}) 
Damit ging die Annahme voraus, dass Platon \enquote{Philosophie als Transzendieren} (Hafwassen 1998) porträtiere. 
Es werden in der Regel den Ideen im Allgemeinen (als Entitäten) oder der Idee des Guten im Besonderen Transzendenz zugesprochen.\\
Raum-Zweittranszendenz und im Verhältnis zu ihren sinnlich wahrnehmbaren Partizipanten. Der Idee des Guten wird speziell noch Seinstranszendenz zugeschrieben.\footcite[vgl.][S. 347]{StrobelTranszendenz}
\enquote{[\dots], dass Ideen nicht räumlich lokalisert werden können und die Prädikate, die auf sie zutreffen, von Zeitbezügen frei sind.}\footcite[vgl.][S. 347]{StrobelTranszendenz}
Es bleibt die Frage nach der Transzendenz gegenüber den sinnlich wahrnehmbaren Partizipanten. Es hängt davon ab, wie man diese These verstehen mag.\\
\emph{Nicht-Immanenz}: Eine gegebene Idee \emph{F} ist nicht in/an den Sinnendingen, die F sind.\\
\emph{Unabhängige Existenz}: Eine gegebene Idee \emph{F} kann existieren, ohne dass ein Sinnesding, das F ist, existiert, aber umgekehrt kann kein Sinnending, das F it, existieren, ohne dass die Idee \emph{F} existiert\footcite[][S.348]{StrobelTranszendenz}

%Es bedarf einer eindeutigen und ausführlichen Erläuterung der Fragestellung, um diese an dem Rest des Textes erarbeiten zu können. Es geht dabei darum, dass die Ausgangsfrage im Grunde alle Aspekte der platonischen Philosophie unter sich fasst, sodass man alles - ausgeklammert sei die Ethik - behandeln müsste.

\section{Die Rede von Platons zwei Welten}
Was motiviert eine solche Rede (Problem der Einheit – Vielheit / Wesen – Akzidenz ….)?
Es macht durchaus Sinn anfangs mit den Begriffen von zwei Welten an die Texte heranzugehen, da offensichtlich von zwei Bereichen gesprochen wird, die aufgestellt und unterschieden werden müssen. Dafür bedarf es, wie später noch deutlicher formuliert, in erster Linie dieser Begrifflichkeiten, um überhaupt Formulierungen zu beginnen.
%\subsection{Einheit-Vielheit Grundproblematiken bei diesem Thema der zwei Welten}
%Es müsste erst einmal grundlegend die Notwendigkeit der Verbindung von der Ideenlehre und dem Einen und Vielen hergestellt werden. 
%Das Grundproblem, das hier leider auch zugrunde liegt ist die Verschränkung des Einen und Vielen, wenn man von den Dingen und Ideen spricht, die sich eben in diesen Begriffen jeweils unterschiedlich ausführen lassen. Dabei kommt auch noch die Frage nach der Verschiedenheit und der Einheit desselbigen hinzu.
%Es sollte hier allerdings nicht zu sehr eingegangen werden, da dies sonst viel zu viel werden würde.
%Siehe hierzu Miglioris Ausführungen zu Philebos und Parmenides\footcite[vgl.][S. 110ff.]{Migliori}
%Die Dinge können auf verschiedenen Ebenen als Eines und Vieles beschrieben werden, bzw. kann Einheit und Vielheit an einem Einzelding beschrieben werden.\footcite[vgl.][S. 112]{Migliori}
%\subsection{Chorismos}
%Wie wird der Begriff des Chorismos wörtlich gebraucht, welche Bedeutung hat das? Wie 


%\subsubsection{Methexis und Parousia von Ideen und Sinnesdingen}
%Hier muss von zwei unterschiedlichen Bereichen gesprochen werden, damit man von einer Bezugnahmen diesen Ausmaßes sprechen kann.
%Wenn man diese zwei Bereiche so definiert, dass es eine Teilhabe und eine Anwesenheit des einen in dem anderen gibt, so entsteht das Problem, dass sich eine Art Schema zwischen diese Bereich schleicht, das wiederum zwei neue Grenzen schafft, die es zu überbrücken oder zusammenzuführen gilt. Dies lässt sich leider nicht lösen, da dies in einen infiniten Regress führt. Daher muss dieses Konzept von neu begonnen werden, um diesem Problem entgehen zu können. Dies ist wirklich nur dann zulässig, wenn ein drittes gegeben oder gesucht wird, das diese Bereiche miteinander zu verbinden, das sich auch noch zwischen diese beiden Bereiche fügt.
%Dies wird von Graeser auf S. 147f. behandelt.
%Zentrale, in der Platonforschung verwendete Textpassagen (Gleichnisse)\\
%Überleitung zum Teil dessen, dass es im Höhlen- und Liniengleichnis zu dem Verstädnis kommt, dass man in vielen Lehrwerken von zwei Welten spricht, da es einfacher ist dies so darzustellen. Es muss sich aber hier dem Thema zugewandt werden, ob es sinnvoll ist von diesen zwei Welten sprechen zu können. Bzw. zu welchem Grad dies möglich ist oder auf welcher Ebene man von zwei Welten sprechen kann und ab welchem Punkt nicht mehr. Es gilt daher diese beiden Positionen deutlich voneinander zu trennen und darzustellen, um sich dann dem Punkt zuzuwenden, dass es nur eine Welt nach diesem Motto geben kann. Dabei soll versucht werden, das jeweilige Verständnis von dem, wie man von zwei Welten spricht und sprechen kann, darzustellen und deutlich zu machen, wie und ab wann man den Begriff der \enquote{zwei} Welten verwenden darf und wann dies nicht mehr zulässig ist.

%Für den Verlauf der Arbeit werden die Originalstellen einmal zitiert, um sie an der gegebene Stelle im vorliegenden Text zu behandeln. Wenn dieselbe Textstelle ein weiteres mal auftritt, wird, sofern nicht eine besondere Betonung auf einer Wiederholung der wichtigen Stelle liegt, lediglich die Notation angegeben und damit darauf verwiesen.