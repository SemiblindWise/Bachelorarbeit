\subsection*{Notizen}
% Der Hauptunterschied zwischen Gnoseologie und Epistemologie besteht also darin, dass Gnoseologie sich stärker auf die Struktur und die Möglichkeiten der menschlichen Erkenntnis konzentriert, während Epistemologie sich mehr auf die Natur des Wissens und die Kriterien für wahres Wissen konzentriert. Es handelt sich im Grunde um zwei verschiedene Perspektiven auf die Erkenntnistheorie, die in verschiedenen philosophischen Traditionen entwickelt wurden.
% Das Gute kann nur Eines sein. Es kann nur ein Gutes geben. Einheit in Vielheit. Wenn es mehrere Gute geben würde, wäre es sinnlos von gut zu sprechen. Erst wenn man die Arete einer Sache findet, kann man von seiner Erfüllung sprechen. Beim Guten gibt es keine Alternative mehr. Es ist und muss vollkommmen sein.
Es wird versucht die Epistemologische, bzw. gnoseologsiche Ebene des Verständnisses von zwei Welten aufzuzeigen. Dies sollte recht einfach werden. Die ontologische Ebene hingegen wird schwieriger. Außerdem wird ein eher pragmatischeres Argument geliefert, das an verschiedenen Stellen auffindbar ist 
Siehe dazu Aufstieg Halfwassen S. 251 das \enquote{intelligible Licht} hat einen Doppelcharakter als Wahrheit und als Sein, einen ontologischen und einen gnoseologischen Aspekt.
Alt-Akademische Erledigung der okzidentalen Metaphysik:
Timaios 27c1-53c4 
Philebos 25a3-b1, 31c8-c11

Wichtige Stellen im Timaios 27c1-53c4
Es gebe ein Schönes an und für sich und ein Gutes und Großes (Phaidon 100b)
Nenne es nun Anwesenheit oder Gemeinschaft [\dots], dass vermöge des Schönen alle schönen Dinge schön werden. (Phaidon 100D)
Das ist schwierig in der Hinsicht, ob hier schon von den schönen Dingen gesprochen wird, also dass bereits in ihrem Sein die Dinge schön sind und erst durch die Idee des Schönen erkannt werden können, dass sie schön sind, weil es hier in dem Satz so strukturiert wurde, dass alle schönen Dinge erst schön werden, also voher eigentlich schon als schön verstanden werden dürften. 
Also die Unterscheidung der Erkenntnis des Schönen in oder an den Dingen gegen die ontologische Deutung, dass die Dinge erst dadurch existieren.

Wenn man den dialektischen Ansprüchen Platons gerecht werden möchte, dahingehend dass man diese bis zum Ende hindurchgeht und sich auch über die der Dialektik inhärenten Dialektik, also die Anwendung der Dialektik auf sich selbst, bewusst wird und dieser eben konsequent bleibt, so bleibt am Ende nichts anderes übrig, als dass es die Aufgabe ist, dass sich die Arbeit der Philosophie gerade damit beschäftigen müsste, dass sich die Dialektik mit dem Thema der Ideen als deren Werkzeug auseinandergehalten werden muss. Dabei gilt es also, dass sich auch in Verbindung mit dem Verständnis von Einem und Vielen ein einheitliches Bild zusammenführen lassen muss.\\
Dazu mehr bei Kutschera 2002 S.186f. \footcite[vgl.][S. 186f.]{Kutschera}
\zitatblock{\enquote{Im \emph{Sophisten} wird betont, dass wir in jeder Aussage Begriffe miteinander verknüpfen und dass es Aufgabe der Dialektik ist, zu untersuchen, welche Begriffe bzw. Ideen sich miteinander verbinden lassen und welche nicht. Es geht dabei um eine Untersuchung von Begriffsverhältnissen, und dazu gehört insbesondere auch die Teilhabe einer Idee an einer anderen.}\footcite[][S. 186f.]{Kutschera}}
Getrennt kann nur werden, was vorher zusammengesetzt war (Phaed. 92E, 78C)

Wie grenzt man die Arbeit von dem Thema der Unsterblichkeit der Seele ab, so dass man zwar das Thema behandel kann, aber nicht in die Frage nach dem Tod und der Schau der Ideen vor dem Eintritt in die Erfahrungswelt abzuschweifen und dennoch das Thema fruchtbar zu machen.

\subsection*{Literatur Notizen. Nicht eingeordnet}
\subsubsection*{Natorp Platons Ideenelehre}
Wichtig bei Gottfired Martin\nocite{NatorpIdeenlehre}
\subsubsection*{Der Staat}
Interne Gliederung nicht im Inhaltsverzeichnis: S. 183 Dialektische Begründung der Lehre von den drei Seelenteilen (436-441), S. 185 Direkte Einführung der Ideenlehre (475-486), S. 188 Die Idee des Guten (502-518), S. 201 Der Erkenntnisweg zum Unbedingten (521-534) Wissenschaften und Dialektik
Es fehlt der letzte Teil 596 im Unterkapitel
\zitatblock{\enquote{Im Phaedo wurde zwar schon alles Sein zuletzt begründet in den Grundlegungen des Denkens. Aber hier sollen wir uns gar etwas denken, das über beides, das Denken und das gedachte Sein hinaus liegt. Aber doch wiederum liegt es im Bereiche, in der Gattung des Denkbaren [\dots]. Es ist das Letzte zwar unter dem Erkennbaren (517B), nur eben noch zu erblicken, aber doch erblickt man es, und muss dann zu dem Schluss kommen, dass es der Grund ist von allem Rechten und Schönen, im sichtbaren Reich der erzeugende Grund des Lichts und des Herrn, der Sonne, im Reich des Denkens selbst als Herrscher Wahrheit und Vernunft verleihend}\footcite[][S. 191]{NatorpIdeenlehre}}
\enquote{Wie also ist es gleichwohl über das Sein und über das Denken hinaus? Jedenfalls insofern es das letzte begründende Prinzip des Seins wie des Erkennens ist.}\footcite[][S. 191]{NatorpIdeenlehre}
\enquote{Es vertritt, nicht eine (besondere) Setzung des Denkens, mithin nicht ein (besonderes) Sein noch eine (besondere) Erkenntnis, sondern die Denksetzung selbst, als letztbegründendes Prinzip alles besonderen Seins, aller besonderen Erkenntnis}\footcite[][S. 192]{NatorpIdeenlehre}
\zitatblock{\enquote{Nicht \emph{ein} letztes logisches Prinzip, sondern \emph{das} Prinzip des Logischen selbst und überhaupt, in welchem alle besondere Denksetzung und damit alles besondere Sein [\dots] zuletzt zu begründen ist. [\dots] Das Gesetz ist es allgemein, welches den Gegenstand konstituiert; dieses Gesetzt selbst, dass im Gesetz der Gegenstand zu begründen, ist somit übergegenständlich, auch über alle besonderen Gesetz, nicht ein, sondern \emph{das} Gesetz; woraus zugleich klar wird, inwiefern dies letzte Prinzip sogar über die Erkenntnis der Wissenschaft hinaus ist.}\footcite[vgl.][S. 194f.]{NatorpIdeenlehre}}
\subsubsection*{Parmenides}
Wenn man auf erkannte Dinge nicht verzichten will, dann schieben sich die Ideen an die Stelle der Dinge, womit aber die Methodenbedeutung, die Natorp in den Ideen sieht, verloren geht, womit man die reinsten Begriffe nicht aus, sondern an der Erfahrung gewinnen kann.\footcite[vgl.][S. 222f.]{NatorpIdeenlehre} \enquote{Wem nun die Erfahrung aufhörte Problem zu sein, wem also auch die Idee selbst sich nicht ferner am Problem der Erfahrung als Wissenschaft entwickeln konnte, dem musste die Ideenwelt erstarren zu etwas wie einer andern Welt gegebener Dinge[\dots]}\footcite[vgl.][S. 223]{NatorpIdeenlehre}
\enquote{Die Fehlmeinung, die aus den Ideen Dinge macht [\dots]}\footcite[][S. 225]{NatorpIdeenlehre}
\subsubsection*{Reine Begriffshandbücher}
\enquote{Trennung und Teilnahme sind die zwei Hauptbegriffe der Dialektik. Getrennt kann nur werden, was vorher zusammengesetzt war (Phaed. 92 E, 78 C)}\footcite[][S. 349]{Perls}
Erklärung zu Phaed. 66B: \enquote{Die doxa ist die Verbindung einer Wahrnehmung mit einer Idee. Also ist ihre eine Hälfte nicht ohne die körperliche Wahrnehmung möglich.}\footcite[][S. 350]{Perls}
\enquote{Allerdings hat sich Platon nirgends näher darüber geäußert, wie er die Teilhabe ontologisch im einzelnen aufgefasst hat.}\footcite[][S. 172]{Gigon75}
Ausdehnung der Relation zwischen Erfahrungswelt und Ideen über den Bereich der ethischen Begriffe hinaus, auf die Totalität der Erfahrungswelt. \footcite[vgl.][S. 172]{Gigon75} \enquote{Überall wo eine Vielheit ähnlicher Dinge oder Phänomene auf ein vorgeordnete Einheit hinweist, deutet er nun diese Einheit als Idee.}\footcite[][S. 172]{Gigon75}
In den Spätdialogen \emph{Sophistes, Politikos} und \emph{Philebos} werden nicht mehr die ontologische Dimension der Ideen betrachtet, sondern nur Allgemeinbegriffe, die lediglich die Geordnetheit der Erfahrungswelt aufgezeigt wird und nicht mehr unveränderliche und urbildlich für sich selbst bestehende Wesenheiten gemeint.\footcite[vgl.][S. 174]{Gigon75}

\zitatblock{\enquote{Die Frage nach der Möglichkeit des Wissensgewinns führt bei Platon also zu einem ontologischen und anthropologischen Dualismus: Aus der Auffassung, dass die Welt des sinnenfälligen Werdens keine sichere Erkenntnis vermitteln kann, folgert er die Gegebenheit eines welttranszendenten Bereichs rein idealer Wesenswahrheiten, die nur der geistigen Erkenntnis des Denkens zugänglich sind und von der Seele immer schon apriorisch gewusst werden.}\footcite[][S. 99]{ThurnerDualismus}}
Es wird von Platon das Wort Chorismos nie in Bezug auf Ideen und Einzeldinge verwendet. Er verwendet es lediglich innerhalb der Seelenlehre in der Bestimmung des Todes als \enquote{Erlösung und Trennung der Seele vom Leib} (Phaidon 67d)\footcite[vgl.][S. 282]{ThurnerTrennung} 
Thurner beginnt mit dem Tiamios Dialog (27d-47e) als Impuls für die zwei Welten. \footcite[vgl.][S. 283]{ThurnerTrennung}
\zitatblock{\enquote{Die Sinnendinge haben den Charakter von Abbildern, weil sie vom Weltbildner (dem DEMIURG) nach dem Vorbild (paradeigma) der rein geistigen Ideen (noêta) gestaltet worden sind (Ti 29a—31 b). Dies impliziert ein selbstständiges Sein dieser Urbilder vor und jenseits der Sinnendinge. Zwischen dem Bereich der rein intelligiblen Urbilder und ihrer sinnlich-materiellen Ähnlichkeiten vermittelt die „WELTSEELE" (psychê tu pantos), die vom Demiurgen durch eine Mischung von unveränderlich Unteilbarem und körperlich Teilbarem zusammengesetzt wurde (Ti 35a).}\footcite[][S. 283]{ThurnerTrennung}}
Fraglich wird allerdings, wenn man es so formuliert, dass \zitatblock{\enquote{[d]ie Unterscheidung zwischen dem geistigen Bereich idealen Seins und dem Bereich des Sinnlichen, das sich zwischen Sein und Nicht-sein befindet}\footcite[vgl.][S. 283]{ThurnerTrennung},} dann wird sehr schnell klar, dass es unmöglich ist, dass etwas zwischen Sein und Nicht-Sein zu verorten ist und damit völlig aus jeglicher Logik fällt. Dabei hat Thurner wohl im Sinn, dass das Sonnengleichnis verschiedene Seinsebenen skizziert. \enquote{[\dots] eine Unterscheidung der Gesamtheit des Seins in zwei unterschiedlich bestimmte Seinsbereiche, die aber durch ein Urbild-Abbild-Verhältnis miteinander verbunden sind}\footcite[vgl.][S. 284]{ThurnerTrennung} Diese Unterscheidung ist nicht auf die Erkenntnisstufen bezogen!
Es gibt eine Stelle in Pol. 477b, wo die Meinung als etwas zwischen Seiendem und Nicht-Seiendem gesetzt wird. Dies kommt aber daher, dass Erkenntnis auf das Seiende sich bezieht und Unkenntnis sich auf das Nichtseiende. Diese Behandlung des Nicht-Seienden und der Möglichkeit von falschen Aussagen (also auch Meinungen) kommt erst im Sophistes zu einem Abschluss, dass das Nicht-Seiende dennoch ist, in Abhängigkeit vom Seienden. Daher ist ein \enquote{Dazwischen} nicht zulässig. 
\enquote{Diese absolute Transzendenz des Guten bestimmt Platon näher, indem er am Schluss des Sonnengleichnisses über das Gute sagt, es sei \enquote{jenseits des Seins, dieses an Würde und Kraft überragend.}(509b)}\footcite[][S. 284]{ThurnerTrennung}
\enquote{Das abschließende Höhlengleichnis veranschaulicht die anthropologische, vor allem das Leben der Philosophen betreffenden Konsequenzen der Platonischen Ontologie zweier Seinsbereiche.}\footcite[][S. 284]{ThurnerTrennung} Dieser Punkt wird nicht weiter ausgeführt, also dass es zwei Seinsbereiche im Höhlengleichnis zu geben scheint. Wird hier der Unterschied im Höhlengleichnis zwischen den Dingen im gesamten Gleichnis und den entsprechenden Erkenntnisstufen gesetzte oder das drinnen und draußen von der Höhle unterschieden?
\zitatblock{\enquote{Nicht zuletzt im Sonnengleichnis wird deutlich, dass die Platonische Tendenz zur geistigen \enquote{Abtrennung} jener bleibenden, idealen Wesenseigenschaften, die auf der Stufe der unklaren Sinneserkenntnis noch als \enquote{unabgetrennt} (achöristos; vgl. Resp 524c) wahrgenommen werden, nur richtig verstanden wird, wenn man sie als die eine komplementäre Hälfte eines dialektischen Gedankenzusammenhanges sieht. Platon bringt dadurch zum Ausdruck, dass die idealen Prinzipien des Seins ihre Begründungsfunktion nur dann erfüllen können, wenn sie dem von ihnen Begründeten ontologisch überlegen sind. Paradoxerweise sind die Ideen nur dann und deshalb als TEILHABE-Ursache in den werdenden Sinnendingen anwesend (vgl. Phlb 26e; Ti 28a), weil sie diese zugleich überragen.}\footcite[][S. 284f.]{ThurnerTrennung}}
Dies ist nicht genau genug formuliert worden, sodass nicht klar ist, was hiermit gemeint sein soll und wie es schlussendlich gemeint und zu verstehen sein soll.
Die idealen Prinzipien sind dem, was sie begründen, ontologisch überlegen. Wie soll das verstanden werden?
methexis/Teilhabe der Dinge an den Ideen


Nochmals einsehen Platonisches Philosophieren 70/CD 3067 K89

\subsubsection*{David Ross: Plato's theory of ideas (1951) 70/CD 3067 R823}
Ideenkritik im Parmenides ist eine Selbstkritik Platons (aus Martin S.151) S. 84



\subsubsection*{Rethinking Plato and Platonism 70/CD 3067 W878 R4}
Es kann nur eine Welt geben, nicht mehrere nebeneinander (Tim. 31a2-31b2) unterstützt von Vogel in der Weise, dass er festhält, \zitatblock{\enquote{As for present-day philosophy, it shows the tendency to eliminate the existence of a \enquote{transcendent} reality, in so far as this is meant to be a reality, existing \enquote{somewhere beyond} the world in which we live: There is one reality only, this one here and now. Certainly, it can be analysed into its intelligible forms, and this may be called a metaphysic of immanency}\footcite[][S. 161]{Vogel}}
35c sehr wichtig!!!
\zitatblock{\enquote{There are four grounds on which Plato is usually qualified as a dualist: (1) his position in metaphysics, usually refered to as the two worlds theory; (2) his radical antithesis of soul and body, as it is commonly understood; (3) the doctrine of two ultimate principles, which he appears to have held at least in his later years; (4) the so-called cosmic dualism, attributed to him by early Christian writers and still ascribed to him by some present-day scholars}\footcite[][S. 159]{Vogel}}
Die Forderung von Vogel ist sehr hoch angesetzt, denn \enquote{[w]hat i \emph{do} want is to give to our philosophers a correct view of what Plato held, and thus, by means of a true picture, contribute something to prevent misunderstandings and clear the way to a true metaphysic}\footcite[vgl.][S. 161]{Vogel}
\subsubsection*{Platons Philosophie II Kutschera 70/CD 3067 K97-2}
Parmenides Kapitel S.164ff
\subsubsection*{Vorhaben Kutscheras}
6.2 und 6.3 Eingehen auf ersten Teil des Parmenides. 6.4 Überleitung zum zweiten Teil. 6.5 Grundgedanken seiner Interpretation des zweiten Teils. 6.7 Probleme mit seiner Interpretation.\\
\enquote{Der \emph{Parmenides} ist neben dem \emph{Timaios} [\dots] sicher jener Dialog, der am schwierigsten zu verstehen ist.}\footcite[][S. 161]{Kutschera}
Der Einstieg in den zweiten Teil des Dialogs ist, dass der junge Sokrates behauptet, dass die Paradoxien von Zenon sich auflösen ließen, wenn man Ideen von ihren empirischen Instanzen unterscheide.\footcite[vgl.][S. 161]{Kutschera}
Es muss sich erst noch weiter in der Philosphie geübt werden, was auch an der Übung getan wird, dass Parmenides dies an dem Beispiel des Einen tut. Dabei stellt Parmenides aus der Existenz des Einen, wie aus der Nicht-Existenz des Einen eine Kette von Widerspruchen ab. Es folgt: \enquote{Es selbst (Das Eine) wie die anderen sind, sowohl für sich wie in Beziehung aufeinandner, ales auf alle Weise und sind es nicht, und scheinen es zu sein und scheinen es nicht zu sein.}\footcite[vgl.][S. 162]{Kutschera}

Kutschera setzt S1: \textbf{Sind F und G gegensätzliche Eigenschaften, so gilt nicht: F hat die Eigenschaften G}
Seite 169. Dem Gedanken des Chorismos liegt zunächst die Unterscheidung der Ideen von ihren empirischen Instanzen zugrunde (130b2-3) (Parmenides). Mit Verweis auf Eutyphron 5d1-2 
\enquote{Für Platon, für den Ideen Gegenstände waren, gilt [eine Idee ist verschieden von all ihren empirischen Instanzen], weil eine Idee das ihren empirischen Instanzen Gemeinsame repräsentiert und der Grund für deren Sosein ist.}\footcite[vgl.][S. 169]{Kutschera} Somit S2:\textbf{Eine Idee ist verschieden von all ihren empirischen Instanzen} und S3:\textbf{Die Existenz und Beschaffenheit der Ideen hängt nicht von der Existenz und Beschaffenheit empirischer Dinge ab.} Unabhängikeit der Ideen vom Entstehen und Vergehen ihrer körperlichen Instanzen.\footcite[vgl.][S. 167]{Kutschera}

Ausschlaggebend ist im Parmenides die Stelle 133c7 ff.\footcite[vgl.][S. 179]{Kutschera}
%Eine interessante Sache, die von Hirschberger hervorgebracht wird, ist die Unterscheidung von zwei Möglichkeiten der Diairesis entweder von oben nach unten, wie es im \emph{Sophistes} durchgeführt worden ist, aber auch von unten nach oben, indem man das Allgemeine aus dem Individuellen heraushebt, um schlussendlich an dem obersten Absoluten anzukommen.\footcite[vgl.][S. 106f.]{Hirschberger} 
%Es geht mit dieser Dialektik darum, dass es um die Erklärung des gesamten Seins durch Aufweis der Strukturidee der Welt geht.\footcite[vgl.][S. 107]{Hirschberger}
%\enquote{Und schließlich geht es in ihr, sofern sie das ganze Sein zusammenschaut und in ihm überall die Parousie der Idee des Guten entdeckt, um den Nachweis der Fußspur Gottes im All.}\footcite[][S. 107]{Hirschberger}